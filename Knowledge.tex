% Options for packages loaded elsewhere
\PassOptionsToPackage{unicode}{hyperref}
\PassOptionsToPackage{hyphens}{url}
%
\documentclass[
  10pt,
  letterpaper,
  twoside]{scrbook}

\usepackage{amsmath,amssymb}
\usepackage{setspace}
\usepackage{iftex}
\ifPDFTeX
  \usepackage[T1]{fontenc}
  \usepackage[utf8]{inputenc}
  \usepackage{textcomp} % provide euro and other symbols
\else % if luatex or xetex
  \usepackage{unicode-math}
  \defaultfontfeatures{Scale=MatchLowercase}
  \defaultfontfeatures[\rmfamily]{Ligatures=TeX,Scale=1}
\fi
\usepackage{lmodern}
\ifPDFTeX\else  
    % xetex/luatex font selection
  \setmainfont[ItalicFont=EB Garamond Italic,BoldFont=EB Garamond
Bold]{EB Garamond Math}
  \setsansfont[]{Europa-Bold}
  \setmathfont[]{Garamond-Math}
\fi
% Use upquote if available, for straight quotes in verbatim environments
\IfFileExists{upquote.sty}{\usepackage{upquote}}{}
\IfFileExists{microtype.sty}{% use microtype if available
  \usepackage[]{microtype}
  \UseMicrotypeSet[protrusion]{basicmath} % disable protrusion for tt fonts
}{}
\usepackage{xcolor}
\usepackage[left=1in, right=1in, top=0.8in, bottom=0.8in,
paperheight=9.5in, paperwidth=6.5in, includemp=TRUE, marginparwidth=0in,
marginparsep=0in]{geometry}
\setlength{\emergencystretch}{3em} % prevent overfull lines
\setcounter{secnumdepth}{5}
% Make \paragraph and \subparagraph free-standing
\ifx\paragraph\undefined\else
  \let\oldparagraph\paragraph
  \renewcommand{\paragraph}[1]{\oldparagraph{#1}\mbox{}}
\fi
\ifx\subparagraph\undefined\else
  \let\oldsubparagraph\subparagraph
  \renewcommand{\subparagraph}[1]{\oldsubparagraph{#1}\mbox{}}
\fi


\providecommand{\tightlist}{%
  \setlength{\itemsep}{0pt}\setlength{\parskip}{0pt}}\usepackage{longtable,booktabs,array}
\usepackage{calc} % for calculating minipage widths
% Correct order of tables after \paragraph or \subparagraph
\usepackage{etoolbox}
\makeatletter
\patchcmd\longtable{\par}{\if@noskipsec\mbox{}\fi\par}{}{}
\makeatother
% Allow footnotes in longtable head/foot
\IfFileExists{footnotehyper.sty}{\usepackage{footnotehyper}}{\usepackage{footnote}}
\makesavenoteenv{longtable}
\usepackage{graphicx}
\makeatletter
\def\maxwidth{\ifdim\Gin@nat@width>\linewidth\linewidth\else\Gin@nat@width\fi}
\def\maxheight{\ifdim\Gin@nat@height>\textheight\textheight\else\Gin@nat@height\fi}
\makeatother
% Scale images if necessary, so that they will not overflow the page
% margins by default, and it is still possible to overwrite the defaults
% using explicit options in \includegraphics[width, height, ...]{}
\setkeys{Gin}{width=\maxwidth,height=\maxheight,keepaspectratio}
% Set default figure placement to htbp
\makeatletter
\def\fps@figure{htbp}
\makeatother
% definitions for citeproc citations
\NewDocumentCommand\citeproctext{}{}
\NewDocumentCommand\citeproc{mm}{%
  \begingroup\def\citeproctext{#2}\cite{#1}\endgroup}
\makeatletter
 % allow citations to break across lines
 \let\@cite@ofmt\@firstofone
 % avoid brackets around text for \cite:
 \def\@biblabel#1{}
 \def\@cite#1#2{{#1\if@tempswa , #2\fi}}
\makeatother
\newlength{\cslhangindent}
\setlength{\cslhangindent}{1.5em}
\newlength{\csllabelwidth}
\setlength{\csllabelwidth}{3em}
\newenvironment{CSLReferences}[2] % #1 hanging-indent, #2 entry-spacing
 {\begin{list}{}{%
  \setlength{\itemindent}{0pt}
  \setlength{\leftmargin}{0pt}
  \setlength{\parsep}{0pt}
  % turn on hanging indent if param 1 is 1
  \ifodd #1
   \setlength{\leftmargin}{\cslhangindent}
   \setlength{\itemindent}{-1\cslhangindent}
  \fi
  % set entry spacing
  \setlength{\itemsep}{#2\baselineskip}}}
 {\end{list}}
\usepackage{calc}
\newcommand{\CSLBlock}[1]{\hfill\break\parbox[t]{\linewidth}{\strut\ignorespaces#1\strut}}
\newcommand{\CSLLeftMargin}[1]{\parbox[t]{\csllabelwidth}{\strut#1\strut}}
\newcommand{\CSLRightInline}[1]{\parbox[t]{\linewidth - \csllabelwidth}{\strut#1\strut}}
\newcommand{\CSLIndent}[1]{\hspace{\cslhangindent}#1}

\setlength\heavyrulewidth{0ex}
\setlength\lightrulewidth{0ex}
\usepackage[automark]{scrlayer-scrpage}
\clearpairofpagestyles
\cehead{
  \leftmark
  }
\cohead{
  \rightmark
}
\ohead{\bfseries \pagemark}
\cfoot{}
\makeatletter
\newcommand*\NoIndentAfterEnv[1]{%
  \AfterEndEnvironment{#1}{\par\@afterindentfalse\@afterheading}}
\makeatother
\NoIndentAfterEnv{itemize}
\NoIndentAfterEnv{enumerate}
\NoIndentAfterEnv{description}
\NoIndentAfterEnv{quote}
\NoIndentAfterEnv{equation}
\NoIndentAfterEnv{longtable}
\renewcommand{\chaptermark}[1]{\markboth{#1}{}}
\renewcommand{\sectionmark}[1]{\markright{#1}}
\makeatletter
\@ifpackageloaded{bookmark}{}{\usepackage{bookmark}}
\makeatother
\makeatletter
\@ifpackageloaded{caption}{}{\usepackage{caption}}
\AtBeginDocument{%
\ifdefined\contentsname
  \renewcommand*\contentsname{Table of contents}
\else
  \newcommand\contentsname{Table of contents}
\fi
\ifdefined\listfigurename
  \renewcommand*\listfigurename{List of Figures}
\else
  \newcommand\listfigurename{List of Figures}
\fi
\ifdefined\listtablename
  \renewcommand*\listtablename{List of Tables}
\else
  \newcommand\listtablename{List of Tables}
\fi
\ifdefined\figurename
  \renewcommand*\figurename{Figure}
\else
  \newcommand\figurename{Figure}
\fi
\ifdefined\tablename
  \renewcommand*\tablename{Table}
\else
  \newcommand\tablename{Table}
\fi
}
\@ifpackageloaded{float}{}{\usepackage{float}}
\floatstyle{ruled}
\@ifundefined{c@chapter}{\newfloat{codelisting}{h}{lop}}{\newfloat{codelisting}{h}{lop}[chapter]}
\floatname{codelisting}{Listing}
\newcommand*\listoflistings{\listof{codelisting}{List of Listings}}
\makeatother
\makeatletter
\makeatother
\makeatletter
\@ifpackageloaded{caption}{}{\usepackage{caption}}
\@ifpackageloaded{subcaption}{}{\usepackage{subcaption}}
\makeatother
\ifLuaTeX
  \usepackage{selnolig}  % disable illegal ligatures
\fi
\IfFileExists{bookmark.sty}{\usepackage{bookmark}}{\usepackage{hyperref}}
\IfFileExists{xurl.sty}{\usepackage{xurl}}{} % add URL line breaks if available
\urlstyle{same} % disable monospaced font for URLs
\hypersetup{
  pdftitle={Knowledge},
  pdfauthor={Brian Weatherson},
  hidelinks,
  pdfcreator={LaTeX via pandoc}}

\title{Knowledge}
\usepackage{etoolbox}
\makeatletter
\providecommand{\subtitle}[1]{% add subtitle to \maketitle
  \apptocmd{\@title}{\par {\large #1 \par}}{}{}
}
\makeatother
\subtitle{A Human Interest Story}
\author{Brian Weatherson}
\date{2024}

\begin{document}
\frontmatter
\maketitle

\renewcommand*\contentsname{Table of contents}
{
\setcounter{tocdepth}{2}
\tableofcontents
}
\setstretch{1.1}
\mainmatter
\bookmarksetup{startatroot}

\chapter*{Preface}\label{sec-preface}
\addcontentsline{toc}{chapter}{Preface}

\markboth{Preface}{Preface}

Over the years I've written many papers defending an idiosyncratic
version of interest-relative epistemology. This book collects and
updates the views I've expressed over those papers.

My original plan was not a collection of papers, that would hardly add
much value over a well designed webpage, but a book that was largely
structured out of different sections of different papers. My thought was
that I had something like a working theory between the papers, and what
would be useful would be to blend the sentences, paragraphs, and even
whole sections from them into a coherent narrative. Some of that plan
has been retained. Most sections in Chapters \hyperref[sec-ratbel]{8}
and \hyperref[sec-evidence]{9} are very similar to sections in one or
other previously published paper. But the bulk of the book is new. In
putting the pieces together, I realised that I'd changed my mind about
enough things, and needed to express myself very differently about
enough other things, so as to make it worth rewriting much of what I
had. The result is that this is about 60\% a new book, 20\% a heavily
edited version of previous material, and 20\% lightly edited
republishing of previous material. Even that last 20\% has some value I
think - it helps to see those points in the context of an overall story
- but this is mostly a new book.

Interest-relative epistemologies all start in roughly the same way. A
big part of what makes knowledge important is that it rationalises
action. But for almost anything we purportedly know, there is some
action that it wouldn't rationalise. I know what I had for breakfast,
but I wouldn't take a bet at billion to one odds about it. Knowledge has
practical limits. The first idiosyncratic feature of my version of
interest-relative epistemology is how those limits are identified. Other
interest-relative philosophers typically say that the limits have to do
with stakes; in \emph{high stakes} situations knowledge goes away.
That's no part of my view. I think knowledge goes away in \emph{long
odds} situations. High stakes situations are almost always are long odds
situations, for reasons to do with the declining marginal utility of
money. But the converse isn't true. On my view, knowledge often goes
away in cases where it is trivial to check before action. This idea,
that interests matter in long odds cases, and not just in high stakes
cases, is the main constant in what I've written on interest-relativity
over the years.

But there are three other respects in which the view I'm going to set
out and defend in this book is very different from the view I set out in
older papers.

I used to identify the practical limits on knowledge with cases where
relying on the purported knowledge would get the wrong answer. I
focused, that is, on the outputs of inquiry. Knowledge goes away if
relying on it would lead one to make a mistake. I now think I was
looking at the wrong end of inquiry. Knowledge goes away if the thinker
starts conducting an inquiry where the purported knowledge is an
inappropriate starting point, and inappropriate for the special reason
that it might be false. Now one way we can tell that something is a bad
starting point is that starting there will mean we end up at the wrong
place. But it's not the only way. Sometimes a bad starting point will
lead to the right conclusion for the wrong reasons. As the Nyāya
philosophers argued, rational inquiry starts with knowledge. If it would
be irrational to start this inquiry with a particular belief, that
belief isn't knowledge.

Not all inquiries are practical inquiries, but many are. And practical
inquiries are usually going to be at the center of attention in this
book. But what is someone trying to figure out when they conduct a
practical inquiry? I used to think that they were trying to figure out
which option maximised expected utility, and to a first approximation
identified knowledge with those things one could conditionalise on
without changing the option that maximised expected utility. As noted in
the previous paragraph, I no longer think that we can identify knowledge
with what doesn't change our verdicts. But more importantly, I no longer
think that expected utility maximisation is as central to practical
inquiry as I once did. There are theoretical reasons from game theory
that raise doubts about expected utility maximisation being the full
theory of rational choice. Weak dominance reasoning is part of our
theory of rational choice, and can't be modelled as expected utility
maximisation. Perhaps some kinds of equilibrium seeking are parts of
practical inquiry, and can't be modelled as expected utility
maximisation. There are also very practical reasons to think that
practical inquiry doesn't aim at expected utility maximisation. When
there are a lot of very similar options - think about selecting a can
from a supermarket shelf - and it's more trouble than it's worth to
figure out which of them maximises expected utility, it's best to ignore
the differences between them and just pick. As I'll argue in
\hyperref[sec-ties]{Chapter 6}, this makes a big difference to how
interests and knowledge interact.

In the version of interest-relativity that I'm defending here,
everything in epistemology is interest-relative. Knowledge, rational
belief, and evidence are all interest-relative. But they are all
interest-relative in slightly different ways. The main aim here is to
defend the interest-relativity of knowledge. A common objection to
interest-relative theories of knowledge is that they can't be extended
into theories of all the things we care about in epistemology. Here I
try to meet that challenge. The way I do so is a little messy. It would
be nice if there was some part of epistemology that's
interest-invariant, as I used to think, or if all the interest-relative
notions were interest-relative in the same way, as other
interest-relative epistemologists argue. For better or worse, that's not
the view I'm defending. Interests matter throughout epistemology, and we
just have to go case by case to figure out how and why they matter.

The ideas from the last three paragraphs are totally absent from my
earliest work - several times they are explicitly rejected - but become
more prevalent as the years go on. This is the first time I've defended
them all in one place. And I think they are all necessary to make the
theory I want to defend hang together.

Here is a list of these papers on interest-relativity that I've
mentioned a few times already.

\begin{itemize}
\tightlist
\item
  ``Can We Do Without Pragmatic Encroachment?'' \emph{Philosophical
  Perspectives} 19 (2005): 417-443.
\item
  ``Defending Interest-Relative Invariantism,'' \emph{Logos and
  Episteme} 2 (2011): 591-609.
\item
  ``Games and the Reason-Knowledge Principle,'' \emph{The Reasoner}
  6(2012): 6-7.
\item
  ``Knowledge, Bets and Interests,'' in \emph{Knowledge Ascriptions},
  edited by Jessica Brown and Mikkel Gerken, Oxford University Press,
  2012, 75-103.
\item
  ``Reply to Blackson'',~\emph{Journal of Philosophical Research} 46
  (2016): 73-75.
\item
  ``Games, Beliefs and Credences,'' \emph{Philosophy and
  Phenomenological Research} 92 (2016): 209-236.
\item
  ``Reply to Eaton and Pickavance,'' \emph{Philosophical Studies} 173
  (2016): 3231-3233.
\item
  ``Interest-Relative Invariantism,'' in \emph{Routledge Handbook of
  Epistemic Contextualism}, edited by Jonathan Jenkins Ichikawa,
  Routledge, 2017, 240-253.
\item
  ``Interests, Evidence and Games,'' \emph{Episteme} 15 (2018): 329-344.
\end{itemize}

I wrote most of this manuscript while on sabbatical at the Australian
National University in the first half of 2019, and I'm very grateful for
their hospitality while I was a visitor there.

Support for that sabbatical came from the Marshall M. Weinberg
Professorship at the University of Michigan. And I'm once again
incredibly grateful for the support Marshall has given to philosophy,
and to many other disciplines, at the University of Michigan.

Many of the papers were drafted, and workshopped, while I was a Visiting
Fellow at the Arché Research Centre at the University of St Andrews. You
could probably fill a book this long with the mistakes I was talked out
of in formal and informal meetings in St Andrews. And it was a real
privilege to have been part of that community for a decade.

In Winter 2020 I taught a graduate seminar based off a draft of this
manuscript at the University of Michigan. I received a lot of valuable
feedback from the students in that seminar. I suspect I would have
received even more valuable feedback had we not had to scramble to
convert the course into a virtual event halfway through the semester.
But I'm still very grateful for what I learned from them over that
course.

I've presented this material at many departments and workshops, and am
very grateful to the feedback I've received on each occasion. Most of
the book was presented in one form or another at Arché. As well, parts
have been presented at the 2012 Rutgers Epistemology Conference, the
2017 \emph{Episteme} Conference, a workshop on pragmatic encroachment
organised by Arizona State University in 2017, the University of Sydney,
the Australian National University, and the 2020 Ranch Metaphysics
Workshop. I've also had valuable feedback on ideas in the book over the
years from Michael Almeida, Charity Anderson, Thomas Blackson, Jessica
Brown, Stewart Cohen, Josh Dever, Tom Donaldson, Tamar Szabó Gendler,
Peter Gerdes, Katherine Hawley, John Hawthorne, Jonathan Ichikawa, Jon
Kvanvig, Jennifer Lackey, Barry Lam, Harvey Lederman, Matthew McGrath,
Sarah Moss, Jennifer Nagel, Shyam Nair, Daniel Nolan, Ángel Pinillos,
Jacob Ross, Mark Schroeder, Kieran Setiya, Ernie Sosa, Levi Spectre,
Robert Stalnaker, Jason Stanley, and Matthew Weiner.

And of course I've got more feedback, and more useful feedback, from
Ishani Maitra than from anyone, or any place, else. She's had to listen
to, and often talk me out of, any number of dead ends, false leads, and
outright mistakes, on this topic for the best part of two decades. If
there's anything in what follows that manages to be true, useful, and
new, it's thanks to her feedback, advice, and support.

\bookmarksetup{startatroot}

\chapter{Overture}\label{sec-overture}

The core thesis of this book is that what a person knows is sensitive to
what their interests are, and in particular to what inquiries they are
engaged in. The thesis is designed to resolve a puzzle about the nature
of inquiry. Inquiry has to start somewhere, and a natural place to start
is with what one knows. If one is planning a meal for friends, and
choosing what to make, it's natural to start with what one knows about
what ingredients are on hand or easily available, what the friends like,
what dietary preferences and restrictions they have, and so on.

Now we face a puzzle. Either we identify knowledge with absolute
certainty or we do not. If we do, then inquiry can barely get started.
If one knows anything with absolute certainty, then it is at most
trivialities like instances of the law of identity. That won't be enough
to get going on planning dinner. So let's say we do not identify
knowledge with absolute certainty, and instead pick some particular
level of certainty below that. Then there will be propositions that are
more certain than that threshold, but which one should not use this
particular inquiry. For instance, there will be cases where one's
evidence that a particular friend is not allergic to peanuts is just
above that threshold, but given the potentially lethal consequences of
getting it wrong, this isn't something that should be taken as a
starting point in inquiry.

The solution is to identify knowledge with a level of certainty which
varies with the nature of the inquiry. In particular, it varies both
with how important it is to get the inquiry right (very important in the
case of the allergy), and with how hard it would be to get further
information relevant to the inquiry.

I'm not the first to defend such a view; there is a thriving literature
on interest-relative theories of knowledge like the one I'm defending
here. But for a long time it was a remarkably curious literature.
Interest-relative theories were discussed everywhere and endorsed
virtually nowhere. It's possible things changed around the time of the
COVID-19 pandemic; after 2020 there were more positive discussions of
interest-relativity.\footnote{See, for instance: Kim
  (\citeproc{ref-Kim2023}{2023}), Gao (\citeproc{ref-Gao2023}{2023}),
  Schmidt (\citeproc{ref-Schmidt2024}{forthcoming}), Steglich-Petersen
  (\citeproc{ref-Steglich-Petersen2024}{2024}), Wu
  (\citeproc{ref-Wu2024}{forthcoming}), and Ye
  (\citeproc{ref-Ye2024}{forthcoming}). That's about as many people
  defending interest-relative theories in one year as defended them for
  the first 15 years since they were introduced in Fantl \& McGrath
  (\citeproc{ref-FantlMcGrath2002}{2002}).} Before then,
interest-relative theories had a ratio of discussion to endorsement that
philosophy hadn't seen since Lewis put concrete modal realism on the
agenda.

The terminology that is used to describe the debate about
interest-relativity is striking. The interest-relative view is usually
opposed to the `purist' or `traditionalist' view. I'm not going to dive
into the literature on which views get described as `pure' or `impure',
but I wanted to pause a bit over `tradition'. This is a particularly
curious choice of word, and I think its curiosity is related to the
strange shape of the literature around interest-relativity.

The recent literature on interest-relativity was kick-started by three
works in the early 2000s. First was Jeremy Fantl and Matthew McGrath's
paper ``Evidence, Pragmatics and Justification'', published in \emph{The
Philosophical Review} in 2002. Then came two books from Oxford
University Press: \emph{Knowledge and Lotteries} by John Hawthorne in
2003, and \emph{Knowledge and Practical Interests} by Jason Stanley in
2004. Now these works are, by standards of recent epistemology, from
quite a long time ago. That is to say, two decades is a long time in
epistemology. Compare, for instance, the literature on the idea that
safety is central to the theory of knowledge. The idea that safety is
important plays a crucial role in a series of works from the late 1990s
and early 2000s by David Lewis, Timothy Williamson, Ernest Sosa, and
Duncan Pritchard.\footnote{I'll have a lot more to say about safety in
  what follows. For now, a rough definition of it will do. A person's
  belief is safe just in case they couldn't easily have gone wrong in
  forming that belief. And safety-relative epistemology says that only
  safe beliefs amount to knowledge, and this plays an important role in
  explaining knowledge.} And it became a central feature of a lot of
epistemological theorising very quickly. But safety-relative
epistemology is really only a few years older than interest-relative
epistemology. So why is one of these traditional and the other not?

One possible answer is that while safety was a new idea, it struck
epistemologists as similar to older ideas. Safety looks a lot like the
sensitivity condition that Robert Nozick
(\citeproc{ref-Nozick1981}{1981}) had argued plays a central role in the
theory of knowledge. Sosa (\citeproc{ref-Sosa1999}{1999}) plays up this
similarity, framing safety as a kind of converse of sensitivity. And
safety looks like a kind of reliability condition, so it is continuous
with twentieth century work on reliabilism. So while safety theories are
new, they have things that look like precursors. But to a lot of
epistemologists, interest-relative theories looked novel. It wasn't just
that they offered a new account of what affects knowledge; it was that
they offered a view that came out of nowhere.

If that was the impression that epistemologists had, it was mistaken.
There are precursors to contemporary interest-relative views, and
looking at them is helpful for thinking about why one might want to
endorse an interest-relative view. I'm going to focus on two of these
precursors, one from Hellenistic philosophy and one from Medieval
philosophy.\footnote{A quick note on sources. This is not at all a work
  of historical scholarship, and I'm not in a position to write such a
  work. So everything I cite here is going to be a contemporary
  secondary (or tertiary) source. I do hope in the future there will be
  more work which looks at the relationship between these historically
  important figures and contemporary views, but that work will have to
  be done by someone with a different skill set to mine.}

Philo of Larissa lived from around 159 BCE to around 83 BCE, and was the
last sceptical head of Plato's Academy.\footnote{My main sources here
  are the Stanford Encyclopedia of Philosophy entry on Philo
  (\citeproc{ref-Brittain2021}{Brittain \& Osorio, 2021}) and the
  chapters on scepticism in Peter Adamson's book on Hellenistic
  philosophy (\citeproc{ref-Adamson2015}{Adamson, 2015}). I'm
  particularly drawing on section 3.3 of the SEP entry, and chapters 16
  and 17 of Adamson's book.} He held a number of views over his life,
but the one that's important here is his `mitigated scepticism'. The
sceptics faced a challenge: if no one knows anything, and indeed no one
should believe anything, then it seems rational action is impossible.
But surely some acts are rational, or at least more rational than other
acts. What can be done?

Philo's response is to say that while it is true that nothing can be
known, it can be rational to assent to certain `persuasive impressions'.
Action that is based in the right way on an impression that is really
persuasive (and not just one that actually persuades) can be rational.
Moreover, says Philo, how much evidence one needs to be properly
persuaded can vary with differences in what's at stake with the action.
As Adamson puts it,

\begin{quote}
Like Arcesilaus, Philo suggests that these impressions will be used as a
practical guide by the Skeptic. But he went further, observing that the
standards we use will differ depending on how high the stakes are. In
the normal course of affairs one bit of evidence will suffice. For
instance, if I'm looking for the giraffes, I'll just ask another
zoo-visitor and follow their directions. But what if it is really
important---if, say, I need to be at the giraffe enclosure in five
minutes to pay a ransom to the giraffe-nappers who are demanding £1
million for the safe return of Hiawatha, who just happens to be my
favorite giraffe? Then I will want to make extra sure.
(\citeproc{ref-Adamson2015}{Adamson, 2015: 112})
\end{quote}

Now Philo (probably) doesn't move from this to an interest-relative
theory of knowledge. But look how close he gets. He thinks that the norm
of belief, or at least the norm of the thing that plays the same role in
his philosophical system as belief plays in ours, is interest-relative.
All you have to add to get an interest-relative theory of knowledge is
that knowledge is the norm of (the thing that plays the functional role
of) belief.

Jumping ahead a millennium and a half, our next stop is with the
epistemology of medieval philosopher Jean Buridan. I'm going to draw
extensively here on Robert Pasnau's discussion of medieval epistemology
in his \emph{After Certainty}. Pasnau credits Buridan with introducing
``what would become the canonical three-level distinction between
absolute, natural, and moral certainty.''
(\citeproc{ref-Pasnau2017}{Pasnau, 2017: 32}). The last of these ``moral
certainty'', is the most important one here. This isn't quite Buridan's
phrase, he talks about moral evidentness, but he seems to be the causal
origin of the introduction of the phrase ``moral certainty'' (or its
equivalent in other languages) into western European discourse. And it's
particularly interesting to the story here to see what kind of problem
this notion is meant to solve.

\begin{quote}
There is still another, weaker evidentness, which suffices for acting
well morally. This goes as follows: if someone, having seen and
investigated all the attendant circumstances that one can investigate
with diligence, judges in accord with the demands of such circumstances,
then that judgment will be evident with an evidentness sufficient for
acting well morally---even if that judgment were false on account of
invincible ignorance concerning some circumstance. For instance, it
would be possible for a judge to act well and meritoriously by hanging
an innocent man because through testimony and other documents it
sufficiently appeared to him in accord with his duty that that good man
was a bad murderer. (Buridan, as quoted in Pasnau
(\citeproc{ref-Pasnau2017}{2017: 34}))
\end{quote}

Note particular the phrase `the demands of such circumstances'.
Buridan's notion here is clearly interest-relative. What it takes to
properly judge a defendant guilty of murder is considerably more than
what it takes to judge that someone broke a promise. The difference
between the misdeeds, while in the first instance a moral difference,
matters to the applicability of this epistemic concept.

Now Buridan does not have an interest-relative account of knowledge.
After all, the very example he uses to introduce this interest-relative
concept is one where the belief is false, and hence not knowledge. This
would change over time. Eventually John Wilkins, writing in the 17th
Century, would take moral certainty to be the standard for knowledge
(\citeproc{ref-Pasnau2017}{Pasnau, 2017: 218}). Wilkins is important to
the history of science as one of the founders of the Royal Society. And
he is important to the history of epistemology because he starts the
tradition of centering epistemology around attainable norms. Here is how
Pasnau puts the point.

\begin{quote}
Wilkins in particular, in his small way, takes what can retrospectively
be seen as a decisive step, because he both rejects the principle of
proportionality in favor of a broad scope for absolute belief and
identifies the whole range of such belief with knowledge. For, even as
he continues to associate knowledge with certainty, he allows that mere
moral certainty is good enough, treating mathematical, physical, and
moral as three different kinds of knowledge and thus locating the
threshold for knowledge not at intellectual compulsion but at the
absence of reasonable doubt: ``that kind of assent which does arise from
such plain and clear evidence as does not admit of any reasonable cause
of doubting is called knowledge or certainty.''
(\citeproc{ref-Pasnau2017}{Pasnau, 2017: 43})
\end{quote}

The `principle of proportionality' here is the idea that the better
one's evidence for a proposition is, the stronger one's belief in that
proposition should be. What's distinctive in Wilkins is that he thinks
one can have absolute belief in a mere moral certainty. This violates
proportionality because if one's belief is a mere moral certainty, then
the evidence for it could be improved. But since it is an absolute
belief, the belief couldn't get stronger.

What's distinctive about Wilkins is not the use of moral certainty in
epistemology. That's there in Buridan 300 years earlier. What's
distinctive is the central role he gives it. And as Pasnau reads the
situation, the approach taken by Wilkins becomes orthodox for the next
300 years. The alternative option, one that Pasnau prefers, is to focus
on what the epistemological ideal is, and on how close we can get to
attaining that ideal. You can read at least some contemporary
probabilists as working in the tradition - one that was common before
Wilkins - of thinking that only maximally supported beliefs get the
maximal level of belief. But this is definitely not the mainstream view
for the last few centuries. The mainstream view is that there are these
important, absolute, concepts that can be attained even though one's
evidential situation could in principle improve further. And these
concepts are closely tied to knowledge. And, most strikingly, the one
that is mostly tied to knowledge is originally introduced as an
interest-relative concept.

In both Philo of Larissa, and in the tradition that runs from Buridan to
Wilkins and beyond, interest-relative epistemic concepts play central
roles. There is no figure here who literally endorses every aspect of
the contemporary interest-relative view. But the precursors are there.
Indeed, they are there at some of the earliest sightings of what we
might, in current terminology, call fallible epistemologies. If
anything, I suspect the idea that an epistemology can be fallibilist and
interest-invariant is the more recent innovation. Rather than dive too
deeply into those historical waters, let's turn to a connection between
Buridan's epistemology and (a particular strand in) Indian epistemology:
the place of action theory in epistemology. What worries Buridan is
whether a certain action, hanging an innocent man, can be given an
epistemological defence. Buridan isn'tthe first philosopher to see a
tight connection between epistemology and action theory.

The fifth century philosopher Vātsyāyana is known for his commentary on
the first or second century Nyāya-sūtra.\footnote{My source for
  everything here is Peter Adamson's and Jonardon Ganeri's
  \emph{Classical Indian Philosophy}
  (\citeproc{ref-AdamsonGaneri2020}{Adamson \& Ganeri, 2020}).} In this
commentary he offers a number of anti-sceptical arguments. This one is
most interesting to the story here.

\begin{quote}
For Vātsyāyana, the purpose of knowledge is indeed crucially important.
He begins his commentary by saying that knowledge is needed in order to
secure any desired objective (artha). Each of us exerts effort only for
the sake of achieving such an objective. Here one might think of an idea
we encountered in Mīmāṃsā, that it is a sacrificer's desire that makes a
ritual incumbent upon the sacrificer. No desire, no action. Now
Vātsyāyana adds: no knowledge, no result! After all, how can you get
what you want when you literally don't know what you're doing?
Vātsyāyana invokes the point again later on, when he responds to the
standard skeptical argument that any means of knowledge must be ratified
by some further means of knowledge, leading to a regress. Thus, the
skeptic is suggesting, we cannot trust a pramāṇa like perception unless
some further perception tells us that it is trustworthy. No, replies
Vātsyāyana. If this were true then ``the activities of practical life''
would be impossible, since the only way we ever achieve anything that we
want is by knowing how to get it. This applies to mundane goals like
wealth and pleasure, and to more exalted goals too. Nyāya competes with
the Buddhists not only on the epistemological front, by refuting
skeptical arguments like the one just mentioned, but also on what we
may, with apologies to Monty Python, call the liberation front. The
elimination of suffering, promised by Buddhists and Naiyāyikas alike, is
one more objective that can be achieved through knowledge and through
knowledge alone. (\citeproc{ref-AdamsonGaneri2020}{Adamson \& Ganeri,
2020: 170})
\end{quote}

More bluntly, the argument is that some actions are rational, only
actions based on knowledge are rational, and so we have some knowledge,
contra scepticism. Unlike Vātsyāyana I'm not in the business of arguing
against scepticism. But this is an excellent anti-sceptical argument.
That's not just because it's sound, and persuasive, though it's both.
It's because it derives anti-sceptical conclusions from the practical
nature of knowledge. It grounds the anti-scepticism where is should be
grounded, in the practical nature of knowledge.

The Nyāya philosophers, like Vātsyāyana, are relevant to this story for
another reason. As well as closely connecting knowledge with action,
they connect it closely with inquiry. And this book, like many
contemporary philosophers, takes the same approach. Jane Friedman
Friedman (\citeproc{ref-Friedman2024}{2024b}) has developed a detailed
account of what inquiry is and how it relates to epistemology. Elise
Woodard (\citeproc{ref-Woodard2021}{2020}) and Arienne Falbo
(\citeproc{ref-Falbo2021}{2021}) have some persuasive criticisms of
particular details of Friedman's views, but enough of the picture
survives, and indeed is developed by both Woodard and Falbo, to be
useful in theorising about knowledge. Guido Melchior
(\citeproc{ref-Melchior2019}{2019}) has developed a detailed account of
a special kind of inquiry, namely checking, and some of what he says
about checking is very useful in resolving some tensions in the
interest-relative picture.

Knowledge seems like it should be related to inquiry. But just what is
the relationship? An inquiry, like any action, has a beginning, a
middle, and an end. And it helps to think about ways in which knowledge
can play a role at each of these three stages. In particular, the
following three theses suggest ways in which knowledge plays a role at
each stage in turn.

\begin{enumerate}
\def\labelenumi{\arabic{enumi}.}
\tightlist
\item
  Inquiry should start with knowledge.
\item
  Inquiry should only be into things one does not know.
\item
  Inquiry should aim at knowledge.
\end{enumerate}

All three of these are plausible, but I'm ultimately only going to
accept 1. I'm going to accept a fairly strong form of it. On the version
I accept, only knowledge is appropriate as a starting point for inquiry,
and any knowledge could (in principle) be appropriate as a starting
point. The latter claim has to be qualified in some ways - it isn't
appropriate to start an inquiry into where the cat is with one's
knowledge about early Roman history. But if, while inquiring into where
the cat is, one knows which year Hannibal crossed the Alps, then that
knowledge is certain enough for use in the inquiry. If it shouldn't be
used, and it probably shouldn't be, that's on grounds of irrelevance,
not on grounds of uncertainty.

But I'm going to reject 2 and 3. I'm disagreeing here with Friedman,
whose theory of inquiry gives an important role to 2. And Woodard argues
convincingly that the failure of 3 implies that 2 has to fail as
well.\footnote{Note that Friedman (\citeproc{ref-Friedman2024}{2024b})
  also rejects 3, but not because she thinks inquiry aims at something
  else; she is sceptical of the metaphor of \emph{aiming} in this
  context. Note also that Falbo and Melchior developed similar arguments
  to Woodard's.} Inquiry might aim at knowledge, but it might aim at any
number of other things. It could aim at understanding, or at
sensitivity, or at developing reasons that convince others. (The latter
aim plays an important role in the explanation Michael Strevens
(\citeproc{ref-Strevens2020}{2020}) offers for some striking features of
contemporary science.) Since one might want to understand something one
knows, or have a more sensitive belief in what one knows, or convince
others of what one knows, it can make sense to inquire into what one
knows.

The next chapter presents a straightforward argument for
interest-relative epistemology. Before I get to that, I want to offer
two motivations for the view. You could try to turn either of these
motivations into a nice, clean, premise-conclusion argument for
interest-relativity. I haven't done that because in both of these cases,
the premise-conclusion format obscures more than it enlightens. The
first motivation comes from the practical nature of belief, and the
second from the thought that knowledge is a natural kind.

Think back to the problem facing Philo of Larissa. He wants to be a
sceptic, so nothing is known. He also wants to be able to act in the
world. Action requires a picture of what reality is like. So we need
some mental state that aims to fit the world, and which can guide
action. Once we have that state, you might well think that it's just
belief. Hugo Mercier (\citeproc{ref-Mercier2020}{2020}) argues that
people do not believe as many conspiracy theories as they say they do;
these apparent endorsements he argues are moves in a complicated
signaling game. His evidence that they don't actually believe the
conspiracy theories is that they act nothing like how they would act
were the theories true. Whether or not the details of Mercier's argument
are right, the form of it seems right. Apparent belief that is out of
sync with action is not really belief at all.

If belief is practical, the norms for belief should be practical too.
This isn't a logically necessary conditional; it is easy to describe
cases where we have non-practical norms for an essentially practical
state. Still, you should expect that the norms of a practical state are
typically practical. So you should expect that epistemology, the study
of norms for belief, will be shot through with practical considerations.
That's what the interest-relative theorist says is in fact the case.

The second motivation comes from reflection on what we're trying to do
in epistemology, and how it relates to the importance of knowledge. I
mentioned earlier that Pasnau regards the turn epistemology took after
Wilkins, where a central focus is on clarifying sub-optimal notions like
knowledge, to be a mistake. (By `sub-optimal' here, I mean merely that
they are standards one can meet while also being in a position to
improve one's doxastic position.) He thinks this is a retreat into mere
lexicography, and away from what was traditionally, and correctly,
viewed as the primary task of epistemology, namely clarifying the nature
of the epistemic ideal. I'm working in this post-Wilkins tradition, so I
probably should say some words in defence of it. This defence ends up
motivating an interest-relative approach.

Firstly, even if one didn't care about threshold standards like
knowledge, the right thing to do isn't to focus on the ideal. We aren't
going to attain the ideal. What we can do is get better and better. But
knowing what the ideal is like is often very little help in figuring out
how to do better. This is a general consequence of the Theory of the
Second Best (\citeproc{ref-LipseyLancaster}{Lipsey \& Lancaster,
1956-1957}). Very often, being like the ideal is a way of being worse
rather than better. For example, the ideal inquirer doesn't forget
anything, so they don't need to take notes while reading. Nevertheless,
it's a good epistemic practice to take notes while reading. So even if
what you ultimately care about is doing better, not meeting thresholds,
it isn't obvious that exploring the ideal is the way to get there. If
our aim is epistemic improvement, we're probably better off exploring
tools that fallible humans have developed for helping other fallible
humans will be more useful than exploring the ideal.

Secondly, and more importantly, the project here is not one of
lexicography. I don't particularly care how the English word `knows' is
used. The fact that a phonologically indistinguishable word is used to
talk both about knowing who won last night and knowing the players on
the winning team is of no relevance to the project we're engaged in
here. The fact that most languages have a word that is very close to
synonymous to the English word `knows' is more relevant. That's not
because it makes the lexicography important. Rather, it's because it
suggests that there is an important concept that English speakers are
picking out with `knows', that French speakers are picking out with
`savoir', and so on for all the other languages in the world. It could
be that all these different language groups agreed to use one of their
limited stock of words for this concept, and it was a mistake in every
case. But as Austin frequently reminded us, that's not the way to bet.

The concept of knowledge is, among other things, scientifically
important. Throughout the social sciences, there are theories that are
grounded in patterns of human behavior. Those patterns are, usually,
best explained in terms of what those humans know. Consider the
(stylized) fact that in a small, open, free market, competing suppliers
of a common good will usually sell goods for the same price. We could
offer an explanation of this in terms of the effective demand for a
supplier's goods given their price and the price of competing suppliers.
The demand curve facing this individual supplier will have a striking
discontinuity; once the price goes above the price others are offering
the good at, demand falls to 0.

Such an explanation will be good as far as it goes, but we can do
better. We can note that there is are mechanisms - in the sense of
mechanism developed by Machamer, Darden and Craver
(\citeproc{ref-MachamerEtAl2000}{2000}) - that underlie this pattern of
effective demand. The mechanisms are individual consumers who will
change their purchasing patterns if they know that someone else is
selling the same good more cheaply. Mechanisms, in this sense, are
things that display a consistent pattern of activity. The activities
have external triggers and reliable outputs given that trigger. Here the
trigger is knowledge that someone else is offering the good more
cheaply, and the output is buying the good elsewhere. The crucial thing
for us is that here, like in many other social science applications, the
trigger needs to be stated in terms of knowledge. It can't just be that
the change in prices leads to a change in behavior; a change in price
that no one knows about won't plausibly bring about any behavioral
change. It can't be that the trigger is stated in terms of what is
absolutely certain because no one can be absolutely certain of
contingent things like the price that a supplier is charging for a good.
Nor can the trigger be stated in terms of high probability. No matter
how probable I think it is that supplier B is cheaper than supplier A,
it might still be rational to buy from supplier A if the rest of the
probability goes to possibilities where B is much much more expensive.

Knowledge alone seems to do the trick. The generalisation that people
buy from suppliers they know to be cheaper seems both true, and to
rationalise their purchasing behavior. What's important for us is that
this places knowledge in the center of our understanding of how this
social arrangement works. That is going to be the general case; you just
can't do social science without talking about how people behave when
they come to know things.

So we have two reasons for thinking knowledge is a reasonably natural
kind: there are more or less synonymous terms for it across languages,
and it plays a key role in scientific explanations. Most fallibilist
theories of knowledge won't make it be particularly natural. (I'll
expand on this point in Section~\ref{sec-lockearb}.) Most such theories
say that to know something is to have a belief that's good enough along
some dimension. So the belief must be justified enough, or safe enough,
or produced by a reliable enough mechanism. Concepts that just pick out
points high enough up some or other scale are not particularly natural.
We should expect that we could do better.

Some fallibilist theories, or at least theories that make knowledge
`sub-optimal' in the sense I used above, do seem to be reasonably
natural. The sensitivity theory that Nozick
(\citeproc{ref-Nozick1981}{1981}) develops, for instance, plausibly
makes knowledge into a natural kind. Whether a belief would be retained
were its content false is not a matter of how well the belief does on
some scale. Alternatively, one could hold that knowledge is primitively
natural, not natural in virtue of its analysis or parts.\footnote{Such a
  view might be inspired by the `knowledge first' program of Williamson
  (\citeproc{ref-Williamson2000}{2000}).} That's not completely
implausible; it isn't obvious that the naturalness of social kinds has
to be explained in virtue of what metaphysically makes it the case that
things satisfy those kinds. Still, it would be nice to have a better
explanation of why knowledge is natural.

On the view defended here, a person knows a proposition if and only if
they properly take it to be settled. What one properly takes to be
settled is interest-relative, hence knowledge is interest-relative. I'm
not putting forward this biconditional an analysis of knowledge, or an
explanation of knowledge. It could be that the order of explanation here
runs from knowledge to proper settling. What I am claiming is that this
biconditional is true, and is part of the explanation of why knowledge
is a natural kind. The way to finish that explanation is to develop a
theory where knowledge interest-relative.

So those are the two big motivations for the interest-relative view: the
practicality of belief and the naturalness of knowledge. Belief is a
practical notion, so the norms of it should be practical. Knowledge is,
at its most essential, a norm of belief. Knowledge is a natural kind, as
evidenced by its cross-linguistic prevalence and its role in science.
This raises a challenge, since knowledge often feels like it requires
the knower do `well enough' along one or other scale, and there is
nothing particularly natural about choosing this point on the scale
rather than that point. The interest-relative theory has an answer to
this problem: a believer has knowledge when their evidence is good
enough to properly settle the inquiries the believer is engaged in, and
that's more than an arbitrary point on a scale.

While these are motivations, neither of them is strictly speaking an
argument. The main argument in this book for the interest-relative
theory is developed in Chapter~\ref{sec-interests}. It is that in some
fairly simple situations, there is a choice between four options.

\begin{enumerate}
\def\labelenumi{\arabic{enumi}.}
\tightlist
\item
  Accepting scepticism about all contingent knowledge.
\item
  Denying some very simple principles connecting knowledge and action -
  and in particular denying that it is rational to take the action one
  knows to do best.
\item
  Denying some very strong intuitions about which actions are rational
  in these simple situations.
\item
  Saying that knowledge is interest-relative.
\end{enumerate}

Since the first three options are implausible, the fourth is correct.
That is, knowledge is interest-relative.

The argument does not turn on intuitions about who knows what in what
situations. The only cases where the interest-relative theory disagrees
with its rivals are ones where intuitions about knowledge seem to me to
be very weak. For what it's worth, and it isn't worth much, I think the
interest-relative theory says the more intuitive thing about most cases.
Ultimately though, I don't particularly care about intuitions about
knowledge, at least in these relatively borderline cases. I will spend
some time defending my version of the interest-relative theory against
the frequently voiced complaint that interest-relative theories get some
clear cases incorrect. When the cases are indeed clear, I'll show that
my version of the theory matches the intuitions, but curve-fitting
around case intuitions will not be my priority.

To know something is to properly take it to be settled. There are two
kinds of practical considerations that might make it improper to take
something to be settled even if the evidence in favor of settling is
strong enough for everyday purposes. The first is that the cost of being
wrong is very high. The second is that the cost of checking whether one
is wrong is very low. The previous literature on interest-relativity has
primarily focussed on the first kind of reason. So the literature is
replete with discussion of `high stakes' cases, where someone stands to
lose a lot if something they have excellent evidence for turns out to be
false. Knowledge is often lost in these cases. There is somewhat less
discussion, however, about how knowledge is also lost in `easy checking'
cases, or about how (as Mark Schroeder
(\citeproc{ref-Schroeder2012}{2012}) notes), knowledge might not be lost
in cases where the stakes are high but checking is impossible. As I'll
put it in Section~\ref{sec-oddsandstakes}, what matters is not the
stakes, but the odds one faces in a particular situation.

Humans engage in both practical and theoretical inquiries. For that
matter, they often engage in inquiries which mix the practical and the
theoretical. A lot of the focus in the literature on interest-relativity
has been on how knowledge interacts with practical inquiry. Indeed the
title of Stanley's defence of an interest-relative account is
\emph{Knowledge and Practical Interests}. I don't impose any such
restriction here. If \emph{p} can't be properly taken to be settled in a
purely theoretical inquiry that someone is engaged in, they don't know
that \emph{p}. This has one striking implication. Let's say the person
is trying to figure out as precisely as possible what the probability of
\emph{p} is. If they can take \emph{p} as settled in that inquiry, then
the answer to the inquiry will be 1. Unless the correct answer to this
inquiry is actually 1, it won't be proper to take \emph{p} as settled.
So in general one easy way to lose knowledge that \emph{p} is to launch
an inquiry into precisely how probable \emph{p} is. I've set this out
using the ideology of probability, but this is unnecessary. Any inquiry
into how well supported \emph{p} is by one's overall evidence will
usually not be allowed to take \emph{p} as a starting point. So engaging
in that inquiry will lead to loss of knowledge. This is what is right in
scepticism, and infallibilism. The Cartesian meditator does, on this
view, lose knowledge in anything when they seriously reflect on how good
their evidence is for it. Happily, this knowledge comes back when they
return to their normal life.

In the middle of the discussion about knowledge and probability there is
a little inference from the premise that taking \emph{p} as settled
would lead to an incorrect answer, to the conclusion it is improper to
take \emph{p} as settled. That's a good inference; that taking \emph{p}
as settled leads to a mistaken conclusion is indeed compelling evidence
that it is improper to take \emph{p} as settled. But it's not the only
reason that it could be improper to take \emph{p} as settled. Among
other things, taking \emph{p} to be settled might get to the right
answer for the wrong reasons. So this principle, don't take something to
be settled if it will lead to the wrong answer, might be good advice,
but isn't a full account of when not to take something as settled. I
will go over this point in much more detail in sections \ref{sec-given}
to \ref{sec-questions}.

What one can properly take as given is a function of one's evidence.
This should be common ground between evidentialists, who think that what
one's evidence is grounds facts about what can be properly taken for
granted, and non-evidentialists who deny this. (On certain coherentist
pictures, for example, it will make sense to talk about someone's
evidence, but the fact that they have some evidence will be ultimately
explained by patterns of coherence among their other beliefs, and will
not be analytically prior to facts about rationality.) It would be
convenient for several purposes if we could have an interest-invariant
notion of evidence that explained why interests caused people to
sometimes lose knowledge. Unfortunately, that's not a viable position.
As I'll argue in Chapter~\ref{sec-evidence}, the arguments that
knowledge is interest-relative generalise into arguments that evidence
itself can be interest-relative.

So far I've sketched in the very broadest outlines the kind of theory
I'm going to propose. Here's the plan for how that theory will be laid
out, and defended, over the coming chapters, as well as some more
details on how the chapters relate to previously published work.

In Chapter~\ref{sec-interests}, I'll set out the main argument for the
interest-relative theory. The argument turns on how to think about a
particular low stakes bet. I argue that every option other than the
interest-relative theory says very implausible things about this case.

The next two chapters set out the fundamentals of the theory. In
Chapter~\ref{sec-belief} I lay out the interest-relative theory of
belief, and how that view differs from the view I developed in ``Can We
Do Without Pragmatic Encroachment''. Then in Chapter~\ref{sec-knowledge}
I extend that to a theory of knowledge, and introduce a problem that
will come up more in later chapters - how this theory interacts with
closure principles.

The following three chapters are, in one way or another, responses to
various objections to interest-relative theories. They are also the most
novel parts of the book; very little of these three chapters draws on
previously published work. Indeed, some of the key arguments build on
work that was unpublished at least when I started work on this book.
Chapter~\ref{sec-knowledge} draws heavily on work by Elise Woodard
(\citeproc{ref-Woodard2021}{2020}) and Chapter~\ref{sec-changes} draws
heavily the doctoral dissertation of Nilanjan Das
(\citeproc{ref-DasThesis}{2016}).

In Chapter~\ref{sec-inquiry} I discuss the role that the concept of
inquiry plays in my theory. On my theory, if something is known, it is
available to use as a starting point in inquiry. I used to think this
meant I was committed to agreeing with Jane Friedman
(\citeproc{ref-Friedman2019a}{2019b}) that it is incoherent to inquire
into something one knows. I've come to see that this isn't right;
depending on what one wants to do in an inquiry one may want to
deliberately set aside some premises. That might mean inquiry into what
one already knows is reasonable. This fact is used to respond to an
influential objection by Jessica Brown (\citeproc{ref-Brown2008}{2008})
to the style of argument I use in Chapter~\ref{sec-interests}.

In Chapter~\ref{sec-ties} I respond to an objection that theories like
mine are committed to implausible closure failures in cases where
choosers have very similar options to choose between. There is a proof
in ``Can We Do Without Pragmatic Encroachment?'' that the theory
developed there is immune to closure failures in these types of cases,
so the objection can't be right as stated. It turns out that the reason
the theory of that paper respects closure is that it has absurdly
sceptical consequences in cases where there are similar objects to
choose between. That's hardly better than a closure failure. In this
chapter I aim to do better.

I show that the objection relies on the assumption that the chooser aims
to maximise expected utility, and this isn't the right criteria of
correctness for decisions in close call situations. It isn't true that
when one is selecting cans off the supermarket shelf, one's selection is
rational iff it is utility maximising. Rather, the rational chooser in
such a situation will adopt a strategy that has the best long-run
consequences. In this case, the strategy will probably be something like
the strategy of picking arbitrarily unless it is clear that one of the
choices is defective. Given a theory of rational choice that emphasizes
the importance of decision making strategies, rather than the importance
of utility maximisation, my preferred epistemological theory gets the
right answers. There are two traps to avoid here: closure failure and
scepticism. And the focus on strategies lets us avoid both.

In Chapter~\ref{sec-changes} I respond to the frequently voiced
objection that interest-relative theories lead to implausible verdicts
about pairs of situations where knowledge is lost or gained due to what
looks like an irrelevant feature of a situation. I have two responses to
these objections. One was first offered in ``Defending Interest-Relative
Invariantism'' (\citeproc{ref-Weatherson2011-WEADIR}{Weatherson, 2011}).
I argue that the intuitions are about what makes it the case that a
person does or doesn't know something, and the arguments from these
examples moves too quickly from a claim about modal variation to a claim
about making. The second response is, I think, more compelling, and it's
essentially a point that Das makes and I'm borrowing. These objections
over-generate. Every theory of how to avoid Dharmottara cases leads to
pairs of cases where a person gains or loses knowledge depending on
factors that seem `irrelevant'. So it's not an objection to my view that
it has the same consequences as every plausible theory of knowledge.

The last two long chapters go into relatively technical details of my
theory of knowledge. I've put them at the end partially because they are
technical - I don't want to lose readers until as late as possible! But
also partially because they are the least changed from earlier work.

Chapter~\ref{sec-ratbel} goes over my theory of rational belief.
Surprisingly, and in contrast to the view defended by Jeremy Fantl and
Matthew McGrath (\citeproc{ref-FantlMcGrath2009}{2009}), interests
affect rational belief in a very different way to how they effect
knowledge. On my view, but not theirs, someone who has mistaken, and
irrational, beliefs about what practical situation they are facing can
easily have a rational, true belief that is not knowledge. This chapter
also tidies up some loose ends from Chapter~\ref{sec-belief} concerning
the so-called `Lockean' theory of belief.

Chapter~\ref{sec-evidence} sets out my interest-relative theory of
evidence. I argue that one's evidence just is what a radical interpreter
would say one's evidence is. In some cases, this means we end up playing
a kind of coordination game with the radical interpreter. What our
evidence is turns on what the right solution to that game. The solution
is interest-relative, but not in the way that knowledge is, nor in the
way that rational belief is.

Chapter~\ref{sec-power} ends with a short note connected
interest-relativity to the familiar saying \emph{Knowledge is Power}. I
argue that this saying only makes sense on an interest-relative view of
knowledge. If interest-relative theories were flawed for one reason or
another, then we'd have to simply concede that the saying is false. We
shouldn't concede that; the saying is true, and interest-relative
epistemology explains why it is true.

\bookmarksetup{startatroot}

\chapter{Interests}\label{sec-interests}

\section{Red or Blue?}\label{sec-redblue}

The key argument that knowledge is interest-relative starts with a
puzzle about a game. Here are the rules of the game, which I'll call the
Red-Blue game.

\begin{enumerate}
\def\labelenumi{\arabic{enumi}.}
\tightlist
\item
  Two sentences will be written on the board, one in red, one in blue.
\item
  The player will make two choices.
\item
  First, they will pick a colour, red or blue.
\item
  Second, they say whether the sentence in that colour is true or false.
\item
  If they are right, they win. If not, they lose.
\item
  If they win, they get \$50, and if they lose, they get nothing.
\end{enumerate}

Our player is Anisa. She has been reading some medieval history, and
last night was reading about the Battle of Agincourt. She was amused to
see that it took place on her birthday, October 25, and in 1415,
precisely 600 years before her own birthday. The book says all these
things about the Battle of Agincourt because they are actually true, and
when she read the book, Anisa believed them. She believed them because
she had lots of independent evidence that the book was reliable (it came
from a respected author and publisher, it didn't contradict her
well-grounded background beliefs), and she was sensitive to that
evidence of its reliability. These beliefs were correct; the book was
reliable and accurate on this point. The Battle of Agincourt was indeed
on October 25, 1415, and everything else the book says about the battle
without qualification is also true.

Anisa comes to know that she is playing the Red-Blue game, and that
these are its rules. She does not come to know any other relevant fact
about the game.\footnote{When presenting this material, some people have
  been puzzled about how this could be possible. It's implausible that
  Anisa knows nothing else about the game; if she didn't know who was
  putting the money up she could hardly trust that she would be paid out
  iff she was correct. More importantly, this extra knowledge might tell
  her something about the sentences. I think it helps assuage these
  worries to imagine this as one round of a repeated game Anisa is
  playing. Every round two sentences from a large stock are drawn at
  random to be the red and blue sentences. Anisa will play 20 such
  rounds, and get paid something between \$0 and \$1000 at the end,
  depending on how many she gets right. Why is she playing this? It
  could be the prize round of a game show that she was the nightly
  winner on. With something like this background, it's plausible that
  what I said in the text is true; she knows 1-6, and nothing else
  relevant. At least, this backstory should be enough to make it
  plausible that the setup is indeed possible.} When the game starts,
the following two sentences are written on the board, the first in red,
the second in blue.

\begin{itemize}
\tightlist
\item
  Two plus two equals four.
\item
  The Battle of Agincourt took place in 1415.
\end{itemize}

Anisa looks at this, thinks to herself, ``Oh, my book said that the
Battle of Agincourt was in 1415, so (given the rules of the game)
playing Blue-True will be as good as any other play, so I'm playing
Blue-True. Playing Red-True would get the same amount, since obviously
two plus two is four, but I'm going to play Blue-True instead''. That's
what she does, and she wins the \$50.

Intuitively, Anisa's move here is irrational, because it creates a
needless risk. There was a simple safe option that she should have
taken, and she declined it. Now it wasn't that much money; it's \$50. To
be sure, she doesn't actually lose it; she gets the answer correct. The
worlds where the risk is costly are somewhat distant; they are worlds
where either she has misremembered something that seems vivid, or where
a book that is clearly reliable has gone wrong. Still, it's sometimes
true that books, even good ones, make mistakes, and memory falters. She
took a risk, one that she didn't have to take, and got no compensation
for taking it. That's irrational.

I'm going to argue, at some length, that the best explanation of why it
is irrational for Anisa to play Blue-True is that knowledge is
interest-relative. When she was at home reading the book and just
thinking about medieval history, Anisa knew that the Battle of Agincourt
took place in 1415. When she was playing the game, and thinking about
winning as much money as possible, Anisa does not know this. When she is
moved into the game situation, she loses some knowledge she previously
had.

In the recent literature, arguments for and against interest-relativity
to date have not focussed on examples like Anisa's, but on examples
involving high-stakes choices. I'll present one example, involving a
character I'll call Blaise, presently. The example involving Anisa does,
however, have a handful of notable predecessors. It's structure is
similar to the examples of low-cost checking that Bradley Armour-Garb
(\citeproc{ref-ArmourGarb2011}{2011}) discusses. (Though he draws
contextualist conclusions from these examples, not interest-relative
ones.) And it is similar to some of the cases of three-way choice that
Charity Anderson and John Hawthorne deploy in arguing against
interest-relativity (\citeproc{ref-AndersonHawthorne2019a}{2019a},
\citeproc{ref-AndersonHawthorne2019b}{2019b}). Still, these are outlier
cases. Most of the literature has focussed on high-stakes cases. Let's
have one on the table.

Last night, Blaise was reading the same book that Anisa was reading. He
too was struck by the fact that the Battle of Agincourt took place on
October 25, 1415. Today he is visited by a representative of the
supernatural world, and offered the following bet. (Blaise knows these
are the terms of the bet, and doesn't know anything else relevant.) If
he declines the bet, life will go on as normal. If he accepts, one of
two things will happen.

\begin{itemize}
\tightlist
\item
  If it is true that the Battle of Agincourt took place in 1415, an
  infant somewhere will receive one second's worth of pure joy, of the
  kind infants often get playing peek-a-boo.
\item
  If it is false that the Battle of Agincourt took place in 1415, all of
  humanity will be cast into The Bad Place for all of eternity.
\end{itemize}

Blaise takes the bet. The Battle of Agincourt was in 1415, and he can't
bear the thought of a lovable baby missing that second of pure joy.

Again, there is an intuition that Blaise did something horribly wrong
here, and one possible explanation of this wrongness is that knowledge
is interest-relative. However, the argument that the interest-relativity
of knowledge is the very best explanation of what's going on is somewhat
weaker in Blaise's case than in Anisa's. It's not that I don't accept
the interest-relative explanation of the case; I do accept it. It's
rather that plausible interest-invariant explanations of the intuitions
about Blaise's case exist. Because these competing explanations exist,
it's hard to argue that interest-relativity is the best explanation of
why Blaise's action is wrong. Without that argument, it's hard to infer
from Blaise's case that knowledge is interest-relative by inference to
the best explanation. So I'll focus on Anisa, not Blaise.

This choice of focus occasionally means that this book is less connected
to the existing literature than I would like. I occasionally infer what
a philosopher would say about cases like Anisa's from what they have
said about cases like Blaise's. I'll probably get some of those
inferences wrong. But I want to set out the best argument for the
interest-relativity of knowledge that I know, and that means going via
the example of Anisa.

Though I am starting with an example, and with an intuition about it, I
am not starting with an intuition about what is known in the example. I
don't have any clear intuitions about what Anisa knows or doesn't know
while playing the Red-Blue game. The intuition that matters here is that
her choice of Blue-True is irrational. It's going to be a matter of
inference, not intuition, that Anisa lacks knowledge.

That inference will largely be by process of elimination. In
Section~\ref{sec-fourfamilies} I will set out four possible things we
can say about Anisa, and argue that one of them must be true. (The
argument won't appeal to any principles more controversial than the Law
of Excluded Middle.) But all four of them, including the
interest-relative view I favour, have fairly counterintuitive
consequences. So something counterintuitive is true around here. This
puts a limit on how we can argue. At least one instance of the argument
\emph{this is counterintuitive, so it is false} must fail. That casts
doubt over all such arguments. This is a point that critics of
interest-relativity haven't sufficiently acknowledged, but it also puts
constraints on how one can defend interest-relativity.

When Anisa starts playing the Red-Blue game, her practical situation
changes. So you might think I've gone wrong in stressing Anisa's
interests, not her practical situation. I've put the focus on interests
for two reasons. One is that if Anisa is totally indifferent to money,
then there is no rational requirement to play Red-True. We need to posit
something about Anisa's interests to even get the data point that the
interest-relative theory explains. The second reason, which I'll talk
about more in Section~\ref{sec-whatinterests}, is that sometimes we can
lose knowledge due to a change not in our practical situation, but our
theoretical interests.

In the existing literature, views like mine are sometimes called
versions of \textbf{subject-sensitive invariantism}, since they make
knowledge relevant to the stakes and salient alternatives available to
the subject. This is a bad name; of course whether a knowledge
ascription is true is sensitive to who the subject of the ascription is.
I know what I had for breakfast and you (probably) don't. The
distinctive feature of theories like mine is that a particular fact
about the subject's situation is relevant: their interests. That should
be reflected in the name. In the past, I've called this view
\textbf{interest-relative invariantism}, or IRI. For reasons I'll say
more about in Section~\ref{sec-neutrality}, I'm not committed to
\emph{invariantism} in this book. So in this book it's just the
interest-relative theory of knowledge, or IRT.

\section{Four Families}\label{sec-fourfamilies}

A lot of philosophers have written about cases like Anisa's and Blaise's
over the last couple of decades. Relatedly, there are a huge number of
theories that have been defended concerning these cases. Rather than
describe them all, I'm going to start with a taxonomy of them. The
taxonomy has some tricky edge cases, and it isn't always trivial to
classify a philosopher from their statements about the cases. It is,
nevertheless, a helpful way to start thinking about the available moves.

Our first family of theories are the \textbf{sceptical} theories. They
deny that Anisa ever knew that the Battle of Agincourt was in 1415. The
particular kind of sceptic I have in mind says that if someone's
epistemic position is, all things considered, better with respect to
\emph{q} than with respect to \emph{p}, that person doesn't know that
\emph{p}. The core idea for this sceptic, which perhaps they draw from
work by Peter Unger (\citeproc{ref-Unger1975}{1975}), is that knowledge
is a maximal epistemic state, so any non-maximal state is not knowledge.
The sceptics say that for almost any belief, Anisa's belief that two
plus two is four will have higher epistemic standing than that belief,
so that belief doesn't amount to knowledge.

Our second family of theories are what I'll call \textbf{epistemicist}
theories. The epistemicists say that Anisa's reasoning is perfectly
sound, and perhaps Blaise's is too. They both know when the Battle of
Agincourt took place, so they both know that the choices they take are
optimal, so they are rational in taking those choices. The intuitions to
the contrary are, say the epistemicist, at best confused. There is
something off about Anisa and Blaise, perhaps, but it isn't that these
particular decisions are irrational.

It's not essential to epistemicism, but one natural form of epistemicism
takes on board Maria Lasonen-Aarnio's point that act-level and
agent-level assessments might come apart.\footnote{See Lasonen-Aarnio
  (\citeproc{ref-Lasonen-Aarnio2010b}{2010},
  \citeproc{ref-Lasonen-Aarnio2014}{2014}) for more details on her view.
  In \emph{Normative Externalism}, I describe the difference between
  act-level and agent-level assessments as the difference between asking
  whether what Anisa does is rational, and whether Anisa's action
  manifests wisdom (\citeproc{ref-Weatherson2019}{Weatherson, 2019:
  124--5}). The best form of epistemicism, I'm suggesting, says that
  Anisa and Blaise are rational but unwise. This isn't Lasonen-Aarnio's
  terminology, but otherwise I'm just coopting her ideas.} On this
version of epistemicism, taking the bet reveals something bad about
Blaise's character, and arguably manifests a vice, but the act itself is
rational. It's that last claim, that the actions like Blaise's are
rational, that is distinctive of epistemicism.

The third family is the family of \textbf{pragmatist} theories, and this
family includes the interest-relative theory that I'll defend. The
pragmatists say that yesterday Anisa knew when the Battle of Agincourt
was, but now she doesn't. The change in her practical situation,
combined with her interest in getting more money, destroys her
knowledge.

And the final family are what I'll call, a little tendentiously, the
\textbf{orthodox} theories. Orthodoxy says that Anisa knew when the
Battle of Agincourt was last night, since her belief satisfied every
plausible criterion for testimonial knowledge. Orthodox also says she
knows it today, since changing practical scenarios or interests like
this doesn't affect knowledge. On the other hand, orthodoxy says that
the actions that Anisa and Blaise take are wrong; they are both
irrational, and Blaise's is immoral. Moreover it says that they are
wrong because they are risky. So knowing that what one is doing is for
the best is consistent with one's action being faulted on epistemic
grounds.

My reading of the literature is that a considerable majority of
philosophers writing on these cases are orthodox. (Hence the name!) But
I can't be entirely sure, because a lot of these philosophers are more
vocal about opposing pragmatist views than they are about supporting any
particular view. There are some views that are clearly orthodox in the
sense I've described, and I really think most of the people who have
opposed pragmatist treatments of cases like Anisa's and Blaise's are
orthodox, but it's possible more of them are sceptical or epistemicist
than I've appreciated.

Calling this last family orthodox lets me conveniently label the other
three families as heterodox. This lets me state what I hope to argue for
in this book: the interest-relative treatment of these cases is correct;
and if it isn't, then at least some pragmatist treatment is correct; and
if it isn't, then at least some heterodox treatment is correct.

It's worth laying out the interest-relative case in some detail, because
we can only properly assess the options holistically. Every view is
going to have some very counterintuitive consequences, and we can only
weigh them up when we see them all laid out. For instance, here are
things that each of them say.

\begin{itemize}
\tightlist
\item
  Sceptical theories say that when Anisa is reading her book, she
  doesn't gain knowledge even though the book is reliable and she
  believes it because of a well-supported belief in its reliability.
\item
  Epistemicist theories say that Anisa and Blaise make rational choices,
  even though they take what look like absurd risks.
\item
  Pragmatist theories say that offering someone a bet can cause them to
  lose knowledge and, presumably, that withdrawing that offer can cause
  them to get the knowledge back.
\item
  Orthodox theories say that it is irrational to do something that one
  knows will get the best result simply because it might get a bad
  result.
\end{itemize}

I'm going to mostly focus on the orthodox theories throughout the book,
and in particular I'll go into much more detail on this last point in
Section~\ref{sec-orthodox}.

Much of what the argumentation in this book, like much of what's in this
literature, will fall into one of two categories. Either it will be an
attempt to sharpen one of these implausible consequences, so the view
with that consequence looks even worse than it does now. Or it will be
an attempt to dull one of them, by coming up with a version of the view
that doesn't have quite as bad a consequence. Sometimes this latter task
is sophistry in the bad sense; it's an attempt to make the implausible
consequence of the theory harder to say, and so less of an apparent flaw
on that ground alone. Sometimes, though, it is valuable drawing of
distinctions. That is, it is scholasticism in the good sense. It turns
out that the allegedly plausible claim is ambiguous. On one
disambiguation we have really good reason to believe it is true, on
another the theory in question violates it, but on no disambiguation do
we get a violation of something really well-supported. I hope that they
work I do here to defend the interest-relative theory is more scholastic
than sophistic, but I'll leave that for others to decide.

Still, if all of the theories are implausible in one way or another,
shouldn't we look for an alternative? Perhaps we should look, but we
won't find any. At least if we define the theories carefully enough, the
truth is guaranteed to be among them. Let's try placing theories by
asking three yes/no questions.

\begin{enumerate}
\def\labelenumi{\arabic{enumi}.}
\tightlist
\item
  Does the theory say that Anisa knew last night that the Battle of
  Agincourt was in 1415? If no, the theory is sceptical; if yes, go to
  question 2.
\item
  Does the theory say that Anisa is rational to play Blue-True? If yes,
  the theory is epistemicist; if no, go to question 3.
\item
  Does the theory say that Anisa still knows that the Battle of
  Agincourt was in 1415, at the time she chooses to play Blue-True? If
  no, the theory is pragmatist; if yes, the theory is orthodox.
\end{enumerate}

That's it - those are your options. There are two two points of
clarification that matter, but I don't think they make a huge
difference.

The first point of clarification is really a reminder that these are
families of views. It might be that one member of the family is
considerably less implausible than other members. Indeed, I've changed
my mind a fair bit about what is the best kind of pragmatist theory
since I first started writing on this topic. There are a lot of possible
orthodox theories. Finding out the best version of these kinds of
theories, especially the last two kinds, is hard work, but it is worth
doing. That doesn't mean that it will lessen the implausibility of
endorsing a view from that family; some of the implausibility flows
directly from how one answers the three questions.

The second point of clarification is that what I've really done here is
classify what the different theories say about Anisa's case. They may
say different things about other cases. A theory might take an
epistemicist stand on Anisa's case, but an orthodox one on Blaise's
case, for example. Or it might be orthodox about Anisa, but would be
epistemicist if the blue sentence was something much more secure, such
as that the Battle of Hastings was in 1066. If this taxonomy is going to
be complete, it needs to say something about theories that treat
different cases differently. So here is the more general taxonomy I will
use.

The cases I'll quantify over have the following structure. Our hero,
called Hero, is given strong evidence for some truth \emph{p}, and they
believe it on the basis of that evidence. There are no defeaters, the
belief is caused by the truth of the proposition in the right way, and
in general all the conditions for knowledge that people worried about in
the traditional (i.e., late twentieth century) epistemological
literature are met. Then they are offered a choice, where one of the
options will have an optimal outcome if \emph{p}, but will not be the
best choice according to normal theories of decision unless the
probability of \emph{p} is incredibly close to one. While Hero's
evidence is strong, it isn't maximally strong. Despite this, Hero takes
the risky option, using the fact that \emph{p} as a key part of their
reasoning. Now consider the following three questions.

\begin{enumerate}
\def\labelenumi{\arabic{enumi}.}
\tightlist
\item
  In cases with this form, does the theory say that when Hero first
  forms the belief that \emph{p}, they know that \emph{p}? If the answer
  is that this is \emph{generally} the case, then restrict attention to
  those cases where they do know that \emph{p}, and move to question 2.
  Otherwise, the theory is sceptical.
\item
  In the cases that remain, is Hero rational in taking the option that
  is optimal iff \emph{p}. If the answer is yes in \emph{every} case,
  the theory is epistemicist. Otherwise, restrict attention to cases
  where this choice is irrational, and move to question 3.
\item
  In \emph{any} of the cases that remain, does the fact that Hero was
  offered the choice destroy their knowledge that \emph{p}? If yes, the
  theory is pragmatic. If no, the theory is orthodox.
\end{enumerate}

So I'm taking epistemicism to be a very strong theory - it says that
knowledge always suffices for action that is optimal given what's known,
and that offers of bets never constitute a loss of knowledge. The
epistemicist can allow that the offer of a bet may cause a person to
`lose their nerve', and hence their belief that \emph{p}, and hence
their knowledge that \emph{p}. Still, if they remain confident in
\emph{p}, they retain knowledge that \emph{p}.

Pragmatism is a very weak theory - it says sometimes the offer of a bet
can constitute a loss of knowledge. The justification for defending such
a weak theory is that so many philosophers are aghast at the idea that
practical considerations like this could ever be relevant to knowledge.
So even showing that the existential claim is true, that sometimes
practical issues matter, would be a big deal.

Orthodoxy is a weak claim on one point, and a strong claim on another.
It says there are some cases where knowledge does not suffice for action
- though it might take these cases to be very rare. It is common in
defences of orthodoxy to say that the cases are quite rare, and use this
fact to explain away intuitions that threaten orthodoxy. The key thing
is that it says that pragmatic factors never matter - so it can be
threatened by a single case like Anisa.

\section{Against Orthodoxy}\label{sec-orthodox}

The orthodox view of cases like Blaise's is that offering him the bet
does not change what he knows, but still he is irrational to take the
bet. In this section, I'm going to run through a series of arguments
against the orthodox view. The reason I am making so many arguments is
not that I lack confidence in any one of them. Rather, it is because the
orthodox view is so widespread that we need to appreciate how many
strange consequences it has.

\subsection{Moore's Paradox}\label{sec-orthodoxmoore}

Start by thinking about what the orthodox view says a rational person in
Blaise's situation would do. Call this rational person Chamari.
According to the orthodox view, offering someone a bit does not make
them lose knowledge. So Chamari still knows when the Battle of Agincourt
was fought. Chamari is rational, so despite having this knowledge,
Chamari will decline the bet. Think about how Chamari might respond when
you ask her to justify declining the bet.

\begin{quote}
You: When was the Battle of Agincourt?\\
Chamari: October 25, 1415.\\
You: If that's true, what will happen if you accept the bet?\\
Chamari: A child will get a moment of joy.\\
You: Is that a good thing?\\
Chamari: Yes.\\
You: So why didn't you take the bet?\\
Chamari: Because it's too risky.\\
You: Why is it risky?\\
Chamari: Because it might lose.\\
You: You mean the Battle of Agincourt might not have been fought in
1415.\\
Chamari: Yes.\\
You: So the Battle of Agincourt was fought in 1415, but it might not
have been fought then?\\
Chamari: Yes, the Battle of Agincourt was fought in 1415, but it might
not have been fought then, and that's why I'm not taking the bet.
\end{quote}

Chamari has given the best possible answer at each point. Yet she has
ended up assenting to a Moore-paradoxical sentence. In particular, she
has assented to a sentence of the form \emph{p, but it might be that not
p}. It is very widely held that sentences like this cannot be rationally
assented to. Since Chamari was, by stipulation, the model for what the
orthodox view thinks a rational person is, this shows that the orthodox
view is false.

There are three ways out of this puzzle, and none of them seems
particularly attractive.

One is to deny that there's anything wrong with where Chamari ends up.
Perhaps in this case the Moore-paradoxical claim is perfectly
assertable. I have some sympathy for the general idea that philosophers
over-state the badness of Moore-paradoxicality
(\citeproc{ref-MaitraWeatherson2010}{Maitra \& Weatherson, 2010}).
Still, it does seem very unattractive to end up precisely here.

Another is to deny that the fact that Chamari knows something licences
her in asserting it. I've assumed in the argument that if Chamari knows
that \emph{p}, she can say that \emph{p}. Maybe that's too strong an
assumption. The conversation, says this reply, goes off the rails at the
very first line. On this way of thinking, it is hard to know what the
point of knowledge is. If knowing something isn't sufficiently good
reason to assert it, it is hard to know what would be.

The orthodox theorist has a couple of choices here, neither of them
good. One is to say that although knowledge is not interest-relative,
the epistemic standards for assertion are interest-relative. Basically,
Chamari meets the epistemic standard for saying that \emph{p} only if
Chamari knows that \emph{p} according to the (false!) interest-relative
theory. At this point, given how plausible it is that knowledge is
closely connected with testimony, it seems we would need an excellent
reason to not simply identify knowledge with this epistemic standard.
The other is to say that there is some interest-invariant standard for
assertion. By running through varieties of cases like Anisa's and
Blaise's, we can show that such a standard would have to be something
like Cartesian certainty. So most everything we say, every single day,
would be norm violating. Such a norm is not plausible.

So we get to the third way out, one that is only available to a subset
of orthodox theorists. We can say that `knows' is context-sensitive,
that in Chamari's context the sentence ``I know when the Battle of
Agincourt was fought'' is actually false, and those two facts explain
what goes wrong in the conversation with Chamari. Armour-Garb
(\citeproc{ref-ArmourGarb2011}{2011}), who points out how much trouble
non-contextualist orthodox theorists get into with these
Moore-paradoxical claims, suggests a contextualist resolution of the
puzzles. While this is probably the least bad way to handle the case,
but it's worth noting just how odd it is.

It's not immediately obvious how to get from contextualism to a
resolution of the puzzle. Chamari doesn't use the verb `to know' or any
of its cognates. She does use the modal `might', and the contextualist
will presumably want to say that it is context sensitive. That doesn't
look like a helpful way to solve the problem though, since her assertion
that the Battle might have been on a different day seems like the good
part of what she says. What's problematic is the unqualified assertion
about when the battle was, in the context of explaining her refusal to
bet. We need some way of connecting contextualism about epistemic verbs
to a claim about the inappropriateness of this assertion.

The standard move by contextualists here is to simply deny that there is
a tight connection between knowledge and assertion
(\citeproc{ref-Cohen2004}{Cohen, 2004};
\citeproc{ref-DeRose2002}{DeRose, 2002}). (So this is really a
sophisticated form of a response I just rejected.) What they say instead
is that there is a kind of meta-linguistic standard for assertion. It is
epistemically responsible to say that \emph{p} iff it would be true to
say \emph{I know that p}. Since it would not be true for Chamari to say
she knows when the Battle of Agincourt was fought, she can't responsibly
say when it was fought.

The most obvious reason to reject this line of reasoning is that it is
implausible that meta-linguistic norms like this exist. Imagine we were
conversing with Chamari about her reasons for declining the bet in
Bengali rather than English, and at every line a contribution with the
same content was made. Would the reason her first answer was
inappropriate be that some English sentence would be false if uttered in
her context, or that some Bengali sentence would be false? If it's an
English sentence, it's very weird that English would have this normative
force over conversations in Bengali. If it's Bengali, then it's odd that
the standard for assertion changes from language to language.

If there were a human language that didn't have a verb for knowledge,
then that last point could be made with particular force. What would the
contextualists say is the standard for assertion in such a language?
Somewhat surprisingly, no such language exists
(\citeproc{ref-Nagel2014}{Nagel, 2014}). It's still a bit interesting to
think about possible languages that do allow for assertions, but do not
have a verb for knowledge. Just what the contextualists would say is the
standard for assertion in such a language is a rather delicate matter.

Rather than thinking about these merely possible languages, let's return
to English, and end with a variant of the conversation with Chamari.
Imagine that she hasn't yet been offered any bet, and indeed that when
the conversation starts, we're just spending a pleasant few minutes idly
chatting about medieval history.

\begin{quote}
You: When was the Battle of Agincourt?\\
Chamari: October 25, 1415.\\
You: Oh that's interesting. Because you know there's this bet that
someone offered my friend Blaise, and I bet I could get them to offer it
to you. If you were to accept it, and the Battle of Agincourt was in
1415, then a small child would get a moment of joy.\\
Chamari: That's great, I should take that bet.\\
You: Well, wait a second, I should tell you what happens if the Battle
turns out to have been on any other date. {[}You explain what happens in
some detail.{]}\\
Chamari: That's awful, I shouldn't take the bet. The Battle might not
have been in 1415, and it's not worth the risk.\\
You: So you won't take the bet because it's too risky?\\
Chamari: That's right, I won't take it because it's too risky.\\
You: Why is it risky?\\
Chamari: Because it might lose.\\
You: You mean the Battle of Agincourt might not have been fought in
1415.\\
Chamari: Yes.\\
You: Hang on, you just say it was fought in 1415, on October 25 to be
precise.\\
Chamari: That's true, I did say that.\\
You: Were you wrong to have said it?\\
Chamari: Probably not; it was probably right that I said it.\\
You: You probably knew when the battle was, but you don't now know it?\\
Chamari: No, I definitely didn't know when the battle was, but it was
probably right to have said it was in 1415.
\end{quote}

And you can probably see all sorts of ways of making Chamari's position
sound terrible. The argument I'm giving here is a version of an argument
against contextualism due to John MacFarlane
(\citeproc{ref-MacFarlane2005-Knowledge}{2005}). He notes that
contextualists have a particular problem with retraction; Chamari's
position sounds much worse than it should if contextualism is right.
Still, I don't want to lean too much weight on how she sounds. Every
position in this area ends up saying some strange things. The very idea
that the epistemic standard for assertion could be meta-linguistic,
either in the version which says some English word determines the
appropriateness conditions for assertions in every language, or that the
appropriateness conditions change from language to language, is even
more implausible than the idea that we should end up where Chamari does.

\subsection{Super Knowledge to the Rescue?}\label{sec-superknow}

Let's leave Blaise and Chamari for a little and return to Anisa. The
orthodox view agrees that it is irrational for Anisa to play Blue-True.
So it needs to explain why this is so. IRT offers a simple explanation.
If she plays Red-True, she knows she will get \$50; if she plays
Blue-True, she does not know that - though she knows she will get at
most \$50. So Red-True is the weakly dominant option; she knows it won't
do worse than any other option, and there is no other option that she
knows won't do worse than any other option.

The orthodox theorist can't offer this explanation. They think Anisa
knows that Blue-True will get \$50 as well. So what can they offer
instead? There are two broad kinds of explanation that they can try.
First, they might offer a structurally similar explanation to the one
IRT gives, but with some other epistemic notion at its centre. So while
Anisa knows that Blue-True will get \$50, she doesn't \emph{super-know}
this, in some sense. Second, they can try to explain the asymmetry
between Red-True and Blue-True in probablistic, rather than epistemic,
terms. I'll discuss the first option in this subsection, and the
probabilistic notion in the next subsection.

What do I mean her by \emph{super-knows}? I mean this term to be a
placeholder for any kind of relation stronger than knowledge that could
play the right kind of role in explaining why it is irrational for Anisa
to play Blue-True. So super-knowledge might be iterated knowledge. Anisa
super-knows something iff she knows that she knows that \ldots{} she
knows it. She super-knows that two plus two is four, but not that the
Battle of Agincourt was in 1415. Or super-knowledge might be (rational)
certainty. Anisa is (rationally) certain that two plus two is four, but
not that the Battle of Agincourt was in 1415. Or it might be some other
similar relation. My objection to the super-knowledge response won't be
sensitive to the details of how we understand super-knowledge.

If a super-knowledge solution is going to work, it had better be that
Anisa does not in fact super-know that the Battle of Agincourt was in
1415. That already rules out some versions of the super-knowledge
solution. In normal versions of the case, Anisa does know that she knows
the Battle of Agincourt was in 1415. She knows that she read this in a
book, that the book had a lot of indicators of reliability, and (at
least according to the orthodox theorist), that what she read was
correct. If she was asked to sort people into whether they do or don't
know that the Battle was in 1415, she would (in normal versions of the
case) be fairly good at doing this, and would sort herself into the
group that does know.\footnote{To be sure, she presumably doesn't know
  for most people what they know about medieval history. What I'm
  imagining is that if she was presented with a bunch of people, asked
  if they know when the Battle of Agincourt was, and was allowed to say
  ``Yes'', ``No'', or ``Don't Know'', then most of the ``Yes'' and
  ``No'' answers would be correct, and she would say ``Yes'' about
  herself.} So she passes all the standard tests for knowing that she
knows when the battle was.

For most versions of what super-knowledge is, it looks like in ideal
cases it should be closed under conjunction. That is, Anisa super-knows
a conjunction (that she is considering) iff she super-knows each of the
conjuncts. I'll come back to one important exception to this, that
super-knowledge is credence above a threshold, in the next subsection.
For now, assume that super-knowledge is closed under conjunction in this
way.

Given that assumption, the fact that Anisa doesn't super-know when the
Battle of Agincourt was can't explain the asymmetry between Red-True and
Blue-True. In particular, it can't explain why Anisa rationally must
choose Red-True. This is because she doesn't super-know that playing
Red-True will win the \$50. If super-knowledge is demanding enough that
she doesn't know when the battle was, it's demanding enough that she
doesn't know the rules of the game. That implies that she doesn't know
that playing Red will win the \$50. She has ordinary testimonial
knowledge of the rules, just like she has ordinary testimonial knowledge
about the Battle of Agincourt. It's just as realistic that she has
misunderstood the rules of the game as that a reliable history book has
gotten a key date wrong. It's not just in evil demon situations that
someone misunderstands a rule. In a very ordinary sense, she can't be
completely certain that she has the rules correct. If testimony from
careful historians can't generate super-knowledge, neither can testimony
from game-show hosts.

In fact, her knowledge of the rules of the game, in the sense that
matters, is probably weaker than her knowledge of history. It is not
unknown for game shows to promise prizes, then fail to deliver them,
either because of malice or incompetence. Knowledge of the game rules,
in particular knowledge that she will actually get \$50 if she selects a
true sentence, requires some knowledge of the future. That seems harder
to obtain than knowledge of what happened in history. After all, she has
to know that there won't be an alien invasion, or a giant asteroid, or
an incompetent or malicious game organiser. (The last two being
considerably more important considerations in normal cases.)

So there is no way of understanding `super-knows' such that 1 is true
and 2 is false.

\begin{enumerate}
\def\labelenumi{\arabic{enumi}.}
\tightlist
\item
  Anisa super-knows that if she plays Red-True, she'll win \$50.
\item
  Anisa does not super-know that if she plays Blue-True, she'll win
  \$50.
\end{enumerate}

If the super-knowledge based explanation of why she should play Red-True
worked, there should be some sense of super-knowledge where 1 is true
and 2 is false. There isn't, so the explanation doesn't work.

The point I'm making here, that in thinking about these games we need to
attend to the player's epistemic attitude towards the game itself, is
not original. Dorit Ganson (\citeproc{ref-Ganson2019}{2019}) uses this
point for a very similar purpose, and in turn quotes Robert Nozick
(\citeproc{ref-Nozick1981}{1981}) making a similar point. I've
belaboured it here because it is so easily overlooked. It is easy to
take things that one is told about a situation, such as the rules of a
game that are being played, as somehow fixed and inviolable. They aren't
the kind of thing that can be questioned. In any realistic case, the
rules will not have such an exalted practical or epistemic status - at
least if one assumes that only what is super-known can be taken as
fixed.

This is why I rest more weight on Anisa's case than on Blaise's. I can't
appeal to your judgment about what a realistic version of Blaise's case
would be like, because there are no realistic versions of cases like
Blaise's. Anisa's case, on the other hand, is very easy to imagine and
understand. We can ask what a realistic version of it would be like.
That version would be such that the player would know what the rules of
the game are, but would also know that sometimes game shows don't keep
their promises, sometimes they don't describe their own games
accurately, sometimes players misinterpret or misunderstand
instructions, and so on. This shouldn't lead us to scepticism: Anisa
knows what game she's playing. But she doesn't super-know what game
she's playing, which means she doesn't super-know she'll win if she
plays Red-True.

\subsection{Rational Credences to the Rescue?}\label{sec-probrescue}

So imagine the orthodox theorist drops super-knowledge, and looks
somewhere else. A natural alternative is to use credences. Assume that
the probability that the rules of the game are as described is
independent of the probabilities of the red and blue sentence. Assume
also that Anisa must, if she is to be rational, maximise expected
utility. Then we get the natural result that Anisa should pick the
sentence that is more probably true.\footnote{Strictly speaking, we need
  one more assumption - namely that for any unexpected way for the game
  to be, the probability of it being that way is independent of the
  truth of both the red and blue sentences. This feels like a safe
  assumption for the orthodox theorist to make.} And that can explain
why she must choose Red-True, which is what the orthodox theorist needed
to explain.

This kind of approach doesn't really have any place for knowledge in its
theory of action. One should simply maximise expected utility; since
doing what one knows to be best might not maximise expected utility, we
shouldn't think knowledge has any particularly special role.

There are many problems with this kind of approach. Several of these
problems will be discussed elsewhere in this book at more length. I will
point to where those problems are discussed rather than duplicate the
discussion here. Some other problems I'll address straight away.

Like the view discussed in Section~\ref{sec-orthodoxmoore} that
separates knowledge from assertion, separating knowledge from action
leads to strange consequences. As Timothy Williamson
(\citeproc{ref-Williamson2005}{2005}) points out, once we break apart
knowledge from action in this way, it becomes hard to see the point of
knowledge. It's worth pausing a bit more over the bizarreness of the
claim that Blaise knows that taking the bet will work out for the best,
but he shouldn't take it - because of its possible consequences.

If one excludes knowledge from having an important role in one's theory
of decision, one ends up having a hard time explaining how dominance
reasoning works. It is, however, a compulsory task for a theory of
decision to explain how dominance reasoning works. Among other things,
we need a good account of how dominance reasoning works in order to
handle Newcomb problems, and we need to handle Newcomb problems in order
to motivate, or even to state, a careful version of expected utility
maximisation. That little argument was very compressed. I'm not going to
expand upon it just yet because there will be so much more discussion of
dominance reasoning throughout this book; a sketch will do for now.

Probabilistic models of reasoning and decision have their limits, and
what we need to explain about the Red-Blue game goes beyond those
limits. So probabilistic models can't be the full story about the
Red-Blue game. To see this, imagine for a second that the Blue sentence
is not about the Battle of Agincourt, but is instead a slightly more
complicated arithmetic truth, like \emph{Thirteen times seventeen equals
two hundred and twenty one}, or a slightly complicated logical truth,
like ¬\emph{q} → ((\emph{p} → \emph{q}) → ¬\emph{p}). If either of those
are the blue sentence, then it is still uniquely rational to play
Red-True, even though the probability of each of those sentences is one.
So rational choice is more demanding than expected utility maximisation.
In sections \ref{sec-lockecoin} and \ref{sec-lockegames} I'll go over
more cases of propositions whose probability is 1, but which should be
treated as uncertain even it is certain that two plus two is four. The
lesson is that we can't just use expected utility maximisation to
explain the Red-Blue game.

Finally, we need to understand the notion of probability that's being
appealed to in this explanation. It can't be some purely subjective
notion, like credence, because that couldn't explain why some decisions
are rational and others aren't. If Anisa was subjectively certain that
the Battle of Agincourt was in 1415, she would still be irrational to
play Blue-True. It can't be some purely physical notion, like chance or
frequency, because that won't even get the cases right. (What is the
chance, or frequency, of the Battle of Agincourt being in 1415?) It
needs to be something like evidential probability. That will run into
problems in versions of the Red-Blue game where the Blue sentence is
arguably (but not certainly) part of the player's evidence. I'll end my
discussion of orthodoxy with a discussion of cases like these.

\subsection{Evidential Probability}\label{sec-orthodoxevidence}

No matter which of these explanations the orthodox theorist goes for,
they need a notion of evidence to support them.\footnote{This subsection
  is based on my (\citeproc{ref-Weatherson2018-WEAIEA-2}{2018: 2}).}
Let's assume that we can find some doxastic attitude D such that Anisa
can't rationally stand in D to \emph{Play Blue-True}, and that this is
why she can't rationally play Blue-True. Then we need to ask the further
question, why doesn't she stand in relation D to \emph{Play Blue-True}?
And presumably the answer will be that she lacks sufficient evidence.
After all, if she had optimal evidence about when the Battle of
Agincourt was, she could play Blue-True.

The orthodox theorist also has to have an interest-invariant account of
evidence. I guess it's logically possible to have evidence be
interest-relative, but knowledge be interest-neutral, but it is very
hard to see how one would motivate such a position.

Now we run into a problem. Imagine a version of the Red-Blue game where
the blue sentence is something that, if known, is part of the player's
evidence. If it is still irrational to play Blue-True, then any orthodox
explanation that relies on evidence sensitive notions (like
super-knowledge or evidential probability) will be in trouble. The aim
of this subsection is to spell out why this is.

So let's imagine a new player for the red-blue game. Call her Parveen.
She is playing the game in a restaurant. It is near her apartment in Ann
Arbor, Michigan. Just before the game starts, she notices an old friend,
Rahul, across the room. Rahul is someone she knows well, and can
ordinarily recognise, but she had no idea he was in town. She actually
thought Rahul was living in Italy. Still, we would ordinarily say that
she now knows Rahul is in town; indeed that he is in the restaurant. As
evidence for this, note that it would be perfectly acceptable for her to
say to someone else, ``I saw Rahul here''. Now the game starts.

\begin{itemize}
\tightlist
\item
  The red sentence is \emph{Two plus two equals four}.
\item
  The blue sentence is \emph{Rahul is in this restaurant}.
\end{itemize}

On the one hand, there is only one rational play for Parveen: Red-True.
She hasn't seen Rahul in ages, and she thought he was in Italy. A
glimpse of him across a crowded restaurant isn't enough for her to think
that \emph{Rahul is in this restaurant} is as likely as \emph{Two plus
two equals four}. She might be wrong about Rahul, so she should take the
sure money and play Red-True. So playing the red-blue game with these
sentences makes it the case that Parveen doesn't know where Rahul is.
This is another case where knowledge is interest-relative, and at first
glance it doesn't look very different to the other cases we've seen.

But take a second look at the story for why Parveen doesn't know where
Rahul is. It can't be just that her evidence makes it certain that two
plus two equals four, but not certain that Rahul is in the restaurant.
At least, it can't be that unless it is not part of her evidence that
Rahul is in the restaurant. If evidence is not interest-relative, then
it is part of Parveen's evidence that Rahul is in the restaurant. This
isn't something she infers; it is a fact about the world she simply
appreciated. Ordinarily, it is a starting point for her later
deliberations, such as when she deliberates about whether to walk over
to another part of the restaurant to say hi to Rahul. That is,
ordinarily it is part of her evidence.

So the orthodox theorist has a challenge. If they say that it is part of
Parveen's evidence that Rahul is in the restaurant, then they can't turn
around and say that the evidential probability that he is in the
restaurant is insufficiently high for her to play Blue-True. After all,
its evidential probability is one. If they say that it is no part of
Parveen's evidence that Rahul is in the restaurant because she is
playing this version of the Red-Blue game, they give up orthodoxy. So
they have to say that our evidence never includes things like Rahul is
in the restaurant.

This can be generalised. Take any proposition such that if the red
sentence was that two plus two is four and that proposition was the
content of the blue sentence, then it would be irrational to play
Blue-True. Any orthodox explanation of the Red-Blue game entails that
this proposition is no part of your evidence - whether you are playing
the game or not. Once we strip all these propositions out of your
evidence, you don't have enough evidence to rationally believe, or even
rationally make probable, very much at all.

Descartes, via a very different route, walked into a version of this
problem. His answer was to (implicitly) take us to be infallible
observers of our own minds, and (explicitly) offer a theistic
explanation for how we can know about the external world given just this
psychologistic evidence. Nowadays, most people think that's wrong on
both counts: we can be rationally uncertain about even our own minds,
and there is no good path from purely psychological evidence to
knowledge of the external world. The orthodox theorist ends up in a
state worse than Cartesian scepticism.

\section{Odds and Stakes}\label{sec-oddsandstakes}

If orthodox views are wrong, then it is important to get clear on which
heterodox view is most plausible.\footnote{This section is based on my
  (\citeproc{ref-Weatherson2016}{2016a: 3}).} I'm defending a version of
the pragmatic view, but it's a different version to the most prominent
versions defended in the literature. The difference can be most readily
seen by looking at the class of cases that have motivated pragmatic
views.

The cases involve a subject making a practical decision. The subject has
a safe choice, which has a guaranteed return of \emph{S}. They also have
a risky choice. If things go well, the return of the risky choice is
\emph{S}~+~\emph{G}, so they will gain \emph{G} from taking the risk. If
things go badly, the return of the risky choice is \emph{S}~‑~\emph{L},
so they will lose \emph{L} from taking the risk. What it takes for
things to go well is that a particular proposition \emph{p} is true. All
of this is known by the subject facing the choice. It's also true (but
not uncontroversially known by the subject) that they satisfy all the
conditions for knowing \emph{p} that would have been endorsed by a
well-informed epistemologist circa 1997. (That is, by a proponent of the
traditional view.) So \emph{p} is true, and things won't go badly for
them if they take the risk. Still, in a lot of these cases, there is a
strong intuition that they should not take the bet, and as I've just
been arguing, that is hard to square with the idea that they know that
\emph{p}. So assuming the traditional view is right about the subject as
they were before facing the practical choice, having this choice in
front of them causes them to lose knowledge that \emph{p}.

But what is it about these choices that triggers a loss of knowledge?
There is a familiar answer to this, one explicitly endorsed by Hawthorne
(\citeproc{ref-Hawthorne2004}{2004}) and Stanley
(\citeproc{ref-Stanley2005}{2005}). It is that they are facing a `high
stakes' choice. Now what it is for a choice to be high stakes is never
made entirely clear, and Anderson and Hawthorne
(\citeproc{ref-AndersonHawthorne2019a}{2019a}) show that it is hard to
provide an adequate definition in full generality. In the simple cases
described in the previous paragraph, however, it is easy enough to say
what a high stakes case is. It just means that \emph{L} is large. So one
gets the suggestion that practical factors kick in when faced with a
case where there is a chance of a large loss.

The version of IRT defended in this book does not care about whether a
subject faces a high stakes bet. Instead, it says that \emph{L} matters,
but only indirectly. What is (typically) true in these cases is that the
subject should maximise expected utility relative to their
evidence.\footnote{This simplifies the relationship between rational
  choice and expected utility maximisation. Later in the book I'll have
  to be much more careful about this relationship. See chapter
  \ref{sec-ties} for many more details.} And taking the risky choice
maximises expected utility only if this equation is true.

\[
\frac{\Pr(\textit{p})}{1 ‑ \Pr(\textit{p})} > \frac{\textit{L}}{\textit{G}}
\]

The left hand side expresses the odds that \emph{p} is true. The right
hand side expresses how high those odds have to be before the risk is
worth taking. If the equation fails to hold, then the risk is not worth
taking. If the risk is not worth taking, then the subject doesn't know
that \emph{p}.

Since the numerator of the right hand side is \emph{L}, then one way to
destroy knowledge that \emph{p} is to present the subject with a
situation where \emph{L} is very high. It isn't, however, the only way.
Since the denominator of the right hand side is \emph{G}, another way to
destroy knowledge that \emph{p} is to present the subject with a
situation where \emph{G} is very low.

In effect, we've seen such a situation with Anisa. To make the parallel
to Anisa's case even clearer, consider the following case, involving a
character I'll call Darja. Darja has been reading books about World War
One, and yesterday read that Franz Ferdinand was assassinated on St
Vitus's Day, June 28, 1914. She is now offered a chance to play a
slightly unusual quiz game. She has to answer the question \emph{What
was the date of Franz Ferdinand's assassination?} If she gets it right,
she wins \$50. If she gets it wrong, she wins nothing. Here's what is
strange about the game. She is allowed to Google the answer before
answering. So here are the two live options for Darja. In the table, and
in what follows, \emph{p} is the proposition that Franz Ferdinand was
indeed assassinated on June 28, 1914.

\begin{longtable}[]{@{}rcc@{}}
\caption{Darja's choice between answering the question, and checking
Google.}\label{tbl-google}\tabularnewline
\toprule\noalign{}
\endfirsthead
\endhead
\bottomrule\noalign{}
\endlastfoot
~ & \emph{p} & ¬ \emph{p} \\
Say ``June 28, 1914'' & 50 & 0 \\
Google the answer & \$50~‑~ε & \$50~‑~ε \\
\end{longtable}

If Darja has her phone near her, and has cheap easy access to Google,
then ε might be really low. In that case she should take the safe
option; it's the one that maximises expected utility. That means she
doesn't know that \emph{p}, even if she remembers reading it in a book
that is actually reliable. Facing a long odds bet can cause knowledge
loss, even in low stakes situations.

So I'm committed to the view that Darja loses knowledge in her
relatively low stakes situation, and indeed I think that's true. That's
not because I have any kind of intuition that she loses knowledge. I
don't have any clear intuition about her case, and I'm certainly not
taking any intuition about the case as a premise. What I am taking as a
premise is that Darja should Google the answer in cases like this one;
doing otherwise is taking a bad risk. The best explanation of why this
is a bad risk is that she doesn't know when Franz Ferdinand was
assassinated. So practical interests can matter even in relatively low
stakes cases.

I'm not the first to focus on these long odds/low stakes cases. Jessica
Brown (\citeproc{ref-Brown2008}{2008: 176}) notes that these cases raise
problems for the stakes-centric version of IRT. Anderson and Hawthorne
(\citeproc{ref-AndersonHawthorne2019a}{2019a}) argue that once we get
beyond the simple two-state/two-option choices, it isn't at all easy to
say what situations are and are not high-stakes choices. These cases are
not problems for the version of IRT that I defend, since this version
gives no role to stakes.

\section{Theoretical Interests Matter}\label{sec-whatinterests}

When saying why I called my theory IRT, one of the reasons I gave was
that I wanted theoretical, and not just practical, interests to matter
to knowledge.\footnote{This section is based on my
  (\citeproc{ref-Weatherson2017-WEAII}{2017: 4}).} This is also
something of a break with the existing literature. After all, Jason
Stanley's book on interest-relative epistemology is called
\emph{Knowledge and Practical Interests}. He defends a theory on which
what an agent knows depends on the practical questions they face. There
are strong reasons to think that theoretical reasons matter as well.

In Section~\ref{sec-oddsandstakes}, I suggested that someone knows that
\emph{p} only if the rational choice to make would also be rational
given \emph{p}. That is, someone knows that \emph{p} only if the answer
to the question \emph{What should I do?} is the same unconditionally as
it is conditional on \emph{p}. My preferred version of IRT generalises
this approach. Someone knows that \emph{p} only if the rational answer
to a question she is interested in is the same unconditionally as it is
conditional on \emph{p}. Interests matter because they determine just
what it is for the person to be interested in a question. Are the
questions, in this sense, always practical questions, or do they also
include theoretical questions? There are two primary motivations for
allowing theoretical interests as well as practical interests to matter.

The first comes from what Jeremy Fantl and Matthew McGrath call the
Unity Thesis~(\citeproc{ref-FantlMcGrath2009}{Fantl \& McGrath, 2009:
73--76}). They argue that whether or not \emph{p} is a reason for
someone is independent of whether they are engaged in practical or
theoretical deliberation. The intuition supporting this is quite clear.
Consider two people with the same background thinking about the question
\emph{What to do in situation S}. One of them is in \emph{S}, the other
is just thinking about it as an idle fantasy. Any reasoning one can
properly do, the other can properly do. Since one is facing a
theoretical question, and the other a practical question, the difference
between theoretical and practical questions can't be relevant.

Let's make that a little less abstract. Imagine Anisa is not actually
faced with the choice between Red-True, Blue-True, Red-False and
Blue-False with these particular red and blue sentences. In fact, she
has no practical decision to make that turns on the date of the Battle
of Agincourt. Instead, she is idly musing over what she would do if she
were playing that game. (Perhaps because she is reading this book.) If
she knows when the battle was, then she should be indifferent between
Red-True and Blue-True. After all, she knows they will both win \$50.
Intuitively she should think Red-True is preferable, both in the
abstract setting and when she's actually making the decision. This seems
to be the totally general case.

The general lesson is that if whether one can take \emph{p} for granted
is relevant to the choice between A and B, it is similarly relevant to
the theoretical question of whether one would choose A or B, given a
choice. Since those questions should receive the same answer, if
\emph{p} can't be known while making the practical deliberation between
A and B, it can't be known while musing on whether A or B is more
choiceworthy.

There is a second reason for including theoretical interests in what's
relevant to knowledge. There is something odd about reasoning from the
premise that the probability of \emph{p} is precisely \emph{x}, to the
conclusion that \emph{p}, in any case where \emph{x}~\textless~1. It is
a little hard to say, though, why this is problematic. We often take
ourselves to know things on grounds that we would admit, if pushed, are
probabilistic. The version of IRT that includes theoretical interests
explains this oddity. If we are consciously thinking about whether the
probability of \emph{p} is \emph{x}, then that's a relevant question to
us. Conditional on \emph{p}, the answer to that question is clearly no,
since conditional on \emph{p}, the probability of \emph{p} is 1. So
anyone who is thinking about the precise probability of \emph{p}, and
not thinking it is 1, is not in a position to know \emph{p}. That's why
it is wrong, when thinking about \emph{p}'s probability, to infer
\emph{p} from its high probability.

Putting the ideas so far together, we get the following picture of how
interests matter. Someone knows that \emph{p} only if the evidential
probability of \emph{p} is close enough to certainty for all the
purposes that are relevant, given their theoretical and practical
interests. Assuming the background theory of knowledge is non-sceptical,
this will entail that interests matter.

\section{Global Interest Relativity}\label{sec-global}

IRT was introduced as a thesis about knowledge. I'm going to argue in
Chapter~\ref{sec-ratbel} that it also extends to rational belief. We
need not stop there. At the extreme, we could argue that every
epistemologially interesting notion is interest-relative. Doing so gives
us a global version of IRT. That is what I'm going to defend here.

Jason Stanley (\citeproc{ref-Stanley2005}{2005}) comes close to
defending a global version. He notes that if one has both IRT, and a
`knowledge first' epistemology
~(\citeproc{ref-Williamson2000}{Williamson, 2000}), then one is a long
way to towards global IRT. Even if one doesn't accept the whole
knowledge first package, but just accepts the thesis that evidence is
all and only what one knows, then one is a long way towards globalism.
After all, if evidence is interest-relative, then probability,
justification, rationality, and evidential support are interest-relative
too.

That's close to the path I'll take to global IRT, but not exactly it. In
Chapter~\ref{sec-evidence} I'm going to argue that evidence is indeed
interest-relative, and so all those other notions are interest-relative
too. That's not because I equate knowledge and evidence. The version of
IRT I defend implies that evidence is a subset of knowledge, and which
subset it is turns out to be interest-relative.

There is a deep puzzle here for IRT. On the one hand, the arguments for
IRT look like they will generalise to arguments for the
interest-relativity of evidence.\footnote{I was first convinced of this
  by conversations with Tom Donaldson some years back. The earlier
  example of Parveen in the restaurant grew out of these conversations.}
On the other hand, the simplest explanation of cases like Anisa's
presupposes that we can identify Anisa's evidence independent of her
interests. That simple explanation says that Anisa shouldn't play
Blue-True because the evidential probability of the blue sentence being
true is lower than the evidential probability of the red sentence being
true. Since she can't rationally play Blue-True, it follows that she
mustn't know that the blue sentence is true. If evidence is identified
independently, this looks like it might generalise into a nice story
about when changes of interests lead to changes of knowledge. The story
looks much less nice if evidence is also interest-relative, and it is.

The aim of Chapter~\ref{sec-evidence} is to tell a story that avoids the
worst of these problems. On the story I'll tell, evidence is indeed
interest-relative, so we can't tell a simple story about precisely when
changes in interests will lead to changes in knowledge. Still, it will
be true that people lose knowledge when the evidential probability of a
proposition is no longer high enough for them to take it for granted
with respect to every question they are interested in.

\section{Neutrality}\label{sec-neutrality}

This book defends, at some length, the idea that knowledge is
interest-relative. I am, however, staying neutral on a number of other
topics in the vicinity.

\subsection{Neutrality about
Contextualism}\label{sec-neutrality-contextualism}

Most notably, I'm not taking any stand on whether contextualist theories
of knowledge are true or false. If you think that contextualism is true,
then what I'm defending is that the view that `knowledge' picks out in
this context, and in most other contexts, is interest-relative.

Contextualist theories of knowledge have a lot in common with
interest-relative theories. The kind of cases that motivate the
interest-relative theories, cases like Anisa's and Blaise's, also
motivate contextualism. They might even be seen as competitors, since
they are offering rival explanations of similar phenomena. They are not,
however, strictly inconsistent. Consider principles A and B below.

\begin{enumerate}
\def\labelenumi{\Alph{enumi}.}
\tightlist
\item
  A's utterance that \emph{B knows that p} is true only if for any
  question \emph{Q?} in which A is interested, the rational answer for B
  to give is the same unconditionally as it is conditional on \emph{p}.
\item
  A's utterance that \emph{B knows that p} is true only if for any
  question \emph{Q?} in which B is interested, the rational answer for B
  to give is the same unconditionally as it is conditional on \emph{p}.
\end{enumerate}

I endorse principle B, and that's why I endorse an interest-relative
theory of knowledge. If I endorsed principle A, then I would be (more or
less) committed to a contextualist theory of knowledge. And principle A
is not inconsistent with principle B.\footnote{There is a technical
  difficulty in how to understand one person answering an infinitival
  question that another person is asking themselves. The points I'm
  making in this section aren't sensitive to this level of technical
  detail.}

It isn't hard to see why cases like Anisa and Blaise can move one to
endorse principle A, and hence contextualism. It would be very odd for
Anisa to say ``This morning, I knew the Battle of Agincourt was in
1415.'' That's odd because she can't now take it as given that the
Battle of Agincourt was in 1415, and in some sense she wasn't in any
better or worse evidential position this morning with respect to the
date of the battle. Perhaps, and this is the key point, it would even be
false for Anisa to say this now. The contextualist, especially the
contextualist who endorses principle A, has a good explanation for why
that's false. The interest-relative theorist doesn't have anything to
say about that. Personally I think it's not obvious whether this would
be false for Anisa to say, or merely inappropriate, and even if it is
false, there may be decent explanations of this that are not
contextualist. (For instance, maybe knowledge is sensitive to what
interests one will have. Or maybe some kind of relativist theory is
true.) But there is clearly an argument for contextualism here, and it
isn't one that I'm going to endorse or reject.

One reason I'm not rejecting contextualism is that I'm not sure really
what it is. Here's a theory about `knows' that I think is interesting,
and I don't know whether it is contextualist. The word `knows' is
polysemous. It has three possible meanings. One of them is something
like Cartesian certainty. In this sense, most knowledge claims are
false. Another is something like information possession. In this sense,
my car might know lots of things, since its systems do quite reliably
store a lot of information. Finally, there is a moderate sense, which is
what we most commonly use. The difference between the three might even
be marked phonologically; the Cartesian sense is often somewhat drawn
out or otherwise emphasised. Is this contextualist? I don't know. Sort
of, I guess. It agrees with the standard contextualist account of the
appeal of scepticism. On the other hand, it denies that `knows' has the
kind of continuous variation that is typical in comparative adjectives
like `rich'. Since I think this kind of polysemy theory might be true,
and (independently) that it might be contextualist, I'm not in a
position to deny contextualism.

\subsection{Other Aspects of Neutrality}\label{sec-neutrality-k-nec}

As I've already noted, I'm making heavy use of the principle that
Jessica Brown calls K-Suff. I'm going to defend that at much greater
length in what follows. What I'm not defending is the converse of that
principle, what she calls K-Nec.

\begin{description}
\tightlist
\item[K-Nec]
An agent can properly use \emph{p} as a reason for action only if she
knows that \emph{p}.
\end{description}

The existing arguments for and against K-Nec are intricate and
interesting, and I don't have anything useful to add to them. All I will
note is that the argument of this chapter doesn't rely on K-Nec, and I'm
mostly going to set it aside.

I'm obviously not going to offer anything like a full theory of
knowledge. I am just defending a particular necessary condition on
knowledge. That condition entails that knowledge is interest-relative
given some common-sense assumptions about how widespread knowledge is.

I will be making one claim about how interests typically enter into the
theory of knowledge. I'll argue that there is a certain kind of
defeater. A person only knows that \emph{p} if the belief that \emph{p}
coheres in the right way with the rest of their attitudes. What's `the
right way'? That, I argue, is interest-relative. In particular, some
kinds of incoherence are compatible with knowledge if the incoherence
concerns questions that are not interesting.

So the impact of interests is (typically) very indirect. Even if the
other conditions for knowledge are satisfied, someone might fail to know
something because it doesn't cohere well with the rest of their beliefs.
What turns out to be most important here is an exception to this
exception clause. Incoherence with respect to uninteresting questions is
compatible with knowledge.

This is going to matter because it affects how we think about what
happen when interests change. It is odd to think that a change in
interests could make one know something. It isn't as odd to think that a
change in interests could block or defeat something that was potentially
going to block or defeat an otherwise well supported belief from being
knowledge. This is something I will return to repeatedly in
Chapter~\ref{sec-changes}.

\bookmarksetup{startatroot}

\chapter{Belief}\label{sec-belief}

\section{Beliefs and Interests}\label{sec-beliefsinterests}

One core premise of this book is that someone knows something iff they
properly take it to be settled. Taking something to be settled is what
we might call believing it. Or, at least, it's a philosophically
significant precisification of the notion of belief. Since belief and
settling will play such an important role in the rest of this book, I'm
going to discuss them here before we turn to knowledge.

The theory in this chapter owes a lot to proposals by Dorit Ganson
(\citeproc{ref-Ganson2008}{2008}, \citeproc{ref-Ganson2019}{2019}). Like
her, I'm going to develop a theory where we first say what it is to have
a belief in normal cases, then include an exception clause for what
happens in special cases, such as high-stakes or long-odds cases. The
details will differ in some respects, but the underlying architecture
will be the same.

And it also owes a lot to work by Jonathan Weisberg
(\citeproc{ref-Weisberg2013}{2013}, \citeproc{ref-Weisberg2020}{2020}).
Believing something is a matter of being willing to use that thing as an
input to deliberation.\footnote{In earlier work I'd identified beliefs
  with something that we computed from the outputs of deliberation. This
  was a mistake; I should have been focussing on the inputs not the
  outputs. I'll say much more in Chapter~\ref{sec-changes} about how my
  views on this point have changed.} If we assume perfect rationality,
it will often be possible to compute what inputs a thinker is using from
the the outputs of their deliberation. But it's a bad idea to assume
perfect rationality in the general case, and without that assumption the
inputs and outputs to deliberation can be arbitrarily far apart. And
when they are, it's the inputs that matter to what someone believes.
Here's how Julia Staffel puts the idea.

\begin{quote}
One of the most important differences between outright beliefs and
credences is how they behave in reasoning. If someone relies on an
outright belief in \emph{p} in reasoning, the person takes p for
granted, or treats \emph{p} as true. The possibility that ¬\emph{p} is
ruled out. By contrast, if someone reasons with a high credence in
\emph{p}, they don't take \emph{p} for granted. The possibility that p
might be false is not ruled out. (\citeproc{ref-Staffel2019}{Staffel,
2019: 939})
\end{quote}

What's essential to belief is that to believe something is to be willing
to use it as a starting point in deliberation. That slogan needs a lot
of qualification to be a theory, but as a slogan it isn't a bad starting
point.

Before we get too deep into this, I need to pause over a terminological
point. When I talk about belief here, I mean to talk about the
psychological aspect of knowledge. Roughly, that is, I'm talking about
the mental state which is such that when things go well the thinker has
knowledge, and which is indistinguishable from knowledge from the
thinker's perspective. I'm not interested here in how closely this
notion tracks the notion we pick out in ordinary language with words
like `believes' or `thinks'.

This caveat is important because of a notable recent argument that
belief is \emph{weak} (\citeproc{ref-HawthorneEtAl2015}{Hawthorne,
Rothschild, \& Spectre, 2016}). Imagine that some panelists on a TV show
are discussing the upcoming Champions League season. They are asked who
will win the League this year, and one of them says ``I think Tranmere
will win''. And without theorising about this too much, assume this is
an appropriate thing to say given their credal states and the situation
they are in. Now see what happens to this case when we adopt two more
premises. First, this is an honest and sincere self-report, they do, as
we'd ordinarily say, think that Tranmere will win. Second, `think' in
English means \emph{believes}. So this person believes Tranmere will
win. Note though that in the circumstances of the TV show, they could
say ``I think Tranmere will win'' even if they think Tranmere is merely
the most likely team to win, which might happen even if they think the
probability of that is very very low. (If there are \emph{n} teams in
the Champions League, and who knows what value \emph{n} will be when
you're reading this, their credence that Tranmere will win could be
maximal even if it is above 1 in \emph{n} by an arbitrarily small
amount.) Yet surely this person would not, at least responsibly, take
Tranmere's winning to be a starting point in deliberation.

Now there are a lot of things we could say about that argument. I
wouldn't want to sign up for either of the two premises that I mentioned
in the middle of the paragraph. I'm sympathetic to the criticisms of the
argument that Timothy Williamson makes in ``Knowledge, Credence, and the
Strength of Belief'' (\citeproc{ref-Williamson2022}{Williamson, 2022}).
For now though I just want to note that this is a discussion of a
separate topic to the one I'm discussing. And in identifying the topic
as I have, I'm working within a very standard, and very long, tradition.
Here's Pasnau, responding to a similar kind of challenge in the context
of interpreting historical figures.

\begin{quote}
I do not know of any historical figure who resists the idea that we can
identify a kind of mental state, in the vicinity of assent, which can
serve as a component in analyzing what it is to be in some more exalted
epistemic state, in the vicinity of knowledge. What that component state
gets called varies from century to century and from author to author.
For Buridan, for instance, it will not be called \emph{opinio}, because
``\emph{opinio} signifies a defect from scientia in some way'' (Summulae
VIII.4.4, trans. p.~710). But this is just a point about that Latin
word, as it gets used at that moment in time, and goes no deeper than
the analogous observation today that a \emph{guess} cannot count as
\emph{knowledge}, no matter what gets added to it. Accordingly,
throughout these lectures, I use `belief' to pick out the mental state
that is a constituent in the epistemically ideal state of
\emph{scientia} and so on, without fussing over whether `belief'
corresponds to \emph{assensus},~\emph{credere},~\emph{opinio}, and so
on. (\citeproc{ref-Pasnau2017}{Pasnau, 2017: 219})
\end{quote}

I agree with all of that except possibly for the clam that belief is
strictly speaking a constituent of scientia, or of knowledge. I want to
leave open, at least at this stage, a knowledge first account where
belief is something like attempted knowledge. If that's right, knowledge
would be a constituent of belief, and not vice versa. What's crucial is
that there are close, even analytic, ties between belief as it's being
used here and knowledge. Since our TV panelist can't know, and can't
reasonably think they know, that Tranmere will win, their expression
can't be an expression of belief, in this philosophically significant
sense, that Tranmere will win.

Here's another way to put the point. It's a starting point in a lot of
work in action theory that there is a true principle somewhere in the
vicinity of the following idea.

\begin{quote}
Zach intends to do some action,~\emph{A}. And he believes that to do
\emph{A}, he must do \emph{B}. Zach bears an interesting and important
normative relationship to \emph{B}. It is an action that he believes to
facilitate his intended end, and something is going wrong, if he intends
\emph{A}, believes \emph{B} to be necessary for \emph{A}, has reflected
clear-headedly on this fact, and yet still fails to intend to do
\emph{B}. (\citeproc{ref-Schroeder2009}{Schroeder, 2009: 223})
\end{quote}

There are challenges about how to make this principle quite right in
cases where Zach shouldn't intend to do \emph{A}. If the `belief is
weak' thesis is correct, however, the whole tradition in action theory
that Schroeder is here joining is fundamentally mistaken. From the
intention to do \emph{A}, and the best guess that the only way to do
\emph{A} is \emph{B}, it does not follow at all that coherence requires
intending to do \emph{B}. Since I don't think that the entire literature
on means-end coherence was based on fundamentally misunderstanding the
nature of belief, I'm going to assume that we have a strong notion of
belief. Just how it relates to the English words `guess', `think' and
even `believes' is left as an issue for another day.

\section{Maps and Legends}\label{sec-mapslegends}

Beliefs, Frank Ramsey famously said, are maps by which we steer
(\citeproc{ref-RamseyGeneralProp}{Ramsey, 1990: 146}). This can be
turned into an argument that belief should be interest-relative as well.
This argument isn't quite right (contrary to my earlier views), but it's
instructive to see why it goes wrong. First let's explore Ramsey's
analogy a bit more closely.\footnote{The picture I'm sketching about the
  map-like nature of belief is similar to the one that Seth Yalcin has
  defended in his (\citeproc{ref-Yalcin2018}{2018}) and, especially,
  (\citeproc{ref-Yalcin2021}{2021}). That's not to say he would endorse
  any of the conclusions here, but simply to note that he has set out
  the the idea that belief is less like a map and more like an atlas,
  and put that idea to philosophical work.}

When I was growing up in car-dependent, suburban Melbourne, the main
street directory that was used was the Melways. This was a several
hundred page thick book that most people kept a copy of in their car. It
largely consisted of page after page of 1:20,000 scale maps of the
Melbourne suburbs, plus more detailed maps of the inner city, and then
progressively less detailed maps of the rural areas around Melbourne,
the rest of Victoria, and finally of the rest of the country. And it was
everywhere. It was common for store advertisements, party invitations
and event announcements to include the Melways page and grid coordinates
of the location. In fact I was a little shocked when I moved to America
and I found it was socially expected (in those pre-Google Maps days)
that you would give people something like turn by turn directions to a
location. I was used to just telling people where something was, i.e.,
giving them the Melways grid coordinates, and letting them use the map
to get themselves there. The Melways really was, collectively, the map
by which we steered.

But you wouldn't want to use it for everything. You wouldn't want to use
it as a hiking map, for example. For one thing, it was much too heavy.
For another, it was patchy on which walking trails it even included, and
had almost no usable topographical information. You steer yourself by
one map when you drive, and another map (or set of maps) when you hike.
What one steers by should be a function of one's interests. And the same
is true of belief. For most people, beliefs are interest-relative
because to believe something is to steer yourself by a map that
represents the world as being that way, and which map one will steer by
is sensitive to one's interests.

Maybe you think this argument leans too heavily on Ramsey's analogy of
beliefs and maps. But once you see the structure of the case, you can
get more purely cognitive examples. (And this in turns helps us see the
brilliance of Ramsey's metaphor.) If you or I were in Anisa's position,
then we would not include the fact that the Battle of Agincourt was in
1415 on the map by which we steer through the Red-Blue game, even if we
would typically include it on our map. When I'm reading the morning
papers and thinking about the effects of some economic policy, such as a
proposed minimum price for alcohol, I'll steer myself by the maps given
in introductory economics texts. That is, I'll just use simple
supply-demand graphs to predict the effects of the policy. Still, I
won't always do that. For example, I won't do it when thinking about
changes in the minimum wage, because systematic changes like that push
simple models beyond their breaking point.\footnote{I've said in the
  text that I believe that simple supply-demand models are right for
  some purposes. At least, I implied that when I said I steer by them,
  and that beliefs are maps by which we steer. Some philosophers think
  this is wrong, and that one only ever accepts these simple models,
  rather than believes them. Once we allow beliefs to be
  interest-relative, this role for the belief/acceptance distinction
  seems to go away. A lot of what are commonly called acceptances are,
  on my theory, beliefs that are highly sensitive to changes in
  interests.} Or we can mix and match the practical and the theoretical.
If there is a proposed price floor on something widely traded (like
electricity), and my predictions about the effects of this change have
even a small practical significance (e.g., I'm thinking about whether my
small business should lock in the price it purchases electricity at for
three years), then I might not use the simple model. In this case the
combination of theoretical and practical interests will change which map
I steer by, i.e., what I believe, even if neither interest on their own
would have been enough to bring about a change.

So it looks like belief is interest-relative, and that's for deep
reasons about the role that belief plays in our cognitive economy. To
believe something is to steer by a map that represents it as true. To
steer by it, in this sense, is to take it as given in our inquiries. For
normal people, what is taken as given is dependent on what question one
is interested in. So for normal people, belief is interest-relative. I
used to think that this could be extended to an argument that it was
part of the metaphysics of belief that it was interest-relative. But as
we'll see in the next section, that isn't quite right. The restriction
to `normal people' a couple of sentences back turns out to be essential,
and this creates complications.

\section{Belief and Stubbornness}\label{sec-stubbie}

Things get complicated when we stop focussing on what normal (or
normal-ish) people do, and think about less common reactions. So
consider a person, call them Stubbie, who uses the same maps and models
for every task. They use the Melways for hiking, they make
macro-economic forecasts using simple supply-demand models, they take
ordinary knowledge for granted in high stakes and long odds cases, and
so on. And they do this even though they know full well that there are
excellent reasons to be more flexible. What should we say about Stubbie?

I think we should say that Stubbie is irrationally stubborn, and part of
his irrationality consists in steering by the same map, in holding onto
the same beliefs, in situations where this is uncalled for. Stubbie
really believes that a simple supply-demand model is predictive even in
complicated cases. He's wrong, and his evidence shows that he is wrong,
but our theory of belief had better allow for some false, irrational
beliefs.

Stubbie's example shows that while one's beliefs should be
interest-relative, they need not be. One should steer by a map suitable
to the circumstances. If one stubbornly steers by the same map come what
may, the fact that it would be advisable to steer by different maps at
different times does not affect what one believes. Stubbie really is
steering himself by the Melways when hiking, and he really believes the
simple economic model he uses.

This shows that one can be a believer, without having those beliefs be
sensitive to one's interests. That suggests that the interest-relativity
of belief comes from the norms - how one should believe - not the
metaphysics - what belief itself is.

There is another complicated variant of this example that raises deeper
questions about the relationship between belief and interests. Imagine
that Stubbie is disposed to keep taking what history books say about
Agincourt for granted. Now he is faced with a decision where a lot rides
on this practice. Perhaps he is playing a version of the Red-Blue game
where the prize is \$50,000, not \$50. And the shock of having that much
at stake causes him to reconsider. So he goes back to thinking it merely
probable that the Battle of Agincourt was in 1415. This is not a case of
interest-relativity of belief. Rather, it is like the kind of case
Jennifer Nagel (\citeproc{ref-Nagel2010}{2010}) discusses, when she
talks about beliefs being causally sensitive to interests. And this
shows we have to be careful to be sure that a case of
interest-sensitivity is really a case where belief is constitutively,
and not merely causally, sensitive to interests.\footnote{In earlier
  work I was not careful on exactly this point. I'll say more about this
  in Chapter~\ref{sec-changes}.}

This version of Stubbie's case opens up the possibility that no beliefs
are really interest-relative. Sometimes a change in circumstances might
cause someone to change the map they steer by, but that's the only way
that interests matter. I don't think this is right, but I'm much less
confident of this than I am of most of the other claims in this book.

There are three significant differences between the way that interests
change the beliefs of normal people to how they change Stubbie's
beliefs. First, they are reversible. Someone who switches to a more
complicated model, or to thinking that a source provides probability
rather than knowledge, can easily switch back. Second, they are
predictable. For a reasonably well-functioning thinker, we can say when
they will switch maps. It will be when the stakes are high, or the odds
are long, or the question pushes on the limitations of their models.
Third, they are not emotionally loaded. The natural way to tell this
variant of Stubbie's story involves shock; he feels the change in his
attitude. But when you or I play the Red-Blue game, we switch from
thinking something is true to thinking it is probable without any
significant phenomenology. I think these three differences are enough to
justify saying that in the normal case, the change of interests
constitutes a change of beliefs, while in Stubbie's case, the change of
interests merely causes a change of beliefs. And if that's right, the
belief itself is interest-relative, in normal cases.

But whether we accept the argument of the last paragraph or not, it
won't affect what we say about Anisa. She believes the Battle of
Agincourt was in 1415. This belief is irrational; she should have
switched to thinking it is merely probable that the battle was in 1415.
The change in the rational status of her belief is constituted by, and
not merely caused by, her change in interests. So interests can be
constitutively relevant to rational belief, even when they don't affect
belief.

The next two sections aims to turn these Ramseyan observations about the
relationship between beliefs and interests into a theory of belief.

\section{Taking As Given}\label{sec-given}

To start towards a positive theory of belief, it helps to think about
the following example, featuring a guy I'll call Sully. (This example is
going to resemble the examples involving Renzo in Ross \& Schroeder
(\citeproc{ref-RossSchroeder2014}{2014}), and at least for a while, my
conclusions are going to resemble theirs as well.) Sully is a fan of the
Boston Red Sox, and one of the happiest days of his life was when the
Red Sox broke their 86 year long curse, and won baseball's World Series
in 2004. He knows, and hence believes, that the Red Sox won the World
Series in 2004. He likes their chances to win again this year, because
in Sully's heart, hope always springs eternal.

It's now the start of a new baseball season, and Sully is offered, for
free, a choice between the following two bets.

\begin{itemize}
\tightlist
\item
  Bet A wins \$50 if the Red Sox win the World Series this year, and
  nothing otherwise.
\item
  Bet B wins \$60 if the same team wins the World Series this year as
  won in 2004, and nothing otherwise.
\end{itemize}

For Sully, this choice is a no-brainer. If the Red Sox win this year, he
wins more money taking B than A. If the Red Sox don't win this year, he
gets nothing either way. So it's better to take B than A, and that's
what he does.

What Sully has done here is use dominance reasoning, in particular weak
dominance reasoning. One option weakly dominates another if it might
have a better return, and can't have a worse return. Weak dominance is
used as an analytical tool in game theory. It is also a form of
inference that non-theorists, like Sully, can use. (Though unless
they've taken a game theory course they might not use this phrase to
describe it.)

Sully's case can be distinguished from that of his more anxious friend
Mack. Mack is also a big Red Sox fan, and also looks back on that
curse-busting World Series win with fondness. But if you offer Mack the
choice between these two bets, he'll hesitate a bit. He'll wonder if
he's really sure it was 2004 that the Red Sox won. Maybe it was 2005 he
thinks. He'll eventually think that even if he's not completely sure
that it was 2004, it was very likely 2004, and so it is very likely that
bet B will do better, and that's what he will take.

Even if Sully and Mack end up at the same point, they have used very
different forms of reasoning. Sully uses weak dominance reasoning, while
Mack uses probabilistic reasoning. Sully takes the fact that the Red Sox
won in 2004 as given, while Mack just takes it to be very likely. The
big thing I want to rely on here is that these are very different
psychological processes. Neither of these guys is doing something that
approximates, or simplifies, the other; they both take bet B, but they
get to that conclusion via very different routes.

There is a theoretical analog to this psychological point. Many game
theorists, perhaps most, think that weak dominance reasoning can be
iterated more or less indefinitely. (That's not to say that they are
right; I'm trying to make a point about conceptual distinctiveness here,
not game theory.) But few if any think that likelihood reasoning can be
iterated indefinitely. This reflects the fact that they are very
different kinds of reasoning. Dominance reasoning is pre-probabilistic.

Sully's reasoning isn't just dominance reasoning. It's dominance
reasoning that relies on a contingent assumption, namely that the Red
Sox won the World Series in 2004. When Sully reasons that A can't do
better than B, he's not drawing any kind of logical or metaphysical
point. It's logically and metaphysically possible that the Red Sox lost
in 2004. For that matter, and this is a point Ganson
(\citeproc{ref-Ganson2019}{2019}) stresses, it's logically and
metaphysically possible that the payouts for A and B are other than what
Sully thinks they are.

And thought he might not make it explicit, at some level Sully surely
knows this. If pushed, he'd endorse the conditional ``If I've
misremembered when the curse-busting World Series win was, and the Red
Sox didn't win in 2004, then bet A might do better than bet B''. So
while he is disposed to use dominance reasoning in deciding whether to
take A or B, this disposition rests on taking some facts about the world
for granted.

Recall the disjunctive way that Sully reasoned. Either the Red Sox will
win this year or they won't. Either way, I won't do better taking bet A,
but I might do better taking bet B. So I'll take bet B. This reasoning -
not just the reasons Sully has but his reasoning - can be appropriately
represented by the kind of decision table that is familiar from decision
theory or game theory.

\begin{longtable}[]{@{}rcc@{}}
\caption{Betting on the Red Sox}\label{tbl-redsox}\tabularnewline
\toprule\noalign{}
~ & Red Sox Win & Red Sox Don't Win \\
\midrule\noalign{}
\endfirsthead
\toprule\noalign{}
~ & Red Sox Win & Red Sox Don't Win \\
\midrule\noalign{}
\endhead
\bottomrule\noalign{}
\endlastfoot
Take Bet A & \$50 & \$0 \\
Take Bet B & \$60 & \$0 \\
\end{longtable}

Focus for now on the columns in this table. Sully takes two
possibilities seriously: that the Red Sox win this year, and that they
don't. The `possibilities' here are possibilities in the sense described
by Humberstone (\citeproc{ref-Humberstone1981}{1981}). They have content
- in one of them the Red Sox win, in the other they don't, but they
don't settle all facts. In the right-hand column, there is no fact of
the matter about which other team wins the World Series. In neither
column is there a fact of the matter about what Sully will have for
lunch tomorrow. If you want to think of these in terms of worlds, they
are both very large sets of worlds, and within those sets there is a lot
of variability.\footnote{Analysing these possibilities as sets of worlds
  is unhelpful when we want to use a model like this to represent modal
  or logical uncertainty. Still, it's often a helpful heuristic, and
  there isn't anything wrong with using a model that breaks down when
  applied outside its appropriate zone.}

But there is more to the content of each column than what is explicitly
represented in the header row. In each column, for example, the Red Sox
won in 2004. That's why Sully can put those monetary payoffs into the
cells. And in each column, the terms of the bet are as Sully knows that
they are. In sets of worlds terms, the sets that are represented by the
columns are exclusive, but far from exhaustive.

Consider those propositions which are true according to all of the
columns in this table. Say a proposition is \emph{taken as given} in a
decision problem when it the decider treats one option as dominating
another, and does so in virtue of a table in which that proposition is
true in every column. Then here is one principle about belief that seems
to be very plausible.

\begin{description}
\tightlist
\item[Given]
S believes that \emph{p} only if there is some possible decision problem
such that S is disposed to take \emph{p} as given when faced with that
problem.
\end{description}

Given is logically weak in one respect, and strong in another. It only
requires that S be willing to take \emph{p} for granted in one possible
choice. It doesn't have to be a likely, or even particularly realistic
choice. Sully is unlikely to have strangers offer him these free money
bets. Given how representationally sparse decision tables are, for
something to be true in all columns of a decision table is a very strong
claim. It doesn't suffice, for instance, for \emph{p} to be true in some
columns and false in none. Each column has to take a stance on \emph{p},
and endorse it.

I will have much more to say about the relationship between decision
tables like Table~\ref{tbl-redsox} in Section~\ref{sec-makdec}. First,
however, I need to say more about belief. I used to think that Given, or
something like it, could be strengthened into a biconditional, and from
there we could get something like a functionalist analysis of belief.
That turns out not quite to be right. Being disposed to sometimes take
\emph{p} as given is not sufficient for belief. If Anisa had played the
Red-Green game rationally, she would have lost any belief about when the
Battle of Agincourt was. To explain cases like that, we need to expand
our theory of belief.

\section{Blocking Belief}\label{sec-block}

Imagine a person, call him Erwan, who is made the offer Blaise is made,
but declines it. He declines on the very sensible grounds that the
Battle of Agincourt might not have been in 1415, and he does not want to
run the risk of sending everyone to the Bad Place. If we stop our theory
of belief with Given, then we have to say that Erwin has some kind of
weird pragmatic incoherence. He believes that \emph{p}, and wants what
is best for everyone, but won't do the thing that will, given his
beliefs, produce what is best for everyone. Declining the bet is not
practically incoherent in this way. So Erwin does not believe that the
Battle of Agincourt was in 1415. At least, he doesn't believe that at
the time he is declining the bet.

So a theory of belief with any hope of being complete needs some
supplementation. The idea I'll use is one that seems prima facie like it
might apply without restriction. A little reflection, however, shows
that it will ultimately need to be restricted, and the most natural
restrictions are pragmatic.

Imagine that we don't ask Erwin whether he is prepared to bet the
welfare of all of humanity on historical claims, but instead ask him a
simple factual question H.

\begin{enumerate}
\def\labelenumi{\Alph{enumi}.}
\setcounter{enumi}{7}
\tightlist
\item
  How many (full) centuries has it been since the Battle of Agincourt?
\end{enumerate}

Erwin will think to himself, ``Well, the Battle of Agincourt was in
1415, and that's a bit over 600 years ago, so that's six centuries. The
answer is six.'' Now compare what happens if we ask him this slightly
more convoluted question.

\begin{enumerate}
\def\labelenumi{\Roman{enumi}.}
\tightlist
\item
  If the Battle of Agincourt was in 1415, how many (full) centuries has
  it been since the Battle of Agincourt?
\end{enumerate}

Erwin will give the same answer, i.e., six. And he will give it for
basically the same reasons. Indeed, apart from the date of the Battle
being one of his reasons in answering H, and not needed to answer I, he
has the same reasons for answering the two questions with six. I mean
that both in the sense that what justifies giving the answer six is the
same for the two questions, and in the sense that what causes him to
answer six is the same for the two questions. (With the exception that
the date of the Battle is a reason in answering H, but not in answering
I.)

Say that a person answers the questions \emph{Q?} and \emph{If p, Q?} in
the same way if they offer the same answer to the two questions, and
their reasons (in both senses) for these answers are the same except
only that \emph{p} is one of the reasons for their answer to \emph{Q?}.
Then here is a plausible principle about belief - albeit one that isn't
going to be quite right.

\begin{description}
\tightlist
\item[Unrestricted Conditional Questions]
If S believes that \emph{p}, then for any question \emph{Q?}, S is
disposed to answer the questions \emph{Q?} and \emph{If p, Q?} the same
way.
\end{description}

Note that in saying these questions are answered the same way, I really
don't just mean that they get the same answers. I will offer the same
answer to the questions \emph{What is one plus one?} and \emph{What is
the largest n such that
x\textsuperscript{n}~+~y\textsuperscript{n}~=~z\textsuperscript{n} has
positive integer solutions?}, but I don't answer these questions the
same way. My reasons for the first answer are quite closely related to
the fact that one plus one does equal two. My reasons for the second
answer are almost wholly testimonial. So in the sense relevant to
Unrestricted Conditional Questions, I do not answer each question the
same way.

I'm understanding what a conditional question is in a particular way,
one I'll describe in the next paragraph. I think this is how conditional
questions usually work in English, so the shorthand \emph{If p, Q?} that
I'm using is not misleading. But I don't intend to defend a particular
claim about the way natural language conditionals work. That would be
another whole book. (Or more.) So I intend to use this shorthand
\emph{If p, Q?} somewhat stipulatively, as follows.

\emph{If p, Q?} is the question \emph{Q?} asked under the assumption
that \emph{p} can be taken as given. So the question \emph{If p, how
probable is q?} is asking for the conditional probability of \emph{q}
given \emph{p}. The question \emph{If p, which option is most useful?}
is asking for a comparison of the conditional utilities of the various
options. And the question \emph{If p, must it be that q?} gets an
affirmative answer if all the (salient) possibilities where \emph{p} is
true are ones where \emph{q} is true. (So it becomes very close to
asking if the material implication \emph{p} ⊃ \emph{q} must be true.)
Now notoriously it is difficult to connect these conditional questions
with questions about the truth of any conditional.\footnote{See Lewis
  (\citeproc{ref-Lewis1976b}{1976}, \citeproc{ref-Lewis1986h}{1986}) on
  the issues about conditional `how probable' questions; Lewis
  (\citeproc{ref-Lewis1988}{1988}, \citeproc{ref-Lewis1996}{1996}) on
  the issues about conditional `how useful' questions; and Gillies
  (\citeproc{ref-Gillies2010}{2010}) on issues about modals in the
  consequent of conditional questions.} But I'm setting all those issues
aside here. Everything that I say about conditional questions I could
say, more verbosely, by making it explicit that they are to be
understood as questions about conditional probability, conditional
utility, conditional modality, and so on.

Now thinking about a few simple cases might make it seem that
Unrestricted Conditional Questions is true. After all, there is
something very odd about a counterexample to it. It would have to be a
case where S believes that \emph{p}, and there is a way they are
disposed to get answer \emph{If p, Q?}, i.e., to get from \emph{p} to an
answer to \emph{Q?}, but they are not disposed to use that to answer
\emph{Q?}. That seems at best rather odd.

There is one potential counterexample that I don't think ultimately
undermines Unrestricted Conditional Questions. There could be a case
where I believe \emph{p}, and \emph{p} is relevant to \emph{Q?}, but I
don't realise its relevance. On the other hand, when I am explicitly
asked \emph{If p, Q?}, being reminded of \emph{p} makes me see the
connection, so I follow the natural path from \emph{p} to an answer to
\emph{Q?}. These kind of one-off performance errors are, sadly, easy to
make. As long as they are one-off, they don't threaten the principle
connecting dispositions.

A bigger problem comes from the two cases that I started the book with.
If the Battle of Agincourt was in 1415, then Anisa maximises expected
utility by playing blue-true, and Blaise maximises expected utility by
taking the bet. So answer to the conditional questions \emph{If the
Battle of Agincourt was in 1415, what options of Anisa's maximse
expected utility?} and \emph{If the Battle of Agincourt was in 1415,
what option of Blaise's maximses expected utility?} have different
answers to the corresponding unconditional questions. Or at least so say
I, and hope you do too. So if Unrestricted Conditional Questions is
true, then none of us have ever believed that the Battle of Agincourt
was in 1415. That can't be right, so there must be some restriction on
the principle.

Happily, a restriction isn't too hard to find. The principle just needs
to be restricted to questions that the subject is currently taking an
interest in. When we're thinking about questions like H and I, then we
do have beliefs about when the Battle of Agincourt was. Were we to be
placed in Anisa or Blaise's situation, or arguably when we even think
about their situation, we lose this belief. So I suggest the following
principle is true, and explains a lot of the cases that have been
discussed so far.

\begin{description}
\tightlist
\item[Relevant Conditional Questions]
If S believes that \emph{p}, then for any question \emph{Q?} that S is
currently taking an interest in, S is disposed to answer the questions
\emph{Q?} and \emph{If p, Q?} the same way.
\end{description}

As I argued in Section~\ref{sec-whatinterests}, whether one is
interested in a question isn't just a matter of one's practical
situation. One can be interested in a question because one is thinking
about what to do should it arise, or because one is just naturally
inquisitive. Many of the questions we're interested in are practical
questions, but not all of them are.

I've argued that Given and Relevant Conditional Questions are necessary
conditions on belief. Very roughly, I think they are jointly sufficient
for belief. I say `roughly' because I don't mean to take a stance on,
say, whether animals have beliefs, or whether one can have singular
thoughts about things one is not acquainted with. A more accurate claim
is that if it is plausible that S is the kind of thing that can have
beliefs, and \emph{p} is the kind of thing it could in principle have
beliefs about, and both Given and Relevant Conditional Questions are
satisfied, then S believes that \emph{p}.

Obviously neither Given nor Relevant Conditional Questions would be
particularly helpful principles to use in providing a reductive
physicalist account of mental content. They say something about
necessary conditions for belief, but the statement of those conditions
makes a lot of assumptions about other content-bearing states of the
agent. So even if these conditions are individually necessary and
jointly sufficient for belief, they wouldn't be any kind of analysis or
reduction of belief.\footnote{Compare: One can consistently deny that
  any analysis or reduction of \emph{knowledge} is possible and say that
  the condition \emph{p is part of S's evidence} is both necessary and
  sufficient for S to know that \emph{p}.} But they could be part of a
theory of belief, and the theory they are part of is helpful for seeing
how beliefs and interests fit together.

\section{Questions and Conditional Questions}\label{sec-questions}

In the previous section I defended this principle:

\begin{description}
\tightlist
\item[Relevant Conditional Questions]
If \emph{S} believes that \emph{p}, then for any question \emph{Q?} that
\emph{S} is currently taking an interest in,~\emph{S} is disposed to
answer the questions \emph{Q?} and \emph{If p, Q?} the same way.
\end{description}

To spell out what that principle amounts to, I need to say something
about what questions are, and what conditional questions are. I'm going
to say just enough about questions to understand the principle. This
won't be anything like a full theory of questions. While much of what I
say will draw on insights from theorists who have worked on questions in
natural language, I'm not primarily interested in how questions are
expressed in natural language. Rather, I'm interested in the contents of
these questions. These contents are interesting because they can be the
contents of mental states. For example, a cat can wonder where a mouse
is hiding. There are deep and fascinating issues about how we can and do
talk about the cat, and the cat's attitudes, but I'm more interested in
the cat's relationship to the question \emph{Where is the mouse hiding?}
than I am in our talk about the cat.\footnote{A useful introduction to
  ways in which questions are relevant to philosophy of language is the
  Stanford Encyclopedia article by Cross \& Roelofsen
  (\citeproc{ref-sep-questions}{2018}). A canonical text on the role of
  questions is Roberts (\citeproc{ref-Roberts2012}{2012}). Roberts
  originally circulated that paper in 1996. Since then it has influenced
  a huge range of works, including this one.}

The simplest questions are true/false questions, like \emph{Did the
Boston Red Sox win the 2018 World Series?}. These won't play a huge role
in what follows, but they are important to have on the table. I am going
to assume that whenever someone considers a proposition, and they don't
take its truth value to be settled, they are interested in the question
of whether it is true.

Next, there are quantitative questions, where the answer is some number
or sequence of numbers.\footnote{I'm including here any question that
  could be answered with a number or sequence of numbers, even if that
  would not be the most usual, or the most helpful, way to answer them.
  So \emph{Where is Fenway Park?} is a quantitative question, because
  \emph{42.3467° N, 71.097° W} is an answer, even if \emph{The corner of
  Jersey St and Van Ness St} is a better answer.} One tricky thing about
quantitative questions is that they may admit of imprecise answers, but
need not. If I ask ``When does tonight's Red Sox game start?'', an
answer of ``Seven'' would usually be acceptable, even if the game
actually starts at a few minutes after seven. That's because, I take it,
the truth conditional content of the utterance ``Seven'' in this context
is that tonight's Red Sox game starts at approximately seven, and I'm
asking a question that admits of an approximate answer. I could have
been asking a question where the only acceptable answer would be the
time that the Red Sox game starts to the nearest minute, or even to the
nearest second. And I could even have asked that question using those
exact same words. (Though if I intended to ask the question about
seconds, using these words would be extremely unlikely to result in
communicative success.)

The main thing that matters for the purposes of this book is that the
questions with different appropriate answers are different questions.
Even if one would normally use the same words in English to express the
questions, the fact that they have different acceptable answers shows
that they are different questions. And as noted above, what really
matters for this book is the mental representation of the contents of
questions. There could be two people who we could report as wondering
when tonight's Red Sox game starts, but one of them will cease wondering
if they find out that it starts around seven, and the other still
wonders which minute near seven it will start at. These people are
wondering about different questions.

The more precise a numerical question one is considering, the fewer
things one can rationally take for granted in trying to answer it. So
the version of IRT I defend implies that the more precise a numerical
question one is considering, the fewer things one knows. Or, to put the
same point another way, the less precise a numerical question one is
considering, the less impact interest-relativity has on knowledge. This
will matter when thinking about how the theory applies to various
examples. If we have ascribe to a thinker an interest in an
unrealistically precise question, we might draw implausible conclusions
about what IRT says about them. But this isn't a consequence of IRT;
it's a consequence of not getting clear about which question a thinker
is considering.

Next, there are questions that ask to identify an individual or a class
of individuals. A striking thing about these questions is that they
often have so-called `mention-some' readings. To understand what this
means, compare these two little exchanges.

\begin{enumerate}
\def\labelenumi{\arabic{enumi}.}
\item
  \begin{enumerate}
  \def\labelenumii{\alph{enumii}.}
  \tightlist
  \item
    Who was in the Beatles?
  \item
    John Lennon was in the Beatles.
  \end{enumerate}
\item
  \begin{enumerate}
  \def\labelenumii{\alph{enumii}.}
  \tightlist
  \item
    Where can I get good coffee in Melbourne?
  \item
    You can get good coffee at Market Lane.
  \end{enumerate}
\end{enumerate}

There is something wrong with 1b as an answer to 1a. It's true that John
Lennon was in the Beatles. But an ordinary use of 1a will be to ask for
the names of everyone in the Beatles, not just one person in them.
(There are exceptions, and it's a fascinating task for another day to
work out when they occur.) On the other hand 2b is a perfectly good
answer to 2a. (Or so I think, but my knowledge of Melbourne coffee is a
little out of date.) It is definitely not necessary to properly answer
2a that one list every place in Melbourne where one can get good coffee.
That could take some time. Moreover, 2b does not (on its most natural
reading) imply that Market Lane is the only place in Melbourne to get
good coffee.

An answer is a `mention-some' answer when it does not imply exhaustivity
in this sense. And a question admits of mention-some answers when it is
properly answered with a mention-some answer. Lots of questions asking
for individuals will be mention-some questions in this sense, but not
all of them will. And, again, it is important to understand what kind of
question is being asked to think about whether it is satisfactorily
answered by an answer that does not imply completeness or
exhaustiveness.

Next, there are questions with infinitivals, such as the following.

\begin{itemize}
\tightlist
\item
  When to visit Venice?
\item
  How to climb Ben Nevis?
\item
  What to do?
\end{itemize}

In most dialects of English, it is rare to use these to simply ask
questions.\footnote{My hunch is that there is quite a bit of dialectical
  variation here, I would need to do much more empirical research to
  back this up.} But they can be the complements of any number of verbs.
Any of the three questions above, like any number of other questions
with infinitivals, can complete sentences like

\begin{itemize}
\tightlist
\item
  A doesn't know \ldots{}
\item
  B is wondering \ldots{}
\item
  C wants D to tell him \ldots{}
\end{itemize}

Mixing and matching the sentence fragments from the last two lists
produces nine different sentences. Some examples of these are

\begin{itemize}
\tightlist
\item
  C wants D to tell him how to climb Ben Nevis.
\item
  A doesn't know what do do.
\item
  B is wondering whether the visit Venice.
\end{itemize}

The philosophical work on these kinds of sentences has been almost
exclusively focussed on just one of the nine sentences I just described:
the one combining a knowledge verb with a `how to' question. I suspect
this is a mistake; what to say about `know how' reports is going to have
a lot in common with what to say about `wondering when' reports. (Here
I'm agreeing with Stanley -Stanley (\citeproc{ref-Stanley2011}{2011}),
though I'm about to disagree with him on a related point.)

There is a puzzle about why, in English, we cannot use these questions
to complete sentences like

\begin{itemize}
\tightlist
\item
  E believes \ldots{}
\item
  F suspects \ldots{}
\item
  G wants H to guess \ldots{}
\end{itemize}

I'm going to set that puzzle aside, as interesting as it is, an just
focus on the sentences we can produce in English.

I'm going to call these questions with infinitivals \emph{practical
questions}. One thing to note about them is that they are are usually
mention-some. When I am wondering what to buy in the supermarket, and I
resolve this by choosing one particular carton of eggs, I don't thereby
imply that there is anything defective about the other cartons. I just
choose some eggs.

For related reasons, answering a practical question like this is
distinct from answering any question, or questions, about the modal
status of different actions. Imagine that in the grip of choice-phobia I
am stuck staring at the cartons of eggs, unable to decide which one to
buy because they are all just alike. In that situation I might know that
there is no carton such that it is what I should buy, and also that
there are many cartons such that I could (rationally, morally) buy any
one of them. But there are so many, and they are so alike and I can't
decide, so I don't know what to buy.\footnote{This discussion will
  probably remind many readers of the story of Buridan's ass, who was
  stuck between two equally appetizing bales of hay. As Peter Adamson
  (\citeproc{ref-Adamson2019}{2019: 453ff}) points out, the connection
  of this example to Buridan is not the one philosophers usually assume.
  That is, it's not Buridan's example. An example of roughly this kind
  was earlier given by al-Ghazālī. And the example involving the ass was
  not given by Buridan at all, but by his opponents, objecting to
  Buridan's own equation of choice with judgment that something is best
  to do. That's the role the example will play a few times in this book,
  as a critique of theories that equate choice with formation of a
  belief about goodness. My earlier versions of IRT, which equated that
  choosing to do something with judging it has highest expected utility,
  will be among the theories thus targeted.}

Resolving this indecision will not involve accepting any modal
proposition like \emph{I should buy this carton in particular}. It
better not, because I really have no reason to accept any such
proposition. Rather, it involves accepting a proposition like \emph{I
will buy this carton in particular}. I can accept that by simply buying
the eggs. There were many other answers I could equally well have
accepted, since there were many other cartons I could buy.\footnote{I'm
  here mildly disagreeing with Jason Stanley
  (\citeproc{ref-Stanley2011}{2011}, Ch. 5) when he says that these
  questions with infinitival complements can be paraphrased using modals
  like `should'. If `will' just is the modal that gets used in the
  paraphrase, as Bhatt (\citeproc{ref-Bhatt1999}{1999}) suggests, the
  spirit of Stanley's view is preserved, even if the letter isn't.}

Practical questions are distinct from questions about modals or
utilities, but there will usually be a correlation between their
answers. Usually, if someone asks you when to visit Venice, and there is
one time in particular such that visiting then maximises expected
utility, that's what you should tell them. That's when they should
visit, and that's what to say when they ask you when to visit.
Relatedly, practical questions can come in conditional form. We can
utter sentences like the following in English.

\begin{itemize}
\tightlist
\item
  J asks K what to do if his patient has hepatitis.
\end{itemize}

And there is one feature of these sentences that needs noting. I don't
know what to do if one's patient has hepatitis, so let's just say that J
tells K to do X. What that means is not that in any situation where the
patient has hepatitis, do X. If the patient's symptoms are confusing, it
might be best to run more tests before doing X. What it does mean is
that if the fact that the patient has hepatitis is taken as given, then
do X. As always, conditional questions should be understood as questions
about what happens in scenarios where the condition in question is taken
as given. And the constraint expressed by Relevant Conditional Questions
is that whatever is known can be taken as given in just this sense.

\section{A Million Dead End Streets}\label{sec-mychanges}

As I've noted already, the view I'm defended here is somewhat different
from my earlier view. And it's helpful to understand the view of this
book to lay out, in one place, the ways in which time has changed my
views. Here is a somewhat simplified version of the view from ``Can We
Do Without Pragmatic Encroachment''. Assume that S is interested in some
quantitative questions and some alethic (i.e., yes/no) questions. Then
the view was that S believes that \emph{p} if and only if these two
conditions are met.

\begin{enumerate}
\def\labelenumi{\arabic{enumi}.}
\tightlist
\item
  For any quantitative question \emph{Q?} that S is interested in, and
  any alethic question \emph{A} that S is interested in, S's answers to
  the question \emph{If A, Q?} and \emph{If A and p, Q?} are the same.
\item
  S's credence in \emph{p} is greater than 0.5.
\end{enumerate}

It was assumed that S is always `interested' in the null question
\emph{Is a tautology true?}, so one special instance of this is that S
answers \emph{Q?} and \emph{If p, Q?} the same way. And it was assumed
that S is an expected utility maximiser, so the practical question of
what to do becomes just the quantitative question \emph{Which of these
options has the highest expected utility?}. There are bells and
whistles, especially in thinking about the level of precision that goes
along with the quantitative questions that S is interested in. (Draw
these too fine, and S doesn't have beliefs, so you have to be a little
careful here.) Even without those complications, I've said enough that
you can see the basic view, and perhaps see its problems.

The biggest change from that view to the one I'm defending here concerns
propositions that are not relevant to any question S is considering. I
used to say in that case belief required credence above 0.5; I now say
that S must be willing, at least sometimes, to take \emph{p} for
granted.

There are other changes too. I no longer presuppose that questions about
what to do just are questions about expected utility. I've stopped
focussing exclusively on answers to (conditional) questions, and moved
to talking about both answers and ways that questions are answered. And
I dropped the requirement that we look at these potentially quite
abstruse questions, such as how to answer \emph{Q?} assuming both
\emph{A} and \emph{p}. The last two changes offset each other; the
reason for including these doubly conditional questions was, in effect,
to look at how S was willing to get to answers about questions with more
practical import.

There are many reasons, most of them due to perceptive critics of my
earlier work, for making these changes. I'll just focus here on the five
that have been most significant.

\subsection{Correctness}\label{sec-mecorrect}

Jacob Ross and Mark Schroeder (\citeproc{ref-RossSchroeder2014}{2014})
note that my earlier theory doesn't have a good story about why false
beliefs are incorrect.\footnote{Fantl \& McGrath
  (\citeproc{ref-FantlMcGrath2009}{2009}) make a similar argument,
  targeted at Lockean theories of belief more than at my theory. I'll
  come back to how this is a problem for Lockean theories in
  Section~\ref{sec-lockecorrect}.} I think that's right. Even if
\emph{p} is false, there is nothing necessarily mistaken about either
having credence in \emph{p} above 0.5, or in having unconditional
preferences match preferences conditional on \emph{p}.

But surely false beliefs are, in a way, incorrect. They may be rational,
they may be well-supported, and so on, but still if you believe that
\emph{p}, and \emph{p} turns out not to be the case, you got it wrong.
There are other mental states that have truth as a correctness
condition. Guesses are correct or incorrect, even if there need be
nothing at all irrational about making a false guess. Indeed, any mortal
who doesn't make false guesses from time to time isn't playing the
guessing game well. Not all mental states are like this. Hoping for
something that doesn't turn out to happen is unfortunate, but not
incorrect. To say that a false belief is incorrect is not to just make
the trivial point that it is false. It is also to say that the belief
failed to meet one important standard of evaluation for beliefs -
correctly representing the world. Credences do not have these
correctness conditions, so the relatively simple reduction I proposed of
belief to credence must be mistaken.

The new theory does not have this problem. Doing dominance reasoning
where all of the situations one considers are non-actual is a mistake.
It's not a mistake because it will inevitably lead to an irrational
decision. Rather, it's a mistake because one draws a conclusion that is
not supported by the premises it is based on. Those premises only say
that one option is better than another conditional on one or other
condition obtaining. That's a bad reason to say the first option is
simply better if there is some extra option that might obtain. And
whatever does obtain, might obtain.

This way of explaining the incorrectness of false belief suggests a
central role for knowledge in norms of beliefs. False beliefs are
mistaken because they lead one to treat the actual situation as one that
could not obtain, yet the actual situation might obtain. One can make
the same mistake by treating a situation that doesn't obtain, but might,
as one that could not obtain. Believing something one doesn't know will
(typically) lead to doing that.

\subsection{Impractical Propositions}\label{sec-meimpractical}

The second clause in my earlier theory was designed to rule out trivial
belief in irrelevant propositions. The first clause on its own has some
absurd consequences. Imagine that I'm relaxing by a stream watching the
ripples without a care in the world. All of the very few questions that
I'm currently interested in have the same answer unconditionally as they
do conditional on the Battle of Agincourt having been fought in 1415. So
according to clause 1, I believe the Battle of Agincourt was in 1415.
That's good, because I do believe that. It's also true that all of the
very few questions that I'm currently interested in have the same answer
unconditionally as they do conditional on the Battle of Agincourt having
been fought in 1416. So if clause 1 was the full theory of belief, then
I would also believe that the Battle of Agincourt was in 1416, which I
do not.

I added clause 2 to the theory to try in order to fix this problem, but
it turned out only to fix a special case. Here's a case it doesn't fix.
Let \emph{p} be the proposition that the next die I roll will land 1, 2,
3 or 4. My credence in that is two-thirds, so it satisfies clause 2. And
conditionalising on it doesn't change the answer to any of the very few
problems that I'm interested while the ripples float down the stream. So
I believe \emph{p}. That's absurd, since I know it is just 2/3 likely.
(This objection is also due in important parts to Ross and Schroeder
(\citeproc{ref-RossSchroeder2014}{2014}), though my presentation differs
from theirs to emphasise just which parts of the objections most worry
me.)

The new theory handles this case easily. There is no context where I
would simply ignore the possibility that this next die roll will land 5
or 6 for the purposes of doing dominance reasoning. So I don't believe
that \emph{p}, as required.

Is there anything we can rule out on purely probabilistic grounds? It's
a little interesting to think this kind of case through. Imagine there
is some salient very large number, and it matters what the remainder is
when that large number is divided by 1000, or 1000000. Could we get to a
point where a choice that is better than some alternative unless that
remainder is, say 537, feel like a dominating choice? I'm not sure
whether that would ever happen. It does seem plausible to say that
whether such a choice ever feels like a dominating choice correlates
with whether we could ever straight up believe that the remainder is not
precisely 537 on purely probabilistic grounds.

\subsection{Choices with More Than Two Options}\label{sec-threeway}

Consider this variant of the Red-Blue game. As well as the four options
Anisa has in the original version of the game, she has a fifth option.
This option says that if she answers some question correctly, she wins
\$100. She's told what the question is, and what the red and blue
sentences are, before she has to choose. And in this case, the question
is, who was the first American woman to win an Olympic gold medal.

Imagine that Anisa just skim reads the red and blue sentences, and
doesn't think about which of them she'd pick, because she knows the
answer to this question. It was, she knows, Margaret Abbott. So she
promptly gives that answer, and wins \$100.

Now she clearly takes an interest in the options Red-True and Blue-True.
She has reasons for preferring to answer the question than take one of
those two options. And she could give those reasons without any
reflection. So Red-True and Blue-True should be in the range of things
that we quantify over when thinking about options she is interested in.
Moreover, she has a stable disposition to choose Red-True over
Blue-True; I think that stable disposition is a strict preference. That
strict preference does not survive conditionalising on the proposition
that the Battle of Agincourt was in 1415. So my earlier theory says that
even in this revised version of the game, Anisa does not believe that
the Battle of Agincourt was in 1415.

This now seems mistaken to me. In any deliberation Anisa does, her
regular disposition to take it for granted that the Battle of Avignon
was in 1415 survives. There is a very nearby deliberation where it does
not survive, namely the deliberation about whether Red-True or Blue-True
is better. But, crucially, she does not have to take an interest in that
question in order to take an interest in the two options Red-True and
Blue-True. If they are both (clearly) suboptimal options in her current
situation, she can simply settle for concluding that they are
suboptimal, and leave it at that.

So I think my old theory made it too easy to lose belief in cases where
one has to choose between many options. Being interested in some
options, because you want to choose the best one of them, does not mean
being interested in all questions about preferences between pairs of
them. The problem was that I'd been focussing largely on two-way
choices, so the distinction between being interested in some choices and
being interested in which of those two is better got elided. That
distinction matters, and the hybrid pragmatic theory handles it better
than my old theory.

\subsection{Hard Times and Close Calls}\label{sec-meties}

In my earlier theory, any practical deliberation was modeled as an
inquiry into which option had the highest expected utility. This was
wrong for a number of reasons, not least that it gives implausible
results in cases involving choices between very similar options. I'll
briefly describe one example that illustrates the problem, and the start
of how I plan to solve it. It turns out to be rather tricky to get the
details right, and I'll come back to this in Section~\ref{sec-andelim}
and again in Chapter~\ref{sec-ties}. The details of the example are new,
but it's a very minor modification of a kind of example that is
discussed in Matthew McGrath and Brian Kim
(\citeproc{ref-McGrathKim2019}{2019}) and credited to a talk by John
Hawthorne ``circa 2007''. Similar examples are also discussed by Alex
Zweber (\citeproc{ref-Zweber2016}{2016}) and by Charity Anderson and
John Hawthorne (\citeproc{ref-AndersonHawthorne2019b}{2019b}), and I'm
drawing on their insights in describing this one.

David is doing the weekly groceries. He needs a can of chickpeas, so he
walks to where the chickpeas are and looks at the shelf. There are two
cans, call them \emph{c}1 and \emph{c}2, that are equally easy to reach
and get from the shelf. Call the actions of taking them \emph{t}1 and
\emph{t}2. David simply assumes, partially on inductive grounds and
partially on grounds of what he knows about supermarkets, that neither
can has passed its expiry date. While it is wildly implausible that
either can has, the probability is not zero. Let
\emph{e\textsubscript{i}} be that can \emph{i} has expired, and assume
that \emph{Pr}(\emph{e}\textsubscript{1}) and
\emph{Pr}(\emph{e}\textsubscript{2}) are low and equal. Call this
probability \emph{e}. Let \emph{h} be the utility of choosing an
unexpired can, and \emph{l} the utility of choosing an expired can,
where obviously \emph{h} \textgreater{} \emph{l}. Then both \emph{t}1
and \emph{t}2 have utility (1-\emph{e})\emph{h}~+~\emph{el}. Conditional
on ¬\emph{e}1, the utility of \emph{t}1 is \emph{h}, which is greater
than (1-\emph{e})\emph{h}~+~\emph{el} as long as \emph{e}~\textgreater~0
and \emph{h}~\textgreater~\emph{l}. So unconditionally, \emph{t}1 and
\emph{t}2 have the same utility, but conditional on ¬\emph{e}1, they
have different utilities. So, according to the theory I used to defend,
when David is making this choice, he does not believe, and hence does
not know ¬\emph{e}1. This seems wrong, and there are even worse
consequences one can draw my thinking about minor variants of the case.

The key part of my response to this will be distinguishing between the
questions \emph{Which can to choose?}, and the question \emph{Which
choice of can has maximal expected utility?}. If David is thinking about
the latter question, then it turns out he really doesn't know
¬\emph{e}1. That's a somewhat surprising result, and I'll turn to
defending it in Chapter~\ref{sec-ties}. But as long as he is focussing
solely on the former question, the argument of the previous paragraph
doesn't go through.

So the big move here is to move from somewhat quantitative questions,
like \emph{Which choice maximises expected utility?}, to practical
questions like \emph{What to do?}. Once we do that, the problem that
Zweber, and Anderson and Hawthorne, raise ceases to be a problem. I
don't intend these brief remarks to be a convincing case that I've got a
good solution to these problems. Rather, the point is to flag that the
theory I'm defending here is distinct from the theory I used to defend,
and this gives me some more resources to handle cases like David and the
chickpeas.

\subsection{Updates and Modals}\label{sec-modalupdate}

The version of IRT that I defend here gives a big role to conditional
attitudes.\footnote{This subsection is based on my
  (\citeproc{ref-Weatherson2016}{2016a: 1}).} That's something that it
has in common with everything I've written about IRT. I used to have a
particular pair of views about how to understand conditional attitudes.
In particular, I took the following two claims to be at least close
approximations to the truth about conditional attitudes.

\begin{enumerate}
\def\labelenumi{\arabic{enumi}.}
\tightlist
\item
  An attitude conditional on \emph{p} is (usually) the same as the
  attitude one would have after updating on \emph{p}.
\item
  The way to update on \emph{p} is to conditionalise.
\end{enumerate}

The first is at best an approximation for familiar reasons. I can think
that no one knows whether \emph{p} is true, and even think that this is
true conditional on \emph{p}. But after updating on \emph{p}, I will no
longer think that. So we have to be a bit careful in applying principle
1; it has counterexamples. Still, it is a useful enough heuristic to
work with.

What wasn't originally obvious to me was that there are counterexamples
to principle 2 as well. They are more significant for the way IRT should
be understood. I used to describe the picture of belief I was defending
as the view that to believe something is to have a credence in it that's
close enough to 1 for current purposes. That's still a decent heuristic,
but it isn't always right. When someone is interested in modal
questions, credence 1 might be insufficient for belief. To see how this
might be so, it helps to start with some points Thony Gillies
(\citeproc{ref-Gillies2010}{2010}) makes about the relationship between
modals, conditionals and updating.

When modal questions are on the table, updating will not be the same as
conditionalising. This is shown by the following example. (A similar
example is in Kratzer (\citeproc{ref-Kratzer2012}{2012: 94}).)

\begin{quote}
I have lost my marbles. I know that just one of them -- Red or Yellow --
is in the box. But I don't know which. I find myself saying things like
\ldots{}``If Yellow isn't in the box, the Red must be.'' (4:13)
\end{quote}

What matters for the purposes of this book is not whether this
conditional is true, but whether its truth is consistent with the Ramsey
test view of conditionals. And Gillies argues that it is.

\begin{quote}
The Ramsey test -- the schoolyard version, anyway -- is a test for when
an indicative conditional is acceptable given your beliefs. It says that
(if \emph{p})(\emph{q}) is acceptable in belief state \emph{B} iff
\emph{q} is acceptable in the derived or subordinate state
\emph{B}-plus-the-information-that-\emph{p}. (4:27)
\end{quote}

And he notes that this can explain what goes on with the marbles
conditional. Add the information that Yellow isn't in the box, and it
isn't just true, but must be true, that Red is in the box.

Note though that while we can explain this conditional using the Ramsey
test, we can't explain it using any version of the idea that
probabilities of conditionals are conditional probabilities. The
probability that Red must be in the box is 0. The probability that
Yellow isn't in the box is not 0. So conditional on Yellow not being in
the box, the probability that Red must be in the box is still 0. Yet the
conditional is perfectly assertable.

There is, and this is Gillies's key point, something about the behaviour
of modals in the consequents of conditionals that we can't capture using
conditional probabilities, or indeed many other standard tools. And what
goes for consequents of conditionals goes for updated beliefs too. Learn
that Yellow isn't in the box, and you'll conclude that Red must be. But
that learning can't go via conditionalisation; just conditionalise on
the new information and the probability that Red must be in the box goes
from 0 to 0.

Now it's a hard problem to say exactly how this alternative to updating
by conditionalisation should work. Very roughly, the idea is that at
least some of the time, we update by eliminating worlds from the space
of possibilities. This affects dramatically the probability of
propositions whose truth is sensitive to which worlds are in the space
of possibilities.

All this matters when we are considering modal questions. For example,
if we are considering the question \emph{Must q be true?}, then it is
plausible that unconditionally the answer is no, and indeed the
unconditional probability that \emph{q} must be true is 0, but that
conditional on \emph{p}, \emph{q} must be true.

We don't even have to be considering modals directly for this to happen.
Assume that actions \emph{A} and \emph{B} have the same outcome
conditional on \emph{q}, but \emph{A} is better than \emph{B} in every
¬\emph{q} possibility. Then if we are considering the question \emph{Is
A better than B?}, it will matter whether it must be the case that
\emph{q}.

Assume that \emph{q} could have probability 1 without it being the case
that \emph{q} must be true. (This is controversial, but I'll offer
arguments in sections \ref{sec-lockecoin} and \ref{sec-lockegames} that
it is possible.) Then unconditionally, \emph{A} is better than \emph{B},
even though they have the same expected utility. That's because weak
dominance is a good principle of practical reasoning: If \emph{A} might
be better than \emph{B} and must not be worse, then \emph{A} is better
than \emph{B}. But by hypothesis, conditional on \emph{p}, \emph{A} is
not better than \emph{B}. So in this case \emph{p} will not be believed;
conditional on \emph{p} the question \emph{Is A better than B} gets a
different answer to what it gets unconditionally.

Note though that all I said to get this example going is that \emph{p}
rules out ¬\emph{q}, and \emph{q} has probability 1. That means \emph{p}
could have any probability at all, up to probability 1. So it's possible
that conditional on \emph{p}, some relevant questions get different
answers to what they get unconditionally, even though \emph{p} has
probability 1. So belief can't be a matter of having probability close
enough to 1 for practical purposes; sometimes even probability 1 is
insufficient.

\section{Ross and Schroeder's Theory}\label{sec-usc}

Jacob Ross and Mark Schroeder (\citeproc{ref-RossSchroeder2014}{2014})
have what looks like, on the surface, a rather different view to
mine.\footnote{This section is based on my
  (\citeproc{ref-Weatherson2016}{2016a: 3}).} They say that to believe
\emph{p} is to have a \emph{default reasoning disposition} to use
\emph{p} in reasoning. Here's how they describe their view.

\begin{quote}
What we should expect, therefore, is that for some propositions we would
have a \emph{defeasible} or \emph{default} disposition to treat them as
true in our reasoning--a disposition that can be overridden under
circumstances where the cost of mistakenly acting as if these
propositions are true is particularly salient. And this expectation is
confirmed by our experience. We do indeed seem to treat some uncertain
propositions as true in our reasoning; we do indeed seem to treat them
as true automatically, without first weighing the costs and benefits of
so treating them; and yet in contexts such as High where the costs of
mistakenly treating them as true is salient, our natural tendency to
treat these propositions as true often seems to be overridden, and
instead we treat them as merely probable.

But if we concede that we have such defeasible dispositions to treat
particular propositions as true in our reasoning, then a hypothesis
naturally arises, namely, that beliefs consist in or involve such
dispositions. More precisely, at least part of the functional role of
belief is that believing that \emph{p} defeasibly disposes the believer
to treat \emph{p} as true in her reasoning. Let us call this hypothesis
the \emph{reasoning disposition account} of belief.
(\citeproc{ref-RossSchroeder2014}{Ross \& Schroeder, 2014: 9--10})
\end{quote}

There are, relative to what I'm interested in, three striking
characteristics of Ross and Schroeder's view.

\begin{enumerate}
\def\labelenumi{\arabic{enumi}.}
\tightlist
\item
  Whether you believe \emph{p} is sensitive to how you reason; that is,
  your theoretical interests matter.
\item
  How you would reason about some questions that are not live is
  relevant to whether you believe \emph{p}.
\item
  Dispositions can be masked, so you can believe \emph{p} even though
  you don't actually use \emph{p} in reasoning now.
\end{enumerate}

The view I'm defending here agrees with them about 1 and 2, though my
theory manifests those characteristics in a quite different way. But
point 3 is a cost of their theory, not a benefit, so it's good that my
theory doesn't accommodate it. (For the record, the theory I put forward
in my (\citeproc{ref-Weatherson2005-WEACWD}{2005a}) did not agree with
them on point 2, and I changed my view because of their arguments.)

I agree with 1 because, as I've noted a few times above, I think
theoretical interests as well as pragmatic interests matter for the
relationship between credence and belief. And I agree with 2 because I
think that whether someone is disposed to use \emph{p} as a premise
matters to whether they believe \emph{p}. Let \emph{p} be some ordinary
proposition about the world that a person believes, such as that the
Florida Marlins won the 2003 World Series. And let \emph{q} be a lottery
proposition that is just as probable as \emph{p}. (That is, let \emph{q}
be a lottery proposition such that if the person were to play the
Red-Blue game with \emph{p} as red and \emph{q} as blue, they would be
rationally indifferent between the choices.) Then on my theory the
person believes \emph{p} but not \emph{q}, and this isn't due to any
features of their credal states. Rather, it is due to their dispositions
to use \emph{p} as a premise in reasoning. (For example, they might use
it in figuring out how many World Series were won by National League
teams in the 2000s.)

Ross and Schroeder argue, and I basically agree, that interest-relative
theories of belief that only focus on practical interests have trouble
with folks who use odd techniques in reasoning. This is the lesson of
their example of \emph{Renzi}. I'll run through a somewhat more abstract
version of that case, because the details are not particularly
important. Start with a standard decision problem. The agent knows that
X is better to do if \emph{p}, and Y is better to do if ¬\emph{p}. The
agent should then go through calculating the relative gains to doing X
or Y in the situations they are better, and the probability of \emph{p}.
But the agent imagined doesn't do that. Rather, the agent divides the
possibility space in four, taking the salient possibilities to be
\emph{p} ∧ \emph{q}, \emph{p} ∧ ¬\emph{q}, ¬\emph{p} ∧ \emph{q} and
¬\emph{p} ∧ ¬\emph{q} and then calculates the expected utility of X and
Y accordingly. This is a bad bit of reasoning on the agent's part. In
the cases we are interested in,~\emph{q} is exceedingly likely.
Moreover, the expected utility of each act doesn't change a lot
depending on \emph{q}'s truth value. So it is fairly obvious that we'll
end up making the same decision whether we take the `small worlds' in
our decision model to be just the world where \emph{p}, and the world
where ¬\emph{p}, or the four worlds this agent uses. But the agent does
use these four, and the question is what to say about them.

Ross and Schroeder say that such an agent should not be counted as
believing that \emph{q}. If they are consciously calculating the
probability that \emph{q}, and taking ¬\emph{q} possibilities into
account when calculating expected utilities, they regard \emph{q} as an
open question. And regarding \emph{q} as open in this way is
incompatible with believing it.

I agree. The agent was trying to work out the expected utility of X and
Y by working out the utility of each action in each of four `small
worlds', then working out the probability of each of these. Conditional
on \emph{q}, the probability of two of them (\emph{p} ∧ ¬\emph{q},
¬\emph{p} ∧ ¬\emph{q}), will be 0. Unconditionally, this probability
won't be 0. So the agent has a different view on some question they have
taken an interest in unconditionally to their view conditional on
\emph{q}. So they don't believe \emph{q}. The agent shouldn't care about
that question, and conditional on each question they should care about,
they have the same attitude unconditionally and conditional on \emph{q}.
But they do care about these probabilistic questions, so they don't
believe \emph{q}. (And again for the record, the theory I defended at
the time Ross and Schroeder wrote their paper did not have the resources
to make this reply; I've changed my views in light of their arguments.)

So far I've been agreeing with Ross and Schroeder. But there is one big
point of disagreement. They think it is very important that a theory of
belief vindicate a principle they call Stability.

\begin{description}
\tightlist
\item[Stability]
A fully rational agent does not change her beliefs purely in virtue of
an evidentially irrelevant change in her credences or preferences.
(\citeproc{ref-RossSchroeder2014}{2014: 20})
\end{description}

Here's the kind of case that is meant to motivate Stability, and show
that views like mine are in tension with it.

\begin{quote}
Suppose Stella is extremely confident that steel is stronger than
Styrofoam, but she's not so confident that she'd bet her life on this
proposition for the prospect of winning a penny. PCR {[}their name for
my old view{]} implies, implausibly, that if Stella were offered such a
bet, she'd cease to believe that steel is stronger than Styrofoam, since
her credence would cease to rationalize acting as if this proposition is
true. (\citeproc{ref-RossSchroeder2014}{2014: 20})
\end{quote}

Ross and Schroeder's own view is that if Stella has a defeasible
disposition to treat as true the proposition that steel is stronger than
Styrofoam, that's enough for her to believe it. And that can be true if
the disposition is not only defeasible, but actually defeated in the
circumstances Stella is in. This all strikes me as just as implausible
as the failure of Stability. Let's go over its costs.

The following propositions are clearly not mutually consistent, so one
of them must be given up. We're assuming that Stella is facing, and
knows she is facing, a bet that pays a penny if steel is stronger than
Styrofoam, and costs her life if steel is not stronger than Styrofoam.

\begin{enumerate}
\def\labelenumi{\arabic{enumi}.}
\tightlist
\item
  Stella believes that steel is stronger than Styrofoam.
\item
  Stella believes that if steel is stronger than Styrofoam, she'll win a
  penny and lose nothing by taking the bet.
\item
  If 1 and 2 are true, and Stella considers the question of whether
  she'll win a penny and lose nothing by taking the bet, she'll believe
  that she'll win a penny and lose nothing by taking the bet.
\item
  Stella prefers winning a penny and losing nothing to getting nothing.
\item
  If Stella believes that she'll win a penny and lose nothing by taking
  the bet, and prefers winning a penny and losing nothing to getting
  nothing, she'll take the bet.
\item
  Stella won't take the bet.
\end{enumerate}

It's part of the setup of the problem that 2 and 4 are true. And it's
common ground that 6 is true, at least assuming that Stella is rational.
So we're left with 1, 3 and 5 as the possible candidates for falsehood.

Ross and Schroeder say that it's implausible to reject 1. After all,
Stella believed it a few minutes ago, and hasn't received any evidence
to the contrary. And I guess rejecting 1 isn't the most intuitive
philosophical conclusion I've ever drawn. But compare the alternatives!

If we reject 3, we must say that Stella will simply refuse to infer
\emph{r} from \emph{p}, \emph{q} and (\emph{p} ∧ \emph{q}) → \emph{r}.
Now it is notoriously hard to come up with a general principle for
closure of beliefs. But it is hard to see why this particular instance
would fail. And in any case, it's hard to see why Stella wouldn't have a
general, defeasible, disposition to conclude \emph{r} in this case, so
by Ross and Schroeder's own lights, it seems 3 should be acceptable.

That leaves 5. It seems on Ross and Schroeder's view, Stella simply must
violate a very basic principle of means-end reasoning. She desires
something, she believes that taking the bet will get that thing, and
come with no added costs. Yet, she refuses to take the bet. And she's
rational to do so! At this stage, I think I've lost what's meant to be
belief-like about their notion of belief. I certainly think attributing
this kind of practical incoherence to Stella is much less plausible than
attributing a failure of Stability to her.

Put another way, I don't think presenting Stability on its own as a
desideratum of a theory is exactly playing fair. The salient question
isn't whether we should accept or reject Stability. The salient question
is whether giving up Stability is a fair price to pay for saving basic
tenets of means-end rationality. And I think that it is. Perhaps there
will be some way of understanding cases like Stella's so that we don't
have to choose between theories of belief that violate Stability
constraints, and theories of belief that violate coherence constraints.
But I don't see one on offer, and I'm not sure what such a theory could
look like.

I have one more argument against Stability, but it does rest on somewhat
contentious premises. There's often a difference between the best
methodology in an area, and the correct epistemology of that area. When
that happens, it's possible that there is a good methodological rule
saying that if such-and-such happens, re-open a certain inquiry. But
that rule need not be epistemologically significant. That is, it need
not be the case that the happening of such-and-such provides evidence
against the conclusion of the inquiry. It just provides a reason that a
good researcher will re-open the inquiry. And, as I've argued above, an
open inquiry is incompatible with belief.

Here's one way that might happen. Like other non-conciliationists about
disagreement, such as Thomas Kelly
(\citeproc{ref-Kelly2010-KELPDA}{2010}), I hold that disagreement by
peers with the same evidence as you doesn't provide \emph{evidence} that
you are wrong. But it might provide an excellent reason to re-open an
inquiry. We shouldn't draw conclusions about the methodological
significance of disagreement from the epistemology of disagreement. So
learning that your peers all disagree with a conclusion might be a
reason to re-open inquiry into that conclusion, and hence lose belief in
the conclusion, without providing evidence that the conclusion is false.
This example rests on a very contentious claim about the epistemology of
disagreement. But any gap that opens up between methodology and
epistemology will allow such an example to be constructed, and hence
provide an independent reason to reject Stability.

\bookmarksetup{startatroot}

\chapter{Knowledge}\label{sec-knowledge}

In Chapter~\ref{sec-belief}, I argued that to believe something is to
take it as given in all relevant inquiries, and in at least one possible
inquiry. I explained what it was to take something as given in terms of
how one answers conditional and unconditional questions. In this chapter
I'm going to argue that whatever is known can be properly taken as given
in all relevant inquiries, where a relevant inquiry is one that one
either is or should be conducting. Since some things that are usually
known cannot be properly taken as given in some inquiries, this implies
that knowledge is sensitive to one's inquiries and hence to one's
interests.

There is an easy argument for the conclusion of this chapter.

\begin{enumerate}
\def\labelenumi{\arabic{enumi}.}
\tightlist
\item
  To believe something is to, inter alia, take it as given for all
  relevant inquiries.
\item
  Whatever is known is correctly believed.
\item
  So, whatever is known is correctly taken as given in all relevant
  inquiries.
\end{enumerate}

I think this argument is basically sound, but both premises are
controversial. Further, it isn't completely obvious that it is even
valid. So I'm not going to rely on this argument. Rather, I'll argue
more directly for the conclusion that whatever is known is correctly
taken as given in all relevant inquiries. This will provide indirect
evidence that the theory of belief in Chapter~\ref{sec-belief} was
correct, since we can now take that theory of belief to be an
explanation for the claim that whatever is known is correctly taken as
given in all relevant inquiries, rather than as part of the motivation
for it.

The argument here will be in two parts. First, I'll focus on practical
inquiries, i.e., inquiries about what to do, and argue that what is
known can be taken as given in all practical inquiries. Then I'll extend
the discussion to theoretical inquiries, and hence to inquiries in
general. Finally, with the argument complete, I'll look at two possible
objections to the argument. One objection is that it has implausible
consequences about the role of logical reasoning in extending knowledge,
and the other is that it leads to implausible results when a source
provides both relevant and irrelevant information.

\section{Ten Decision Commandments}\label{sec-makdec}

A practical inquiry can often be represented by the kind of decision
table that we use in decision theory courses.\footnote{This section and
  the next are loosely based on my (\citeproc{ref-Weatherson2012}{2012:
  1.1}).} Table~\ref{tbl-walkbus}, for instance, is a table for the
problem faced by a person, call him Ragnar, choosing how to get to work.

\begin{longtable}[]{@{}rcc@{}}
\caption{Ragnar's trip to work}\label{tbl-walkbus}\tabularnewline
\toprule\noalign{}
\endfirsthead
\endhead
\bottomrule\noalign{}
\endlastfoot
~ & Rain & Dry \\
Walk & 0 & 5 \\
Bus & 3 & 4 \\
\end{longtable}

If we tell the students that the probability of rain is 0.4, we expect
them to figure out that the expected utility of walking is 3, and the
expected utility of taking the bus is 3.6, so it is better to take the
bus. And that's a little surprising, since it probably won't rain, and
if it doesn't, it is better to walk. The key point is that walking is
risky, and in this case expected utility theory say that it isn't a risk
worth taking.

Table~\ref{tbl-walkbus} can serve two related philosophical purposes,
which we can helpfully distinguish using terminology from Peter Railton
(\citeproc{ref-Railton1984}{1984}). The table can provide a
\emph{criterion of rightness} for Ragnar's actions. It is rational for
him to take the bus because of the expected utility calculation. The
table can do more than that though. In simple cases like this one, it
can provide a \emph{deliberation procedure}. Ragnar can, in theory and
in simple cases, use a table like this to decide what to do. There are
limits to when tables can be used in this way, and as I'll argue in
Chapter~\ref{sec-ties}, those limits end up suggesting limits to how
often the tables even provide criteria of rightness. In simple cases
though, the table isn't just something the theorist can use to
understand Ragnar, it is something Ragnar himself can use to deliberate.
This is especially true in cases where one of the options is dominated,
either strictly or weakly, by another.\footnote{An option is strictly
  dominated by another if it does worse than that option in every state.
  It is weakly dominated by another if it does worse than that option in
  some states, and never does better than it.} I've appealed to the fact
that the tables can be deliberation procedures, and not just criteria of
rightness, already, in the discussion of Sully and Mack in
Section~\ref{sec-given}. There the focus was on how tables like these
related to belief; here I want to relate them to knowledge.

There are (at least) ten ways in which Table~\ref{tbl-walkbus} could
misrepresent Ragnar's situation. To put the same point another way,
there are (at least) ten ways in which it could correctly represent his
situation. One way to think about the core project of this book is to
say what it means for a table to correctly represent a decision
situation in one of these ten respects. It is a little easier to think
about the misrepresentations, so I'll start with them.

First, the numbers in the table might be wrong. The table says that,
conditional on catching the bus, Ragnar is better off it is dry than if
it rains. Maybe that isn't true. The theory of well-being
(\citeproc{ref-sep-well-being}{Crisp, 2021}) is about, among other
things, when the numbers in the cells of tables like this are correct.
That's a big topic, and not one I'm going to have anything to say about
here.\footnote{As well as questions about well-being, there are also
  questions here about what one should do in cases where the outcome is
  itself a kind of gamble. Imagine that chooser is trying to decide
  whether to bet on a basketball game, and it is known how much money
  they will win or lose in the four states. The value to the chooser of
  those outcomes depends on any number of further things, like the rate
  of inflation in the near term, and the ``position of wealth holders in
  the social system'' (\citeproc{ref-Keynes1937}{Keynes, 1937: 214})
  some years hence. Just how these uncertainties should be accounted for
  is a difficult question, especially for any theorist who deviates in
  any way from orthodox expected utility theory. I would like to have a
  better theory of how the account of decision making with deliberation
  costs that I discuss in Chapter~\ref{sec-ties} interacts with this
  question.}

Second, the probabilities might be wrong. Maybe it isn't the case that
the probability of rain is 0.4, and in fact it is 0.2. There is an
enormous question here about what it even means for one to misrepresent
the probabilities. Is the correct representation one that tracks
objective chances, or Ragnar's evidence, or Ragnar's beliefs, or
something else, or some combination of these? One upside of focussing on
dominance arguments is that these questions can be temporarily set
aside.

The next four questions concern the rows, and here we have less
philosophical work to draw on. Brian Hedden
(\citeproc{ref-Hedden2012}{2012}) has a paper arguing that the options
should all be decisions, rather than actions. So the first row should
say ``Ragnar decides to walk'' rather than ``Ragnar walks''. This would
be a fairly radical change from practice in decision theory, though one
worth taking seriously. The more conservative option would be to link
the rows to some or other philosophical theory of abilities
(\citeproc{ref-sep-abilities}{Maier, 2022}). In some sense it seems
right to say that there should be a row for all and only the actions
that Ragnar is able to perform. The details are going to be tricky
though. This book is focussed on the columns rather than the rows, but I
want to briefly mention four important topics about the rows, which will
constitute our third through sixth ways the table might misrepresent
Ragnar's situation.

Third, the table might leave off an option that should be there. Perhaps
Ragnar should, or at least should consider, driving to work. Or perhaps
it should include the option of quitting his job immediately, and hence
not going to work.

Fourth, the table might include an option that should not be there. If
the bus route near Ragnar's house has just been cancelled, perhaps the
table should not include a row for the bus.

Fifth, the table might have merged multiple options that should be
separated. Perhaps it should have separate rows for walking with an
umbrella, and walking without an umbrella. This differs from the third
point, because it does not say that Ragnar should do (or consider)
something wholly distinct from what is already there, but rather that it
should separate out different ways of bringing about something that is
considered.

Sixth, the table might have separated multiple options that should be
merged. It's hard to see how Table~\ref{tbl-walkbus} could have made
this mistake, but if we had separate rows for walking while wearing a
red shirt, and walking while wearing a blue shirt, it would be arguable
that this is too fine a grain, and the right table would not distinguish
these.

The final four questions concern the columns, and they mirror the four
questions about the rows. These questions will be central to the
narrative of this chapter, and of this whole book.

Seventh, the table might leave off a state that should be there. Perhaps
Ragnar should consider the possibility that it will snow, or that there
will be an ice storm. Taking the only two states to be rain and dry
excludes those possibilities\footnote{As noted back in
  Section~\ref{sec-given}, I'm using `possibilities' here in the sense
  described by I. L. Humberstone (\citeproc{ref-Humberstone1981}{1981}).},
and perhaps they should be included.

Eighth, the table might include a state that should not be there. If it
is bucketing down as Ragnar is preparing to leave, including a state
where it is dry might be a mistake.

Ninth, the table might have merged states that should be separated.
Perhaps the column that simply says \emph{Dry} should have been split
into two: one being \emph{Dry and Sunny}, the other being \emph{Dry and
Cloudy}.

Tenth, the table might have split states that should be merged. It's
unlikely that a two state table will do this, but if we had made the
split suggested in the previous paragraph, one could easily argue that
it was a mistake, and that Ragnar should have treated these as a single
state.

That gives us ten ways that the table could go wrong. It's helpful to
have them in a simple list.

\begin{enumerate}
\def\labelenumi{\arabic{enumi}.}
\tightlist
\item
  The values could be wrong.
\item
  The probabilities could be wrong.
\item
  An option could be improperly excluded.
\item
  An option could be improperly included.
\item
  The options might be too coarse-grained.
\item
  The options might be too fine-grained.
\item
  A state could be improperly excluded.
\item
  A state could be improperly included.
\item
  The states might be too coarse-grained.
\item
  The states might be too fine-grained.
\end{enumerate}

For every one of these ten possible mistakes, there is a prior
philosophical question about what it means for the table to have made,
or not made, that mistake. Every one of those ten questions is, at least
to my mind, incredibly philosophically important. Even someone who
thought, like Foxwell, that books should only be written for ``grave
cause'' (\citeproc{ref-Keynes1936Foxwell}{Keynes, 1936: 599}), should
concede that a clear answer to any one of the ten would be sufficient
grounds to warrant a scholarly monograph.

This book is primarily concerned with the seventh, though the argument
touches to some extent on the eighth as well. It is proper to exclude a
possibility from the table if the chooser knows that possibility does
not obtain. If that conditional could be turned into a biconditional,
we'd have an answer to the eighth question, too, but that is a more
delicate question.\footnote{Back in Section~\ref{sec-neutrality} I said
  I was staying neutral on that question, and I'm not changing that
  position here.} In any case, the conditional will be enough.

\section{Knowing Where the Ice Cream Goes}\label{sec-icecream}

The aim of this section is to argue for the following principle.

\begin{description}
\tightlist
\item[Knowledge Allows Exclusion (KAE)]
If a chooser knows that a possibility does not obtain, then it is
permissible to use a decision table where that possibility is excluded,
i.e., is incompatible with the possibilities in each of the columns.
\end{description}

Knowledge Allows Exclusion is Jessica Brown's principle K Suff applied
to practical decision making using tables. That's a fairly central case
for K Suff, so if KAE is true, then it seems plausible that K Suff will
be true too. I'll come back to the more general case for K Suff in later
sections, though; here the focus is KAE. I'm going to build up to KAE in
stages; first I'm going to talk about ice cream.

The contemporary theory of duopoly starts with Harold Hotelling's paper
``Stability in Competition'' (\citeproc{ref-Hotelling1929}{1929}).
Hotelling describes how a duopoly that does not maximise consumer
welfare can be stable if the two parties have the ability to
differentiate their product along one dimension. Surprisingly, the
equilibrium is that they do not in fact take advantage of this ability,
and instead provide the very same product. Hotelling's observation is
that if both parties could differentiate, neither party has the
incentive they would normally have to reduce prices to the point where
consumer surplus is maximised. Hotelling is interested in possible
equilibria, and he doesn't focus on how the parties might calculate the
equilibria. (The impression one gets from the paper is that it will
involve a good chunk of trial-and-error.) Subsequent work revealed that
it turns out that in some duopoly situations, not much is needed to get
to the equilibrium; just iterated deletion of dominated strategies.

Here is the standard way Hotelling's model is introduced in
textbooks.\footnote{This particular example isn't in Hotelling, but it
  is in so many textbooks that I haven't been able to find out where it
  was first introduced. It differs from his examples in that the parties
  do not have the capacity to compete on price.} Imagine that two ice
cream trucks have to choose (simultaneously) where they will be located
on a beach. The beach has seven locations, numbered 1 to 5. The distance
between location \emph{m} and location \emph{n} is \textbar{}\emph{m} -
\emph{n}\textbar. Assume for simplicity that the price of ice cream is
fixed, the trucks just compete on location. There are two beach-goers at
each of locations 1 to 5, so 10 in total. Each beach-goer will buy an
ice cream from the nearest truck. If two trucks are equidistant from a
location, the two people there will head off in either direction, one
buying from each truck. Question: Where should the two trucks go,
assuming that it is common knowledge that each truck owner is rational,
and simply wants to maximise their own sales?

This puzzle can be solved using just the idea that strictly dominated
strategies can be iteratively deleted. Table~\ref{tbl-hotelling} shows w
many sales each truck will make for each choice of location. The choice
of the first truck determines which row of the table we're in, the
choice of the second truck determines which column of the table we're
in, and the resulting cell lists first the sales of the first truck,
then the sales of the second truck.

\clearpage

\begin{longtable}[]{@{}rccccc@{}}
\caption{Payouts in the Hotelling
game}\label{tbl-hotelling}\tabularnewline
\toprule\noalign{}
\endfirsthead
\endhead
\bottomrule\noalign{}
\endlastfoot
& \textbf{1} & \textbf{2} & \textbf{3} & \textbf{4} & \textbf{5} \\
\textbf{1} & 5,5 & 2,8 & 3,7 & 4,6 & 5,5 \\
\textbf{2} & 8,2 & 5,5 & 4,6 & 5,5 & 6,4 \\
\textbf{3} & 7,3 & 6,4 & 5,5 & 6,4 & 7,3 \\
\textbf{4} & 6,4 & 5,5 & 4,6 & 5,5 & 8,2 \\
\textbf{5} & 5,5 & 4,6 & 3,7 & 2,8 & 5,5 \\
\end{longtable}

Assume that it is common knowledge, in the sense of Lewis
(\citeproc{ref-Lewis1969a}{1969}), that Table~\ref{tbl-hotelling} is the
payout table, and that each player will not make choices that are
strictly dominated. That is, for each \emph{n}, the proposition we get
by having \emph{n} iterations of \emph{each player knows} in front of
\emph{this is the game table, and each player is rational}, is true.
Then the theorist, and each player, can reason as follows.

Row's option 1 is strictly dominated by option 2; option 2 gets 1 more
sale in three possible states, and 3 more sales in the other two, so it
should be excluded. The same goes for option 5, which is strictly
dominated by option 4. Since the game is symmetric, the same goes for
Column's options 1 and 5. By the common knowledge assumption, this means
we can delete those rows, and columns, from the table. The result is
Table~\ref{tbl-hotelling-iterated}.

\begin{longtable}[]{@{}lccc@{}}
\caption{The Hotelling game after one
iteration}\label{tbl-hotelling-iterated}\tabularnewline
\toprule\noalign{}
\endfirsthead
\endhead
\bottomrule\noalign{}
\endlastfoot
& \textbf{2} & \textbf{3} & \textbf{4} \\
\textbf{2} & 5,5 & 4,6 & 5,5 \\
\textbf{3} & 6,4 & 5,5 & 6,4 \\
\textbf{4} & 5,5 & 4,6 & 5,5 \\
\end{longtable}

For both players, option 3 dominates the other two options, so it will
be chosen. Moreover, the reasoning here generalises. If there are 7
options to start with, we need to do two rounds of deleting dominated
options to get the players to the middle of the beach. If there are 9
options to start with, we need to do three rounds of deletion. In
general, if there are 2\emph{k}+1 options, we get the players to the
middle of the beach after \emph{k} rounds of deletion. Since common
knowledge licences all these iterations, the players will always end up
in the middle of the beach if there are an odd number of options.

At this point you might be worried for two reasons. Practically, this
seems like it proves too much. Contra the conclusion of Hotelling's
paper, it's not true that shoes, churches, and cider mills are as
homogenous as this argument would suggest. Theoretically, there are
plenty of reasons to be worried about common knowledge as Lewis
understood it. Harvey Lederman (\citeproc{ref-Lederman2018}{2018}) shows
that assuming common knowledge, in Lewis's sense, of dominance avoidance
leads to paradoxes. Let's see whether we can get by with less.

Assume that it is not common knowledge, but merely mutual knowledge that
the payout table is as in Table~\ref{tbl-hotelling}, and the players do
not take dominated options. That is, each player knows both those
things. That is all we'll assume. Since knowledge is factive, we can
still rule out the extreme options, i.e., 1 and 5. Given that each
player knows the other will not take dominated options, each player
knows that it is only options 2 through 4 that are relevant. So given
just the mutual knowledge assumption, we can show that from each
player's perspective, they are playing the game depicted in
Table~\ref{tbl-hotelling-iterated}. In that game, option 3 is strictly
dominant. So this assumption is enough to get us back to the middle of
the beach. Note, however, that this reasoning does not generalise. Given
merely mutual knowledge of non-dominance, we can show that neither
player will take options 1 or 2, or the second-last or last options, but
we can't show any more than that. So in the 7 option game, we can only
show that they will both end up somewhere between options 3 and 5. In
the games with much larger numbers of options, we can't show much at
all. That seems both empirically and theoretically more plausible.

The argument of the last paragraph is meant to serve two distinct, but
related, philosophical purposes.\footnote{I'm indebted here to
  conversations with Eric Swanson.} First, it is meant to show that we
theorists can deduce what the players will in fact do, given their
evidence, and the assumptions about rationality. Second, it is meant to
show that it would be rational for the players themselves to get to that
conclusion via just that reasoning. It is important, in general, to
distinguish between what is entailed by some assumptions, and what can
be reasonably inferred from those assumptions Harman
(\citeproc{ref-Harman1986}{1986}). In this case, though, I want to claim
that the reasoning I've set out in that paragraph plays both roles. As
theorists, we can tell that the players will not player either the
extreme, or the next to extreme, option, and no more. The players
themselves will not go to any of those 4 spots, given our assumptions,
but we can't know more about their actions without more knowledge of
their mental states.

But wait a minute! Without KAE, the last two paragraphs consist of one
fallacious step after another. The player knows that the other player
will not play an extreme option. Also, they know that if the extreme
options are excluded, option 2 is strictly dominated. Without KAE, it
doesn't follow that they can simply delete the extreme options. To
delete an option just is to exclude it from the table. Without KAE, the
fact that the player knows an option doesn't obtain isn't a sufficient
reason to make this deletion. Since it is, in practice, a sufficient
reason, it follows that KAE is true. Or, at least, that a restricted
version of KAE applied to this case is true. Since the case seems
arbitrary, it follows that KAE is true in general.

That's my primary argument for KAE. In general, it is reasonable to do
as many rounds of deletion of dominated strategies as we have iterations
of mutual knowledge of rationality and the structure of the game table.
That is, it is reasonable for the theorist to do exactly as many rounds
of deletion as there are iterations of mutual knowledge of rationality
among the players. Without KAE, that match up isn't guaranteed, so KAE
must be true.

\section{Other Answers}\label{sec-other-answers}

If KAE is false, what should go in its place? What could be the state
which does allow exclusion?

\subsection{None of the Above}\label{sec-none-of-the-above}

One might object to the presupposition of that question. Maybe exclusion
is never allowed. Perhaps every table should partition the possibility
space. In any table, the last state should be \emph{None of the above},
so (assuming classical logic) it must always be true that some state in
the table obtains.

If one is not completely convinced that classical logic is correct, this
move won't seem particularly appealing. I suspect, however, that most
readers are completely convinced that classical logic is correct, so I
won't investigate that line. Instead I'll look at two more pressing
objections to the idea that decision tables should always have a none of
the above option.

First, in many cases there is no sensible way to determine the
probabilities or utilities that would go in this column. Imagine that
I'm making a decision whose consequences are sensitive to which team
wins the next Super Bowl. (Perhaps I'm planning a giant Super Bowl
party, or I'm setting the odds for season long bets at a sports book.) I
work out the probabilities that each of the 32 teams in the NFL will win
this year, and what the consequences of my various options would be in
each case. If it's never permissible to exclude states from a decision
table, if decision tables always have to be logically complete, I need a
33rd state: that none of these teams win. But how could that be? Maybe
the league might be cancelled? Maybe a new team could be introduced
mid-season and could win? There is not really a sensible way to even
assign probabilities to these options. Worse still, there is no way to
assign utilities to actions given that state. The expected return of an
action given this state will depend on the probabilities of the
different ways it could come about. The error bars on those
probabilities are bigger than the probabilities themselves. There is
simply no sensible value to put in the cell as the value of the pair;
\emph{Schedule a large Super Bowl party in Las Vegas}, \emph{None of
these 32 teams win the league}. If that state comes about because the
Super Bowl is cancelled, it's terrible. If it comes about because a new
team gets added, that would create so much interest that it would be
great. If I don't have any way of figuring out the relative
probabilities of these events, I have no idea what the expected value
is. So this approach makes decision tables useless.

Second, one should only be unwilling to exclude states from decision
tables if one is so sceptical that one is unwilling to take any
contingent proposition to be evidence. After all, taking something to be
evidence involves excluding possibilities where it doesn't obtain from
one's reasoning. If one doesn't take anything to be evidence, then it is
unclear how one's probabilities can update. It can't be by regular
conditionalisation. It could be by Jeffrey conditionalisation, if one
thought that somehow it was impossible to ever learn that \emph{p}, but
sometimes possible to learn what \emph{p}'s probability is. Personally,
I've never had a learning experience that told me the precise
probability of some proposition without learning for sure some other
proposition. I have never seen reason to think anyone else has either.

This is a quite general point about interest-relative epistemology, and
one that will keep coming up in different ways throughout the book. If
one wants to do without knowledge, and just use probabilities (or
credences), one owes us a story of how those probabilities change. The
best stories about how probabilities change all involve some kind of
interest-relativity.

\subsection{Evidence}\label{sec-kae-evidence}

These considerations suggest a different answer to this exclusion
problem; perhaps the decision maker can exclude \emph{p} iff \emph{p} is
part of their evidence. Call this view EAE, for Evidence Allows
Exclusion.

It isn't obvious that this is an alternative to KAE. If evidence and
knowledge are co-extensive, as Williamson
(\citeproc{ref-Williamson2000}{2000}) argued, it will not be. Since I'm
going to argue in Chapter~\ref{sec-evidence} that Williamson is wrong
about this, I'm committed to EAE and KAE being distinct. So I need an
argument against EAE.

My argument will be by cases. That \emph{p} is part of one's evidence
either entails that one knows \emph{p} or it does not. Either way, EAE
doesn't pose a problem for my overall argument.

If it does, then whether EAE or KAE is true won't matter for the overall
argument. I'm going to argue that some propositions that are known in
typical situations might not be properly excluded if one's interests
change. That will imply interest-relativity given KAE, but it will also
imply interest-relativity given EAE plus the thesis that evidence
entails knowledge.

If evidence doesn't entail knowledge, then EAE is implausible. If
evidence isn't strong enough to let the decision maker know that
propositions inconsistent with it are false, it surely isn't strong
enough to let the decision maker know they can ignore propositions
inconsistent with it.

The view I'll defend in Chapter~\ref{sec-evidence} is that evidence does
entail knowledge. There is a really simple argument for this view. One
way to know that \emph{p} is by properly deducing \emph{p} from one's
evidence. The deduction \emph{p, therefore p} can be properly carried
out. So one can know anything in one's evidence. I'm not relying on this
argument here, and instead on the point that if evidence doesn't suffice
for knowledge, it surely doesn't suffice for exclusion.

The same considerations show that CAE, the view that Certainty Allows
Exclusion, doesn't threaten the larger argument for interest-relativity.
Either certainty entails knowledge or it doesn't. If it does, then CAE
can be used in place of KAE below to derive interest-relativity. If it
does not, and this might happen if certainty just is subjective
certainty, then it is implausible that it suffices for proper exclusion.

\subsection{Sufficiently High Probability}\label{sec-kae-high-prob}

Perhaps one can exclude those propositions whose falsity is sufficiently
high that treating them as definitely false doesn't make a difference to
the decision one makes. Call this view PAE, for sufficiently high
Probability Allows Exclusion.

The first thing to note is that if this is to be plausible, the notion
of sufficiency here must be interest-relative. It's often fine to ignore
propositions that have a one in 500 chance of being true. When planning
what to do a fine sunny day with a clear weather forecast, I simply
ignore the chance that there will be a passing shower, even though that
still has a 1 in 500 chance. On the other hand, it's absurd to ignore
one in 500 chances when deciding what insurance to buy. About one house
in 500 has a fire in a given year; that's not a reason to skip fire
insurance for the year.

Second, as stated this view has the odd consequence that decision makers
can ignore situations that actually obtain. This doesn't seem very
plausible. At least, it would be very odd to have a textbook
representation of a decision problem where the actual world wasn't in
one of the columns. So probably the best way to interpret PAE is as
saying that falsehoods can be excluded iff they have sufficiently high
probability.

Third, once one does that, PAE starts to look suspiciously like a form
of KAE. In particular, it looks like the view I'll call IRT-CP in
Chapter~\ref{sec-ties}. That means (a) that it isn't obviously an
alternative to KAE, and (b) the objections to IRT-CP are also objections
to it. Since I'll go over those objections in detail in
Section~\ref{sec-lockecoin} and Section~\ref{sec-lockegames}, I won't
double them up here, but assume that they work against PAE.

\subsection{Wrapping Up}\label{sec-kae-wrapping-ip}

I've argued that the states we can exclude from a decision table are the
states that the agent knows not to obtain. The argument is largely by
elimination. One might object that I haven't excluded \emph{all}
alternatives. We could keep going asking whether one can exclude all and
only those things that are justifiably believed to be false, or which
are known to be known, or any number of other alternatives.

At this point, it is natural to object to alternatives that they are too
complicated to warrant much confidence. What we can properly take for
granted in decision making is a very important fact about our doxastic
states. If one is sympathetic to a broadly functionalist picture of
mind, it might be the most important fact. If so, it isn't surprising
that the most common form of appraisal of doxastic states, that they are
knowledge, is the norm for appropriate exclusion. It would be very
surprising if something considerably more complicated was the correct
norm instead.

That's hardly a conclusive argument, but it seems like a good enough one
to leave off the survey here, and return to the main narrative of asking
what follows if Knowledge Allows Exclusion.

\section{From KAE to Interest-Relativity}\label{sec-from-kae-to-irt}

If KAE. Knowledge Allows Exclusion, is true then there is a simple
argument that Anisa loses knowledge when playing the Red-Blue game.
Table~\ref{tbl-assume-agincourt} would be a bad table for Anisa to use
when deciding what to do.

\begin{longtable}[]{@{}rcc@{}}
\caption{What the Red-Blue game looks like if Anisa assumes that the
Battle of Agincourt was in
1415.}\label{tbl-assume-agincourt}\tabularnewline
\toprule\noalign{}
\endfirsthead
\endhead
\bottomrule\noalign{}
\endlastfoot
~ & 2+2=4 & 2+2 ≠ 4 \\
Red-True & \$50 & 0 \\
Red-False & 0 & \$50 \\
Blue-True & \$50 & \$50 \\
Blue-False & 0 & 0 \\
\end{longtable}

If she used that table, then it would look like Blue-True is the weakly
dominant option. That would mean that Blue-True is at least a rational
choice, and perhaps the rational choice. Since Blue-True is not a
rational choice, this table must be wrong. If Anisa knows that the
Battle of Agincourt was in 1415, and knowledge structures decision
tables, then everything on this table is correct. So Anisa does not know
that the Battle of Agincourt was in 1415. Since she does know this when
not playing the game, her knowledge is interest-relative.

\section{Theoretical Knowledge}\label{sec-theoreticalknowledge}

Knowledge structures proper practical deliberation. Because what things
can be taken as structural assumptions differs between different pieces
of practical reasoning, knowledge is sensitive to the interests of the
inquirer. But this isn't the only way in which knowledge is sensitive to
interests. It is also sensitive to which purely theoretical questions
the inquirer is taking an interest in.

I've already mentioned one way in which this has to be true. One kind of
theoretical question is \emph{What should I do in this kind of
situation?}. If actually being in that kind of situation and having to
decide what to do affected what one knows, then thinking abstractly
about it should affect what one knows as well.

This kind of comparison, between practical deliberation about what to
do, and theoretical deliberation about what one should in just that
situation, suggests a few things. It suggests that if practical
interests affect knowledge, then so do theoretical interests. It also
suggests that they should do so in more or less the same way. So it
would be good to have a story that assigns to knowledge the role of
structuring theoretical deliberation, in just the way that it structures
practical deliberation. That's more or less the story I'm going to tell,
though there are some complications along the way.

The story I like starts with an observation by Pamela Hieronymi.

\begin{quote}
A reason, I would insist, is an item in (actual or possible) reasoning.
Reasoning is (actual or possible) thought directed at some question or
conclusion. Thus, reasons must relate, in the first instance, not to
states of mind but to questions or conclusions.
(\citeproc{ref-Hieronymi2013}{Hieronymi, 2013: 115--6})
\end{quote}

So to a first approximation the inquirer knows that \emph{p} only if
they can properly use \emph{p} as a reason in ``thought directed at the
question'' they are considering. That is, they can use \emph{p} as a
step in this reasoning. This way of putting things connects Hieronymi's
view of reasons to the idea present in both Hawthorne and Stanley
(\citeproc{ref-HawthorneStanley2008}{2008}) and Fantl and McGrath
(\citeproc{ref-FantlMcGrath2009}{2009}) that things known are reasons.
While I'm going to spend the rest of this section quibbling about
whether this is quite right, it's a good first step.

It's enough to get us a fairly strong, but also fairly natural, kind of
interest-relativity. In normal circumstances, Anisa knows that the
Battle of Agincourt was in 1415. Now imagine not that she's playing the
Red-Blue game, but thinking about how to play it. And she wonders what
to do if the red sentence says that two plus two is four, and the blue
sentence says that the Battle of Agincourt was in 1415. It would be a
mistake for her to reason as follows: Well, the Battle of Agincourt was
in 1415, so playing Blue-True will get me \$50, and nothing will get me
more than \$50, so I should play Blue-True. The mistake is the first
step; she just can't take this for granted in this very context.

This is a very obscure kind of question to wonder about, but there are
more natural questions that lead to the same kind of result. Imagine
that the day after reading the book, but before playing any weird game,
Anisa starts wondering how likely it is that the book was correct.
History books do make mistakes, and she wants to estimate how likely it
is that this was a mistake. Again, it would be an error to reason as
follows: Well, the Battle of Agincourt was in 1415, and that's what the
book says, so the book is certainly correct. Again, the problem is the
first step; she just can't take this for granted in this very context.

But it's not like she can only take for granted in that context things
that are certain. If that were true, she couldn't even start inquiry
into how likely it is the book got this wrong. She has to take a bunch
of stuff as beyond the scope of present inquiry. She should not question
that the book says that the battle was in 1415, or that there was a
Battle of Agincourt, or that it is a widely written about (but also
widely mythologised) battle, or that 1415 is before the invention of the
moveable type printing press and so records from 1415 might be less
reliable, and so on. None of these things are things that she knows with
Cartesian certainty. Indeed, some of them are probably
all-things-considered less likely than that the Battle of Agincourt was
in 1415.\footnote{When I was editing this book I realised I wasn't sure
  when the moveable type printing press was invented, and had to double
  check it was after 1415.} So it's not like there is some threshold of
likelihood, or of evidential support, and inquiring into the likelihood
of this statement implies that one can take for granted all and only
things that clear this threshold. Rather, individual inquiries have
their own logic, their own rules about what can and can't be taken for
granted.

There is an interesting analogy here with the rules of evidence in
criminal trials. Whether some facts can be admitted at a trial depends
in part on what the trial is. For example, some jurisdictions allow
evidence obtained in a search that illegally violated X's rights to be
used in a trial of Y, though it could not be used when X was on trial.
The picture I have of knowledge is similar; what one knows is what one
can use in inquiry, and what one can use changes depending on the
question under discussion. I'll have much more to say about this in
Chapter~\ref{sec-inquiry}.

So the starting point is that what's known is what can be used. What I'm
going to ultimately defend is a much more restricted thesis. Using what
is known provides immunity from a particular criticism: that your
starting point might not be true. I'm going to say a little bit about
why this immunity claim is correct, and then say much more about why I
prefer this way of talking about the role of knowledge in reasoning.

When one says that it is good to use what one knows in reasoning, there
are two natural ways to interpret this. One is that using what one knows
is all-things-considered good unless there is some independent reason to
the contrary. The other is to say that there is a kind of badness in
reasoning one avoids if one uses what one knows. I'm going to be
defending the second kind of reading. That's what I mean by saying that
using what one knows provides immunity from a certain kind of criticism.
The alternative requires that we can specify all the ways in which one
might go wrong while using what one knows - those are the ``independent
reasons to the contrary''. I don't think that's something we're now in a
position to do.

The justification for the immunity claim is quite straightforward. It's
incoherent to say of someone that they know that \emph{p}, but they
shouldn't have used \emph{p} in reasoning because it might be false.
That's Moore-paradoxical, if not outright contradictory. If it is
incoherent to say \emph{A, and X shouldn't have done B because C}, then
\emph{A} is a defence to the criticism of \emph{X} that she shouldn't
have done \emph{B} because \emph{C}. So knowing that \emph{p} is a
defence to the criticism that one shouldn't have used \emph{p} in
reasoning because it might be false.

Can we say something stronger? Can we say that knowing that \emph{p}
immunises the reasoner from all criticisms? Surely not; using irrelevant
facts in inquiry is a legitimate criticism, even if the facts are known.
Is there a true claim that's a bit more qualified, but still stronger
than the immunity claim that I make?

One possibility would be to say that reasoning that starts with what is
known is immune from all criticisms except those on a specified list.
What might be on the list? I've already mentioned one thing - using
irrelevant facts. Another thing might be that the reasoning itself is
irrelevant to what one should be doing. If there is a drowning child in
front of me, and I start idly musing about what the smallest prime
greater than a million might be, I can be criticised for that reasoning.
That criticism can be sustained even if my mathematical reasoning is
impeccable, and I get the correct answer.\footnote{As it turns out,
  that's 1,000,003.}

Some facts are irrelevant to an inquiry. Others are relevant, but not
part of the best path to resolving the inquiry. This can be a ground for
criticism as well. It's in some cases a mild criticism. If one follows
an obvious path to solving a problem, when there is an alternative
quicker way to solving the problem using a clever trick, it isn't much
of a complaint to say that the reasoning wasn't maximally efficient.
There are many quicker proofs of a lot of things Euclid proved, but this
hardly detracts from the greatness of Euclid's work. And, interestingly
for what is to follow, using an inefficient means of inquiry does not
prevent the inquiry ending in knowledge. After all, Euclid knew a lot of
geometry, even though he rarely had maximally efficient proofs. There is
a general lesson here - the fact that an inquirer was imperfect isn't in
itself a reason to deny that they end up with knowledge.

Inefficiency in inquiry is often not a big deal; other mistakes in
inquiry are more serious. Sometimes the premises do not support the
conclusion. It's notoriously hard to say what is meant by support here.
It seems to have some rough relationship to logical entailment, but it's
hard to say more than that. Sometimes premises support a conclusion they
do not entail - that's what happens in all inductive inquiry. Sometimes
premises do not support a conclusion they do entail. If I reason, ``3 is
the first odd prime greater than 0, so 1,000,003 is the first odd prime
greater than 1,000,000, and there are no even primes greater than 2, so
1,000,003 is the first prime greater than 1,000,000'', I reason badly. I
can't know on that basis that 1,000,003 is the first prime greater than
1,000,000. But the premise, that 3 is the first odd prime greater than
0, entails the next step. It just fails to support it, in the relevant
sense.

Maybe now we might suspect we've got enough criticisms on the table. Is
there anything wrong about an inquiry where the following criteria are
met?

\begin{itemize}
\tightlist
\item
  It is worthwhile to conduct the inquiry.
\item
  It is sensible, and efficient enough, to choose these particular
  starting points.
\item
  The starting points are all things that are known to be true.
\item
  Every step after the starting point is supported by the steps
  immediately preceding it.
\end{itemize}

An inquiry with these features looks pretty good. If there is really
nothing to complain about in such an inquiry, then the following is
true. An inquirer who starts an inquiry with what they know is immune
from all criticisms except perhaps (a) that they shouldn't be conducting
this inquiry at all, (b) that their starting points are irrelevant (or
perhaps inefficient) for reaching their conclusion, or (c) that their
later steps are not supported by their earlier steps. While those are
fairly non-trivial exception clauses, that's still a fairly strong claim
about the role of knowledge in inquiry.

Unfortunately, there are puzzle cases that suggest that even an inquiry
with those four features may be flawed. I'll just mention two such cases
here. The point of these cases is that they suggest inquiry can be
flawed in ever so many ways, and we should not be confident about
putting together a complete list of the ways inquiry can go wrong.

First, there might be moral constraints on inquiry. Consider the
following example, drawn from Basu and Schroeder
(\citeproc{ref-BasuSchroeder2019}{2019}). Casey is at a fancy
fundraising party, where the guests and the wait staff are all wearing
suits. The person next to Casey is black, and Casey reasons as follows.

\begin{enumerate}
\def\labelenumi{\arabic{enumi}.}
\tightlist
\item
  Almost all the black people here are on the wait staff.
\item
  The person next to me is black.
\item
  So, the person next to me is on the wait staff.
\end{enumerate}

That's not valid, but one might argue that it's a rational inductive
inference. Alternatively, we can consider the case where Casey
explicitly concludes that the person next to them is probably black. We
can imagine that all of the following things are true. It is reasonable
for Casey to think about whether the person in question is on the wait
staff; it matters for the reasonable practical purpose of getting a
drink. The wait staff are not wearing distinctive clothes, so seeing
what observational characteristics correlate with being on the wait
staff is a reasonable approach to that inquiry. Casey knows that the
premises of the inquiry are true, and the premises support the
conclusion of the inquiry.

And yet, it seems something goes badly wrong if Casey reasons this way.
If the conclusion is false, it doesn't seem like mere inductive bad
luck. Arguably, there is a moral prohibition on reasoning in this way.
Furthermore, this moral prohibition plausibly prevents Casey's reasoning
from providing knowledge.

Now one might well question just about every step of the last two
paragraphs. It's one thing to regret the lack of signals from attire as
to who is on the wait staff; it's another thing to jump to using skin
colour as the best proxy. Given how many other things Casey can see
about this person (such as how they are moving, what they are carrying,
how they are engaging with others), it isn't clear that the premises
support the conclusion, even inductively.

Even if all those things are not true, it might be that Casey can get
knowledge this way; the inquiry might be morally wrong without having
any epistemic flaws that prevent it generating knowledge. Other examples
of morally problematic inquiry suggest that there is no simple
connection between an inquiry being morally bad, and it not generating
knowledge. Many inquiries are morally problematic because they involve,
or even constitute, privacy violations. But that doesn't mean the
privacy violator doesn't come to know things about their victim. Indeed,
part of the wrongness of the privacy violation is that they do come to
know things about their victim.

Still, Casey can be criticised for inquiring in this way, even if the
criticism does not imply that the inquiry produced no knowledge. That
suggests that there are possible criticisms of inquiries that satisfy
the four bullet points listed earlier.

Another source of trouble comes from holistic constraints on reasoning.
What I have in mind here are rules that allow for a natural resolution
of the puzzles of ``transmission failure'' that Crispin Wright
(\citeproc{ref-Wright2002}{2002}) discusses. Start with one of Wright's
examples. Ada is walking by a park with a football pitch. It clearly
isn't just a practice; the players are in uniforms and occupying
familiar positions on the pitch, there is a referee and a crowd, and so
on. One of the players kicks the ball into the net, the referee points
to the centre of the ground, and half the players and crowd celebrate.
After this happens, Ada reasons as follows.

\begin{enumerate}
\def\labelenumi{\arabic{enumi}.}
\tightlist
\item
  The ball was kicked into the net, and no foul or violation was called.
\item
  So, a goal was scored.
\item
  So, a football match is being played, as opposed to, e.g., an ersatz
  match for the purposes of filming a movie.
\end{enumerate}

As Wright points out, there is something wrong with the step from 2 to 3
here. As he also points out, it isn't trivial to say just what it is
that's wrong. After all, 2 entails 3, and Ada knows that 2 entails 3.
But it seems wrong to make just this \emph{inference}.

Here's one natural suggestion about what's wrong.\footnote{This is far
  from an original suggestion. See Weisberg
  (\citeproc{ref-Weisberg2010}{2010}) for discussion of it, and of
  related proposals, and for more discussion of the literature on
  Wright's examples.} It's too simple to be the full story, but it's a
start. The transition Ada makes from 1 to 2 presupposes 3, and 1 is her
only evidence for 2. When those two conditions are met, it is wrong to
infer from 2 to 3. More generally, there is something wrong with
inferring a conclusion from an intermediate step in reasoning if that
conclusion must be presupposed in order to even reach that intermediate
step.

This is too rough as it stands to be a full theory of what is going on
in cases like Ada's, but the details aren't important at this point.
What is important is that there might be some kind of holistic
constraint on reasoning. In some sense, Ada goes wrong in taking 2 for
granted when she infers 3. This doesn't intuitively undermine her claim
to know both 2 and 3.

One important commonality between the last two cases, the moral
encroachment and the transmission failure cases, is that the reasoning
is not subject to the following kind of criticism. The speaker can't be
criticised for taking as a premise something that might be false. Maybe
there is something wrong with inferring something is probably true of an
individual because it is true of most people in the group the individual
is part of. But this restriction applies to the inference; not to the
premises. We wouldn't say to the person who made this inference, ``You
shouldn't reason like that; it might not be true that most people in the
group have this feature.'' If we did say that, they would have an easy
reply. If Ada does do the problematic reasoning, it would be wrong to
reply to her ``You shouldn't reason like this; it might not have been a
goal.'' She could simply, and correctly, say that it quite clearly was a
goal.

This is the key to the correct rule linking knowledge and reasoning. If
the inquirer uses as a step in reasoning something that she knows to be
true, then she is immune to a certain kind of criticism. She is immune
to the criticism that the premise she used might not be true.

What I started this section doing was saying that such a reasoner is
immune to all criticism, then trying to work out exceptions to that
principle. So an exception needed to be included to allow that the
reasoner might be criticised for using an irrelevant reason. The hope
was that eventually a full list of such exceptions could be found. This
project turned out to be wildly optimistic. I don't know that we need to
include further exceptions to handle the moral encroachment or
transmission failure cases. But I also don't know that we don't need to
include extra exceptions. And I have no idea, and no idea how to find
out, whether we need yet more exceptions.

Rather than say knowledge provides immunity to criticism except in these
cases, and then try to fill out the list of cases, it's better to say
that knowledge provides a particular kind of immunity. If the reasoner
knows that the premise they use is true, they can't be criticised on the
grounds that it might be false. This isn't a trivial claim. There were
several examples involving Anisa where she could be criticised for using
a premise that might be false. All of those seemed like legitimate
criticisms even though the premise was one she knew before starting the
inquiry. That criticism does not seem appropriate in the moral
encroachment case, or the transmission failure case, or other cases like
them that may be discovered.

I am assuming here that there is no trivial connection between \emph{It
might be that not-p}, and \emph{The inquirer does not know that p}. If
these claims express the same thing, at least in the particular context
of evaluating the inquirer, then it would be trivial to say that
knowledge provides immunity to criticism on the grounds that one's
premises might not be true. The recent literature on epistemic modals,
however, does not inspire confidence that any such trivial connection
exists.\footnote{See Holliday \& Mandelkern
  (\citeproc{ref-HollidayMandelkern2024}{2024}) for a survey of how
  differently the two claims behave in embeddings and inferences, and a
  radical claim about how to best account for those differences.} So
this immunity seems like a non-trivial claim.

So the key principle I'll be working with is that One cannot be
criticised for using what one knows in an inquiry on the grounds that
one is using what might be false. That's a bit of a mouthful, so
sometimes I'll simply say that one can rationally take for granted what
one knows. I'll have a lot more to say about this principle in the rest
of this book, especially in Chapter~\ref{sec-evidence}.

I'll spend the rest of this chapter talking about how this principle
relates to the idea that knowledge is closed under competent deduction.
There are interesting examples that seem to show that the principle
leads to several distinct kinds of violations of that principle. I'll
argue that this is not right, and for any plausible closure principle,
adding the idea that one can take for granted what one knows does not
yield a new objection to that principle.

The principle as stated is a little ambiguous, and to defend it I need
to resolve that ambiguity. Surprisingly, I need to resolve it by taking
the logically stronger disambiguation. Normally if a principle is
ambiguous, and might lead to problems, the trick is to insist on the
weaker reading. That's not what's about to happen.

When I say that an inquirer can rationally take for granted the things
they know, this should be understood collectively. That's to say, I
endorse the collective and not (merely) the individual version of the
immunity to criticism principles stated here.

\begin{description}
\tightlist
\item[Take for Granted (Individual)]
If an inquirer knows some things, then each of those things are such
that they can take that thing for granted in conducting the inquiry.
\item[Take for Granted (Collective)]
If an inquirer knows some things, then they can take all of those things
for granted in conducting the inquiry.
\end{description}

I'll come back to the difference between these principles, and why I
need to endorse the collective version, in Section~\ref{sec-andintro}.
Until then I'll be talking about single pieces of knowledge at a time.

\section{Knowledge and Closure}\label{knowledge-and-closure}

Here are two very plausible principles about knowledge, both due to John
Hawthorne (\citeproc{ref-Hawthorne2005}{2005}).

\begin{description}
\tightlist
\item[Single Premise Closure]
If one knows \emph{p} and competently deduces \emph{q} from \emph{p},
thereby coming to believe \emph{q}, while retaining one's knowledge that
\emph{p}, one comes to know that \emph{q}.
(\citeproc{ref-Hawthorne2005}{Hawthorne, 2005: 43})
\item[Multiple Premise Closure]
If one knows some premises and competently deduces \emph{q} from those
premises, thereby coming to believe \emph{q}, while retaining one's
knowledge of those premises throughout, one comes to know that \emph{q}.
(\citeproc{ref-Hawthorne2005}{Hawthorne, 2005: 43})
\end{description}

Hawthorne endorses the first of these, but has reservations about the
second for reasons related to the preface paradox. I'm similarly going
to endorse the first and have reservations about the second. But my
reasons don't have anything to do with the preface paradox. I argued in
``Can We Do Without Pragmatic Encroachment''
(\citeproc{ref-Weatherson2005-WEACWD}{Weatherson, 2005a}) that concerns
about the preface paradox are over-rated, and I think those arguments
still hold up. But I have a slightly different qualification than
Hawthorne does to Multiple Premise Closure, and I will discuss that more
in Section~\ref{sec-andintro}.

It is not trivial to prove that my version of IRT satisfies these
closure conditions. One reason for this is that I have not stated a
sufficient condition for knowledge. all that I have said is that
knowledge is incompatible with a certain kind of caution. So in
principle I cannot show that if some conditions obtain then someone
knows something. What I can show is that introducing new conditions
linking knowledge with relevant questions does not introduce new
violations of the closure conditions.

\subsection{Single Premise Closure}\label{sec-andelim}

But it turns out that even showing this is not completely trivial.
Imagine yet another version of the Red-Blue game.\footnote{This game
  will resemble the examples that Zweber
  (\citeproc{ref-Zweber2016}{2016}) and Anderson and Hawthorne
  (\citeproc{ref-AndersonHawthorne2019b}{2019b}) use to raise doubts
  about whether pragmatic theories like mine really do endorse single
  premise closure.} In this game, both of the sentences are claims about
history that are well supported without being certain. And both of them
are supported in the very same way. It turns out to be a little
distracting to use concrete examples in this case, so just call the
claims A and B. Imagine that the player read both of these claims in the
same reliable but not infallible history book, and she knows the book is
reliable but not infallible, and she aims to maximise her expected
returns. Then all four of the following things are true about the game.

\begin{enumerate}
\def\labelenumi{\arabic{enumi}.}
\tightlist
\item
  Unconditionally, the player is indifferent between playing Red-True
  and playing Blue-True.
\item
  Conditional on \emph{A}, the player prefers Red-True to Blue-True,
  because Red-True will certainly return \$50 while Blue-True is not
  completely certain to win the money.
\item
  Conditional on \emph{B}, the player prefers Blue-True to Red-True,
  because Blue-True will certainly return \$50 while Red-True is not
  completely certain to win the money.
\item
  Conditional on \emph{A}~∧~\emph{B}, the player is back to being
  indifferent between playing Red-True and playing Blue-True.
\end{enumerate}

From 1, 2 and 3, it follows in my version of IRT that the player does
not know either \emph{A} or \emph{B}. After all, conditionalising on
either one of them changes her answer to a relevant question. The
question being,~\emph{Which option maximises my expected returns?},
where this is understood as a mention-all question.

Now look what happens at point 4. Conditionalising on
\emph{A}~∧~\emph{B} does not change the answer to that question. So,
assuming there is no other reason that the player does not know
\emph{A}~∧~\emph{B}, arguably she does know \emph{A}~∧~\emph{B}. That
would be absurd; how could she know a conjunction without knowing either
conjunct?

Here is how I used to answer this question. Define a technical notion of
interest. Say that a person is interested in a conditional question
\emph{If p, Q?} if they are interested, in the ordinary sense, in both
the true-false question \emph{p?} and they are interested in the
question \emph{Q?}. If conditionalising on a proposition changes (or
should change) their answer to any question they are interested in in
this technical sense, then they don't know that proposition. This solves
the problem because conditionalising on \emph{A}~∧~\emph{B} does change
their answer to the question \emph{If A, which option maximises expected
returns?} on its mention-some reading. So even though 4 is correct, this
does pose a problem for closure.

This was not a great solution for two reasons. One is that it seems
extremely artificial to say that someone is interested in these
conditional questions that they have never even formulated. Another is
that it is hard to motivate why we should care that conditionalisation
changes (or should change) one's answers to these artificial questions.

There was something right about the answer I used to give. It is that we
should not just look at whether conditionalisation changes the answers a
person gives to questions they are interested in. We should also look at
whether it changes things `under the hood'; whether it changes how they
get to that answer. The idea of my old theory was that looking at these
artificial questions was a way to indirectly look under the hood. What I
got wrong was trying to find some other question whose answer changed
when and only when what was under the hood changed. I should have just
looked under the hood.

So let's look again at the two questions that are relevant. This time,
don't think about what answer the player gives, but about how they get
to that answer.

\begin{enumerate}
\def\labelenumi{\arabic{enumi}.}
\setcounter{enumi}{4}
\tightlist
\item
  Which option maximises expected returns?
\item
  If \emph{A}~∧~\emph{B,} which option maximises expected returns?
\end{enumerate}

On the most natural way to understand what the player does, there will
be a step in her answer to 5 that has no parallel in her answer to 6.

She will note, and rely on, the fact that she has equally good evidence
for \emph{A} as for \emph{B}. That is why each option is equally good by
her lights. The equality of evidence really matters. If she had read
that \emph{A} in three books, but only one of those books added that
\emph{B}, then the two options would not have the same expected returns.
She should check that nothing like this is going on; that the evidence
really is equally balanced.

But nothing like this happens in answering 6. In that case,
\emph{A}~∧~\emph{B} is stipulated to be given. So there is no question
about how good the evidence for either is. When answering a question
about what to do if a condition obtains, we don't ask how good the
evidence for the condition is. We just assume that it holds. So in
answering 6, there is no step that acknowledges the equality of the
evidence for both \emph{A} and \emph{B}.

So in fact the player does not answer the two questions the same way.
She ends up with the same conclusion, but she gets there by a different
means. And that is enough, I say, to make it a different answer. If she
knew \emph{A}~∧~\emph{B} she could follow exactly the same steps in
answering 5 and 6, but she cannot.

What should we say if she does follow the same steps? If this is
irrational, nothing changes, since what matters for knowledge is which
questions should be answered the same way, not which questions are
answered the same way. (It does matter for belief, but that is not the
current topic.) So I will assume that it is possible for the player to
rationally answer both questions the same way. (I will have much more to
say about why this is a coherent assumption in Chapter~\ref{sec-ties}.)

The way she should answer 6 is to take \emph{A}~∧~\emph{B} as given.
Hence she will take either option, Red-True or Blue-True, as being
equivalent to just taking \$50, which she knows that is the best she can
do in the game. So in answering question 6, she will take it as given
that both of these options are maximally good.

By hypothesis, she is answering question 5 and question 6 the same way.
So she will take it to be part of the setup of question 5 that both
options return a sure \$50 After all, that is part of the setup of
question 6. But if she takes that as given, then conditionalising on
either \emph{A} or \emph{B} does not change her expected returns. So now
claims 2 and 3 are wrong; conditionalising on either conjunct won't make
a difference because she treats each conjunct as given.

That is the totally general case. Assume that someone has competently
deduced \emph{Y} from \emph{X}, and they know \emph{X}. So they are
entitled to answer the questions \emph{Q?} and \emph{If X, Q?} by the
same method. Since the method for the latter takes \emph{X} as given, so
can the method for the former. So they can answer \emph{Q?} taking
\emph{X} as given. What one can appropriately take as given is closed
under competent deduction? (Why? Because in the answer to \emph{Q?} that
starts with \emph{X}, you can just go on to derive \emph{Y}, and then
see that it is also a way to answer \emph{If Y, Q?}.) So they can answer
\emph{Q?} taking \emph{Y} as given. So they can answer \emph{Q?} in the
same way they answer \emph{If Y, Q?}.

So assuming there is no other reason to deny \textbf{Single Premise
Closure}, adding a clause about how one may answer questions does not
give us a new reason to deny it.

\subsection{Multiple Premise Closure}\label{sec-andintro}

That shows that IRT satisfies Single Premise Closure. The argument that
it satisfies Multiple Premise Closure starts with the observation that
Multiple Premise Closure more or less follows from Single Premise
Closure plus a principle I'll call \textbf{And-Introduction Closure}.

\begin{description}
\tightlist
\item[And-Introduction Closure]
If one knows some propositions, and one competently infers their
conjunction from those propositions, while retaining one's knowledge of
all those propositions, then one knows the conjunction.
\end{description}

Start with the standard assumption that a conclusion is entailed by some
premises iff it is entailed by their conjunction. (It would take us way
too far afield to investigate what happens if we dropped that
assumption.) Given that assumption, in principle the only inferential
rule one needs with multiple premises is And-Introduction. In practice,
people do not generally reason via conjunctions in this way. Someone who
knows \emph{A}~∨~\emph{B}, and who knows ¬\emph{A}, does not first infer
(\emph{A} ∨ \emph{B})~∧~¬\emph{A}, and then infer \emph{B} from that.
They just infer \emph{B}. It's a harmless enough idealisation, however,
to model them as first inferring the conjunction whenever they use
multiple premises. So I will assume that if I can show that IRT does not
cause problems for And-Introduction Closure, and I've already argued
that it does not cause problems for Single Premise Closure, then it does
not cause problems for Multiple Premise Closure.

Here is the quick argument that IRT does not cause problems for
And-Introduction Closure.

\begin{enumerate}
\def\labelenumi{\arabic{enumi}.}
\tightlist
\item
  The key feature of IRT, the one that potentially causes problems for
  And-Introduction closure, is that one knows that \emph{p} only if one
  can take \emph{p} for granted in one's current inquiry.
\item
  If, in the course of an inquiry, one knows some premises, then one can
  take them for granted in that inquiry.
\item
  If one can take some premises for granted in an inquiry, then one can
  take their conjunction for granted in that inquiry.
\item
  So, there is no IRT-based reason that And-Introduction Closure fails.
\end{enumerate}

Premise 1 is just a restatement of my version of IRT, and premise 3
should be uncontroversial. If one can take some premises for granted,
then one (rationally) is ruling out possibilities where they are false.
To rule out possibilities where they are false just is to take their
conjunction for granted. So those premises should be fairly
uncontroversial. What is controversial is that the argument is sound,
and, in particular, that premise 2 is correct.

The conclusion is not that Multiple Premise Closure holds. Maybe you
think it fails for some independent reason, distinct from IRT. I don't
think the other reasons that have been offered in the literature are
compelling, but I am not building the failure of these reasons into IRT.
So the main assumption behind the argument is that if adding the `take
for granted' clause to our theory of knowledge does not lead to closure
violations, then nothing else in the theory does. The argument for that
is basically that there isn't much more to the theory. So I think the
argument is sound.

Still, it might look like the argument must be wrong. After all, it is
easy to cook up cases where it looks like IRT leads to a closure
failure. Here is one such example. It is another version of the Red-Blue
game. In this version, the red sentence is, once again,~\emph{Two plus
two equals four}. This time the blue sentence is a conjunction
\emph{A~and~B}, where both \emph{A} and \emph{B} express historical
facts that the player has excellent, but not perfect, evidence
for.\footnote{If you want to make this more concrete, pick a random
  history book off the shelf and choose two claims that are both
  reasonably specific - so there could easily be a mistake about the
  details - and not independently warranted.} Now the following four
claims all seem true.

\begin{enumerate}
\def\labelenumi{\arabic{enumi}.}
\tightlist
\item
  Unconditionally, the only rational play is Red-True.
\item
  Conditional on \emph{A}, the only rational play is Red-True. Even
  given \emph{A}, playing Blue-True requires betting that \emph{B} is
  true, and that's a pointless risk to run when playing Red-True only
  requires that two and two make four.
\item
  Conditional on \emph{B}, the only rational play is Red-True. Even
  given \emph{B}, playing Blue-True requires betting that \emph{A} is
  true, and that's a pointless risk to run when playing Red-True only
  requires that two and two make four.
\item
  Conditional on \emph{A}~∧~\emph{B}, Blue-True is rationally
  permissible, and arguably rationally mandatory, since it weakly
  dominates Red-True.
\end{enumerate}

So conditionalising on either one of \emph{A} or \emph{B} doesn't change
anything, but conditionalising on \emph{A}~∧~\emph{B} does change how
the player answers a question. So it looks like in this case the player
might know \emph{A}, know \emph{B}, and for all I've said be fully aware
that these two things entail \emph{A}~∧~\emph{B}, but not know
\emph{A}~∧~\emph{B}. So what's happened? How is this not a
counterexample to premise 2?

The key thing to note is that when the player is choosing what to do,
the following things are all true about them.

\begin{itemize}
\tightlist
\item
  They can take \emph{A} for granted. That is, they are rationally
  permitted to take \emph{A} for granted in resolving their inquiry
  about what to do.
\item
  Similarly, they can take \emph{B} for granted.
\item
  But they cannot both take \emph{A} for granted and take \emph{B} for
  granted. If both those things are taken for granted, then they can
  rationally infer that Blue-True will have a maximal payout, and hence
  that it is a rational play. And they cannot infer that.
\end{itemize}

It is cases like this one that required the clarification that I made at
the end of Section~\ref{sec-theoreticalknowledge}. The player here
cannot take both of \emph{A} and \emph{B} for granted. So they don't
know both those things. So this is not a case where they know \emph{A},
know \emph{B}, and don't know \emph{A}~∧~\emph{B}. Since they cannot
take both \emph{A} and \emph{B} for granted, they do not know both of
those things.

The picture I'm presenting here is similar to the picture Thomas Kroedel
(\citeproc{ref-Kroedel2012}{2012}) offers as a solution to the lottery
paradox.\footnote{Different writers take different things to be
  \emph{the} lottery paradox. In all cases, they concern what kind of
  non-probabilistic attitude an ideal agent would take towards the
  proposition that a particular ticket in a large, fair, lottery will
  lose. It seems unintuitive to say that they will not believe this,
  since the ticket might win. And this will lead to an inconsistency,
  since they will believe of every ticket that it will not win, but also
  believe that a ticket will win. But if you say it is not belief, you
  seem to either get scepticism, or the view that the ideal agent can
  believe \emph{p}, and not believe \emph{q}, even though they think
  \emph{q} is more probable than \emph{p}. Which of the four problems I
  just mentioned is most salient to a writer tends to depend on their
  background commitments, but most people defend views on which at least
  one of the problems is genuinely problematic.} He argues that we can
solve the lottery paradox if we take justification to be a kind of
permissibility, not a kind of obligation. And just as we can have
individual permissions that don't combine into a collective permission,
we can have individually justified beliefs that are such that we can't
justifiably believe each of them. This isn't exactly how I'd put it. For
one thing, I'm talking about knowledge not justification. For another,
it's not that knowledge is a species of permission, as much as it
behaves like permission in certain contexts, and those are just the
contexts where counterexamples to And-Introduction Closure arise. These
are minor points of difference though; I'm still basically relying on
Kroedel's ideas.

Thinking of things the way Kroedel suggests helps say something positive
about what is going on in this game. So far I've said something negative
- the player does not know both that \emph{A} and that \emph{B}. That's
enough to show that the case is not a counterexample to And-Introduction
Closure. A counterexample would, after all, have to be a case where the
player knows both \emph{A} and \emph{B}. But saying what's not the case
is not a helpful way to say what is the case. To say something more
positive, it helps to think about other cases where permissions do not
agglomerate. To that end, I'll talk through one case involving
professional norms.

Professor Paresseux is, like most academics, in a situation where
professional morality requires he do his fair share, but is fairly open
about what tasks he does that will constitute doing his fair share.
Right now he has two requests for work, R1 and R2, and while he is not
obliged to do both, he is obliged to do at least one. So he may turn
down R1, and he may turn down R2, but he may not turn down both. So as
not to keep the reader in suspense, let's say up front that he is going
to turn down both. Our question will be, what exactly does Professor
Paresseux do that's wrong?

To make this a little more concrete, and a little more complicated, I
want to add two features to the case. First, accepting R1 would be
better than accepting R2. He is uniquely well placed to do R1, and it
would create more work for others if he turns it down. (As, indeed, he
will.) But the norms governing Professor Paresseux are not maximising
norms, and he does not violate them if he accepts R2 and rejects R1.
Second, Professor Paresseux first turns down R1, let's say in the
morning, and then later that day, let's say after a hearty lunch, turns
down R2. Given that, there are three models we can have for the case,
all of which have some plausibility.

The first model says that he was wrong to turn down R1. Here's a little
argument for that, using language that seems natural. He should have
accepted one of the requests. Since he was well placed to perform R1,
it's also true that if he did one of them, it should have been R1. So he
should have accepted R1, and turning it down was the mistake. Oddly, it
turns out to have been made true that he did the wrong thing in turning
down R1 by his latter decision to turn down R2, but that's just an odd
feature of the case.

The second model says that odd feature is intolerably odd. It says he
was wrong to turn down R2. Here's a little argument for that. At
lunchtime, he hadn't done anything wrong. True, he had turned down R1,
but he had moral permission to do that. It was only after lunch that he
made it the case that he violated a norm. So the violation must have
been after lunch. So the violation was in turning down R2.

A third model says that both of these arguments are inconclusive. What's
really true is simply that Professor Paresseux should not have turned
down both requests. Which one individually was wrong? That, says the
third model, is indeterminate. One of them must be, since he could not
permissibly turn down both. But there is no fact of the matter about
which it is.

If I had to choose, I would say that the third is the most plausible
model. The arguments for the first two models are not terrible - indeed
I think both are plausible models - but the arguments are equally
compelling, and incompatible. So I suspect neither is entirely right.
The third model, which says both of them are partially right - there is
something not quite ok about both refusals - seems to better fit the
scenario. But what I more strongly think is that each of these models is
more plausible than either of the following two.

The fourth model is that there is a strong kind of agglomeration
failure. It is determinately true that Professor Paresseux acted
permissibly it turning down R1, and it is determinately true that he
acted permissibly in turning down R2, but overall he acted
impermissibly. It's true that in the abstract Professor Paresseux could
have turned down each one. But in the particular context he is in, where
these are the options to fulfil his duty to do his share of the work,
and he does neither, is not a context where he can (determinately) avail
himself of both of these permissions.

The fifth model says that since he had to do his share and did not, and
both refusals are ways of not doing his share, both of them are
impermissible. This seems like overkill. It is much more intuitive that
Professor Paresseux has done one wrong thing than that he has done two
wrong things.

I hope I haven't traumatised too many readers with tales of people
shirking professional responsibilities, because having Professor
Paresseux's example on the table helps us lay out the options for what
to say about Player. Player plays the version of the Red-Blue game I
just described, where the blue sentence is the conjunction of two
plausible (and true) claims from a well regarded history book he just
read, and the red sentence is that two plus two is four. Player looks at
the rules, infers via his historical knowledge that playing Blue-True
will have a maximal return, and so plays Blue-True. I think that this
play is irrational, and if Player knew the conjunction it would be
rational, so Player does not know the conjunction. But what do we say
about Player's knowledge of each conjunct? It turns out that there are
five somewhat natural options that correspond to the five models I
offered about Professor Paresseux. I'll simply list them here.

\begin{enumerate}
\def\labelenumi{\arabic{enumi}.}
\tightlist
\item
  Player knows the conjunct for which he has better evidence, and does
  not know the conjunct for which he has less good evidence. It was
  impermissible to take for granted the thing that was less well
  supported. This parallels the idea that Professor Paresseux did
  something wrong in turning down the request he was better placed to
  fulfil.
\item
  Player knows the conjunct that he first took for granted, and not the
  conjunct that he took for granted second. When he first took one of
  the conjuncts for granted, that was a permissible mental act, but
  given that he had done it, it was impermissible to take the second for
  granted. This parallels the idea that whichever request Professor
  Paresseux turns down second is the impermissible turn-down, because
  it's then he becomes in violation of his duty.
\item
  It is indeterminate which conjunct Player knows. He doesn't know both,
  because if he did then he could take both for granted, and he cannot
  take both for granted. Given both conjuncts, Blue-True is a rational
  play. So he must not know one, but there is no reason to say it is
  this one rather than that one, so it is indeterminate which he doesn't
  know. This parallels the indeterminacy solution to Professor
  Paresseux's puzzle.
\item
  Player does know both conjuncts, since knowledge requires permissible
  taking for granted, and each of his takings for granted are
  individually permissible. But he doesn't know the conjunction, and so
  And-Introduction Closure fails.
\item
  Player does not know either conjunct.
\end{enumerate}

The fifth model seems like the least plausible. Somewhat unfortunately,
it is also the model I defended (or at least committed myself to) in
``Can We Do Without Pragmatic Encroachment''. There I said knowledge
requires that conditionalising on the known doesn't change any answers
to interesting questions, and any question taken conditional on an
interesting proposition is interesting. So each of the questions
\emph{What should I play given the first conjunct is true?} and
\emph{What should I play given the second conjunct is true?} are both
interesting questions (in this technical sense of `interesting').
Inquiring into the first question is incompatible with knowing the
second conjunct, while inquiring into the second question is
incompatible with knowing the first conjunct. This was a fun way out the
problem, but it was also overkill. Player loses one bit of knowledge,
not two, so my earlier view must be wrong.

Which of the other four models is correct? I think the fourth, which
violates And-Introduction Closure, is the least plausible. That's
largely because it violates And-Introduction Closure. But the other
three are all plausible, and are all consistent with And-Introduction
Closure. (And note that all five are consistent with IRT. IRT itself
says very little about this puzzle.) My preferred version of IRT says
that typically the third option is correct - usually in cases like this
it is indeterminate what is known.

There are mix-and-match options available. Perhaps if Player's evidence
for the first conjunct is (much) stronger than their evidence for the
second conjunct, and it was the first one that they took for granted in
reasoning, then they (determinately) know the first but not the second
conjunct. I don't need to take a stance on whether cases like this ever
arise to defend And-Introduction Closure. That's because all I need is
that for any case like this, one of the first three models is right.
That can be true even if it is different models in different cases.

\section{Summary}\label{sec-closuresummary}

Putting all that together, IRT is consistent with Single Premise Closure
and with And-Introduction Closure. Assuming that it is a harmless
idealisation to treat anyone who uses multiple premises in reasoning as
reasoning from the truth of all the conjunction of their premises, it
follows that IRT is consistent with Multiple Premise Closure.

But this isn't quite the end of the story. Even if the arguments of the
last two sections work, what they show is that there must be some way to
explain away any apparent conflict between IRT and closure principles.
The arguments do not, on their own, tell us what that explanation will
look like, or whether it will have unacceptable consequences. Without
such an explanation, we might be sceptical of the arguments of this
chapter, and indeed of IRT itself. So I'll come back several times to
issues about closure. In Chapter~\ref{sec-ties}, I'll go over what IRT
says about cases like Zweber's, and Anderson and Hawthorne's, more
thoroughly.

Before I get to that though, it is time to say more about a notion that
has done a lot of work so far but which has not been adequately
investigated: inquiry.

\bookmarksetup{startatroot}

\chapter{Inquiry}\label{sec-inquiry}

The next three chapters are primarily defensive; they are responding to
the three objections to IRT that seem to me most serious. But they
aren't just defensive. I'm not just saying why the theory from the
chapters to date is immune to these arguments. I'm also developing the
theory. That's especially true in this chapter, which is why it is
first. So what are these objections?

The first is what I'll call the \emph{objection from double checking}.
As Jessica Brown (\citeproc{ref-Brown2008}{2008}) argued, there are
plenty of cases where intuitively a person knows that \emph{p}, but
should check whether \emph{p} is true. This seems to be a problem for
IRT, since it is motivated by the thought that what's known is an
appropriate starting point in inquiry. At first glance, it's very weird
to have an inquiry into \emph{p}, when the inquirer is in a position to
simply say \emph{p},~therefore \emph{p}. I used to think that in these
cases the defender of IRT would have to either say that they are not
really cases of knowledge, or not really cases of appropriate inquiry.
Unfortunately, neither of these options was particularly successful. I
now think the objection should be addressed head on. It is possible to
properly conduct an inquiry into \emph{p}, even when one knows that
\emph{p}, and even when knowledge provides appropriate starting points
for inquiry. That's because it is often appropriate to deliberately
restrict oneself in inquiry, and use fewer resources than are otherwise
available. The aim of this chapter is to defend the claims made in the
last two sentences, and to show how they provide a response to the
objection from double checking.

The second is what I'll call the \emph{objection from close calls}. As
Alex Zweber (\citeproc{ref-Zweber2016}{2016}) and, separately, Charity
Anderson and John Hawthorne
(\citeproc{ref-AndersonHawthorne2019a}{2019a}) showed, some simple
versions of IRT say implausible things about cases where a person is
choosing between very similar options. What I'm going to argue is that
the problem their cases raise is not due to IRT, which is correct, but
to the background assumption that choosers should maximise expected
utility. My response is going to be that in the cases they describe,
choosers should not maximise expected utility. That might sound like an
absurdly radical view, since expected utility theory is at the heart of
all contemporary decision theory. But expected utility theory has fairly
implausible things to say about \emph{close call} cases. A better
theory, one that takes account of deliberation costs, is both more
plausible, and consistent with IRT. I'll say much more about this in
Chapter~\ref{sec-ties}.

The third is what I'll call the \emph{objection from abominable
conjunctions}. This is the IRT-equivalent of the blank stare objection
to modal realism. Many people find it simply implausible that knowledge
could depend on something like interests, which are not relevant to the
truth of what is purportedly known. The defender of IRT owes a reply to
this widespread feeling. Part of my reply came back in
Chapter~\ref{sec-overture}. I think this feeling is a result of being in
a very strange place in the history of epistemology, where the focus is
on fallibilist, interest-invariant, concepts. But we can do better than
that. It is hard to articulate the intuition behind the unhappiness with
IRT without lapsing into the JTB theory of knowledge. Most plausible
solutions to the problems with the JTB theory end up introducing kinds
of interest-relativity for independent reasons. I'll go over these
responses in Chapter~\ref{sec-changes}.

So those are the three objections I'm going to spend a lot of time on.
There are three other classes of objection I'm not going to spend much
time on.

The first class are objections to IRT that assume that knowledge changes
when and only when one is in a `high stakes' situation. Since I don't
assume that, those objections don't raise problems for my version of
IRT.

The second class are objections to IRT that assume that some parts of
epistemology are interest-invariant, while some are interest-relative. I
used to endorse such a theory, but I don't any more. This book defends a
global interest-relativism where knowledge, belief, rationality and
evidence are all interest-relative (in different ways). So these
objections don't raise problems for my version of IRT either.

The third class are objections to IRT that only apply to versions of IRT
that add on an opposition to contextualism or relativism. With this
addition, IRT becomes what has been called \emph{interest-relative
invariantism}, or IRI. While I've defended that in the past, I'm not
going to defend it here. The thesis of this book is that knowledge is
interest-relative. If you want to understand the word `knowledge' in the
previous sentence in a contextualist or relativist way, go right ahead.
Whatever metasemantic theory you have about the kind of words `knows'
and `knowledge' are, I will be willing to defend the claim that
knowledge is interest-relative.

\section{Starting and Settling}\label{sec-settling}

At the heart of the influential picture of inquiry developed by Jane
Friedman (\citeproc{ref-Friedman2017}{2017},
\citeproc{ref-Friedman2019a}{2019b},
\citeproc{ref-Friedman2019b}{2019a}, \citeproc{ref-Friedman2020}{2020},
\citeproc{ref-Friedman2024}{2024b}) is the view that humans are capable
of a number of distinctive attitudes. To be inquiring into some
question, she argues, is to have a \emph{questioning attitude} towards
that question. That's to say, she does not identify inquiry with
particular actions, or at least with particular bodily movements. An
actor might mimic the movements an inquirer makes without actually
inquiring; a genuine inquirer might be sitting in an armchair quietly
synthesizing their evidence. So particular movements are neither
sufficient nor necessary for real inquiry. Rather, inquiry is a state of
mind, a questioning state of mind.

The contrast to having a questioning attitude is having a \emph{settled}
attitude.\footnote{These are contrasts, but they don't exhaust the
  space. One might not have an attitude to a question. I'd also say that
  one might not treat a question as settled while not inquiring into it,
  because one treats the question as unworthy of effort, or impossible
  to make progress on. As Friedman
  (\citeproc{ref-Friedman2024debate}{2024a}) notes, it gets complicated
  to say something coherent about these cases while allowing for the
  possibility of inquiry to be reopened.} Friedman holds that to believe
something is to treat the question of whether it is true as being
affirmatively settled, and I'm adopting the same position here. This
attitude is deeply related to inquiry. Typically things are settled as
the result of inquiry. Also typically, one does not inquire into
something one has settled. Friedman holds a further claim: if one does
inquire into something one has settled, this is a kind of mistake. It is
incoherent to both have a questioning and a settled attitude towards the
same question. I'm going to disagree with this further claim, while
mostly adopting the broad picture she develops.

The main difference between her picture and the picture of inquiry I'm
using concerns where beliefs go in inquiry. I think that treating
something as settled is most fundamentally about willingness to use it
as the beginning of a new inquiry. The essential feature of belief is
that it starts inquiry, not that it ends inquiry. What makes an attitude
a belief is not that inquiry into it is settled, it's that it can be
used in the process of settling open questions. I used to think that
whether one identified beliefs with settled states, or with the inputs
to inquiry, was only a difference of emphasis, and a pretty minor one at
that. After all, beliefs are typically the outputs of one inquiry and
then serve as inputs to another; whether one takes one or other of these
roles to be more fundamental seems like a pretty esoteric question. But
I've come to think that actually quite a bit turns on it. If you think
beliefs are fundamentally the things that inquiry start with, then there
is a little gap in the argument that one should not inquire into what
one already believes.

That argument, the one to the conclusion that one should not inquire
into what one already believes, seems pretty simple. Assume one believes
that \emph{p} and is inquiring into the question \emph{p?}. Our theory
is that beliefs are appropriate starting points for inquiry, so it looks
like this one should end pretty quickly. One can just argue \emph{p},
therefore \emph{p}, and close the inquiry. If the inquiry stays open
longer than that, one is doing it wrong.

This looks like a pretty strong argument for a conclusion that a number
of people have reached via different routes.\footnote{These quotes were
  compiled by Elise Woodard (\citeproc{ref-Woodard2021}{2020}).}.

\begin{quote}
If one knows the answer to some question at some time then one ought not
to be investigating that question, or inquiring into it further \ldots{}
at that time. (\citeproc{ref-Friedman2017}{Friedman, 2017: 131})
\end{quote}

\begin{quote}
There is something to be said for the claim that the person who knows
they have turned the coffee pot off should not be going back to check.
(\citeproc{ref-HawthorneStanley2008}{Hawthorne \& Stanley, 2008} ,587)
\end{quote}

\begin{quote}
Any such cases {[}of believing while inquiring{]} involve peculiarities
(such as irrationality or fragmentation).
(\citeproc{ref-McGrath2021}{McGrath, 2021} ,482n37)
\end{quote}

So how could that argument fail? It could fail if there are reasons for
adopting constraints on an inquiry. If there are reasons to not use all
the tools at our disposal, there could be cases where an inquiry into
\emph{p} gets started, and we have reasons not to just say \emph{p},
therefore \emph{p}. At the highest possible level of abstraction, this
doesn't sound very likely. It seems at first like there should be
something like a principle of total evidence for inquiry, saying that
you can use whatever tools, whatever evidence, you have to hand. Such a
principle, however, turns out to be false.

To warm up to this, consider an analogy to legal inquiries. There we are
all familiar with the idea that some evidence might be inadmissible in
some inquiries. Now the reasons for this are typically not epistemic.
It's rather that we think the system as a whole will be more just if
some kinds of evidence are excluded from some inquiries. That looks a
bit different to the situation where an individual inquirer is just
trying to find what's true. But we'll see that the analogy here is not
quite as bad as it first looks.

In the rest of this section, I'll go over six kinds of cases where one
can sensibly inquire into what one already knows. I don't think any of
these examples constitute knock-down proofs of the possibility of
rational inquiry into what one knows, and for reasons I'll get to later
in the chapter, I don't really need them to. It is helpful to see the
range of cases where inquiry into what one knows is useful.

\subsection{Sensitivity Chasing}\label{sec-sensitiveinquiry}

Guido Melchior (\citeproc{ref-Melchior2019}{2019}) argues that the point
of \emph{checking} is to establish a sensitive belief in the checked
proposition. To motivate this, think about the following case. Florian
has just weighed out the coffee beans for his morning pot of coffee.
Naturally he uses the best scales he has for this purpose; it's
important to get the coffee right. He starts wondering whether his
scales have recently stopped being reliable. What does he do next?
Here's one thing he doesn't do. He doesn't look at the beans on the
scale, note that the scale says 24g, note that he knows they are 24g
(via that excellent scale), and conclude that the scale is still
working. That's no good at all; he has to use some other scale to check
this one.

This is like the Problem of Easy Knowledge
(\citeproc{ref-Cohen2002}{Cohen, 2002}), but note that it doesn't rely
on the scale being a source of basic knowledge. Florian might have lots
of independent evidence that the scale is good; it's from a good
manufacturer and has been producing plausible results for a while.
Still, if he wants to check it, he has to use something else. Here's the
part that seems most surprising to me. Add to the story that he has a
backup scale, one that he thinks is pretty good but not as good as his
best scale. It's fine to use the backup scale to check the main scale,
and not fine to use the scale to check itself. The best explanation for
this is that checking requires sensitivity. Using the scale to test
itself is a method that isn't sensitive to whether the scale is working.
Using some other scale, even a less reliable one, to check whether it is
working, is at least somewhat sensitive. Checking is, at least in part,
a matter of \emph{sensitivity chasing}. One reason it is often good to
check what one knows is that sensitivity chasing is often sensible.

Sensitivity chasing is perfectly acceptable goal in inquiry. One might
inquire into \emph{p} for the purpose of making one's belief in \emph{p}
more sensitive. Now assume, as most epistemologists believe, that one
can know \emph{p} even if one's belief is insensitive in various ways.
One can know \emph{p} even if one would still believe \emph{p} were
\emph{p} false.\footnote{One simple example from Saul Kripke
  (\citeproc{ref-KripkeNozick}{2011}): I know that I do not falsely
  believe that I was born on the Galapagos Islands. But while this is
  knowledge, it is not a sensitive belief.} If one has insensitive
knowledge, it might be worthwhile to inquire into what one knows with
the aim of generating sensitive knowledge. Indeed, this seems like a
primary aim of what we call \emph{checking}. Inquiring into \emph{p} by
saying \emph{p} therefore \emph{p} will not increase one's sensitivity
to whether \emph{p} is true. So it's worthwhile to not allow that move
in the inquiry, if the aim is to increase sensitivity.

There are other examples that show the difference between knowing and
checking. Slightly modifying an example from Frank Jackson
(\citeproc{ref-Jackson1987}{1987}), imagine that someone wants to know
what \emph{The Age} said was the result of last night's game. One way to
learn what \emph{The Age} said would be to look up the result in
\emph{The Guardian}, and use one's background knowledge that they both
report the same (correct) result. That's a way to come to know what
\emph{The Age} said. But it's not a way to check what \emph{The Age}
said. It's not a way to check because had \emph{The Age} said anything
different, you wouldn't have known. That's a kind of insensitivity. It's
an insensitivity that's consistent with knowledge; one can know what a
newspaper says by knowing the truth and that the newspaper reports the
truth. This insensitivity is is removed by proper checking. So checking
aims for sensitivity that goes beyond belief, and beyond knowledge.
Given that checking, i.e., chasing this kind of sensitivity, is
rational, so is inquiring into what one knows.

\subsection{Rules}\label{sec-rulesinquiry}

It's hard to always be perfectly rational. Sometimes it makes sense to
not think too hard about things where getting the right answer would be
quite literally more trouble than it's worth. I'll have much more to say
about this point in Chapter~\ref{sec-ties}, where I make much of this
insight from Frank Knight.

\begin{quote}
It is evident that the rational thing to do is to be irrational, where
deliberation and estimation cost more than they are worth.
(\citeproc{ref-Knight1921}{Knight, 1921: 67fn1})
\end{quote}

Knight is interested in the case where the rational thing to do is not
inquire when inquiry would have minimal gains. There is another case
that is more relevant here. Sometimes it is worth having a simple rule
that says \emph{Always inquire in these situations}, rather than having
a meta-inquiry into whether inquiry is worthwhile right now. To make
this a little less abstract, it might be worthwhile always checking that
the door is locked when one closes it, even if one frequently knows that
one has just locked the door. As Hawthorne \& Srinivasan
(\citeproc{ref-HawthorneSrinivasan2013}{2013}) point out, given the
non-luminosity of evidence and knowledge, a simple rule like this might
do better any other realistic rule.

Often following rules about when to inquire will be part of one's
professional responsibilities. I presented an example like this in
chapter 7 of \emph{Normative Externalism} - an inspector who is sent to
do a random check of an establishment he had checked just a few days
before. He knows everything is working well; he just checked it! But
it's his job to check, and it's good to have random spot checks on top
of regular checks, so it's good to run this inquiry. That's true even
though the inspector knows how it will end.

\subsection{Understanding}\label{sec-understandinginquiry}

There is a famous puzzle about moral testimony. Something seems off
about a person who simply believes moral principles on the basis of
testimony, even from a trusted testifier. It's odd to convert to
vegetarianism simply because someone you trust says that's what morality
requires. There is also a famous answer to this puzzle, due to Alison
Hills (\citeproc{ref-Hills2009}{2009}). (There are other answers too,
including ones that deny the puzzle exists. To avoid going down too many
rabbit holes, I'm going to assume for now the answer Hills gives is
correct.) Hills says that moral testimony can give us moral knowledge,
like any kind of testimony can provide knowledge, but it can't provide
understanding. What's weird about the person who becomes a vegetarian on
testimonial grounds alone is that the can't explain their actions, since
they don't know why they are acting this way.

Beyond moral testimony, there seem to be many everyday cases of
knowledge without understanding. One can know that Franz Ferdinand was
assassinated in Sarajevo on June 28, 1914, without knowing why that
happened. Or, indeed, one can know why one part of that is true, e.g.,
why it was that \emph{Franz Ferdinand} was assassinated in Sarajevo on
June 28, 1914, without knowing why he was assassinated \emph{in
Sarajevo}; or why he was assassinated \emph{on June 28, 1914}. Given
those facts, it is possible to seek understanding of something that one
already knows.

In many cases, but not all, the search for understanding will look like
a somewhat different inquiry to the search for knowledge. If one wants
to know why Franz Ferdinand was assassinated \emph{in Sarajevo}, one
will inquire into the role that city plays in the history of relations
between Austria-Hungary and Serbia. That will be a different kind of
inquiry to determining whether the assassination really happened. But in
the moral case things aren't this clear. Imagine again our person who
hears from a trusted source that meat eating is wrong, but doesn't
understand why this is so. They should do some moral inquiry. The
inquiry will look, as far as I can see, very similar to the inquiry they
will do in case they are working out whether meat eating is wrong. That
is, it will look just like an inquiry into whether meat eating is wrong.

I think the best way to systematise things here is to take appearances
at face value. Even once one is convinced meat eating is in fact wrong,
if one doesn't know why it is, one will continue to inquire into the
morality of meat eating. This inquiry is justified by the aim of coming
to understand the wrongness of meat eating.

\subsection{Defragmentation}\label{sec-defraginquiry}

Recall Professor Paresseux from Section~\ref{sec-andintro}. He's told
that the visiting speaker this week is his old graduate school colleague
Professor Assidue. But he puts no effort into remembering this fact, and
it slips from the front of his mind. The talk is approaching, and
Paresseux wonders to himself, who's talking to us this afternoon? So he
Googles the department talk schedule, sees that it is Assidue, and then
says to himself ``Ah, I knew that, I saw the email the other day.''

It is very hard to fit the category of information that has `slipped
one's mind' into familiar epistemological categories.\footnote{The point
  here is related to the discussion in
  Section~\ref{sec-neutrality-contextualism} about how sometimes
  \emph{knows} seems to just mean \emph{possesses the information}.} I
think we should say that Paresseux is correct, and he did indeed know
the answer to his inquiry before he started looking. After all, he could
have retrieved the information by simply thinking hard about what had
happened this week. The best explanation for why that's possible is that
he did still know that Professor Assidue would be the speaker. But I
also think it made sense for him to conduct an inquiry into this thing
that he knew. It's much easier to Google something than to trawl one's
memory for the answer. More reliable too. So this looks like a sensible
inquiry for him to have conducted.

Following Andy Egan (\citeproc{ref-Egan2008}{2008}), I treat this as a
case where Paresseux's mind is `fragmented', in the sense of Lewis
(\citeproc{ref-Lewis1982c}{1982}) and Stalnaker
(\citeproc{ref-Stalnaker1984}{1984}). There is a part that contains the
information about who the speaker is. That part isn't at the front of
his attention, so he doesn't act on it. Still it is a part of him; he
knows that stuff. Still, it is better to conduct an inquiry, i.e., a
Google search, than to rely on this knowledge. So it is rational to
inquire into something one knows.

\subsection{Public Reason}\label{sec-rawlsinquiry}

One unfortunate position an inquirer can find themselves in is knowing
something is true, even understanding why it is true, and being unable
to convince anyone of their result. At this point one needs more
reasons, but where to find them? Often, the way to find them will be to
do what anyone else would do if they were trying to find out if the
thing itself were true. Here are two such examples, drawn from rather
different parts of philosophy.

Michael Strevens (\citeproc{ref-Strevens2020}{2020}) argues that the
effectiveness of science in the last 350 years is partially due to the
fact that scientists have adopted an ``iron rule'': \emph{only empirical
evidence counts}. There are any number of ways one might come to
rationally believe a scientific theory other than evidence. It might
follow from broadly metaphysical principles one holds (at least in the
early modern sense of metaphysical), it might be more elegant than any
other theory, it might promise to unify seemingly disparate phenomena.
But if you want to convince the scientific community, meaning convince
both the collective community and most of the scientists who make it up,
you need data. So you go looking for data, even for theories you know
are true on non-empirical grounds. Strevens thinks this is individually
irrational, but collectively for the best. It's irrational for any one
person to have just one way to come to believe things. But by
incentivising the search for data in this way, we've collectively
created an institution that has taken the measure of the world in ways
previously unimaginable. There is something else valuable about data -
it's available, at least in principle, to everyone. So even if you can't
recreate my metaphysical intuitions, you can rerun my experiments. The
iron rule doesn't just lead to more measurements being taken, it imposes
a kind of public reason constraint on science. Only evidence that
everyone can accept as evidence, and indeed that they could (at least in
theory) create for themselves, counts.

This way of putting the point should remind us of an important strand in
contemporary political philosophy, namely that political rules should
satisfy a \emph{public reason} constraint. As Jonathan Quong puts it

\begin{quote}
Public reason requires that the moral or political rules that regulate
our common life be, in some sense, justifiable or acceptable to all
those persons over whom the rules purport to have authority.
(\citeproc{ref-Quong2017}{Quong, 2018})
\end{quote}

Now as a matter of fact, we haven't had as much uptake of this meta-rule
in politics as in science. But we can imagine a society where there is,
in practice, a kind of public reason constraint. If you want your
favorite rule to be part of the regulation of society, you have to come
up with a justification of it that satisfies this constraint. In such a
society, there will be people who have idiosyncratic ideas for rules
that would be good rules for the community, ideas that they don't have
public justifications for. In practice, the vast majority of these ideas
will be bad ones. But some of them will not be. Indeed, a handful will
even know that their ideas are good. Still, if this knowledge comes via
idiosyncratic sources, they will need to come up with more public
reasons if they want to see their rule implemented. As I suggested in
the previous subsection, the way to find reasons for a moral claim is
generally to inquire into whether that claim is true. Or, at least, to
act like that's what one is doing.

\subsection{Evidence Gathering}\label{sec-evgather}

In Section~\ref{sec-cutelim} I'm going to argue that having \emph{p} as
part of one's evidence might license inductive inferences that are not
licensed by a smaller evidence set that doesn't include \emph{p}, even
if one knows \emph{p} on the basis of that smaller set. If that's right,
evidence gathering could be epistemically useful even if one already
knows the evidence to be gathered.

\subsection{Possible Responses}\label{sec-friedmaninquiry}

If this was a paper dedicated to proving that it is rational to inquire
into what one knows, at this stage I'd have to show that a philosopher
who denies that is ever rational has no good story to tell about these
six cases. That would be a lot to show, since actually there is plenty
that such a philosopher could say. They could deny that the inquiries
are indeed rational. They could deny that the inquirers in question
really do know the thing they are inquiring into, perhaps using IRT to
back up that denial. They could deny that these are real inquiries, as
opposed to some kind of ersatz inquiry. Or they could deny that this is
really an inquiry into the very thing known, as opposed to an inquiry
into some related proposition, like what the causal history of that
thing was. They wouldn't even have to choose between these four; they
could mix-and-match to deal with the putative counterexamples.

At the end of the day, I don't think these responses will cover all the
cases. But it would be a massive digression to defend that claim, and it
isn't necessary for what's going to happen in the rest of this chapter.
All I need is that there are people who very much look like they are
conducting rational, genuine inquiries into things they already know. If
there is a subtle way of explaining away that appearance, that won't
matter for the story that's to come, since such subtleties will end up
being good news for my side of the debate about IRT. The worry we're
building up to is that IRT has no good explanation of what's happening
in cases where someone seems to rationally, genuinely inquire into
something they already know. If there are in fact no such cases, that
can't be a problem!

One reason for thinking that some of these cases will work is that there
is a fairly general recipe for constructing the cases. It's due to Elise
Woodard (\citeproc{ref-Woodard2021}{2020}) and (independently) Arienne
Falbo (\citeproc{ref-Falbo2021}{2021}). Start with the following two
assumptions. First, inquiry is not just about collecting knowledge, but
generally about improving one's epistemic position.\footnote{When I say
  inquiry is about improving one's epistemic position, I don't mean that
  that's how inquirers represent what they are doing to themselves. That
  would be to over-intellectualise things. Rather, inquiry is about
  doing things that are, as a matter of fact, things that improve one's
  epistemic position. One can be improving one's epistemic position even
  if one self-represents one's actions in a more mundane way, e.g., as
  looking up when the coffee shop opens.} Second, given fallibilism, one
can know \emph{p} but have a sub-optimal epistemic position. So one can
know \emph{p}, but (rationally) want to improve one's epistemic position
with respect to \emph{p}. If one acts to address that want, one will be
inquiring into what one knows, and doing so rationally. Given IRT you
should worry about whether every step in the last few sentences really
does follow from the ones before it. But I suspect the general picture
is right, especially, as Melchior (\citeproc{ref-Melchior2019}{2019})
stresses, in checks aimed at increasing sensitivity.

Looking ahead a little, the primary aim of the rest of the chapter will
be to defuse some potential counterexamples to IRT that involve someone
rationally inquiring, especially checking, what they know. My response
will be disjunctive. Either inquiry solely aims at knowledge, or it does
not. If inquiry does solely aim at knowledge, appearances in this cases
are deceiving, and the inquiry is not in fact rational. If, as I think,
inquiry does not solely aim at knowledge, then the cases are not in fact
counterexamples to IRT.

\section{Using Knowledge in Inquiry}\label{sec-irtinquiry}

Sometimes an inquirer has reasons to deliberately hobble their own
inquiry. They have reasons to conduct an inquiry with one hand tied
behind their back. Perhaps those reasons come from the social norms of
the enterprise they are engaged in, as Strevens suggests. Perhaps those
reasons come from the fact that they are sensitivity chasing, as
Melchior suggests, and only a restricted inquiry will increase
sensitivity. Perhaps those reasons come from the fact that they are
trying to follow rules, and the rules do not allow certain kinds of
tools to be used. The unifying theme is that sometimes the inquirer
wants not just to run an inquiry, but to run it in a particular way.

The core principle in my version of IRT is that someone who uses what
they know in inquiry is immune to criticism on the grounds that what
they are doing is epistemically risky. Equivalently, they are immune to
criticism on the grounds that their premises might be false. That's
compatible with saying that someone can know \emph{p}, and be properly
criticised for using \emph{p} in inquiry. I motivated that restriction
in Section~\ref{sec-theoreticalknowledge} by looking at people whose use
of \emph{p} in inquiry can be criticised on relevance grounds. In this
chapter we see several more reasons. Someone who has reasons to perform
a restricted inquiry, especially someone whose aims can only be realised
by conducting a properly restricted inquiry, can be criticised for
overstepping those restrictions. That's fine, and totally consistent
with IRT, as long as we pay attention not just to whether someone is
being criticised, but why they are being criticised.

It isn't just my idiosyncratic version of IRT that escapes this
criticism. Jeremy Fantl and Matthew McGrath defend a version of IRT that
uses the following principle.

\begin{quote}
When you know a proposition \emph{p}, no weaknesses in your epistemic
position with respect to \emph{p}---no weaknesses, that is, in your
standing on any truth-relevant dimension with respect to
\emph{p}---stand in the way of \emph{p} justifying you in having further
beliefs. (\citeproc{ref-FantlMcGrath2009}{Fantl \& McGrath, 2009: 64})
\end{quote}

I'm going to come back in Section~\ref{sec-weakness} to why I don't
quite think that's right. But my disagreement turns on a fairly small
technical point; I'm following Fantl and McGrath's lead much more than
I'm diverging from them. These examples of properly restricted inquiry
show how they too can accept rational inquiry into what one already
knows.

Consider a person who is sensitivity chasing; they know \emph{p} but
want to have a more sensitive belief that \emph{p}. So they conduct an
inquiry into \emph{p}, and reason to themselves \emph{p}, therefore
\emph{p}. This closes the inquiry. Something has gone wrong. It isn't
bad reasoning; can't go wrong with identity. And it isn't that they use
something they know as a premise; anything one knows can be used as a
premise. It's that they had an aim that could only be met by a
restricted inquiry, and they violated those restrictions. That's the
incoherence here.

There is a way to read Fantl and McGrath's principle so that this case
is a problem for them, but I don't think it's the right reading. The
sensitivity of one's belief is, in their terms, part of the strength of
one's epistemic position. So if one's belief was more sensitive, one
wouldn't have a reason to be chasing sensitivity. So in this case, you
might think it's weakness of epistemic position that's relevant; the
weakness of epistemic position explains why the inquiry is being
conducted in the first place. But I don't think that's fair. The
principle only talks about how inquiry should be conducted, not about
whether the inquiry should be conducted. So Fantl and McGrath could say,
and I think this is the right way to read what they do say, that
knowledge is compatible with the weakness in one's epistemic position
explaining why an inquiry is in order. It's just that knowledge is not
compatible with weakness of epistemic position preventing the knowledge
being used once the inquiry starts.

\section{Independence}\label{sec-independence}

These reflections on the nature of inquiry help tidy up a loose end from
\emph{Normative Externalism} (\citeproc{ref-Weatherson2019}{Weatherson,
2019}). In that book I argued against David Christensen's Independence
principle, but I didn't offer a fully satisfactory explanation for why
the principle should seem plausible. Here's the principle in question.

\begin{quote}
\textbf{Independence}: In evaluating the epistemic credentials of
another's expressed belief about P, in order to determine how (or
whether) to modify my own belief about P, I should do so in a way that
doesn't rely on the reasoning behind my initial belief about P.
(\citeproc{ref-Christensen2011}{Christensen, 2011: 1--2}).
\end{quote}

This is expressly stated as a principle about disagreement, but it is
meant to apply to any kind of higher-order evidence. (This is made clear
in ``Formulating Independence''
(\citeproc{ref-Christensen2019}{Christensen, 2019}), which also includes
some new thoughts about how Christensen now thinks the principle should
be stated.) I argued that this couldn't be right in general; it gives
the wrong results in clear cases, and leads to regresses. Still, it
seems plausible that something like this should be right. In
\emph{Normative Externalism} I hinted at an inquiry-theoretic proposal
about what that similar truth might be. (See, for example, the response
to Littlejohn (\citeproc{ref-Littlejohn2015}{2018}), at the top of page
178.) But I never really spelled it out. Here's what I now think the
right thing to say is. \footnote{The picture I'm about to give is really
  similar to the one laid out by Andy Egan
  (\citeproc{ref-Egan2008}{2008}). We're interested in different kinds
  of cases, but the idea that a cognitive system might work best by
  allowing one part to check on another using just the evidence the
  first part has endorsed is one I'm just taking from him. If I'd seen
  this connection when writing \emph{Normative Externalism} I would have
  connected it to the discussion of Madisonian moral psychology in part
  I of that book.}

Peer disagreement, or really any other kind of higher order evidence,
gives a thinker a reason to conduct an inquiry into whether their
earlier thinking was correct. Further, it gives them reason to conduct
an inquiry that is restricted in a particular way. The restriction is
that they should not rely on the reasoning from their earlier thinking.
Putting those two things together, we get that disagreement about
\emph{p} gives someone who believes \emph{p} reason to inquire into
\emph{p} using a different approach, any different approach, from what
they previously used.

Once we've got a principle about reasons, we could try formulating this
as a defeasible rule. It's plausible that one should adopt the
defeasible rule of conducting such an inquiry whenever one sees a
disagreement, or some other kind of potentially defeating higher-order
evidence. As long as one builds enough into the defeasibility clause,
such a rule won't be subject to the counterexamples I described, or the
ones that have caused Christensen (\citeproc{ref-Christensen2019}{2019})
to have second thoughts about the right formulation of the rule. After
all, every counterexample will naturally fall into the defeasibility
clause.

Such a rule could be justified by the observation that it will probably
be beneficial in the long run for people like us to adopt it. Double
checking isn't that hard, and can be very useful. Getting stuck in a bad
epistemic picture can have devastating consequences; it's good to step
back from time to time to look if that's happening to us. Disagreements
with peers are a natural trigger for that kind of inquiry. Those same
benefits can explain why disagreement, or other kinds of higher-order
evidence, give us reason to double check.

But why should one conduct a restricted inquiry here? Given the stakes,
we're trying to work out whether we've got ourselves into a bad
epistemic state, shouldn't we through everything we have at the problem?
That would be bad, since Independence expressly bars the thinker from
using some of the tools at their disposal. It requires them to not do
the same kind of inquiry they did before, which presumably was the one
they thought best suited to the problem. That's a big restriction, and
needs some justification. I can offer two kinds of justification, not
entirely distinct.

The point of having a rule like this, a rule like \emph{Double-check
your reasoning when a peer disagrees}, is to prevent us falling into
epistemic states that are local but not global equilibria. The states
we're worried about are ones where any small change will make the
epistemic state worse, but large changes will make things better.
Picturesquely, we've reached the top of a small hill when we want to
climb a mountain. We should be somewhere higher, but any step will be
downhill. It's good to not get stuck in places like this, and nudges
from friends are a way out.

If we want to check whether we're in such a bad situation, we want a
test that is sensitive to whether we are. That is, we want a test that
would say something different if we were in that situation to what it
would say if we were doing well. (This is Melchior's point about the aim
of tests.) Just conducting the same inquiry we previously conducted will
typically not be sensitive in this way. Or, more precisely, it will be
sensitive to something like performance errors, but not competence
errors. We need something more sensitive if the aim is to avoid getting
stuck in local equilibria, and that requires setting aside the work
we've previously done.

One of the reasons that local equilibria can be sticky is that we know
our way around them well. We know all the ways in which one part of the
picture we have supports the other parts. We typically don't know how to
think about other pictures so clearly. We don't know, don't see, the
ways in which other pictures might `hang together' as well as ours does.
We are inevitably going to be biased towards our own ways of thinking.
So it's worthwhile to try to level the playing field, by looking at how
things would seem if we didn't have our own distinctive way of thinking.

None of this is to take back anything I said in \emph{Normative
Externalism}. Disagreement with a peer known to have the same evidence
does not give someone a reason to reject a well-formed belief. It gives
them a reason to double-check that belief. As I've been stressing all
chapter, one can double-check one's beliefs, and even one's knowledge.
That is what should happen here.

Finally, thinking of disagreement as providing a reason to double check
provides a nice explanation of one of the harder examples in
\emph{Normative Externalism}, the case of Efrosyni on page 222. She does
a calculation, then double checks it by a different technique, then
hears that a peer disagrees. What should she do now? I think typically
she should do nothing. The disagreement gives her a reason to double
check each calculation she did, but she's already carried out that
double check. This is, I think, the intuitively right result. If someone
has already double checked their work, they should infer that someone
who disagrees is wrong. Perhaps in some rare case they could get reason
to double check the `combined' inquiry, consisting of the initial
inquiry plus the double check. But that's rare; usually they should just
point out their work.

With this picture of the relationship between knowledge, inquiry, and
checking in place, it's time (at last) to return to potential
counterexamples to IRT.

\section{Double Checking}\label{sec-doublecheck}

In her 2008 paper ``Subject-Sensitive Invariantism and the Knowledge
Norm for Practical Reasoning, Jessica Brown
(\citeproc{ref-Brown2008}{2008}) runs through a bunch of cases where,
she says, intuitively someone knows a proposition but they cannot use it
in practical deliberation. The first of these cases has been frequently
cited as a problem for the kind of view I'm defending.

\begin{quote}
A student is spending the day shadowing a surgeon. In the morning he
observes her in clinic examining patient A who has a diseased left
kidney. The decision is taken to remove it that afternoon. Later, the
student observes the surgeon in theatre where patient A is lying
anaesthetised on the operating table. The operation hasn't started as
the surgeon is consulting the patient's notes. The student is puzzled
and asks one of the nurses what's going on:

\emph{Student}: I don't understand. Why is she looking at the patient's
records? She was in clinic with the patient this morning. Doesn't she
even know which kidney it is?\\
\emph{Nurse}: Of course, she knows which kidney it is. But, imagine what
it would be like if she removed the wrong kidney. She shouldn't operate
before checking the patient's records.
\end{quote}

I think there are pretty good arguments that checking the chart is the
right thing to do even if the surgeon knows which kidney is diseased, so
this case isn't a problem for the views about knowledge and action that
I'm defending.

In medical contexts, intuitions about appropriate action very rarely
track expected utility maximisation.\footnote{Jonathan Ichikawa
  (\citeproc{ref-Ichikawa2017}{2017: 152ff}) makes this point well in
  responding to Brown.} This is one reason why it is so easy to come up
with medical counterexamples to act utilitarianism for intro ethics
classes. Instead, intuitions about appropriate actions here are more
likely to track with rule utilitarianism. The rule \emph{Double-check
the notes before removing an organ} seems like it will on average
maximise utility, even if it would not help in this case.

To connect this to the discussion in Section~\ref{sec-settling}, the
surgeon here is doing a bit of mostly harmless sensitivity chasing.
Before checking the notes, their belief that the left kidney was
diseased was not sensitive to the possibility that they'd misremembered
the morning meeting; after checking the notes it is. Since busy surgeons
do sometimes misremember meetings some hours earlier. So this is a
reasonable bit of sensitivity for the surgeon to chase, and for the
rule-makers to require be chased.

These considerations don't just defend IRT against the example, they
show how IRT can be used to resolve a puzzle about a related case.
Continue Brown's story by imagining that every time the surgeon raises
the scalpel to make the first incision, they instead go back to look at
the notes to check they are removing the correct kidney. Now we have the
following conversation.

\begin{quote}
\emph{Student}: I don't understand. Why is she looking at the patient's
records for the seventeenth time? She just looked at the notes each
minute for the last sixteen minutes; she knows which kidney it is.\\
\emph{Nurse}: Of course, she knows which kidney it is. But, imagine what
it would be like if she removed the wrong kidney. She shouldn't operate
before checking the patient's records.
\end{quote}

This is a really bad defence of the surgeon's actions. We are owed a
story about why it is a bad defence. My story starts with the point that
Student is right to ask why she is inquiring into something she knows.
While as we've seen there are cases where that is appropriate, these
cases are somewhat unusual. It's a reasonable default assumption that
inquiry into something one knows is mistaken. That assumption is only
defeated if there is some other worthwhile epistemic good that can be
attained. In this case, there isn't, since sensitivity to whether one
misread the chart the last sixteen times isn't a worthwhile kind of
sensitivity to get.

In general, anyone who wants to separate out knowledge from action, and
do so on account of the fact that sometimes we double check things we
know, owes a story about why we don't also triple-check,
quadruple-check, and so on. I suspect such a story won't be easy to
tell.

Brown has another example that hasn't attracted nearly as much attention
in the literature. This is unfortunate since I think it's a more
pressing problem for the view Brown is attacking.

\begin{quote}
A husband is berating his friend for not telling him that his wife has
been having an affair even though the friend has known of the affair for
weeks.

\emph{Husband}: Why didn't you say she was having an affair? You've
known for weeks.\\
\emph{Friend}: Ok, I admit I knew, but it wouldn't have been right for
me to say anything before I was absolutely sure. I knew the damage it
would cause to your marriage.
\end{quote}

In this case, the tricks I was deploying in Section~\ref{sec-settling}
don't seem to help. There is no further epistemic good that Friend
obtains by waiting further.

That said, my intuition here is that Friend's speech is just incoherent.
Or, at least, it is incoherent if we take the final statement at face
value. My best guess as to what's going on here is that we really
shouldn't do that; Friend didn't really know about the
affair.\footnote{The particular versions of IRT Brown was responding to
  in the 2008 paper were heavily motivated by intuitions about cases.
  Brown argues, quite correctly I think, that those theories aren't
  entitled to appeal to arguments that the intuitions which go against
  them are mistaken. After all, if IRT is just motivated by intuitions,
  the argument that knowledge is not sensitive to interests is just as
  good an argument against those intuitions as the arguments that IRT
  defenders can make about this example. Happily, my version of IRT is
  not motivated just by intuitions about cases, so I don't have to worry
  about this dialectical point.}

There are two things that might be going on in this case. My best guess
is that the explanation for why Friend's statement seems so natural
relies on both of them.

First, we do sometimes use `know' in a purely informational sense. We
saw this in Section~\ref{sec-defraginquiry} with Paresseux's claim that
he \emph{knew} Assidue was visiting. He possessed the information,
though little more than that. Still, in context this can be enough to
ascribe knowledge.

Second, we can be very flexible about past-tense knowledge claims when
we, the current speakers, know how things turned out. After our sports
team loses a game they should have won, we might say ``I had a bad
feeling about today, I knew we were going to mess it up.'' In most cases
it would be weird to say the speaker even thought their team would mess
up, let alone believed it. (Why didn't they bet on the opposition if
they thought the result was a foregone conclusion?) But even if they did
believe it, we really don't think \emph{bad feelings} are appropriate
grounds for knowledge. And yet, the speakers claim that they knew the
team would mess up sounds fine.

Is Friend's statement like Paresseux's knowing (i.e., possessing the
information) that Assidue would be visiting, or the sports fan's knowing
(i.e., having an accurate premonition) that their team would mess up? My
guess is that it's a bit of both. Either way, the Friend didn't know, in
the sense of \emph{know} relevant to epistemology, about the affair.

The general methodological point is that these last two senses of
knowledge do seem different to what we typically talk about in
epistemology. It's possible, as I noted in
Section~\ref{sec-neutrality-contextualism}, that considering the
information-possession sense of knowledge is important for thinking
through whether any kind of contextualism is true. I don't think the
`bad feeling' cases are relevant to anything in epistemology, save for
cases where we might need to explain away intuitions that they are
involved in. Maybe that's what's happening in Brown's second case.

\section{The Need to Inquire}\label{sec-need}

So far I've mostly talked about inquiries that a person is actually
conducting. But we should also think about the inquiries that they
should conduct. Consider the following two abstractly described
possibilities.

A person believes \emph{p} for good reasons, and it is true, and there
are no weird things happening that characterise typical gaps between
rational true belief and knowledge. There is some action 𝜑 they are
considering that will have mildly good consequences if \emph{p}, and
absolutely catastrophic consequences if ¬\emph{p}. One of the
alternatives to 𝜑 is first checking whether \emph{p}, which would be
trivial, and then doing 𝜑 iff \emph{p}. We've seen lots of these cases
before, but here's the new twist. The person absolutely does not care
about the catastrophic consequences. They will all fall on people the
person could not care less about. So they are planning to simply do 𝜑,
for the good consequences. Since \emph{p} is true, nothing bad will
happen. Still, it seems something has gone wrong. We want to say that
they've been reckless, that they've taken an immoral risk. But it isn't
risky to do something that you know won't have bad consequences. So they
do not know that \emph{p}, and for similar reasons to why Anisa doesn't
know that \emph{p}. Yet the version of IRT that I've given so far
doesn't say that they don't know that \emph{p}.

The second case has the same initial structure as the first. The person
believes \emph{p} for good reasons, it's true, and there is no funny
business going on - no fake barns or the like blocking knowledge. They
are thinking about doing 𝜑. They know that if \emph{p} is true, 𝜑 will
have a small benefit. They also know that it would be completely trivial
to verify whether \emph{p} is true. They also in some sense know that if
they do 𝜑, and \emph{p} is false, it will be absolutely catastrophic.
And they care about the catastrophe. But they've sort of forgotten this
fact about 𝜑. It's not that it has totally vanished from their mind. But
they aren't attending to it, and it doesn't form any part of their
deliberation when thinking about 𝜑. So they do 𝜑, nothing bad happens,
and later when someone asks them whether they were worried about the
possible catastrophe, they are shocked that they would do something so
reckless. They are shocked, that is, that they forgot that it was
important to confirm whether \emph{p} was true before doing 𝜑. It feels,
from the inside, like they got away with taking a terrible risk. But if
they knew \emph{p}, it should not seem like a risk, it should seem like
rational action. (Just like they would think doing 𝜑 after checking
whether \emph{p} was rational action.) So this too should be a case
where we say knowledge fails for practical reasons. (I'm going to come
back to a version of this case in Section~\ref{sec-atomism}, where it
will be useful for highlighting one of the few points where I disagree
with the theory that Jeremy Fantl and Matthew McGrath
(\citeproc{ref-FantlMcGrath2002}{2002},
\citeproc{ref-FantlMcGrath2009}{2009}) endorse.)

The natural thing to say here is that in each case, the person should
conduct an inquiry. They should check whether \emph{p} is true. In that
inquiry, they shouldn't take \emph{p} for granted. They shouldn't take
it for granted for a very particular reason, because it might be false.
If they knew \emph{p}, they could take it for granted, or, at least, if
they couldn't, it would be for some reason other than that \emph{p}
might be false. So they don't know that \emph{p}.

What these two types of case show is that knowledge is not just
sensitive to what one is actually inquiring into, it is also sensitive
to what one should be inquiring into. If one should inquire into Q, and
were one to inquire into Q, one shouldn't take \emph{p} for granted
because it might be false, one doesn't know \emph{p}.~

This is a kind of moral encroachment in the sense of Basu \& Schroeder
(\citeproc{ref-BasuSchroeder2019}{2019}). What one knows might be
sensitive to one's moral obligations in inquiry. Imagine two people both
take \emph{p} for granted in making a decision that affects other
people. This is mostly fine because \emph{p} is true, and they had good
reasons to take it for granted. Still, there was some risk to others,
and they could have checked whether \emph{p} was actually true before
acting, but in each case they had other things they would rather be
doing than checking \emph{p}. What differs between the two people is
what they would rather be doing. The first could have checked, but it
would have taken them away from a rescue operation in progress; the
second could have checked, but it would have taken them away from their
social media feed. If the theory I've developed so far is correct, then
the first knows that \emph{p}, and the second does not, and the
difference comes down to the differing moral importance of contributing
to rescue operations and social media.

It's worth recalling here that the methodology I'm using in this book is
perhaps a little different to a common methodology in this area. I don't
think that if you fill out the two cases from the last paragraph in full
detail, it will be intuitively obvious that one person knows and the
other doesn't, and that's evidence for IRT. Rather, I think that it's
plausible that one isn't being reckless by acting on what one knows, and
this principle, combined with anti-sceptical principles and judgments
about which acts are indeed reckless, leads to IRT. As always, these
cases allow for four broad classes of response: the sceptic who denies
there is knowledge even in the low-stakes case; the epistemicist who
denies the intuitions about which actions are reckless; the orthodox
theorist who says that acting on what one knows can be reckless; and the
pragmatist, who accepts both the intuitions about which acts are
reckless and how knowledge connects to recklessness, and infers that
knowledge is sensitive to pragmatic, and in this case moral, factors.

\section{Multiple Inquiries}\label{sec-multiple}

IRT says that what one knows is a function of what one is inquiring
into. It would be very convenient if there was a position in the logical
form of knowledge ascriptions for inquiries. That is, it would be very
convenient if the logical form of \emph{S knows that p} was something
like \emph{Ktspi}, where \emph{t} is the time, \emph{s} is the knower,
\emph{p} is what's known, and \emph{i} is the inquiry it is known in.
Then we could say that one condition on such a knowledge claim being
true is that at \emph{t}, \emph{s} can properly use \emph{p} as a
starting point in inquiry \emph{i}.\footnote{More precisely, as I said
  in Section~\ref{sec-theoreticalknowledge}, if they use \emph{p} in
  \emph{i}, that won't be subject to criticism on the grounds that
  \emph{p} might be false. I'll use the more informal version in the
  text in what follows to increase readability.} Unfortunately for IRT,
that's not the logical form of knowledge ascriptions. The \emph{t},
\emph{s}, and \emph{p} are there all right, but not the \emph{i}.
Fortunately for IRT, the logical form does have reference to a knower,
that \emph{s}. Since knowers undertake inquiries, we can bring in the
inquiries via the knower. All knowledge is inquiry-relative, we say, and
it is relative to the inquiries of the person knowledge is being
ascribed to.

If every person was, at each time, undertaking precisely one inquiry,
everything would fall into place very nicely. Given \emph{t} and
\emph{s}, we could guarantee the unique existence of an \emph{i}, and it
would be as if there was an \emph{i} in the logical form, as IRT would
like. Unfortunately, that's not close to being true. Some people at some
times are making no inquiries, e.g., when they are asleep. And some
people at some times are making many inquiries. The former case is no
problem for IRT. If the person is making no inquiries, then what they
know is determined by `traditional' factors, such as what they believe,
whether those beliefs are true, grounded in the evidence, safe, and so
on. The case where someone is engaged in multiple inquiries is a little
harder.

The view I'll defend is that the person knows \emph{p} only if \emph{p}
can properly be used as a starting point in all the inquiries the person
is engaged in. This has a surprising, and not entirely welcome, side
effect. It means that some people don't know \emph{p}, and hence can't
use \emph{p} in an inquiry \emph{i}, even though they could use \emph{p}
as a starting point to \emph{i} if \emph{i} were the only inquiry they
were engaged in. This is a somewhat more sceptical result than I like,
but I suspect it's the best choice out of a bad lot. The only other
options I can see are to either try to find ways to get \emph{i} back
into the logical form of knowledge ascriptions, or to adopt a novel form
of relativism that says knowledge claims are true or false relative to
inquiries, or to say that the person conducting multiple inquiries is
fragmented, and each of the fragments has their own knowledge. None of
these moves strikes me as remotely plausible, and so we're forced to
have some kind of view where we quantify over the inquiries a person is
engaged in.

In the rest of this section, I have three aims. First, to make what I've
said so far less abstract, by describing a case where someone has
multiple inquiries, and this matters in surprising ways. Second, to say
why it isn't great that IRT is forced to say that someone doesn't know
something that is otherwise usable in an inquiry they are engaged in.
Third, to say why this isn't a devastating result, even though it's not
exactly a happy one.

Our example of someone with multiple inquiries will be a historian
called Tori. She has been taught, like everyone else, that the Battle of
Hastings was in 1066. For most purposes she takes that to be one of the
fixed points in the historical record. But she's noticed some anomalies
in some the documents from around that time, anomalies that would be
explained by the battle being in 1067. She's seen enough documents to
know that the overwhelming likelihood is that these anomalies have some
simple explanation, like a transcription error. But in her spare time
over the last few years, she has been investigating off and on whether
the best explanation might be that everyone else has the date of the
battle wrong, and in fact it was in 1067.

If it is worth inquiring into the date of the Battle of Hastings, it is
not sensible to take the date of the battle as fixed. That would make
the inquiry very short. So if it's reasonable for Tori to conduct this
inquiry, then while she is conducting it, she does not know when the
Battle of Hastings took place.

If this inquiry into the date is something she has been working on in
her spare time for years, she has presumably had other jobs that did not
involve trying to overturn the historical record about one of the
central events in British history. In some of those jobs, it will have
been sensible to take as given when the Battle of Hastings, and hence
the Norman rule over Britain, took place. So there will be contexts,
ones where her primary focus is on an everyday question where one takes
for granted the common assumptions about British history, but she still
has as a background project this idea that maybe the Battle of Hastings
took place a year later, where IRT seems to get into trouble. It wants
to say that for the purposes of her everyday inquiries, Tori knows the
Battle of Hastings took place in 1066. After all, this is a true,
rational, belief, that is based in the right way in the facts, and which
is reasonably taken as a starting point for this very inquiry. That
looks like, relative to that inquiry, it is knowledge. But for the
purposes of finding out the best explanation of the anomalies, she does
not know when the battle took place, on pain of not being able to
rationally investigate one possible explanation.

My version of IRT says that knowledge is relative to inquirers, not to
inquiries, so I can't say that she knows the date relative to one
inquiry but not another. That's not great. In the everyday inquiry Tori
is exactly like someone who knows when the Battle of Hastings was it
what look like all the relevant features, and yet she doesn't know. How
can we explain away this anomaly?

The first thing to note is that even if Tori loses the knowledge that
the Battle of Hastings was in 1066, she keeps her voluminous evidence
that the Battle happened then. In most inquiries, anything she might
infer from a claim about the Battle's date, she can infer from that
evidence. So she'll still, on the whole, be able to draw the same
conclusions in other inquiries as if she kept that knowledge.

Usually there are two reasons for keeping the conclusions of one's
inquiries and not one's evidence. First, it helps with clutter avoidance
(\citeproc{ref-Harman1986}{Harman, 1986: 12}). If a knowledge of history
required knowing not just a bunch of things about what happened, when it
happened, and ideally why it happened, but also knowing how and where
one learned these facts, then even the most basic knowledge of history
would be beyond most of us. Second, it makes certain kind of inferences
much smoother to go through various steps rather than applying something
like cut-elimination and getting rid of the middle steps. That is, it's
easier for Tori to infer from some evidence that the Battle of Hastings
was in 1066, and then from that and some other evidence to draw further
conclusions, than it is to draw inferences directly from the underlying
evidence. But while both of these considerations are very powerful ones
in general, one would definitely not like to never store or rely on
intermediate conclusions in inquiry, they aren't nearly as powerful in
any specific case. If there's one step in an inquiry that one is unsure
of on other grounds, it's not a huge effort to retain one's evidence for
\emph{that step}, and replace inferences that rely on it with inferences
that rely on the underlying evidence.

The other thing to note is that we can explain Tori's behaviour in
inquiries without positing more knowledge to her than IRT allows. The
key thing is to replace the familiar \textbf{Knowledge Norm of
Assertion} with the slightly more complicated \textbf{Sufficient
Evidence Norm of Assertion}.

\begin{description}
\tightlist
\item[Knowledge Norm of Assertion]
One must: Assert \emph{p} only if one knows that \emph{p}.
\item[Sufficient Evidence Norm of Assertion]
One must: Assert \emph{p} only if one's evidence is sufficient for one's
audience to know that \emph{p}.
\end{description}

If one identifies evidence with knowledge, then it's hard to see any
space between these two. I don't quite endorse that identification for
reasons that I'll go over more in Chapter~\ref{sec-evidence}, but I
mention it here just to note that this need not be a radical revision.

If the norms do come apart, then the latter seems to play better with
IRT. Imagine that \emph{S} is talking to some people who are facing a
long-shot bet on whether \emph{p}. These people would not be best off,
in expectation, taking \emph{p} for granted. Unfortunately, \emph{S}
doesn't care about the welfare of these people, though for some reason
they do care about being a good informant and testifier. Further imagine
that \emph{S}'s evidence for \emph{p}, while strong, isn't quite strong
enough to justify the audience in taking this long-shot bet. Then it is
wrong for \emph{S} to simply say that \emph{p}.

The picture behind the Sufficient Evidence Norm of Assertion is that one
should say \emph{p} only if one's audience can take \emph{p} as a
starting point in inquiry. Sometimes one might violate this norm without
much blame attaching, as when it turns out one's audience has an
unexpectedly long-odds bet on \emph{p}. In normal cases, however, where
one knows at least something about one's audience, one should calibrate
one's assertions to the projects of one's audience.

This picture seems to get two possible cases where Tori is involved in a
group inquiry just right.

In the first (more normal) case, Tori is working with a group of people
who do not share her worry about the anomalies in the dating of the
Battle of Hastings. They think the date is a settled fact. In their
presence Tori can speak as if it is settled. After all, her evidence
suffices for her audience to know when the Battle was, given their lack
of interest in odd anomalies.

In the second (somewhat odder) case, Tori is working with a group of
people one of whom shares her concerns about these anomalies. In the
context of the other inquiry (i.e., not the inquiry into the date of the
Battle), Tori says ``The Battle of Hastings was in 1066.'' It would be
reasonable for the other person who shares her concerns about the
anomalies to conclude that Tori had satisfied herself that the anomalies
were just mistakes, and the Battle really was in 1066. That's because, I
say, the unqualified assertion would be improper unless Tori had
resolved these concerns to a standard that would be satisfying to the
two of them. This case is a bit odd, it does require the coincidental
presence of two people with unusual interests, but I think the
Sufficient Evidence Norm plus IRT gets them right.

Two final notes about this case.

First, I've crafted the Sufficient Evidence Norm to be the variation on
the Knowledge Norm that a defender of IRT should like. But one might
suspect the Knowledge Norm on independent grounds, e.g., because it gets
the cases in Maitra \& Weatherson
(\citeproc{ref-MaitraWeatherson2010}{2010}) wrong. I think the
Sufficient Evidence Norm should be tinkered with to handle those cases,
but I'm not exactly sure how this should go. Still, the tinkering
shouldn't undermine the way IRT handles these cases.

Second, there is a really interesting historical question around here.
Imagine you have a community that governs itself by the Sufficient
Evidence Norm. And then someone comes along and invents the scientific
journal, and all of a sudden it's possible to assert things with no
knowledge of what is at take for one's audience. How should one react,
especially given the usefulness of the scientific journalfor conducting
inquiries that are widely distributed over space and time?

A natural move would be to develop some new interest-invariant standards
for printed assertion, and hopefully make it clear to both writers
\emph{and readers} what these standards are.

Once upon a time I had hoped this book would include an argument that
the development of interest-invariant epistemology was just such a
reaction to the invention of the printing press and, somewhat later, to
the adoption of scientific journals as important conduits for sharing
information in distributed inquiries. I still think something like this
is arguably true, at least if we mean the development of
interest-invariant norms for what I called in Chapter~\ref{sec-overture}
`sub-optimal' epistemology. But defending this claim would require a
different book, and a writer with very different skills, to this one.

So I'll just leave this as a conjecture for future research. What most
philosophers call `traditional' epistemological views, i.e., fallibilism
plus interest-invariance, might just be a response to a relatively
recent technological innovation.

\bookmarksetup{startatroot}

\chapter{Ties}\label{sec-ties}

I have mentioned a couple of times that a natural version of IRT leads
to unpleasant closure failures. Adam Zweber
(\citeproc{ref-Zweber2016}{2016}) and, separately, Charity Anderson and
John Hawthorne (\citeproc{ref-AndersonHawthorne2019a}{2019a}), showed
that the following principle cannot be the only way interests enter into
our theory of knowledge.

\begin{description}
\tightlist
\item[Conditional Preferences]
If S knows that \emph{p}, and is trying to decide between X and Y, then
her preferences over X and Y are the same unconditionally as they are
conditional on \emph{p}.
\end{description}

They show if you add this principle and nothing else to a natural
interest-relative theory of knowledge, you get a theory where a person
can know \emph{p} ∧ \emph{q} but not know \emph{p}. Further, they argue
that the natural ways to modify IRT to avoid this result make the theory
implausibly sceptical. The various ways I've defended IRT over the years
are not vulnerable to the first objection, since I was always careful to
avoid this kind of closure failure. But they were vulnerable to the
second objection, since they did lead to some very sceptical results in
the cases that Zweber, and Anderson and Hawthorne, discuss. So the point
of this chapter is to describe a version of IRT that avoids their
challenge.

Surprisingly, the response will not involve making any particularly
dramatic changes to the theory of knowledge. What it will involve is
making a fairly dramatic change to the underlying decision theory.
That's one reason I'm spending a whole chapter on this objection; the
changes you need to make to respond to it run fairly deep. In
particular, they involve breaking the tight connection most theorists
assume holds between rational action and expected utility maximisation.
The other reason for spending so much time on these examples is that
thinking through them reveals a lot about the relationship between
reasons, rational action, and knowledge.

\section{An Example}\label{sec-frankielee}

Let's start with an example from a great thinker. It will require a
little exegesis, but that's not unusual when using classic texts.

~~Well Frankie Lee and Judas Priest ~\\
\strut ~~They were the best of friends ~\\
\strut ~~So when Frankie Lee needed money one day ~\\
\strut ~~Judas quickly pulled out a roll of tens ~\\
\strut ~~And placed them on the footstool ~\\
\strut ~~Just above the potted plain ~\\
\strut ~~Saying ``Take your pick, Frankie boy, ~\\
\strut ~~My loss will be your gain.'' ~\\
\strut ~~~``The Ballad of Frankie Lee and Judas Priest'', 1968.\\
\strut ~~~~~~~Lyrics from Dylan (\citeproc{ref-Dylan2016}{2016: 225})

On a common reading of this, Judas Priest isn't just asking Frankie Lee
how much money he wants to take, but which individual notes. Let's
simplify, and say that it is common ground that Frankie Lee should only
take \$10, so his choice is which note to take. This will be enough to
set up the puzzle.

Assume something else that isn't in the text, but which isn't an
implausible addition to the story. The world Frankie Lee and Judas
Priest live in is not completely free of counterfeit notes. It would be
bad for Frankie Lee to take a counterfeit note. It won't matter just how
common these notes are, or how bad it would be. The puzzle will be most
vivid if each of these are relatively small quantities. So there aren't
that many counterfeit notes in circulation, and the (expected)
disutility to Frankie Lee of having one of them is not great. There is
some chance that he will get in trouble, but the chance isn't high, and
the trouble isn't any worse than he's suffered before. Still, other
things exactly equal, Frankie Lee would prefer a genuine note to a
counterfeit one.

Now for some terminology to help us state the problem Frankie Lee is in.
Assume there are \emph{k} notes on the footstool. Call them
\emph{n}\textsubscript{1}, \ldots, \emph{n\textsubscript{k}}. Let
\emph{c\textsubscript{i}} be the proposition that note
\emph{n\textsubscript{i}} is counterfeit, and its negation
\emph{g\textsubscript{i}} be that it is genuine. Let \emph{g}, without a
subscript, be the conjunction
\emph{g}\textsubscript{1}~∧~\ldots~∧~\emph{g\textsubscript{n}}; i.e.,
the proposition that all the notes are genuine. Let
\emph{t\textsubscript{i}} be the act of taking note
\emph{n\textsubscript{i}}. Let \emph{U} be Frankie Lee's utility
function, and \emph{Cr} his credence function.

In our first version of the example, we'll make two more assumptions.
Apart from the issue of whether the note is real or counterfeit, Frankie
Lee is indifferent between the notes, so for some \emph{h}, \emph{l},
\emph{U}(\emph{t\textsubscript{i}}~\textbar~\emph{g\textsubscript{i}})~=~\emph{h}
and
\emph{U}(\emph{t\textsubscript{i}}~\textbar~\emph{c\textsubscript{i}})~=~\emph{l}
for all \emph{i}, with of course \emph{h}~\textgreater~\emph{l}. Frankie
Lee thinks each of the banknotes is equally likely to be genuine, so for
some \emph{p}, \emph{Cr}(\emph{g\textsubscript{i}})~=~\emph{p} for all
\emph{i}. (The probability of any of them being a counterfeit is
independent of the probability of any of the others being counterfeit.)

That's enough to get us three puzzles for the form of IRT that just uses
Conditional Preferences. I'm going to refer to this form of IRT a lot,
so let's give it the memorable moniker IRT-CP. That is, IRT-CP is what
you get by taking a standard theory of knowledge, adding Conditional
Preferences as a further constraint on knowledge, and stopping there. I
don't know that anyone endorses IRT-CP, but it's a good theory to have
on the table. It says a number of implausible things about Frankie Lee,
and the big challenge, as I see it, is to craft a version of IRT that
doesn't fall into the same traps.

First, Frankie Lee doesn't know of any note that it is genuine. As
things stand, Frankie is indifferent between \emph{t\textsubscript{i}}
and \emph{t\textsubscript{j}} for any \emph{i}, \emph{j}. But
conditional on \emph{g\textsubscript{i}}, Frankie prefers
\emph{t\textsubscript{i}} to \emph{t\textsubscript{j}}. Right now, the
expected utility of taking either \emph{i} or \emph{j} is
\emph{ph}~+~(1-\emph{p})\emph{l}. If Frankie Lee conditionalises on
\emph{g\textsubscript{i}}, then the utility of \emph{t\textsubscript{j}}
doesn't change, but the utility of \emph{t\textsubscript{i}} now becomes
\emph{h}, and that's higher than \emph{ph}~+~(1-\emph{p})\emph{l}. Since
IRT-CP says that one doesn't know \emph{p} if conditionalising on
\emph{p} changes one's preferences over pragmatically salient options,
and \emph{t\textsubscript{i}} and \emph{t\textsubscript{j}} are really
salient to Frankie Lee, it follows that he doesn't know
\emph{g\textsubscript{i}}. Since \emph{i} was arbitrary in this proof,
so he doesn't know of any of the notes that they are genuine. That's not
very intuitive, but worse is to follow.

Second, Frankie Lee does know that all the notes are genuine, although
he doesn't know of any note that it is genuine. Conditional on \emph{g},
Frankie Lee's preferences are the same as they are unconditionally. He
used to be indifferent between the notes; after conditionalising he is
still indifferent. So the one principle that IRT-CP adds to a standard
theory of knowledge does not rule out that Frankie Lee knows \emph{g}.
So he knows \emph{g}; but doesn't know any of its constituent conjuncts.
This is a very unappealing result.

To generate the third problem, we need to change the example a bit. Keep
that the probabilities of each note being genuine are equal and
independent. But this time assume that the notes are laid out in a line,
and Frankie Lee is at one end of that line. So to get a note that is
further away from him, he has to reach further. And this has an ever so
small disutility. Let \emph{r\textsubscript{i}} be the disutility of
reaching for note \emph{i}. And assume this value increases as \emph{i}
increases, but is always smaller than (1-\emph{p})(\emph{h}-\emph{l}).
That last quantity is important, because it is the difference between
the utility of taking an arbitrary note (with no penalty for the cost of
reaching for it), and the utility of taking a genuine banknote.

If all these assumptions are added, Frankie Lee knows one more thing. He
knows \emph{g}\textsubscript{1}. That's because as things stand, he
prefers \emph{t}\textsubscript{1} to the other options. Conditional on
\emph{g\textsubscript{i}} for any \emph{i} ≥ 2, he prefers
\emph{t\textsubscript{i}} to \emph{t}\textsubscript{1}. So if
\emph{i}~≥~2, conditionalising on \emph{g\textsubscript{i}} changes
Frankie's preferences, so he doesn't know \emph{g\textsubscript{i}}.

This third puzzle is striking for two reasons. One is that it involves a
change of strict preferences. Unconditionally, Frankie strictly prefers
\emph{t}\textsubscript{1} to \emph{t\textsubscript{i}}; conditional on
\emph{g\textsubscript{i}} he strictly prefers \emph{t\textsubscript{i}}
to \emph{t}\textsubscript{1}. When I first saw these puzzles, I thought
we could possibly get around them by restricting attention to cases
where conditionalisation changes a strict preference. This example shows
that way of rescuing IRT-CP won't work. The other reason is that it
heightens the implausibility of the sceptical result that Frankie
doesn't know \emph{g\textsubscript{i}}. It's one thing to say that the
weird situation that Judas Priest puts Frankie Lee makes Frankie Lee
lose a lot of knowledge he ordinarily has. That's just IRT in action;
change the practical situation and someone might lose knowledge. It's
another to say that within this very situation, Frankie Lee knows of
some notes that they are genuine but does not know that others are
genuine, even though his evidence for the genuineness of each note is
the same.

So we have three puzzles to try to solve, if we want to defend anything
like IRT-CP.

\begin{enumerate}
\def\labelenumi{\arabic{enumi}.}
\tightlist
\item
  In the case where Frankie Lee has no reason to choose one note rather
  than another, he doesn't know of any note that it is genuine. This is
  surprisingly sceptical.
\item
  In the case where he has a weak reason to choose one note, he knows
  that note is genuine, but not the others. This retains the
  surprisingly sceptical consequence of the first puzzle, and adds a
  surprising asymmetry.
\item
  In both cases, there seems to be a really bad closure failure, with
  Frankie Lee knowing that all the notes are genuine, but not knowing of
  all or most individual notes that they are genuine.
\end{enumerate}

Before we leave Frankie Lee for a while, let's note one variation on the
case that somewhat helps IRT. Imagine that the country they are in has
just reached the level of technological sophistication where it can mass
produce plastic banknotes. Further, no one in the country has yet
figured out how to produce plausible forgeries of plastic banknotes, and
Frankie Lee knows this. Finally, assume that one of the notes, lucky
\emph{n}\textsubscript{8}, is one of the new plastic notes, while the
others are the old paper notes. If Frankie Lee cares about counterfeit
avoidance at all, he should take \emph{n}\textsubscript{8}. He should do
so because it definitely isn't a counterfeit, while each of the others
might be. So in that case, Frankie Lee doesn't know that the notes other
than \emph{n}\textsubscript{8} are genuine, at least if whatever might
be false isn't known.

Now we have a case where IRT-CP gives the right answers for the right
reasons. A theory that disagrees with IRT-CP about this case has to
either (a) deny this intuition that the uniquely rational choice for
Frankie Lee is \emph{n}\textsubscript{8}, or (b) say that Frankie Lee
should choose \emph{n}\textsubscript{8} because the other choices are
too risky, even though he knows the risk in question will not eventuate.
Neither option is particularly appealing, at least if one is unhappy
with making Moore-paradoxical assertions, so this is a good case for
IRT-CP. Or, more carefully, it's good news for some version of IRT. This
case is some evidence that the problem is not with the very idea of
interest-relativity, but with the implementation of it. We'll see more
such evidence as the chapter goes along.

\section{Responding to the Challenge, Quickly}\label{sec-tiesresponse}

The second half of this chapter is going to get into the weeds a bit
about how choices do and should get made in cases like Frankie Lee's.
Before we do that, I am going to outline how my version of IRT, which
differs from IRT-CP, handles these cases.

Let's start with closure, and assume that Frankie Lee doesn't know of
any note that it is genuine. And assume that's because the conditional
utility of a salient act is importantly different, conditional on that
note being genuine, to what its unconditional utility. Now we can avoid
the closure problem by stressing that what matters is not that the
conditional and unconditional questions end up with the same verdict,
but that the process of getting to that verdict is the same. This is why
if Frankie Lee doesn't know of any note that it's genuine, he also
doesn't know \emph{g}. Right now, when choosing a note (and trying to
maximise expected utility), he should be indifferent because the risk
that any note is counterfeit, given his evidence, is more or less the
same as the risk that any other note is counterfeit. When he is choosing
conditional on \emph{g}, he doesn't have to attend to risks, or his
evidence, or anything that might be more or less equal to anything else.
He just takes it as fixed, for purposes of answering the question of
what to choose conditional on \emph{g}, that the notes are genuine. He
ends up in the same place both times, indifference between the notes,
but he gets there via different pathways. That's enough to defeat
knowledge that \emph{g}.

I'm appealing again here to a point I first made back in
Section~\ref{sec-block}. In English, saying that two questions are
answered the same way is ambiguous. It might mean that we end up in the
same place when answering the two questions. Or it might mean that we
get to that place the same way. There are any number of examples of
this. The questions \emph{What is three plus two}, and \emph{How many
Platonic solids are there}, get answered the same way in the first
sense, but not the second sense. Conditional Preference stresses that
certain conditional and unconditional questions get answered the same
way in this first sense. My version of IRT says that what matters is
that these conditional and unconditional questions get answered the same
way in the second sense.

That deals with the closure problem satisfactorily, but it does not help
with the sceptical problem. To solve that problem we need to rethink our
theory of decision. I added, almost as an aside, an assumption in the
earlier discussion that Frankie Lee was trying to maximise expected
utility. That's a mistake; he shouldn't do that. In a lot of cases like
Frankie Lee's, the rational thing to do is to simply ignore the
possibility that the notes are counterfeit. This will sometimes lead to
taking a choice that doesn't maximise either actual or expected utility.
But choice making procedures can be costly. Difficult choice making
procedures involve computational, hedonic, and investigative costs. It
is worth giving up some expected utility in the outcome to use a cheaper
decision procedures. One way to do that is to simply ignore some risks.

If Frankie Lee ignores the risk that the notes are counterfeit, then the
argument that he doesn't know \emph{g}\textsubscript{1},
\emph{g}\textsubscript{2}, etc., doesn't get off the ground. Given that
he's ignoring the risk that the notes are counterfeit, conditionalising
on them not being counterfeit changes precisely nothing. So there is no
pragmatic argument that he does not know they are genuine. This approach
will avoid the sceptical problems if, but only if, this kind of
`ignoring' is rational and widespread. I aim to make a case that it is.
But first I want to make things if anything worse for IRT, by stressing
how quotidian examples with the structure of Frankie Lee's are. This
will prevent me from being able to dismiss the example as a theorist's
fantasy, but will ultimately help see why ignoring the downside risks is
so natural, and so rational.

\section{Back to Earth}\label{sec-backearth}

The Frankie Lee and Judas Priest case is weird. Who offers someone
money, then asks them to pick which note to take? Intuitions about such
weird cases cases are sometimes deprecated. Perhaps the contrivance
doesn't reveal deep problems with a philosophical theory, but merely a
quirk of our intuitions. I am not going to take a stand on any big
questions about the epistemic significance of intuitions about weird
cases here. Rather, I'm going to note that cases with the same structure
as the story of Frankie Lee and Judas Priest are incredibly common in
the real world. Thinking about the real world examples can show us how
pressing are the problems these cases raise. It also helps us see the
way out of these problems.

So let's leave Frankie Lee for now, just above the potted plain, and
think about a new character. We will call this one David, and he is
buying a few groceries on the way home from work. In particular, he has
to buy a can of chickpeas, a bottle of milk, and a carton of eggs. To
make life easy, we'll assume each of these cost the same amount:
\$5.\footnote{If that sounds implausible to you, make the
  can/bottle/carton a different size, or change the currency to some
  other dollars than the one you're instinctively using. I think this
  example works tolerably well when understand as involving, for
  example, East Caribbean dollars.} None of these purchases is entirely
risk free. Canned goods are pretty safe, but sometimes they go bad. Milk
is normally removed from sale when it goes bad, but not always. And eggs
can crack, either in transit or just on the shelf. In David's world,
just like ours, each of these risks is greater than the one that came
before.

David has a favorite brand of chickpeas, of milk, and of eggs, and he
knows where in the store they are located. So his shopping is pretty
easy. But it isn't completely straightforward.

First he gets the chickpeas. That's simple; he grabs the nearest can,
and unless it is badly dented, or leaking, he puts in in his basket.

Next he goes onto the milk. The milk bottles have sell-by dates printed
in big letters on the front. \footnote{This kind of labeling is common
  for milk in Australian supermarkets, but not, typically, in American
  supermarkets.} David checks that he isn't picking up one that is about
to expire. His store has been known to have adjacent bottles of milk
with sell-by dates 10 days apart, so it's worth checking. But as long as
the date is far enough in the future, he takes it and moves on.

Finally, he comes to the eggs. (Nothing so alike as eggs, he always
thinks to himself, a little anachronistically.) Here he has to do a
little more work. He takes the first carton, opens it to see there are
no cracks on the top of the eggs, and, finding none, puts that in his
basket too. He knows some of his friends do more than this; flipping the
carton over to check for cracks underneath. But the one time he tried
that, the eggs ended up on the floor. And he knows some of his friends
do less; just picking up the carton by the underside, and only checking
for cracks if the underside is sticky where the eggs have leaked. He
thinks that makes sense too, but he is a little paranoid, and likes
visual confirmation of what he's getting. All done, he heads to the
checkout, pays his \$15, and goes home.

The choice David faces when getting the chickpeas is like the choice
Frankie Lee faces. In a normal store, it will be more like the version
where Frankie Lee has to reach further for some notes than others, but
sometimes there will be multiple cans equidistant from David. More
normally though, some of the cans will be towards the front, and others
towards the back, and it will be easier to grab one of the ones from the
front. That's why it is weird to get one from the back; reaching incurs
costs without any particular payoff.

Ignore this complication for now and focus on the ways in which David's
options in the supermarket are like Frankie Lee's. He has to choose from
among a bunch of very similar seeming options. In at least the chickpeas
example, there is something you'd want to say that he knows: canned
goods sold at reputable stores are safe. But the arguments above seem to
show that David does not know this, at least if IRT-CP is true. Indeed,
it seems to show this as long as Conditional Preferences is true, even
if it isn't the full story of how interests matter to knowledge.
Assuming there is some positive probability of the chickpeas not being
safe, and the costs of reaching for some other can are low enough, David
is in exactly the same situation as Frankie Lee. Right now, he maximises
utility by taking the front-most can. But conditional on one of the
other cans being safe, he maximises utility by taking it. So he does not
know of any of the other cans that they are safe.

Frankie Lee's situation is weird. Who lays out some ten dollar bills and
asks you to pick one? (Judas Priest, I guess.) But David's situation is
not weird. Looking at a fully stocked shelf of industrially produced
food, and needing to pick one can out of an array of similar items, is a
very common experience. If a theory of knowledge yields bizarre verdicts
about a case like this, it is no defence at all to say the situation is
too obscure. In this modern world, it's an everyday occurrence.

\section{I Have Questions}\label{sec-supermarketquestions}

So far in this chapter I've mostly assumed that these two questions are
equivalent:

\begin{enumerate}
\def\labelenumi{\arabic{enumi}.}
\tightlist
\item
  Which option has highest expected utility?
\item
  What to do?
\end{enumerate}

In doing this, I've faithfully reproduced the arguments of some critics
of IRT. Those critics were hardly being unfair to proponents of IRT in
treating these questions as being alike. They are explicitly treated as
being interchangable in, for example, my ``Can We Do Without Pragmatic
Encroachment''. But this was a mistake I made in defending IRT, and the
beginning of a solution to the problems raised by Frankie Lee is to
separate the questions out. I already mentioned one respect in which
these questions differ back in Section~\ref{sec-questions}. I'll
rehearse that difference, briefly mention a second difference, then
spend some time on a third difference.

The point I made much of back in Section~\ref{sec-questions} was that
someone might know the utility facts, but not know what to do. When
Frankie sits down, with his fingers to his chin, and tries to decide
which of the tens to take, it's possible he knows that they each have
the same utility. But he still has to pick one, and with his head
spinning he can't decide which one to take. In cases like these
answering questions about utility comparisons won't settle questions
about what to do.\footnote{James Joyce (\citeproc{ref-Joyce2018}{n.d.})
  suggests the following terminology. If Frankie is rational, then
  utility considerations settle questions about what to \emph{choose},
  but not questions about what to \emph{pick} in the case of a tie. I
  haven't quite followed that terminology; I've let Frankie pick and
  choose more freely than that. But I'm following Joyce in stressing
  this conceptual distinction.}

A second reason for not treating the questions alike is that to treat
them alike assumes away something that should not be assumed away. It
simply assumes that risk-sensitive theories of choice, as defended by
Quiggin (\citeproc{ref-Quiggin1982}{1982}) and Buchak
(\citeproc{ref-BuchakRisk}{2013}), are mistaken. We probably shouldn't
simply assume that. It turns out the difference between expected utility
theory and these heterodox alternatives isn't particularly relevant to
Frankie's or David's choices, so I'll leave this aside for the rest of
the chapter.

The third way in which the treating the questions as equivalent is wrong
takes a little longer to set up. The short version is that rational
people are satisficers, and for a satisficer you can answer the question
\emph{What to do} without taking a stand on questions about relative
utility. The longer version is set out in the next section.

\section{You'll Never Be Satisfied (If You Try to
Maximise)}\label{sec-satisfied}

The standard model of practical rationality that we use in philosophy is
that of expected utility maximisation. But there are both theoretical
and experimental reasons to think that this is not the right model for
choices such as that faced by Frankie or David. Maximising expected
utility is resource intensive, especially in contexts like a modern
supermarket, and the returns on this resource expenditure are
unimpressive. What people mostly do, and what they should do, is choose
in a way that is sensitive to the costs of adopting one or other way.

There are two annoying terminological issues around here that I mostly
want to set aside, but need to briefly address in order to forestall
confusion.

I'm going to assume maximising expected utility means taking the option
with the highest expected utility given facts that are readily
available. So if one simply doesn't process a relevant but
observationally obvious fact, that can lead to an irrational choice. I
might alternatively have said that the choice was rational (given the
facts the chooser was aware of), but the observational process was
irrational. But I suspect that terminology would just add needless
complication.

I'm going to come back to another point that is partially
terminological, and partially substantive. That's whether we should
identify the choice consequentialists recommend in virtue of the fact
that it maximises expected utility with one of the options (in the
ordinary sense of option), or something antecedent.

I'm going to call any search procedure that is sensitive to resource
considerations a satisficing procedure. This isn't an uncommon usage.
Charles Manski (\citeproc{ref-Manski2017}{2017}) uses the term this way,
and notes that it has rarely been defined more precisely than that. But
it isn't the only way that it is used. Mauro Papi
(\citeproc{ref-Papi2013}{2013}) uses the term to exclusively mean that
the chooser has a `reservation level', and they choose the first option
that crosses it. This kind of meaning will be something that becomes
important again in a bit. And Chris Tucker -Tucker
(\citeproc{ref-Tucker2016}{2016}), following a long tradition in
philosophy of religion, uses it to mean any choice procedure that does
not optimize. Elena Reutskaja et al (\citeproc{ref-Reutskaja2011}{2011})
contrast a `hybrid' model that is sensitive to resource constraints with
a `satisficing' model that has a fixed reservation level. They end up
offering reasons to think ordinary people do (and perhaps should) adopt
this hybrid model. So though they don't call this a satisficing
approach, it just is a version of what Manski calls satisficing. Andrew
Caplin et al (\citeproc{ref-Caplin2011}{2011}), on the other hand,
describe a very similar model to Reutskaja et al's hybrid model - one
where agents try to find something above a reservation level but the
reservation level is sensitive to search costs - as a form of
satisficing. So the terminology around here is a mess. I propose to use
Manski's terminology: agents satisfice if they choose in a way that is
sensitive to resource constraints. Ideally they would maximise, subject
to constraints, but saying just what this comes to runs into obvious
regress problems (\citeproc{ref-Savage1967}{Savage, 1967}). Let's set
aside this theoretical point for a little, and go back to David and the
chickpeas.

When David is facing the shelf of (roughly equidistant) chickpeas, he
can rationally take any one of them - apart perhaps from ones that are
seriously damaged. How can expected utility theory capture that fact? It
says that more than one choice is permissible only if the choices are
equal in expected utility. So the different cans are equal in expected
utility. But on reflection, this is an implausible claim. Some of the
cans are ever so slightly easier to reach. Some of the cans will have
ever so slight damage - a tiny dint here, a small tear in the label
there - that just might indicate a more serious flaw. Of course, these
small damages are almost always irrelevant, but as long as the
probability that they indicate damage is positive, it breaks the
equality of the expected utility of the cans. Even if there is no
visible damage, some of the labels will be ever so slightly more faded,
which indicates that the cans are older, which ever so slightly
increases the probability that the goods will go bad before David gets
to use them. Of course in reality this won't matter more than one time
in a million, but one in a million chances matter if you are asking
whether two expected utilities are strictly equal.

The common thread to the last paragraph is that these objects on the
shelves are almost duplicates, but the most careful quality control
doesn't produce consumer goods that are actual duplicates. This is
particularly true in Frankie Lee's choice situation. If all the notes he
looks at are really duplicates, down to the serial numbers, he should
run away. There are always some differences. It is unlikely that these
differences make precisely zero difference to the expected utility of
each choice. Even if they do, discovering that is hard work.

So it seems likely that, according to the expected utility model, it
isn't true that David could permissibly take any can of chickpeas that
is easily reachable and not obviously flawed. Even if that is true, it
is extremely unlikely that David could know it to be true. But one thing
we know about situations like David's is that any one of the (easily
reached, not clearly flawed) cans can be permissibly chosen, and David
can easily know that. So the expected utility model, as I've so far
described it, is false.

I'll return in the next section to the question of whether this is a
problem for theories of decision based around expected utility
maximisation broadly, or whether it is just a problem for the particular
way I've spelled out the expected utility theory. But for now I want to
run through two more arguments against the idea that supermarket
shoppers like David should be maximising expected utility (so
understood).

In all but a vanishingly small class of cases, the different cans will
not have the same expected utility. Indeed, that they have the same
expected utility is a measure zero event. One way to note that expected
utility maximisation can't be the right theory of choice-worthiness is
that cases where multiple cans are equally choice-worthy is not a
measure zero event; it's the standard case. And figuring out which can
has the highest expected utility is a going to be work. It's possible in
principle, I suppose, that someone could be skilled at it, in the sense
that they could instinctively pick out the can whose shape, label
fading, etc., reveal it to have the highest expected utility. Such a
skill seems likely to be rare - though I'll come back to this point
below when considering some other skills that are probably less rare.
For most people, maximising expected utility will not be something that
can be done through effortless skill alone; it will take effort. This
effort will be costly, and almost certainly not worth it. Although one
of the cans will be ever so fractionally higher in expected utility than
the others, the cost of finding out which can this is will be greater
than the difference in expected utility of the cans. So aiming to
maximise expected utility will have the perverse effect of reducing
one's overall utility, in a predictable way.

The costs of trying to maximise expected utility go beyond the costs of
engaging in search and computation. There is evidence that people who
employ maximising strategies in consumer search end up worse off than
those who don't. Schwartz et al. (\citeproc{ref-SchwartzEtAl2002}{2002})
reported that consumers could be divided in `satisficers' and
`maximizers'. And once this division is made, it turns out that the
maximizers are less happy with individual choices, and with their life
in general. This finding has been extended to work on career choice
(\citeproc{ref-IyengarEtAl2006}{Iyengar, Wells, \& Schwartz, 2006})
where the maximisers end up with higher salaries but less job
satisfaction, and to friend choice
(\citeproc{ref-NewmanEtAl2018}{Newman, Schug, Yuki, Yamada, \& Nezlek,
2018}), where again the maximizers seem to end up less satisfied.

There is evidence in those works I just cited that maximizing is bad at
what it sets out to achieve. But there are both empirical and
theoretical reasons to be cautious about accepting these results at face
value.

Whether maximizers are worse off seems to be tied up to the `paradox of
choice' (\citeproc{ref-Schwartz2004}{Schwartz, 2004}), the idea that
sometimes giving people even more choices makes them less happy with
their outcome, because they are more prone to regret. But it is unclear
whether such a paradox exists. One meta-analysis
(\citeproc{ref-ScheibehenneEtAl2010}{Scheibehenne, Greifeneder, \& Todd,
2010}) did not show the effect existing at all, though a later
meta-analysis finds a significant mediated effect
(\citeproc{ref-ChernevEtAl2015}{Chernev, Böckenholt, \& Goodman, 2015}).
But it could also be that the result is a feature of an idiosyncratic
way of carving up the maximizers from the satisficers. Another way of
dividing them up produces no effect at all
(\citeproc{ref-DiabEtAl2008}{Diab, Gillespie, \& Highhouse, 2008}).

The theoretical reasons relate to Newcomb's problem. Even if we knew
that maximizers were less satisfied with how things are going than
satisficers, it isn't obvious that any one person would be better off
switching. They might be like a two-boxer who would get nothing if they
took one-box. There is a little evidence in Iyengar et al.
(\citeproc{ref-IyengarEtAl2006}{2006}) that tells against this
explanation of what is happening, but not nearly enough to rule it out
conclusively.

The upshot of all this, I think, is that there are potentially two kinds
of cost of engaging in certain kind of search and choice procedures.
Some procedures are more costly to implement than others: they take more
time, or more energy, or even more money. But further, some procedures
might have a hedonic cost that extends beyond the time that the
procedure is implemented. There is no theoretical or empirical guarantee
that choosing widget W by procedure P1 will produce the same amount of
happiness as choosing widget W by procedure P2. And especially for
choices that are intended to produce happiness, this kind of factor
should matter to us. In short, there are many more ways to assess a
consumer choice procedure than the quality of the products in ends up
choosing. This will be the key to our resolution of the puzzles about
closure.

\section{Deliberation Costs and Infinite
Regresses}\label{sec-deliberationcosts}

The idea that people should reason by choosing arbitrarily between
choices that are close enough is not a new one. Experimental work by
Reutskaja et al. (\citeproc{ref-Reutskaja2011}{2011}) suggests this is
how people do reason. But the idea that people should reason this way
goes back much further. It is often traced back to a footnote of
Knight's. Here is the text that provides the context for the note.

\begin{quote}
Let us take Marshall's example of a boy gathering and eating berries
\ldots{} We can hardly suppose that the boy goes through such mental
operations as drawing curves or making estimates of utility and
disutility scales. What he does, in so far as he deliberates between the
alternatives at all*, is to consider together with reference to
successive amounts of his ``commodity,'' the utility of each increment
against its ``cost in effort,'' and evaluate the net result as either
positive or negative (\citeproc{ref-Knight1921}{Knight, 1921: 66--7})
\end{quote}

The footnote attached to `at all' says this:

\begin{quote}
Which, to be sure, is not very far. Nor is this any criticism of the
boy. Quite the contrary! It is evident that the rational thing to do is
to be irrational, where deliberation and estimation cost more than they
are worth. That this is very often true, and that men still oftener
(perhaps) behave as if it were, does not vitiate economic reasoning to
the extent that might be supposed. For these irrationalities (whether
rational or irrational!) tend to offset each other.
(\citeproc{ref-Knight1921}{Knight, 1921: 67fn1})
\end{quote}

Knight doesn't really give an argument for the claim that these effects
will offset. As John Conlisk (\citeproc{ref-Conlisk1996}{1996}) shows in
his fantastic survey of the late 20th century literature on bounded
rationality, it very often isn't true. Especially in game theoretic
contexts, the thought that other players might think that ``deliberation
and estimation cost more than they are worth'' can have striking
consequences. That's not relevant to us though; we're just interested in
the claim about rationality.

There is something paradoxical, almost incoherent, about Knight's
formulation. If it is ``rational to be irrational'', then being
``irrational'' can't really be irrational. There are two natural ways to
get out of this paradox. One, loosely following David Christensen
(\citeproc{ref-Christensen2007}{2007},) would be to say that ``Murphy's
Law'' applies here. Whatever one does will be irrational in some sense.
Still, some actions are less irrational than others, and the least
irrational will be to decline to engage in deliberation that costs more
than it is worth. I suspect what Knight had in mind though was something
different (if not obviously better). He is using `rational' as more or
less a rigid designator of the the property of choosing as a Marshallian
maximiser does. What he means here is that the disposition to not choose
in that way will be, in the long run, the disposition with maximal
returns.

This latter idea is what motivates the thought that rational agents will
take what Conlisk calls ``deliberation costs'' into account. Conlisk
thinks that this is what rational agents will do, but he notes that
there is a problem for it.

\begin{quote}
However, we quickly collide with a perplexing obstacle. Suppose that we
first formulate a decision problem as a conventional optimization based
on the assumption of unbounded rationality and thus on the assumption of
zero deliberation cost. Suppose we then recognize that deliberation cost
is positive; so we fold this further cost into the original problem. The
difficulty is that the augmented optimization problem will itself be
costly to analyze; and this new deliberation cost will be neglected. We
can then formulate a third problem which includes the cost of solving
the second, and then a fourth problem, and so on. We quickly find
ourselves in an infinite and seemingly intractable regress. In rough
notation, let P denote the initial problem, and let F(.) denote the
operation of folding deliberation cost into a problem. Then the regress
of problems is P, F(P), F\textsuperscript{2}(P), \ldots{}
(\citeproc{ref-Conlisk1996}{Conlisk, 1996: 687})
\end{quote}

Conlisk's own solution to this problem is not particularly satisfying.
He notes that once we get to F\textsuperscript{3} and
F\textsuperscript{4}, the problems are `overly convoluted' and seem to
be safely ignored. This isn't enough for two reasons. First, even a
problem that is convoluted to state can have serious consequences when
we think about solving it. (What would \emph{Econometrica} publish if
this weren't true?) Second, as is often noted, F\textsuperscript{2}(P)
might be a harder problem to solve than P, so simply stopping the
regress there and treating the rational agent as solving this problem
seems to be an unmotivated choice.

As Conlisk notes, this problem has a long history, and is often used to
dismiss the idea that folding deliberation costs into our model of the
optimising agent is a good idea. I use `dismiss' advisedly. Conlisk
points out that there is very little \emph{discussion} of the infinite
regress problem in the literature before his paper in 1996. The same
remains true after 1996. Instead people appeal to the regress in a
sentence or two to set aside approaches that incorporate deliberation
cost in the way that Conlisk suggests.

Up to around the time of Conlisk's article, the infinite regress problem
was often appealed to by people arguing that we should, in effect,
ignore deliberation costs. After his article, the appeals to the regress
comes from a different direction. The appeals now typically come from
theorists arguing that deliberation costs are real, but the regress
means it will be impossible to consistently incorporate them into a
model of an optimizing agent. So we should instead rely on experimental
techniques to see how people actually handle deliberation costs; the
theory of optimization has reached its limit. This kind of move is found
in writers as diverse as Gigerenzer \& Selten
(\citeproc{ref-GigerenzerSelton2001}{2001}), Odell
(\citeproc{ref-Odell2002}{2002}), Pingle
(\citeproc{ref-Pingle2006}{2006}), Mangan, Hughes, \& Slack
(\citeproc{ref-ManganEtAl2010}{2010}), Ogaki \& Tanaka
(\citeproc{ref-OgakiTanaka2017}{2017}) and Chakravarti
(\citeproc{ref-Chakravarti2017}{2017}). Proponents of taking
deliberation costs seriously within broadly optimizing approaches, like
Miles Kimball (\citeproc{ref-Kimball2015}{2015}), say that solving the
regress problem is the biggest barrier to having such an approach taken
seriously by economists.

It really matters for the story of this book that there is a solution to
the infinite regress problem within a broadly optimizing framework. More
precisely, IRT needs there to be a solution to the regress problem that
does not defeat knowledge. At least some of the time, the fact that a
belief was formed by a rationally problematic procedure means that the
belief is not a piece of knowledge. As we might say, the irrationality
of the procedure is a defeater of the claim to knowledge. But perhaps if
the procedure is optimal (even if not rational) that defeats the
defeater. `Optimal' here need not mean rationally optimal; it means
optimal given the computational limitations on the agent. But now I've
said enough to suggest that the regress problem will arise.

Here's how I plan to solve the regress problem. What matters for
optimality is that the thinker is following the procedure that is the
optimal solution to F(P). It doesn't matter that they compute that it is
the optimal solution, or even that they are following it because it is
the optimal solution. It is an external, success oriented condition,
that does not require that it be followed in the right way, e.g., by
computing the optimal answer. The thinker just has to do the right
thing. This kind of externalism solves the regress problem by denying it
gets started. There is no higher order problem to solve, because the
thinker doesn't have to solve that problem in order to act rationally.
They just have to have dispositions that mean they mimic the correct
solution.

This solution to the regress problem is easy to state, but a little
harder to motivate. There are two big questions to answer before we can
say it is really motivated.

\begin{enumerate}
\def\labelenumi{\arabic{enumi}.}
\tightlist
\item
  Why should we allow this kind of unreflective rule-following in our
  solution to the regress?
\item
  Why should we think that F(P) is the point where this consideration
  kicks in, as opposed to P, or anything else?
\end{enumerate}

There are a few ways to answer 1. One motivation traces back to the work
by the artificial intelligence researcher Stuart Russell
(\citeproc{ref-Russell1997}{1997}). (Although really it starts with the
philosophers Russell cites as inspiration, such as Cherniak
(\citeproc{ref-Cherniak1986}{1986}) and Harman
(\citeproc{ref-Harman1973}{1973}).) He stresses that we should think
about the problem from the outside, as it were, not from inside the
agent's perspective. How would we program a machine that we knew would
have to face the world with various limitations? We will give it rules
to follow, but we won't necessarily give it the desire (or even the
capacity) to follow those rules self-consciously. What's more useful is
giving it knowledge of the limitations of the rules. That can be done
without following the rules as such. It just requires having good
dispositions to complicate the rules one is following in cases where
such complication will be justified.

Another motivation is right there in the quote from Knight that set this
literature going. Most writers quote the footnote, where Knight suggests
it might be rational to be irrational. But look back at what he's saying
in the text. The point is that it can be perfectly rational to use
considerations other than drawing curves and making utility scales. What
one has to do is follow internal rules that (non-accidentally) track
what one would do if one was a self-consciously perfect Marshallian
agent. That's what I'm saying too.

Finally, there is the simple point that on pain of regress any set of
rules whatsoever must say that there are some rules that are simply
followed. This is one of the less controversial conclusions of the
debates about rule-following that were started by Wittgenstein
(\citeproc{ref-Wittgenstein1953}{1953}). That we must at some stage
simply follow rules, not follow them in virtue of following another
rule, say the rule to compute how to follow the first rule and act
accordingly, is an inevitable consequence of thinking that finite
creatures can be rule followers.

So question 1 is not really a big problem. But question 2 is more
serious. Why F(P), and why not something else? The short answer will be
that any reason to think that rational actors maximize \emph{expected}
utility, as opposed to actual utility, will also be a reason to think
that they solve F(P) and not P. The longer answer is a bit more
roundabout, but it helps us to see what a solution to F(P) will look
like.

Start by stepping back and thinking about why we cared about
\emph{expected} utility in the first place. Why not just say that the
best thing to do is to produce the best outcome, and be done with it?
Well, we don't say that because we take it as a fixed point of our
inquiry that agents are informationally limited, and that the best thing
to do is what is best given that limitation. Given some plausible
assumptions, the best thing for the informationally limited agent to do
would be to maximise expected utility. This is a second-best option, but
the best is unavailable given the limitations that we are treating as
unavoidable.

Agents are not just informationally limited, they are computationally
limited too. We could treat computational limits as the core limitation
to be modelled. As Conlisk says, it is ``entertaining to imagine''
theorists who worked in just this way
(\citeproc{ref-Conlisk1996}{Conlisk, 1996: 691}). So let's imagine we
meet some Martian economists, and they take computational, and not
informational, limitations as the core constraint on rational choosers.
So in their models, every agent has all the information relevant to
their choice, but can't always compute what to do with that information.

Conlisk doesn't spell out the details of this thought experiment, and
it's a little tricky to say exactly how it should work. (I'm indebted
here to Harvey Lederman.) After all, you might think that `information'
should include things like information about the results of various
computations, or about what would be best to do given their information.
So how can we make sense of a being that is computationally but not
informationally limited?

Here's one way to make sense of what Conlisk's Martians might be like.
Assume that the Martians are very strict positivists. (This isn't going
to make them optimal social scientists, but presumably we never thought
they were.) So the truths can be divided up into observation sentences,
and things derived from observation sentences by definition and
deduction. In their preferred models, every agent knows every true
observation sentence - including those about observations that have not
yet been made. But they don't know all the results of deriving further
truths from the observation sentences by definition and deduction. So
such an agent might know precisely all the points she has to drive to
today, and know the cost of traveling between any two points, but not
know the optimal route to take on her travels. That last claim won't be
`information' in the relevant sense since it is not an observation
sentence.

The point is not that the Martian economists think that every agent
knows every observation sentence, any more than human economists think
that every agent has a solution to every traveling salesman problem in
their back pocket. Rather, it's that they that this is a good modeling
assumption. Conlisk has some fun imagining what Martian economists who
make this modeling assumption might say in defence of their practice.
They might disparage their colleagues who take informational limitations
seriously as introducing ad hoc stipulations into theory. They might
argue that informational limitations are bound to cancel out, or be
eliminated by competition. They might argue that apparent informational
limitations are really just computational ones, or at least can be
modelled as computational ones. (Here it might be helpful to think of
the Martian economists as positivists, and in particular as positivists
who think that the notion of observation sentence is flexible enough to
behave differently in different theoretical contexts.) And so on,
replicating almost every complaint that human economists have ever made
about theorists who want to take computational limitations seriously.

What Conlisk doesn't add is that they might suggest that there is a
regress worry for any attempt to add informational constraints. Imagine
that inside one of these models, an agent is deciding what to have for
dinner. Let Q be the initial optimisation problem as the Martians see
it. That is, Q is the problem of finding the best outcome, the best
dinner, given full knowledge of the situation, but the actual
computational limitations of the agent. Then we suggest that we should
also account for the informational limitations. Let's see if this will
work, they say. Let I be the function that transforms a problem into one
that is sensitive to the informational limitations of the agent. But if
we're really sensitive to informational limitations, we should note that
I(Q) is also a problem the agent has to solve under conditions of less
than full information.\footnote{At this point the Martians might note
  that all they are relying on here is that agents in their model
  violate negative introspection: sometimes they don't know something
  without knowing that they don't know it. They could cite L.
  Humberstone (\citeproc{ref-Humberstone2016}{2016: 380--402}) for why
  this is a sensible modeling assumption.} So the informationally
challenged agent will have to solve not just I(Q), but
I\textsuperscript{2}(Q), and I\textsuperscript{3}(Q) and so
on.\footnote{At this point, some of the Martians note that the existence
  of Elster (\citeproc{ref-Elster1979}{1979}) restored their faith in
  humanity.}

Orthodox defenders of (human versions of) rational choice theory have to
think this is a bad argument. I think most of them will agree with
roughly the solution I'm adopting. The right problem to solve is I(Q),
on a model where Q is in fact the problem of choosing the objectively
best option. Put in philosophers' terms, we should think of Q as
rigidly, and transparently, designating the problem the agent is facing.
So I(Q) is not the problem of doing what's best given how little one
knows about both the world and one's place in it. Rather, it's the
problem of how to do the best one can in this very situation, given
one's ignorance about the world. Even if one doesn't know precisely the
situation one is in, and one doesn't know what utility function one has,
or for that matter what knowledge one has, one should maximise expected
utility given actual expectations and actual utility. The problem to
solve is I(Q), not I\textsuperscript{2}(Q).

But the bigger thing to say is that neither we nor the Martians really
started with the right original problem. The original problem, O, is the
problem of choosing the objectively best option; i.e., choosing what to
have for dinner. The humans start by considering the problem I(O), i.e.,
P, and then debate whether we should stick with that problem, or move to
F(I(O)). The Martians start by considering the problem F(O), i.e., Q,
then debate whether we should stick with that or move to I(F(O)). And
the answer in both cases is that we should move.

Given the plausible commutativity principle that introducing two
limitations to theorising has the same effect whichever order we
introduce them, I(F(O))~=~F(I(O). That is, F(P)~=~I(Q). And that's the
problem that we should think the rational agent is solving.

But why solve that, rather than something more or less close to O? Well,
think about what we say about an agent in a Jackson case who tries to
solve O not I(O). (A Jackson case, in this sense, is a case where the
choice with highest expected value is known to not have the highest
objective value. So trying to get the highest objective value will mean
definitely not maximising expected value.) We think it will be sheer
luck if they succeed. We think in the long run they will almost
certainly do worse than if they tried to solve I(O). And in the rare
case where they do better, we think it isn't a credit to them, but to
their luck. In cases where the well-being of others is involved, we
think aiming for the solution to O involves needless, and often immoral,
risk-taking.

The Martians can quite rightly say the same things about why F(O) is a
more theoretically interesting problem than O. Assume we are in a
situation where F(O) is known to differ from O. For example, imagine the
decision maker will get a reward if they announce the correct answer to
whether a particular sentence is a truth-functional tautology, and they
are allowed to pay a small fee to use a computer that can decide whether
any given sentence is a tautology. The solution to O is to announce the
correct answer, whatever it is. The solution to F(O) is to pay to use
the computer. The Martians might point out that in the long run, solving
F(O) will yield better results. That if the agent does solve problems
like O correctly, even in the long run, this will just mean they were
lucky not rational. That if the reward is that a third party does not
suffer, then it is immorally reckless to not solve F(O), i.e., to not
consult the computer. In general, whatever we can say that motivated
``Rational Choice Theory'', as opposed to ``Choose the Best Choice
Theory'', they can say too.

Both the human and the Martian arguments look good to me. We should add
in both computational and informational limitations into our model of
the ideal agent. But note something else that comes from thinking about
these Jackson cases. In solving a limitation sensitive problem, we
aren't trying to approximate a solution to the limination insensitive
problem. This is part of why the regress can stop here. To solve F(X),
we don't have to solve X, and then see how close the various
computationally feasible solutions get to this solution. That's true in
general because of Jackson cases, but it's especially true when X is
itself a complex problem. In trying to solve F(I(O)), i.e., I(F(O)), we
aren't trying to maximise expected value, and then approximate that
solution given computational limitations. Nor are we trying to be
optimal by Martian standards (i.e., solve F(O)), then approximate that
given informational limitations. We're just trying to get as good an
outcome as we can, given our limitations. Doing that does not require
solving any iterated problem about how well we can solve F(I(O)) given
various limitations, any more than rationally picking berries requires
drawing Marshallian curves.

So that's the solution to the regress. It is legitimate to think that
there is a rule that rational creatures follow immediately, on pain of
thinking that all theories of rationality imply regresses. And thinking
about the contingency of how Rational Choice Theory got to be the way it
is suggests that the solution to what Conlisk calls F(P), or what I've
called F(I(O)), will be that point.

What might that stopping point look like in practice? In his discussion
of the regress, Miles Kimball (\citeproc{ref-Kimball2015}{2015})
suggests a few options. I want to focus on two of them.

\begin{quote}
Least transgressive are models in which an agent sits down once in a
long while to think very carefully about how carefully to think about
decisions of a frequently encountered type. For example, it is not
impossible that someone might spend one afternoon considering how much
time to spend on each of many grocery-shopping trips in comparison
shopping. In this type of modelling, the infrequent computations of how
carefully to think about repeated types of decisions could be
approximated as if there were no computational cost, even though the
context of the problem implies that those computational costs are
strictly positive. (\citeproc{ref-Kimball2015}{Kimball, 2015: 174})
\end{quote}

That's obviously relevant to David in the supermarket. He could, in
principle, spend one Saturday afternoon thinking about how carefully to
check each of the items in the supermarket before putting it in his
shopping cart. Then in future trips, he could just carry out this plan.
This isn't terrible, but I don't think it's optimal. For one thing,
there are much better things to do with Saturday afternoons. For
another, it suggests we are back in the business of equating solving
F(P) with approximately solving P. And that's a mistake. Better to just
say that David is rational if he just does the things that he would do
were he to waste a Saturday afternoon this way, and then plan it out.
That thought leads to Kimball's more radical suggestion for how to avoid
the regress,

\begin{quote}
{[}M{]}odelling economic actors as doing constrained optimization in
relation to a simpler economic model than the model treated as true in
the analysis. This simpler economic model treated as true by the agent
can be called a ``folk theory'' (\citeproc{ref-Kimball2015}{Kimball,
2015: 175})
\end{quote}

It's this last idea I plan to explore in more detail. (It has some
similarities to the discussion of small worlds in
(\citeproc{ref-Joyce1999}{J. M. Joyce, 1999: 70--77}).) The short
version is that David can, and should, have a little toy model of the
supermarket in his head, and should optimize relative to that model. The
model will be false, and David will know it is false. And that won't
matter, as long as David treats the model the right way.

\section{Ignorance is Bliss}\label{sec-ignorancebliss}

There are a lot of things that could have gone wrong with a can of
chickpeas. They could have gone bad inside the can. They could have been
contaminated, either deliberately or through carelessness. They could
have been sitting around so long they have expired. All these things
are, at least logically, possible.

These possibilities, while serious, are rare and hard to detect. It is
unheard of for someone to deliberately contaminate canned chickpeas,
even though other grocery products like strawberries have been targeted.
To check for expiry dates, one must scan each can, which is
time-consuming due to the small type. A badly dented can may increase
the risk of unintentional contamination, but most cans have no dents or
only minor ones.

Given the rarity of these problems and the difficulty in obtaining
evidence that significantly increases the probability of them occurring,
the rational choice is to act in a way that is not affected by whether
these problems actually occur. It is best to be vigilant, in the sense
of Sperber et al. (\citeproc{ref-SperberEtAl2010}{2010}). In this
context, that means considering only those problems for which there is
evidence that they are worth considering, and ignoring the rest. To
ignore a potential problem is to choose in a way that is insensitive to
evidence for the problem. That makes sense for both the banknotes and
the chickpeas, because engaging in a choice procedure that is sensitive
to the probability of the problem will, in the long run, make you worse
off.

In Kimball's terms, the rational shopper will have a toy model of the
supermarket in which the contents of undamaged cans are safe to eat.
This model is defeasible, but typically not defeated. (In Joyce's terms,
the small worlds are all ones in which the undamaged cans are safe.) A
thinker who uses that toy model won't change their view by
conditionalising on the fact that a particular can is safe. So it is
consistent with IRT that they know the can is safe. That gets us out of
the worst of the sceptical challenges. By similar reasoning, Frankie Lee
knows all of the banknotes are genuine.

This chapter started with the problem that cases like Frankie Lee's
seemed to lead to rampant scepticism given pragmatic theories like IRT.
The solution to this problem was more pragmatism. Rational choosers
typically do not use a model where the probability of a forgery or
contamination is 0.99999. This model is more trouble than it's worth,
since there is no actionable difference between it and one where the
probability is 1. In cases where one can do something about the risk,
like taking the plastic banknote, or checking inside the egg carton, is
is often worthwhile to do something. In those cases, but only those
cases, IRT does have sceptical consequences. In general, the simpler
model is the best choice, and when it is, IRT is consistent with the
chooser having a lot of knowledge.

So David does know that the chickpeas are safe. He believes this on the
basis of evidence that is connected in the right way to the truth of the
proposition that the chickpeas are safe. There is a potential pragmatic
defeater from the fact that Conditional Preference seems to rule out
this knowledge. But there is a pragmatic defeater of that pragmatic
defeater. Conditional Preferences only implies scepticism in David's
case if David is insensitive to deliberation costs when choosing. He
shouldn't be, on practical grounds. He should use a toy model that says
all safe looking cans are safe. Once he uses that toy model, there is no
pragmatic defeat of his well-supported, well-grounded true belief. He
knows the chickpeas are safe.

On the other hand, David doesn't know the eggs aren't cracked. The toy
model that says all available eggs are uncracked is bad. It isn't bad
because it's wrong. It's bad because there is a model that will yield
better long run results even once we account for its complexity. That's
the model that says that only eggs that have been visually inspected are
certain to be uncracked; all other eggs are at best probably uncracked.
So David doesn't know the eggs aren't cracked. Note this would be true
even if improvements in the supply chain made the probability of cracked
eggs much lower than it is today. What matters in the canned goods case
is not just that the risk of contamination is low, it's also that there
isn't anything to do about it. As long as it remains easy to flip the
lid of egg cartons to check whether they are cracked, it will be hard to
know without flipping they aren't cracked.

This is another illustration of how the form of IRT I endorse really
doesn't care about stakes. The stakes in this case are not zero - buying
cracked eggs wastes money and that's why David should check. But it
isn't `high stakes' in anything like the sense that phrase is used. The
stakes are exactly the same as in the chickpeas case. What matters is
not the cost of being wrong about an assumption, but rather the relative
cost of being wrong compared to the probability that one is wrong and
the cost of checking.

The milk case is only slightly more complicated. At least in some
places, the expiry date for milk is written in very large print on the
front of the bottle. In those cases, it is worth checking that you
aren't buying milk that expires tomorrow. So before you check, you don't
know that the milk you pick up doesn't expire tomorrow. (And, like in
the eggs case, that's true even if the shop very very rarely sells milk
that close to the expiry date.) But there is no way to check whether a
particular container of milk, far from its expiration date, has gone
bad. You can't easily open a milk bottle in the supermarket and smell
it, for example. So that's the kind of rare and uncheckable problem that
the sensible chooser will ignore. Their toy model will include that in a
well functioning store, all milk that is well away from the expiry date
is safe. So once they've checked the expiry date, they know it is safe
(assuming it is safe).

And in the normal case, Frankie Lee knows that the notes aren't
forgeries. His toy model of the currency, like ours, should be that all
bank notes are genuine unless there is a clear sign that they are
not.\footnote{Or at least some clear enough sign. Arguably, the fact
  that a note is a high value one that someone is trying to use in the
  betting ring half an hour before the Melbourne Cup is in itself a sign
  that it is not genuine. A sceptical theory that says no one in that
  betting ring knows whether they are passing on forged bank notes is
  not a problematic sceptical theory.} So we have a solution from within
IRT to both the closure problems and the sceptical problems.

In the next chapter, I'll look at problems that can be addressed without
taking this many detours into decision theory.

\bookmarksetup{startatroot}

\chapter{Changes}\label{sec-changes}

My version of IRT version shares defects with more familiar versions of
IRT. For instance, it is subject to the criticism that Crispin Wright
makes here.

\begin{quote}
{[}A{]} situation may arise \ldots{} when we can truly affirm an `ugly
conjunction' like:

\begin{quote}
~X didn't (have enough evidence to) know P at t but does at t* and has
exactly the same body of P-relevant evidence at t* as at t.
\end{quote}

Such a remark seems drastically foreign to the concept of knowledge we
actually have. It seems absurd to suppose that a thinker can acquire
knowledge without further investigation simply because his practical
interests happen so to change as to reduce the importance of the matter
at hand. Another potential kind of ugly conjunction is the synchronic
case for different subjects:

\begin{quote}
~X knows that P but Y does not, and X and Y have exactly the same body
of P-relevant evidence.
\end{quote}

when affirmed purely because X and Y have sufficiently different
practical interests. IRI, as we noted earlier, must seemingly allow that
instances of such a conjunction can be true.
(\citeproc{ref-Wright2018}{Wright, 2018: 368})
\end{quote}

That's right; I do allow that instances of such a conjunction can be
true. A similar objection has been made by Gillian Russell and John
Doris (\citeproc{ref-RussellDoris2008}{2009}), by Michael Blome-Tillmann
(\citeproc{ref-BlomeTillmann2009}{2009}), and by David Eaton and Timothy
Pickavance (\citeproc{ref-EatonPickavance2015}{2015}). My main reply to
these objections is that they overgenerate and would be successful
objections to any theory that separates knowledge from rational true
belief. Since knowledge does not equal rational true belief, no such
objection can work.\footnote{This reply was first made in my
  (\citeproc{ref-Weatherson2016-WEARTE}{2016b}), and earlier replies to
  Russell and Doris, as well as Blome-Tillman, were made in my
  (\citeproc{ref-Weatherson2011-WEADIR}{2011}), although I now believe
  that those replies did not quite get to the heart of the matter.}

\section{Overview of Replies}\label{sec-overview}

I'm going to quickly go over five responses to this objection. I think
at some level all five are correct. The first two, however, would
probably do little to persuade anyone not already committed to IRT. The
last three are more persuasive, and I'll develop each of them in a
subsequent section.

The first thing one could say about these objections is that since they
just state a prominent feature of the view, that it allows knowledge to
turn on non-alethic features, and object to that very feature, the
objections are blatantly question-begging. One could say that, but
really that and \$2.90 will get you a ride on the New York subway. The
opponents think that this view is radical. And of course the objections
to radical views will end up being question-begging
(\citeproc{ref-Lewis1982c}{Lewis, 1982}). Saying that one's opponents
are begging the question might make you feel better - you don't have to
be persuaded by their arguments - but doesn't actually move the debate
forward. We can, and must, do better.

A second thing to say is that on some versions of IRT, it will be very
hard to state the objection. Consider a version of IRT that also accepts
E=K, the thesis that one's evidence is all and only what one knows. This
is hardly an obscure version of the view; it's what is defended by Jason
Stanley (\citeproc{ref-Stanley2005}{2005}). Now it will not be true on
such a view that there are, as Wright suggests, two people who have the
same evidence but different knowledge. That's impossible, since having
different knowledge literally entails, on this view, that they have
different evidence. But does this make the objection go away, or does it
just make it harder to state? I'm mostly inclined to think it's the
latter. There is still something weird about people who have the same
input from the world, and the same reactions to that input, but who
differ in what they know about the world. So this response, while more
useful than the last one, i.e., not totally useless, won't quite work
either.

A third response challenges head on the intuition about `weirdness'
mentioned in the previous paragraph. One of the consequences of the vast
Gettier literature is that there are any number of cases where people
have the same inputs, the same true beliefs based on those inputs, but
different knowledge. It's trivial to get these inter-world versions of a
case like this, and maybe that's enough to undermine the intuition. More
generally, it's hard to state, and endorse, the intuition that the
interest-relative theory violates without committing oneself to
something very much like the JTB theory of knowledge. And since that
theory is false, that's kind of bad news for the intuition. Or, perhaps
more carefully, either that theory is false, or justification is
understood in terms of knowledge, as on the E=K picture. And appealing
to E=K might be an independent way to respond to the challenge. I'll
spell out this response more fully in Section~\ref{sec-gettier}.

A fourth response aims to undermine the intuition in a different way.
There is something fundamentally right about the JTB theory of
knowledge, at least if we don't presuppose that the justification, the
J, gets an internalist spin. But it can't be that the theory is
extensionally correct. What is it? My conjecture is that knowledge is
\emph{built}, in the sense described by Karen Bennett
(\citeproc{ref-Bennett2017}{2017}), out of those three components,
justification, truth and belief. Now this needs a notion of building
that doesn't involve necessitation, and spelling that out would be a
task for a different (and longer!) book. I'll try and say enough in
Section~\ref{sec-building} to make it at least minimally plausible that
this conjecture is true, and that it is consistent with IRT.

The fifth response, and the one I want to lean on the most, comes from
Nilanjan Das (\citeproc{ref-DasThesis}{2016}). On the most plausible
ways of articulating what the differences are between JTB and knowledge,
it's not just that the differences will depend on `non-standard'
factors, it's that they will often depend on interests. Whether a belief
is safe, or sensitive, or produced by a reliable method, or apt, or
virtuous, or any other plausible criteria you might want, depends in
part on the interests of the believer. More carefully, whether a belief
satisfies any one of those properties can be counterfactually dependent
on the interests of the believer. So I conclude these objections
massively over-generate. If they are right, they show that practically
every theory of knowledge produced in the last several decades is false.
But it's really implausible that these kinds of considerations could
show that. So the objection fails. I'll end in Section~\ref{sec-das} by
spelling out this response.

\section{So Long JTB}\label{sec-gettier}

The story of investigations into knowledge over the last sixty years is
the story of making the list of things knowledge is sensitive to ever
longer. The thesis of this book is that human interests, in particular
the interests of the would be knower, should be added to that list. But
to defend that thesis, and especially to defend it from the kind of
blank stare objection that I'm worrying about in this chapter, it helps
to have the list in front of us. So I'm going to describe a mundane case
of knowledge, then discuss various ways in which that knowledge could be
lost if the world were different.

Our protagonist, Charlotte, is reading a book about the build up to
World War One. In the base case, the book is Christopher Clark's
\emph{The Sleepwalkers} (\citeproc{ref-Clark2012}{Clark, 2012}), though
in some of the variants we'll discuss she reads a less impressive book.
In it she reads the remarkable story of Henriette Caillaux, the second
wife of anti-war French politician Joseph Caillaux. As you may already
know, Henriette Caillaux shot and killed Gaston Calmette, the editor of
\emph{Le Figaro}, after \emph{Le Figaro} published a string of damaging
articles about Joseph Caillaux. The killing took place on March 16,
1914, and the trial was that July. It ended on July 28 with her
acquittal.

Charlotte reads all of this and believes it. And indeed it is true. And
the book is reliable. Although Charlotte does believe what the book says
about Henriette Caillaux, she is not credulous. She is an attentive
enough, and skilled enough, reader of contemporary history to know when
historians are likely to be going out on a limb, and when they are not
being as clear as one might like in reflecting how equivocal the
evidence is. But Clark is a good historian, and Charlotte is a good
reader, and the beliefs she takes from the book are both true and
supported by the underlying evidence.

Focus for now on this proposition

\begin{quote}
Henriette Caillaux's trial for the murder of Gaston Calmette ended in
her acquittal in late July 1914.
\end{quote}

Call this proposition \emph{p}. In this base case, Charlotte knows that
\emph{p}. But there are ever so many ways in which Charlotte could fail
to have known it. The following three are particularly important .

\begin{quote}
\emph{Variant J}

Charlotte didn't finish the book. She only got as far as the start of
Caillaux's trial, but lost interest in the machinations of the diplomats
in the late stages of the July crisis. Still, she had a strong hunch
that Caillaux would be acquitted and, on just this basis, firmly
believed that she would be.
\end{quote}

\begin{quote}
\emph{Variant T}

Charlotte is in a world where things went just as in the actual world up
to the trial, but then Caillaux was found guilty. Despite this,
Charlotte reads a book that is word-for-word identical to Clark's book.
That is, it falsely says that Caillaux was acquitted, before quickly
moving back to talking about the war. Charlotte believes, falsely, that
\emph{p}.
\end{quote}

\begin{quote}
\emph{Variant B}

Charlotte reads the book to the end, but she can't believe that Caillaux
would be acquitted. The evidence was conclusive, she thought. She is
torn because she also can't really believe a historian would get such a
clear fact wrong. But she also can't believe anyone would be acquitted
in such a trial. So she withholds judgment on the matter, not sure what
actually happened in Caillaux's trial.
\end{quote}

Charlotte does not know that \emph{p} in all three scenarios. These
cases are good evidence that knowledge requires justification, truth,
and belief. In variant J, Charlotte's belief in p is not justified, but
rather a mere hunch, so she doesn't know. In variant T, Charlotte's
belief is incorrect, making it an honest mistake and hence not
knowledge. In variant B, Charlotte lacks knowledge because she doesn't
even believe \emph{p}; she has the evidence, but does not accept it.

There are philosophers who argue that the conditions in all three cases
are not strictly necessary. However, I won't be discussing these points
as it would take us too far afield. Instead, I'll assume that Variant J
demonstrates the need for justification or some form of rationality for
knowledge. Variant T shows that knowledge requires truth, and Variant B
shows that belief or strong acceptance is necessary for knowledge.

For a short while in the mid-20th century, some philosophers thought
these conditions were not merely necessary for knowledge, but jointly
sufficient. To know that \emph{p} just is to have a justified, true
belief that \emph{p}. This became known, largely in retrospect, as the
JTB theory of knowledge. It fell out of fashion dramatically after a
short but decisive criticism was published by Edmund Gettier
(\citeproc{ref-Gettier1963}{1963}). But Gettier's criticism was not
original; he had independently rediscovered a point made by the 8th
century philosopher Dharmottara (\citeproc{ref-Nagel2014}{Nagel, 2014}).
Here is a version of the kind of case Dharmottara discovered.

\begin{quote}
\emph{Variant D}

Charlotte stops reading before the denouement. She thinks Caillaux was
acquitted, not on a hunch, but because she read in another book that
official France was too disorganized in July 1914 to convict any
murderers. This is untrue, but Charlotte used it to arrive at the
correct conclusion that \emph{p}.
\end{quote}

In Variant D, Charlotte lacks knowledge of \emph{p} because basing one's
reasoning on a falsehood typically does not establish knowledge. So
whether one knows is influenced by the accuracy of the grounds for one's
belief. The subsequent variations may not be as straightforward, as
determining whether Charlotte knows \emph{p} will be more controversial.
They are all instances where it is plausible that knowledge is sensitive
to more factors than we've seen so far. The first case is a version of
an example due to Gilbert Harman (\citeproc{ref-Harman1973}{1973:
143ff}).

\begin{quote}
\emph{Variant H}

Charlotte's unfamiliarity with Henriette Caillaux is surprising, because
in her world Caillaux is as infamous as killers like Ned Kelly, Jack the
Ripper, and Lee Harvey Oswald. Her killing of Calmette has been the
subject of numerous novels, plays, and movies. But all these renditions
have a fictionalized ending: Caillaux is convicted and executed. The
authorities were so embarrassed by the actual ending of the trial, where
Caillaux was acquitted, that they successfully conspired to convince the
public that this never happened. Charlotte, coincidentally, is the only
person who hasn't heard of Caillaux's story. When she reads a
word-for-word copy of Clark's book, she doesn't realize it's
controversial and believes that \emph{p}. If she had encountered any of
these older books or plays, she would have assumed her book was mistaken
since it's ``common knowledge'' that Caillaux was convicted.
\end{quote}

Intuitions may vary on this, but in Variant H, I don't think Charlotte
knows that \emph{p}. If that's right, then whether Charlotte knows that
\emph{p} is sensitive not just to the evidence she has, but to the
evidence that is all around her. If she's swimming in a sea of evidence
against \emph{p}, and by the sheerest luck has not run into it, the
evidence she does not have can block knowledge that \emph{p}.

The previous example relied on the possibility of counter-evidence being
everywhere. Possibly all that matters is that the counter-evidence is in
just the right \emph{somewhere}.

\begin{quote}
\emph{Variant S}

In this world, an over-zealous copy-editor makes a last minute change to
the very first printing of Clark's text. Not able to believe that
Caillaux was acquitted - the evidence was so conclusive - they change
the word `acquittal' to `conviction' in the sentence describing the end
of the trial. Happily, this error is quickly caught, and only the first
printing of the book contains the mistake. Charlotte discovered the book
in a second-hand shop, which had two copies - one from the flawed first
printing and one from a later printing. She bought the later one simply
because it was the first one she saw. If she had entered the history
section from the other direction, she would have bought the first
printing and believed that \emph{p} was false.
\end{quote}

Plausibly, Charlotte doesn't know that \emph{p} because it was a matter
of luck that she purchased the later printing instead of the earlier
one. Her method of forming beliefs, which involves buying a seemingly
authoritative history book and accepting its plausible and
well-supported claims, fails in this particular instance in a nearby
possible world where she obtains the other copy. This type of luck is
not compatible with knowledge. In contemporary terminology, a belief
forming method yields knowledge only if it is \emph{safe}. A method is
safe only if it doesn't go wrong in nearby, realistic, scenarios
(\citeproc{ref-Williamson2000}{Williamson, 2000}). So whether one knows
is sensitive to not just the evidence one has, but the evidence one
could easily have had.

Safety in this sense is a tricky notion. In Variant K, it seems to me
that Charlotte does know that \emph{p}.

\begin{quote}
\emph{Variant K}

Charlotte detests reading books on paper, and only ever reads on her
Kindle (an electronic book-reading device). Just like in Variant S,
there was an error in the first printing of Clark's book. But the Kindle
version never contained this error, and in any case, Kindle versions are
updated frequently so even if it had, the error would have been quickly
corrected. Charlotte reads the book on her Kindle, and comes to believe
that \emph{p}.
\end{quote}

In this case, Charlotte believes \emph{p} on good evidence from a
trustworthy source, and there is no realistic possibility where she goes
wrong on this question by trusting this source. That seems to me like
enough for knowledge. I'll return to the difference between Variants S
and K in Section~\ref{sec-das}, but first I want to look at two more
cases.

\begin{quote}
\emph{Variant C}

Charlotte reads Clark's book and believes \emph{p}. But like in Variant
B, she was sure that Caillaux would be convicted. And she still thinks
it is absurd that someone would be acquitted given this evidence. Rather
than responding to these conflicting pressures by withholding judgment,
she responds by both believing that \emph{p} is true, and believing it
is false. She is just inconsistent, like so many of us are in so many
ways.
\end{quote}

It seems to me that in this case, Charlotte does not know that \emph{p}.
The incoherence in her beliefs on this very point undermines her claim
to knowledge. With one more change, we get to the case that motivates
this book.

\begin{quote}
\emph{Variant I}

Charlotte reads the book, and believes that \emph{p}. She is then
offered a bet by a curiously benevolent deity. If she takes the bet, and
\emph{p} is true, she wins a dinner at her favourite bistro, Le Temps
des Cerises. If she takes the bet, and \emph{p} is false, she is cast
into The Bad Place for eternity. If she declines the bet, life goes on
as normal. Now she's deciding what to do.
\end{quote}

By this stage you won't be surprised to hear that I think Variant I is
just like Variant C in being a case where Charlotte lacks knowledge.
What I want to defend is something even stronger than that. In Variants
C and I Charlotte lacks knowledge for just the same reason; it would be
incoherent to believe \emph{p}. Knowledge requires coherence and
rationality, and in Variant I, if Charlotte believes \emph{p}, she is
either irrational or incoherent. I'll come back to this point about the
relationship between Variants C and I in Section~\ref{sec-building}.
First I want to reflect a bit on what we've seen in the earlier cases.

Most of the people who think that it is implausible that interests
matter to knowledge are happy acknowledging the varieties of sensitivity
that are revealed by Variants J, T, B, D, H, S, K and C. (Or at least
they acknowledge most of these; maybe they have idiosyncratic objections
to including one or other kind of sensitivity.) They just think this one
new kind of sensitivity is a bridge too far. It is a bit of a puzzle to
me why we should think sensitivity to interests is more philosophically
problematic than the other kinds of sensitivity we've seen so far. It
might help to get you to share my puzzlement by starting with what looks
like a simple question. What should we call the class of factors
knowledge is sensitive to which revealed by these variants, but which
does not include interests?

One option is to call them the `traditional' factors. Now since
discussion of, say, safety only really became widespread in the 1990s,
the tradition of including it in one's theory of knowledge is quite a
new one. But I don't mind calling new things traditional. I'm
Australian, and we have great traditions like the traditional
Essendon-Collingwood Anzac Day match, which also dates to the 1990s.
This terminology is a bit unstable though. After all, we've been
discussing the role of interests in epistemology since at least 2002
(\citeproc{ref-FantlMcGrath2002}{Fantl \& McGrath, 2002}), so that's
almost long enough to be traditional as well.

Another option is to say that they are the factors that are
truth-connected, or truth-relevant. But there's no way to make sense of
this notion in a way that gets at what is wanted. For one thing, it's
really not obvious that coherence constraints (like we need for Variant
C) are connected to truth. For another, all Variant I suggests is that
we need a principle like the following in our theory of knowledge.

\begin{quote}
Someone knows something only if their evidence is strong enough for them
to rationally treat the thing as a fixed starting point in their
inquiries.
\end{quote}

On the face of it, that's at least as truth-connected as the relatively
uncontroversial requirement that knowledge be based on evidence. It just
says knowledge requires strong evidence. Now, of course, it also says
just how strong the evidence must be depends on what their inquiries
are. Is that problematic? It might be if you think that every aspect of
a requirement on knowledge is truth-relevant.

That last claim really can't be right. Or, at least, it can't be right
unless you believe the JTB theory of knowledge. If the JTB theory is
false, then any premise one might use in a Wright-style argument against
IRT is bound to hav counterexamples. Recall the particular way Wright
argued against IRT

\begin{quote}
X didn't (have enough evidence to) know P at t but does at t* and has
exactly the same body of P-relevant evidence at t* as at t.
(\citeproc{ref-Wright2018}{Wright, 2018: 368})
\end{quote}

If evidence primarily affects justification, then similarity of evidence
at t and t* should just tell us that X is rational in believing P at
both times or neither. Let's say that it's both times. Then as long as
one could be in a JTB-but-not-knowledge-situation at t and a
knowledge-with-the-same-evidence-situation at t*, Wright's conjunction
should be possible. Here's one way that could happen.

\begin{quote}
\emph{Variant S*}

Charlotte reads the book on her Kindle, and believes that \emph{p} at
t\textsubscript{0}. The next day, at t, she can't believe she read that
\emph{p} and reads the book again. It still says that \emph{p}, but this
is bizarre because a new version of the book that says ¬\emph{p} was
pushed out to all Kindles. Due to a network failure, Charlotte's Kindle
was the only one not to get the push. She now doesn't know that
\emph{p}; this case is just like the safety cases and the Harman cases.
The next day at t* a corrected version of the book that says \emph{p} is
pushed out to all Kindles, including Charlotte's. Again perplexed, she
triple checks, and comes to believe, and know, that \emph{p}.
\end{quote}

The ugly conjunction that IRT endorses is something that theories that
are sensitive to safety considerations, or evidential availability
considerations, also endorse. And the true theory is sensitive to one or
other kind of these considerations.

\section{Making Up Knowledge}\label{sec-building}

All that said, I've come to think there is something right about the JTB
theory. Or, as I'd prefer, the RTB theory; as in Rational True
Belief.\footnote{I think it's strange to apply the notion of
  justification to beliefs, and much more natural to talk about rational
  beliefs.} It isn't extensional adequacy; Dharmottara refuted that 1300
years ago. But it can be expressed using the modern\footnote{Well,
  modern if you think it's not the same notion as Meister Eckhart's
  notion of grounding. I'm a little agnostic on that.} notion of
grounding. Or, as I'd prefer, using the notion of a \emph{building
relation} that Karen Bennett (\citeproc{ref-Bennett2017}{2017})
describes.

Consider a very abstractly described case where all of 1-4 are true.

\begin{enumerate}
\def\labelenumi{\arabic{enumi}.}
\tightlist
\item
  S knows that \emph{p}.
\item
  \emph{p}.
\item
  S's attitude to \emph{p} is rational.
\item
  S believes that \emph{p}.
\end{enumerate}

I think that when 1 is true, it is made true by 2-4. Following Bennett,
we might say that the fact expressed in 1 is built from the facts
expressed in 2-4. Now to make this work, we need a notion of building
(or grounding) that's contingent, since 2-4 do not collectively entail
1. Defending the coherence of such a notion in detail would make for a
very different book to this one. But I'll say a few words about why I
think such a notion is going to be needed.

When I say that 1 is made true by 2-4, I mean that it is metaphysically
explained by 2-4. They provide a complete explanation of 1's truth. Now
here's the key step. A complete explanation need not be an entailing
explanation. I'll give a relatively uncontroversial example of this
involving causal explanation, then suggest a different philosophical
example.

It is, famously, hard to explain the origins of World War One. But
without settling all the causal and explanatory issues about the war's
origins, we can confidently make the following two claims.

\begin{description}
\tightlist
\item[C]
Had a giant asteroid struck Sarajevo on June 27, 1914, the war would not
have started when it did.
\item[NE]
It is no part of the explanation of the start of the war that no such
giant asteroid struck Sarajevo on June 27, 1914.
\end{description}

The counterfactual claim, C, can easily be verified by thinking about
the consequences of giant asteroid strikes. (See, for example, the
extinction of the dinosaurs.)

The claim about explanation, NE, can be verified by thinking about how
absurd the task of explanation would be if it were false. For every
possible event that could have changed history, but didn't, we'd have to
include its non-happening in our explanation of the war. The
non-occurrence of every possible alien invasion, mass pandemic, or tulip
mania that could have happened, and would have made a difference, would
be part of our explanation.

So the origins of the war are sensitive to whether there was a giant
asteroid strike, but the lack of a giant asteroid strike is no part of
the complete explanation for why the war took place. Complete causal
explanations can leave out things that are counterfactually relevant to
whether the event took place. That means that they aren't entailing
explanations, since if everything in the complete explanation happened,
but so did an asteroid strike, the war wouldn't have taken place.

We see the same thing in commonsense morality. This is one of the key
points behind Bernard Williams's ``One Thought Too Many'' argument
(\citeproc{ref-Williams1976}{Williams, 1976}). If one's child is
drowning in a pool, one has a reason to dive in and rescue them.
Moreover, it's a complete reason. When someone asks ``Why did you do
that?'', you've given them a complete reason if you say ``My child was
drowning''. And you should accept that answer even if you think there
are cases where that would be the wrong thing to do. Set up your
preferred horror story moral example where diving in to rescue the child
would lead to the destruction of the world. Had that horror story been
actual, it perhaps would not have been morally required to dive into the
pool. But in reality, a complete explanation of why it was required was
that one's child was drowning.

The same thing is true about the relationship between knowledge and
interests. What one knows is always (in principle) sensitive to what
one's interests are. But in cases where one knows, one's knowledge is
not explained by what one's interests are. Rather, it is explained just
by the factors that go into RTB, and perhaps the interplay between them.

Some of the objections to IRT might rely on running together building
and of counterfactual dependence. In their critique of IRT, Gillian
Russell and John Doris (\citeproc{ref-RussellDoris2008}{2009})
repeatedly talk about how implausible it is that a change in interests
can ``make'' one have knowledge. Strictly speaking, I don't think a
change in interests does make one have knowledge. It's true that one
might have knowledge, and not have had that knowledge had one's
interests been different. But it doesn't follow that facts about
interests stand in a making, or building, relationship to facts about
knowledge. They could be, and should be, treated as things relevant to
whether facts about truth, belief and rationality suffice in the
circumstances for knowledge. Those factors, and only those factors, make
for knowledge. That's true whether we're talking about familiar
counterexamples to the JTB (or RTB) theories, or whether we're talking
about interest-relativity.

The distinction between building and counterfactual sensitivity explains
part of why the verdicts of IRT can sound implausible, but it doesn't
explain all of it. To defend IRT from the claim that it renders
implausible verdicts, we need something more. So I'll end this chapter
with an argument by Nilanjan Das that responds to this kind of
objection. The argument is going to be that every plausible theory of
knowledge is committed to some kinds of interest-relativity, and so the
intuitions that my version of IRT violates are violated by every
plausible theory of knowledge. Such intuitions must be wrong, so can't
form the basis of a good objection.

\section{Every Theory is Interest-Relative}\label{sec-das}

Think about the difference between Variant S and Variant K.\footnote{Though
  they are making somewhat different points, there is a resemblance
  between these cases and the cases that Gendler and Hawthorne
  (\citeproc{ref-Gendler2005}{2005}) use to raise trouble for fake barn
  intuitions.} Variant S was meant to be a simple case where Charlotte
does not know something because of a safety violation. Knowledge is
incompatible with a certain kind of luck. To know something is to do
better than make a lucky guess. Charlotte isn't guessing, but she seems
to be lucky in a similar kind of way to the guesser, so she doesn't
know. But in Variant K, she isn't lucky. It's no coincidence that her
book said the correct thing. There is no serious possibility of her
being misled on this point.

Since Charlotte knows that \emph{p} in Variant K, but not in Variant S,
knowledge is sensitive to one's preferred format for reading books. This
is hardly a `truth-relevant' feature, so knowledge isn't only relevant
to truth-relevant features. Knowledge generally depends on whether one
was lucky, and the factors that determine whether one was lucky on an
occasion need not be truth-relevant.

The same patterns recurs in other cases. In Variant H, Charlotte lacked
knowledge because of evidence around her. But imagine a variant of that
variant where Charlotte recently emigrated to a country where no one
ever talks about Henriette Caillaux. In the variant, Charlotte knows
that \emph{p}. So her knowledge of French history is sensitive to her
emigration status. And emigration status isn't truth-relevant or
truth-connected.

If knowledge is sensitive to external factors, and it isn't required
that knowledge be infallible, then knowledge will be sensitive to things
that are not particularly truth-relevant. Any fallibilist, externalist,
theory of knowledge will have to face a version of a reference class
problem, in order to say whether a particular true belief was a matter
of luck. In general, the things that make one be in this reference class
rather than that are not truth-relevant, but they are relevant to
whether one knows.

That's enough to argue against sweeping generalisations about what
knowledge could or could not be sensitive to. Knowledge could be
sensitive to anything, because anything could matter to which reference
class one is in. Nilanjan Das (\citeproc{ref-DasThesis}{2016: 116})
shows that we can say something stronger. Cases like these can be used
to directly argue for interest-relativity, even if one rejects all the
other arguments in the existing literature on IRT.

Knowledge requires not getting it right just by luck. Making that
intuition precise is a lot of work, but it means at least that the
following is true. If the method the person used to form their belief
frequently goes wrong in their actual environment, then even on
occasions that the method gets the right answer, it isn't knowledge. But
what's their environment? It's not just spaces within a fixed distance
from them. Rather, it's spaces that they could easily have ended up
being. It's spaces where it's a matter of luck that they are or aren't
in them. So my environment, in the relevant sense, consists of a network
of college towns and universities throughout the globe, and excludes any
number of places a short drive away. But should I become more interested
in nearby suburbs than far away colleges, my environment would change.
That is to say, environment is an interest-relative notion.

If knowledge is sensitive to what one's environment is like, and one's
environment in the relevant sense is interest-relative, then knowledge
is going to be interest-relative. That's what is going on with Charlotte
and the Kindle. Two people can be alike in what signals they get from
the world, and alike in what the world is like immediately around them,
but be in different environments because of their different interests.
If the method they use to form beliefs on the basis of that signal has
differing levels of success in different environments, then whether they
have knowledge will be sensitive to which environment they are in. That
will depend on any number of `non-traditional' factors, including their
interests.

Now this isn't the only way, or even the main way, that interests matter
to knowledge. But it is a way. And it shows that objections that rely on
the very idea of knowledge being interest-relative must over-generate.
Unless such objections are tied to a rejection of the idea that safety
or reliability or any other external factor matters to knowledge, they
rule out too much.

That concludes the defence of IRT over the last three chapters. The
final two chapters of the book return to setting out the view, going
over two important, but technical, points. First, I argue that rational
belief is not sensitive to interests in quite the same way that
knowledge is. And second, I argue that evidence is interest-relative,
but also in not quite the same way that knowledge is.

\bookmarksetup{startatroot}

\chapter{Rationality}\label{sec-ratbel}

This chapter is about rational belief. My version of IRT allows a new
kind of gap between rational true belief and knowledge, and I'll argue
we should treat this as a philosophical discovery, not a refutation of
the view. Then I'll present two arguments for the possibility of
rationally having credence 1 in a proposition without believing it. The
first is due to Timothy Williamson; the second is new. These arguments
refute two claims about the relationship between belief and credence.
One is a descriptive claim: to believe \emph{p} just is to have credence
in \emph{p} at or above some threshold. The other is a normative claim:
one rationally believes \emph{p} just in case one rationally has
credence in it at or above some threshold. Even if those two arguments
concerning belief and credence one don't work, and rational credence one
does entail rational belief, there are independent arguments against the
descriptive and normative claims if the `threshold' in them is
non-maximal. I'll end the chapter by noting how the view of rational
belief that comes out of IRT is immune to the problems associated with
understanding belief in terms of a credal threshold.

\section{Atomism about Rational Belief}\label{sec-atomism}

In Chapter~\ref{sec-belief} I suggested that the following two
conditions were individually necessary for belief that \emph{p}, and
suggested they might be jointly sufficient.\footnote{This section is
  based on my (\citeproc{ref-Weatherson2012}{2012} ).}

\begin{enumerate}
\def\labelenumi{\arabic{enumi}.}
\tightlist
\item
  In some possible decision problem, \emph{p} is taken for granted.
\item
  For every question the agent is interested in, the agent answers the
  question the same way (i.e., giving the same answer for the same
  reasons) whether the question is asked unconditionally or conditional
  on \emph{p}.
\end{enumerate}

At this point one might think that offering a theory of rational belief
would be easy. It is rational to believe \emph{p} just in case it is
rational to satisfy these conditions. Unfortunately, this nice thought
can't be right. It can be irrational to satisfy these conditions while
rationally believing \emph{p}.

Coraline is like Anisa and Chamari, in that she has read a reliable book
saying that the Battle of Agincourt was in 1415. And she now believes
that the Battle of Agincourt was indeed in 1415, for the very good
reason that she read it in a reliable book.

In front of her is a sealed envelope, and inside the envelope a number
is written on a slip of paper. Let \emph{X} denote that number,
non-rigidly. (So when I say Coraline believes \emph{X}~=~\emph{x}, it
means she believes that the number written on the slip of paper is
\emph{x}, where \emph{x} rigidly denotes some number.) Coraline is
offered the following bet:

\begin{itemize}
\tightlist
\item
  If she declines the bet, nothing happens.
\item
  If she accepts the bet, and the Battle of Agincourt was in 1415, she
  wins \$1.
\item
  If she accepts the bet, and the Battle of Agincourt was not in 1415,
  she loses \emph{X} dollars.
\end{itemize}

For some reason, Coraline is convinced that \emph{X}~=~10. This is very
strange, since she was shown the slip of paper just a few minutes ago,
and it clearly showed that \emph{X}~=~10\textsuperscript{9}. Coraline
wouldn't bet on when the Battle of Agincourt was at odds of a billion to
one. But she would take that bet at 10 to 1, which is what she thinks
she is faced with. Indeed, she doesn't even conceptualise it as a bet;
it's a free dollar she thinks. Right now, she is disposed to treat the
date of the battle as a given. She is disposed to lose this disposition
should a very long odds bet appear to depend on it. But she doesn't
believe she is facing such a bet.

So Coraline accepts the bet; she thinks it is a free dollar. Since
that's when the battle took place, she wins the dollar. All's well that
end's well. But it really was a wildly irrational bet to take. You
shouldn't bet at those odds on something you remember from a history
book. Neither memory nor history books are that reliable. Coraline was
not rational to treat the questions \emph{Should I take this bet?}, and
\emph{Conditional on the Battle of Agincourt being in 1415, should I
take this bet?} the same way. Her treating them the same way was
fortunate - she won a dollar - but irrational.

Yet it seems odd to say that Coraline's belief about the Battle of
Agincourt was irrational. What was irrational was her belief about the
envelope, not her belief about the battle. To say that a particular
disposition was irrational is to make a holistic assessment of the
person with the disposition. But whether a belief is rational or not is,
relatively speaking, atomistic.

That suggests the following condition on rational belief.

\begin{quote}
S's belief that \emph{p} is irrational if

\begin{enumerate}
\def\labelenumi{\arabic{enumi}.}
\tightlist
\item
  S irrationally has one of the dispositions that is characteristic of
  belief that \emph{p}; and
\item
  What explains S having a disposition that is irrational in that way is
  her attitudes towards \emph{p}, not (solely) her attitudes towards
  other propositions, or her skills in practical reasoning.
\end{enumerate}
\end{quote}

Intuitively, Coraline's irrational acceptance of the belief is explained
by her (irrational) belief about what's in the envelope, not her
(rational) belief about the Battle of Agincourt. We can take the
relevant notion of explanation as a primitive if we like; it's in no
worse philosophical shape than other notions we take as a primitive. But
it is possible to spell it out a little more.

Coraline has a pattern of irrational dispositions related to the
envelope. If you offer her \$50 or \emph{X} dollars, she'll take the
\$50. Alternatively, if you change the bet so it isn't about Agincourt,
but is instead about any other thing she has excellent but not quite
conclusive evidence for, she'll still take the bet. On the other hand,
she does not have a pattern of irrational dispositions related to the
Battle of Agincourt. She has this one, but if you change the payouts so
they are not related to this particular envelope, then for all we have
said so far, she won't do anything irrational.

That difference in patterns matters. We know that it's the beliefs about
the envelope, and not the beliefs about the battle, that are explanatory
because of this pattern. We could try and create a reductive analysis of
explanation in clause 2 using facts about patterns, like the way Lewis
tries to create a reductive analysis of causation using similar facts
about patterns in ``Causation as Influence''
(\citeproc{ref-Lewis2004a}{Lewis, 2004}). But doing so would invariably
run up against edge cases that would be more trouble to resolve than
they are worth. There are ever so many ways in which someone could have
an irrational disposition about any particular case. We can imagine
Coraline having a rational belief about the envelope, but still taking
the bet because of any of the following reasons:

\begin{itemize}
\tightlist
\item
  It has been her life goal to lose a billion dollars in a day, so
  taking the bet strictly dominates not taking it.
\item
  She believes (irrationally) that anyone who loses a billion dollars in
  a day goes to heaven, and she (rationally) values heaven above any
  monetary amount.
\item
  She consistently makes reasoning errors about billions, so the
  prospect of losing a billion dollars rarely triggers an awareness that
  she should reconsider things she normally takes for granted.
\end{itemize}

The last one of these is especially interesting. The picture of rational
agency I'm working with here owes a lot to the notion of epistemic
vigilance, as developed by Dan Sperber and co-authors
(\citeproc{ref-SperberEtAl2010}{Sperber et al., 2010}). The rational
agent will have all these beliefs in their head that they will drop when
the costs of being wrong about them are too high, or the costs of
re-opening inquiry into them are too low. They can't reason, at least in
any conscious way, about whether to drop these beliefs, because to do
that is, in some sense, to call the belief into doubt. And what's at
issue is whether they should call the belief into doubt. So what they
need is some kind of disposition to replace a belief that \emph{p} with
an attitude that \emph{p} is highly probable, and this disposition
should correlate with the cases where taking \emph{p} for granted will
not maximise expected utility. This disposition will be a kind of
vigilance. As Sperber and his collaborators show, we need some notion of
vigilance to explain a lot of different aspects of epistemic evaluation.
I think that notion can be usefully pressed into service
here.\footnote{Kenneth Boyd (\citeproc{ref-Boyd2015}{2016}) suggests a
  somewhat similar role for vigilance in the course of defending an
  interest-invariant epistemic theory. Obviously I don't agree with his
  conclusions, but my use of Sperber's work does echo his.}

If you need something like vigilance in your theory of belief, then you
have to allow that vigilance might fail. Maybe some irrational
dispositions can be traced to that failure, and not to any propositional
attitude the decider has. For example, if Coraline systematically fails
to be vigilant when exactly one billion dollars is at stake, then we
might want to say that her belief in \emph{p} is still rational, and she
is practically, rather than theoretically, irrational. (Why could this
happen? Perhaps she thinks of Dr Evil every time she hears the phrase
``One billion dollars'', and this distractor prevents her normally
reliable skill of being vigilant from kicking in.)

If one tries to turn the vague talk of patterns of bets involving one
proposition or another into a reductive analysis of when one particular
belief is irrational, one will inevitably run into hard cases where a
decider has multiple failures. We can't say that what makes Coraline's
belief about the envelope, and not her belief about the battle,
irrational is that if you replaced the envelope, she would invariably
have a rational disposition. After all, she might have some other
irrational belief about whatever we replace the envelope with. Or she
might have some failure of practical reasoning, like a vigilance
failure. Any kind of universal claim, like that it is only bets about
the envelope that she gets wrong, won't do the job we need.

In ``Knowledge, Bets and Interests'', I tried to use the machinery of
credences to make something like this point. The idea was that
Coraline's belief in \emph{p} was rational because her belief just was
her high credence in \emph{p}, and that credence was rational. I still
think that's approximately right, but it can't be the full story. For
one thing, beliefs and credences aren't as closely connected
metaphysically as this suggests. To have a belief in \emph{p} isn't just
to have a high credence, it's to be disposed to let \emph{p} play a
certain role. (This will become important in the next two sections.) For
another thing, it is hard to identify precisely what a credence is in
the case of an irrational agent. The usual ways we identify credences,
via betting dispositions or representation theorems, assume away all
irrationality. But an irrational person might still have some rational
beliefs.

Attempts to generalise accounts of credences so that they cover the
irrational person will end up saying something like what I've said about
patterns. What it is to have credence 0.6 in \emph{p} isn't to have a
set of preferences that satisfies all the presuppositions of such and
such a representation theorem, which in turn maps ones preferences onto
a probability function and a family of utility functions such that
\emph{Pr}(\emph{p})~=~0.6. That can't be right because some people have
credence about 0.6 in \emph{p} while not uniformly conforming to these
constraints. But what makes them intuitive cases of credence roughly 0.6
in \emph{p} is that generally they behave like the perfectly rational
person with credence 0.6 in \emph{p}, and most of the exceptions are
explained by other features of their cognitive system other than their
attitude to \emph{p}.

In other words, we don't have a full theory of credences for irrational
beings right now, and when we get one, it won't be much simpler than the
theory in terms of patterns and explanations I've offered here. So it's
best for now to just understand belief in terms of a pattern of
dispositions, and say that the belief is rational just in case that
pattern is rational. And that might mean that on some occasions
\emph{p}-related activity is irrational even though the pattern of
\emph{p}-related activity is a rational pattern. Any given action, like
any thing whatsoever, can be classified in any number of ways. What
matters here is what explains the irrationality of a particular
irrational act, and that will be a matter of which patterns of
irrational dispositions the actor has.

However we explain Coraline's belief, the upshot is that she has a
rational, true belief that is not knowledge. This is a novel kind of
Dharmottara case. (Or Gettier case for folks who prefer that
nomenclature.) It's not the exact kind of case that Dharmottara
originally described. Coraline doesn't infer anything about the Battle
of Agincourt from a false belief. But it's a mistake to think that the
class of rational, true beliefs that are not knowledge form a natural
kind. In general, negatively defined classes are disjunctive; there are
ever so many ways to not have a property. An upshot of this discussion
of Coraline is that there is one more kind of Dharmottara case than was
previously recognised. But as, for example, Williamson
(\citeproc{ref-WilliamsonLofoten}{2013}) and Nagel
(\citeproc{ref-Nagel2013-Williamson}{2013}) have shown, we have
independent reason for thinking this is a very disjunctive class. So the
fact that it doesn't look anything like Dharmottara's example shouldn't
make us doubt it is a rational, true belief that is not knowledge.

\section{Coin Puzzles}\label{sec-lockecoin}

So rational belief is not identical to rationally having the
dispositions that constitute belief. But nor is rational belief a matter
of rational high credence. In this section and the next I'll argue that
even rational credence 1 does not suffice for rational belief. Then in
the next section I'll run through some relatively familiar arguments
that no threshold short of 1 could suffice for belief. If the argument
of this section or the next is successful, those `familiar arguments'
will be unnecessary. But the two arguments I'm about to give are
controversial even by the standards of a book arguing for IRT, so I'm
including them as backups.

The point of these sections is primarily normative, but it should have
metaphysical consequences. I'm interested in arguing against the
`Lockean' thesis that to believe \emph{p} just is to have a high
credence in \emph{p}. Normally, this threshold of high enough belief for
credence is taken to be interest-invariant, so this is a rival to IRT.
But there is some variation in the literature about whether the phrase
\emph{The Lockean Thesis} refers to a metaphysical claim, i.e., belief
is high credence, or a normative claim, i.e., rational belief is
rational high credence. Since everyone who accepts the metaphysical
claim also accepts the normative claim, and usually takes it to be a
consequence of the metaphysical claim, arguing against the normative
claim is a way of arguing against the metaphysical claim. This section
and the next argue that no matter how high the Lockean sets the
threshold, their theory fails, since rational credence one does not
entail rational belief. In Section~\ref{sec-lockepuzzles} I'll go over
puzzles that arise for Lockean theories that set the threshold below
one.

The first puzzle for Lockeans comes from an argument that Timothy
Williamson (\citeproc{ref-Williamson2007}{2007}) made about certain
kinds of infinitary events. A fair coin is about to be tossed. It will
be tossed repeatedly until it lands heads twice. The coin tosses will
get faster and faster, so even if there is an infinite sequence of
tosses, it will finish in a finite time. (This isn't physically
realistic, but this need not detain us. All that will really matter for
the example is that someone could believe this will happen, and it's
physically possible that someone has that belief.)

Consider the following three propositions

\begin{enumerate}
\def\labelenumi{\Alph{enumi}.}
\tightlist
\item
  At least one of the coin tosses will land either heads or tails.
\item
  At least one of the coin tosses will land heads.
\item
  At least one of the coin tosses after the first toss will land heads.
\end{enumerate}

So if the first coin toss lands heads, and the rest land tails, B is
true and C is false.

Now consider a few versions of the Red-Blue game (perhaps played by
someone who takes this to be a realistic scenario). In the first
instance, the red sentence says that B is true, and the blue sentence
says that C is true. In the second instance, the red sentence says that
A is true, and the blue sentence says that B is true. In both cases, it
seems that the unique rational play is Red-True. But it's really hard to
explain this in a way consistent with the Lockean view.

Williamson argues that we have good reason to believe that the
probability of all three sentences is 1. For B to be false requires C to
be false, and for one more coin flip to land tails. So the probability
that B is false is one-half the probability that C is false. But we also
have good reason to believe that the probabilities of B and C are the
same. In both cases, they are false if a countable infinity of coin
flips lands tails. Assuming that the probability of some sequence having
a property supervenes on the probabilities of individual events in that
sequence (conditional, perhaps, on other events in the sequence), it
follows that the probabilities of B and C are identical. The only way
for the probability that B is false to be half the probability that C is
false, while B and C have the same probability, is for both of them to
have probability 1. Since the probability of A is at least as high as
the probability of B (since it is true whenever B is true, but not
conversely), it follows that the probability of all three is 1.

Since betting on A weakly dominates betting on B, and betting on B
weakly dominates betting on C, we shouldn't have the same attitudes
towards bets on these three propositions. Given a choice between betting
on B and betting on C, we should prefer to bet on B since there is no
way that could make us worse off, and some way it could make us better
off. Given that choice, we should prefer to bet on B (i.e., play
Red-True when B and C are expressed by the red and blue sentences),
because it might be that B is true and C false.

Assume (something the Lockean may not wish to acknowledge) that to say
something might be the case is to reject believing its negation. Then a
rational person faced with these choices will not believe \emph{Either B
is false or C is true}; they will take its negation to be possible. But
that proposition is at least as probable as C, so it too has probability
1. So probability 1 does not suffice for belief. This is a real problem
for the Lockean - no probability suffices for belief, not even
probability 1.

\section{Playing Games}\label{sec-lockegames}

Some people might be nervous about resting too much weight on infinitary
examples like the coin sequence. So I'll show how the same puzzle arises
in a simple, and finite, game.\footnote{This section is based on
  material from my (\citeproc{ref-Weatherson2016}{2016a: 1}).} The game
itself is a nice illustration of how a number of distinct solution
concepts in game theory come apart. (Indeed, the use I'll make of it
isn't a million miles from the use that Kohlberg and Mertens
(\citeproc{ref-KohlbergMertens1986}{1986}) make of it.) To set the
problem up, I need to say a few words about how I think of game theory.
This won't be at all original - most of what I say is taken from
important works by Robert Stalnaker (\citeproc{ref-Stalnaker1994}{1994},
\citeproc{ref-Stalnaker1996}{1996}, \citeproc{ref-Stalnaker1998}{1998},
\citeproc{ref-Stalnaker1999}{1999}). But the underlying philosophical
points are important, and it is easy to get confused about
them.\footnote{At least, I used to get these points all wrong, and
  that's got to be evidence they are easy to get confused about, right?}
So I'll set the basic points slowly, and then circle back to the puzzle
for the Lockeans.\footnote{I'm grateful to the participants in a game
  theory seminar at Arché in 2011, especially Josh Dever and Levi
  Spectre, for very helpful discussions that helped me see through my
  previous confusions.}

Start with a simple decision problem, where the agent has a choice
between two acts \emph{A}\textsubscript{1}and \emph{A}\textsubscript{2},
and there are two possible states of the world,
\emph{S}\textsubscript{1} and \emph{S}\textsubscript{2}, and the agent
knows the payouts for each act-state pair are given by
Table~\ref{tbl-underspecified}.

\begin{longtable}[]{@{}lcc@{}}
\caption{An underspecified decision
problem.}\label{tbl-underspecified}\tabularnewline
\toprule\noalign{}
\endfirsthead
\endhead
\bottomrule\noalign{}
\endlastfoot
& \emph{S}\textsubscript{1} & \emph{S}\textsubscript{2} \\
\emph{A}\textsubscript{1} & 4 & 0 \\
\emph{A}\textsubscript{2} & 1 & 1 \\
\end{longtable}

What to do? I hope you share the intuition that it is radically
underdetermined by the information I've given you so far. If
\emph{S}\textsubscript{2} is much more probable than
\emph{S}\textsubscript{1}, then \emph{A}\textsubscript{2} should be
chosen; otherwise \emph{A}\textsubscript{1} should be chosen. But I
haven't said anything about the relative probability of those two
states.

Now compare that to a simple game. The players are Row and Column; Row
will choose a row, Column will choose a column, and then the payouts
will be given by the cell at the row and column's intersection. Row has
two choices, which I'll call \emph{A}\textsubscript{1} and
\emph{A}\textsubscript{2}. Column also has two choices, which I'll call
\emph{S}\textsubscript{1} and \emph{S}\textsubscript{2}. It is common
knowledge that each player is rational, and that the payouts for the
pairs of choices are given in Table~\ref{tbl-simple-game}. (As always,
Row's payouts are given first.)

\begin{longtable}[]{@{}lcc@{}}
\caption{A simple game.}\label{tbl-simple-game}\tabularnewline
\toprule\noalign{}
\endfirsthead
\endhead
\bottomrule\noalign{}
\endlastfoot
& \emph{S}\textsubscript{1} & \emph{S}\textsubscript{2} \\
\emph{A}\textsubscript{1} & 4, 0 & 0, 1 \\
\emph{A}\textsubscript{2} & 1, 0 & 1, 1 \\
\end{longtable}

What should Row do? This one is easy. Column gets 1 for sure if she
plays \emph{S}\textsubscript{2}, and 0 for sure if she plays
\emph{S}\textsubscript{1}. So she'll play \emph{S}\textsubscript{2}. And
given that she's playing \emph{S}\textsubscript{2}, it is best for Row
to play \emph{A}\textsubscript{2}.

The game in Table~\ref{tbl-simple-game} is just a variant on the
decision problem in Table~\ref{tbl-underspecified}. The relevant states
of the world are choices of Column. Unlike the decision problem, there
is a determinate answer to what Row should do in the game. More
importantly for present purposes, the game can be solved without
explicitly saying anything about probabilities. This is because we
deduce all we need to know about probabilities from the assumption that
Column is rational. Since Column is rational, they will play
\emph{S}\textsubscript{2}. Since Column will play
\emph{S}\textsubscript{2}, Row should play \emph{A}\textsubscript{2}.

Looking at games this way helps understand why theorists sometimes think
of game theory as `interactive epistemology'
(\citeproc{ref-Aumann1999}{Aumann, 1999}). The theorist's work is to
solve for what a rational agent should think other rational agents in
the game should do. This is why game theory makes heavy use of
equilibrium concepts. As theorists, we adopt a theory of rational
choice, and see what happens if that theory is common ground amongst the
players. In effect, we treat \emph{rationality} as an unknown variable
that we solve for given premises about which choices are rational in
which games.\footnote{If we're solving for a variable, what are the
  equations we're using as input. The standard methodology is to say
  they are intuitions. Game theorists make as much use of intuitions
  analytic philosophers. See, for example, Cho \& Kreps
  (\citeproc{ref-ChoKreps1987}{1987}).} Not surprisingly, there are
going to be multiple solutions to the puzzles we face.

This way of thinking naturally leads to the epistemological
interpretation of mixed strategies. The most important solution concept
in modern game theory is the Nash equilibrium. A set of moves is a Nash
equilibrium if no player can improve their outcome by deviating from the
equilibrium, conditional on no other player deviating. In many simple
games, the only Nash equilibria involve mixed strategies.
Table~\ref{tbl-death-damascus} is one simple example.

\begin{longtable}[]{@{}lcc@{}}
\caption{Death in Damascus as a
game.}\label{tbl-death-damascus}\tabularnewline
\toprule\noalign{}
\endfirsthead
\endhead
\bottomrule\noalign{}
\endlastfoot
& \emph{S}\textsubscript{1} & \emph{S}\textsubscript{2} \\
\emph{A}\textsubscript{1} & 0, 1 & 10, 0 \\
\emph{A}\textsubscript{2} & 9, 0 & -1, 1 \\
\end{longtable}

The only Nash equilibrium for this game is that Row plays a mixed
strategy playing both \emph{A}\textsubscript{1} and
\emph{A}\textsubscript{2} with probability ½, while Column plays the
mixed strategy that gives \emph{S}\textsubscript{1} probability 0.55,
and \emph{S}\textsubscript{2} with probability 0.45.

Now what is a mixed strategy? The \emph{metaphysical} interpretation of
mixed strategies is that players use some randomising device to pick
what to do. This interpretation is often implicit in the way many
textbooks introduce mixed strategies.

But the understanding of game theory as interactive epistemology
naturally suggests an \emph{epistemological} interpretation of mixed
strategies, as Stalnaker argues.

\begin{quote}
One could easily \ldots{} {[}model players{]} \ldots{} turning the
choice over to a randomizing device, but while it might be harmless to
permit this, players satisfying the cognitive idealizations that game
theory and decision theory make could have no motive for playing a mixed
strategy. So how are we to understand Nash equilibrium in model
theoretic terms as a solution concept? We should follow the suggestion
of Bayesian game theorists, interpreting mixed strategy profiles as
representations, not of players' choices, but of their beliefs.
(\citeproc{ref-Stalnaker1994}{Stalnaker, 1994: 57--8})
\end{quote}

For our purposes, the important thing about the epistemological
interpretation of mixed strategies is that it allows us to make sense of
the difference between playing a pure strategy and playing a mixed
strategy where one of the `parts' of the mixture is played with
probability one.

With that in mind, consider the game I'll call Up-Down.\footnote{In
  earlier work I'd called it Red-Green, but this is too easily confused
  with the Red-Blue game that plays such an important role in
  Chapter~\ref{sec-interests}.} Informally, in this game \emph{A} and
\emph{B} must each play a card with an arrow pointing up, or a card with
an arrow pointing down. I will capitalise \emph{A}'s moves, i.e.,
\emph{A} can play UP or DOWN, and italicise \emph{B}'s moves, i.e.,
\emph{B} can play \emph{up} or \emph{down}. If at least one player plays
a card with an arrow facing up, each player gets \$1. If two cards with
arrows facing down are played, each gets nothing. Each cares just about
their own wealth, so getting \$1 is worth 1 util. All of this is common
knowledge. More formally, the payouts are given in
Table~\ref{tbl-up-down}, with \emph{A} on the row and \emph{B} on the
column.

\begin{longtable}[]{@{}lcc@{}}
\caption{The Up-Down game.}\label{tbl-up-down}\tabularnewline
\toprule\noalign{}
\endfirsthead
\endhead
\bottomrule\noalign{}
\endlastfoot
& \emph{up} & \emph{down} \\
UP & 1, 1 & 1, 1 \\
DOWN & 1, 1 & 0, 0 \\
\end{longtable}

I'll first work through Up-Down assuming Uniqueness: the epistemological
theory that there is precisely one rational credence to have in any
salient proposition about how the game will play. Some philosophers
think that Uniqueness always holds
(\citeproc{ref-White2005-WHIEP}{White, 2005}). I join with those such as
North (\citeproc{ref-North2010}{2010}) and Schoenfield
(\citeproc{ref-Schoenfield2013}{2013}) who don't. I'll assume it for now
because it's easiest to understand the analysis I'll offer with this
assumption in place; later we'll relax the assumption.

Up-Down is symmetric. So given Uniqueness, \emph{A} and \emph{B} should
have the same probability of playing UP/\emph{up}. Call this common
probability \emph{x}. It cannot be that x\emph{~\textless~1. }A\emph{'s
expected return from UP is 1, while the expected return from DOWN is
}x\emph{. If }x* \textless{} 1 and \emph{A} is rational, they'll
definitely play UP. If \emph{A} will definitely play UP, the probability
they'll play UP is 1, contradicting the assumption that
\emph{x}~\textless~1.

So we know \emph{x}~=~1. Arguably, we don't know that \emph{A} will play
UP. Assume we could know this. Whatever reason we would have for
concluding that would be a reason for any rational person to conclude
that \emph{B} will play \emph{up}. \emph{A} is rational, so \emph{A}
will conclude this. So \emph{A}'s expected return from either strategy
is 1. So \emph{A} should be indifferent between UP and DOWN. Since all
we know about \emph{A} is that they are rational, and we know they are
indifferent between UP and DOWN, we can't conclude, i.e., can't know,
they will play UP.

There is an obvious objection to this argument. At one point I moved
from the claim that \emph{A}'s expected return from UP and DOWN is the
same, to the conclusion that \emph{A} has just as much reason to play UP
and DOWN. That looks like it is assuming that expected utility
maximisation is the full theory of rationality. That, in turn, is
something we might want to question.

In Chapter~\ref{sec-ties} I said that expected utility maximisation
can't be the right theory of decision for agents who face non-trivial
comptutational costs. This shouldn't be relevant here. \emph{A} and
\emph{B} face pretty simple computations, and we can assume that the
cost of those computations is negligible for each of them.

A more serious objection is that \emph{A} has a reason beyond utility
maximisation to play UP, namely that UP weakly dominates DOWN. After
all, there's one possibility on the table where UP does better than
DOWN, and none where RED does better. So perhaps even if UP and DOWN
have the same expected utility, there is a reason to play UP.

At least as I've set up this game, this isn't an extra reason \emph{A}
has. To see this, it helps to compare the case to the kinds of games
where Stalnaker (in the papers cited above) thinks that weak dominance
does provide a distinct reason to make a choice. He is talking about
games where the agents' attitude towards the possible payouts is
different to their attitude towards each other. For example, the players
may have common knowledge of the payouts, but only common belief in the
rationality of each other. Or perhaps they even have rational, true
belief in the rationality of each other, but crucially not knowledge. If
that's right, but only if that's right, then it makes sense to use weak
dominance reasoning.

The key motivation behind weak dominance reasoning is that taking a
weakly dominated option is a needless risk. If UP will definitely return
1, while DOWN may return 0, then DOWN is risky in a way that UP is not.
The notion of \emph{risk} here need not be understood probabilistically.
Even if it the probability that DOWN will return 1 is 1, there is still
that payout of 0 sitting on the table, and so there is a risk.

Here we need to slow down. There is no outcome on the table where UP
returns 1. But if the table is wrong, then UP might return 0. It might
return anything at all. The only way that DOWN is risky while UP is not
is if there is no risk that the table is mistaken.

Now one might object, we stipulated that \emph{A} knows the table is
correct, there is no way it can be mistaken. We also stipulated that
\emph{A} knows that \emph{B} is rational. So if rationality implies
playing UP/\emph{up}, there is no way that DOWN can return 0.

This is why Stalnaker's assumption that there is an asymmetry between
the players' attitude towards the table and towards each other matters.
If the players have a stronger attitude towards the rationality of each
other than towards the correctness of the table, there is a sense in
which irrational outcomes on the table are more of a risk than outcomes
that are not on the table.

However, if the players think the players being irrational is exactly as
live a possibility as the table being mistaken, then it is unreasonable
to treat outcomes on the table which are only reached when the players
are irrational as more relevant to decisions than outcomes not on the
table at all.

That's why weak dominance reasoning is inappropriate in the Up-Down
game. In some sense there is a risk DOWN could lead to a payout of 0.
\emph{B} might make an irrational move, even though, by stipulation,
\emph{A} knows that they will not. In the very same sense, there is a
risk UP could lead to a payout of 0. The table could be wrong, even
though \emph{A} knows that it is not.

That's why the possibility of weak dominance reasoning doesn't undermine
the reductio argument I've offered against UP/\emph{up} being the
uniquely rational play. It also helps us see why ultimately we don't
need the assumption of Uniqueness to generate the objection.

Let's state the argument more carefully without Uniqueness. Assume,
again for reductio, that some rational person \emph{C} has credence
ε~\textgreater~0 that \emph{A} will play DOWN. (It could be that
\emph{C} is a theorist, like us, or they could be one of the players.)
We will now try to build a full model of \emph{C}'s attitudes towards
the game.

Since it is common ground that \emph{A} is an expected utility
maximiser, \emph{C} must have at least credence ε that \emph{A} has
credence 1 that \emph{B} will play \emph{up}. Is this coherent?

One reason to think not is that even without Uniqueness, it is strange
to think that one rational agent could regard a possibility as
infinitely less likely than another, given the exact same evidence
evidence.

Another reason to think this combination of views is incoherent is that
without Uniqueness, the possibility of weak dominance reasoning comes
back. If \emph{C} has credence ε that \emph{A} will play DOWN, then it
is consistent with \emph{B}'s rationality that \emph{B} has credence ε
that \emph{A} will play DOWN. Somehow \emph{C} must have credence 1 that
\emph{B} does not have the same credences they do about what \emph{A}
will do, even though they and \emph{B} have exactly the same evidence.

Uniqueness implies that \emph{C} should have credence 1 that \emph{B}
will have the same credences as they do. I think Uniqueness is wrong, so
I don't think that's a plausible constraint. But it's another thing to
say that \emph{C} should have credence 0 that someone in the same
evidential situation as them has the same credences.

So even without Uniqueness, there are two reasons to think that it is
wrong to have credence ε~\textgreater~0 that \emph{A} will play DOWN.
Further, the argument that we can't know \emph{A} will play UP did not
rely on Uniqueness. So this is a case where credence 1 doesn't imply
knowledge, and since the proof is known to us, and full belief is
incompatible with knowing that you can't know, this is a case where
credence 1 doesn't imply full belief. So whether \emph{A} plays UP, like
whether the coin will ever land tails, is a case where belief comes
apart from high credence, even if by high credence we literally mean
credence one. This is a problem for the Lockean, and, like Williamson's
coin, it is also a problem for the view that belief is credence 1.

\section{Puzzles for Lockeans}\label{sec-lockepuzzles}

I've already mentioned two classes of puzzles, those to do with infinite
sequences of coin tosses and those to do with weak dominance in games.
But there are other puzzles that apply especially to the kind of Lockean
who identifies belief with credence above some non-maximal,
interest-invariant, threshold.

\subsection{Arbitrariness}\label{sec-lockearb}

The first problem for the Lockeans, and in a way the deepest, is that it
makes the boundary between belief and non-belief arbitrary. This is a
point that was well made some years ago now by Robert Stalnaker
(\citeproc{ref-Stalnaker1984}{1984: 91}). Unless these numbers are made
salient by the environment, there is no special difference between
believing \emph{p} to degree 0.9876 and believing it to degree 0.9875.
But if the belief threshold is 0.98755, this will be the difference
between believing \emph{p} and not believing it, which is an important
difference.

The usual response to this is to say that the boundary is
vague.\footnote{Versions of this response are made by Richard Foley
  (\citeproc{ref-Foley1993}{1993} Ch. 4), David Hunter
  (\citeproc{ref-Hunter1996}{1996}) and Matthew Lee
  (\citeproc{ref-Lee2017b}{2017b}).} This won't help at all on theories
of vagueness which endorse classical logic, like epistemicism
(\citeproc{ref-Williamson1994}{Williamson, 1994}), or supervaluationism,
or my preferred comparative truth theory
(\citeproc{ref-Weatherson2005b}{Weatherson, 2005b}). On any of those
theories there will still be a true existential claim that the threshold
exists and is unimportant.

Even without settling what the right theory of vagueness is, we can see
why this can't be right by thinking about what it means to say that a
boundary is a vague point on a scale. Most comparative adjectives are
vague, and the vagueness consists in which vague point on a scale is the
boundary for their application. For example, whether a day is hot
depends on whether it is above some vague point on a temperature scale.
Vague comparative adjectives like `hot' don't enter into non-trivial
lawlike generalisations. There are laws involving the underlying scale,
i.e., temperature, but no laws that are distinctively about the days
that are hot. The most you can do is give some kind of generic claim.
For instance, you can say that hot days are exhausting, or that
electricity use is higher on hot days. But these are generics, and the
interesting law-like claims will involve degrees of heat, not the
hot/non-hot binary.

It's a fairly central presupposition of this book that belief is more
connected to lawlike psychological generalisations than these mere
generics. Folk psychology is full of lawlike generalisations that are
essentially about belief. These are social science laws, not laws of
fundamental physics, so the laws in question with be exception-ridden,
ceteris paribus laws. But they are laws nonetheless; they are
explanatory and counterfactually resilient.

The Lockean fundamentally doesn't believe that these generalisations of
folk psychology are anything more than generics, so this is a somewhat
question-begging argument. The Lockean thinks the real laws are about
credences, just like the real laws about hot days concern the underlying
temperature scale. So my assumption that there are folk psychological
laws about belief is strictly speaking question-begging. Nonetheless, it
is true. I suspect any argument I could give for it would be less
plausible than simply stating the claim, so I won't really try to argue
for it. What I will do is illustrate why I believe it, and hopefully
remind you why you believe it too.

Start by considering this generalisation.

\begin{itemize}
\tightlist
\item
  If someone wants an outcome O, and they believe that doing X is the
  only way to get O, and they believe that doing X will neither incur
  any costs that are large in comparison to how good O is, nor prevent
  them being able to do something that brings about some other outcome
  that is comparatively good, then they will do X.
\end{itemize}

This isn't a universal - some people are just practically irrational.
But it's stronger than just a generic claim about high temperatures. It
would still be true if the world were different in ever so many ways,
and in cases where the person does X, this generalisation is part of the
explanation for why they do X.

The Lockean denies almost all of that. They say this principle has
widespread counterexamples, even among rational agents. Even when it is
true, it isn't explanatory. Rather, it is a summary of some genuinely
explanatory claims about the relationship between credence and action.

For example, the Lockean thinks that someone in Blaise's situation
satisfies all the antecedents and qualifications in the principle. They
want the child to have a moment of happiness. They believe (i.e., have a
very high credence that) taking the bet will bring about this outcome,
will have no costs at all, and will not prevent them doing anything
else. Yet they will not think that people in Blaise's situation will
generally take the bet, or that it would be rational for them to take
the bet, or that taking the bet is explained by these high credences.

That's what's bad about making the belief/non-belief distinction
arbitrary. It means that generalisations about belief are going to be
not particularly explanatory, and are going to have systematic (and
highly rational) exceptions. We should expect more out of a theory of
belief.

\subsection{Correctness}\label{sec-lockecorrect}

I've talked about this one a bit in Section~\ref{sec-mecorrect}, so I'll
be brief here. Beliefs have correctness conditions. To believe \emph{p}
when \emph{p} is false is to make a mistake. That might be an excusable
mistake, or even a rational mistake, but it is a mistake. On the other
hand, having an arbitrarily high credence in \emph{p} when \emph{p}
turns out to be false is not a mistake. So having high credence in
\emph{p} is not the same as believing \emph{p}.

Matthew Lee (\citeproc{ref-Lee2017a}{2017a}) argues that the versions of
this argument by Ross and Schroeder
(\citeproc{ref-RossSchroeder2014}{2014}) and Fantl and McGrath
(\citeproc{ref-FantlMcGrath2009}{2009}) are incomplete because they
don't provide a conclusive case for the premise that having a high
credence in a falsehood is not a mistake. But this gap can be plugged.
Imagine a scientist, call her Marie, who knows the correct theory of
chance for a given situation. She knows that the chance of \emph{p}
obtaining is 0.999. (If you think the belief/non-belief threshold is
greater than 0.999, just increase this number, and change the resulting
dialogue accordingly.) And her credence in \emph{p} is 0.999, because
her credences track what she knows about chances. She has the following
exchange with an assistant.

\begin{quote}
ASSISTANT: Will \emph{p} happen?\\
MARIE: Probably. It might not, but there is only a one in a thousand
chance of that. So \emph{p} will probably happen.
\end{quote}

To their surprise,~\emph{p} does not happen. But Marie did not make any
kind of mistake here. Indeed, her answer to assistant's question was
exactly right. But if the Lockean theory of belief is right, and false
beliefs are mistakes, then Marie did make a mistake. So the Lockean
theory of belief is not right.

\subsection{Moorean Paradoxes}\label{sec-lockemoore}

The Lockean says other strange things about Marie. By hypothesis, she
believes that \emph{p} will obtain. Yet she certainly seems sincere when
she says it might not happen. So she believes both \emph{p} and it might
not be that \emph{p}. This looks like a Moore-paradoxical belief, yet in
context it seems completely banal.

The same thing goes for Chamira. Does she believe the Battle of
Agincourt was in 1415? Yes, say the Lockeans. Does she also believe that
it might not have been in 1415? Yes, say the Lockeans, that is why it
was rational of her to play Red-True, and it would have been irrational
to play Blue-True. So she believes both that something is the case, and
that it might not be the case. This seems irrational, but Lockeans
insist that it is perfectly consistent with her being a model of
rationality.

Back in Section~\ref{sec-orthodoxmoore} I argued that this kind of thing
would be a problem for any kind of orthodox theory. And in some sense
all I'm doing here is noting that the Lockean really is a kind of
orthodox theorist. But the argument that the Lockean is committed to the
rationality of Moore-paradoxical claims doesn't rely on those earlier
arguments; it's a direct consequence of their view applied to simple
cases like Marie and Chamira.

\subsection{Closure and the Lockean Theory}\label{sec-closure}

The Lockean theory makes an implausible prediction about
conjunction.\footnote{This subsection draws on material from my
  (\citeproc{ref-Weatherson2016}{2016a}).} It says that someone can
believe two conjuncts, yet actively refuse to believe the conjunction.
Here is how Stalnaker puts the point.

\begin{quote}
Reasoning in this way from accepted premises to their deductive
consequences (\emph{p}, also \emph{q}, therefore \emph{r}) does seem
perfectly straightforward. Someone may object to one of the premises, or
to the validity of the argument, but one could not intelligibly agree
that the premises are each acceptable and the argument valid, while
objecting to the acceptability of the conclusion.
(\citeproc{ref-Stalnaker1984}{Stalnaker, 1984: 92})
\end{quote}

On the Lockean view, this happens all the time, and is intelligible.
According to the Lockeans, it is easy to find
triples〈\emph{S},~\emph{A},~\emph{B}〉such that

\begin{itemize}
\tightlist
\item
  \emph{S} is a rational agent.
\item
  \emph{A} and \emph{B} are propositions.
\item
  \emph{S} believes \emph{A} and believes \emph{B}.
\item
  \emph{S} does not believe \emph{A} ∧ \emph{B}.
\item
  \emph{S} knows that she has all these states, and consciously
  reflectively endorses them.
\end{itemize}

One argument against the Lockean is that there are no such triples, at
least when \emph{S} is rational. That's what I think. Even if I'm wrong,
there is a separate argument against the Lockean. The Lockean doesn't
just think these triples are possible, they think they are common.
That's because for any \emph{t}~∈~(0, 1) you care to pick, triples of
the form〈\emph{S},~\emph{C},~\emph{D}〉are common.

\begin{itemize}
\tightlist
\item
  \emph{S} is a rational agent.
\item
  \emph{C} and \emph{D} are propositions.
\item
  \emph{S}'s credence in \emph{C} is greater than \emph{t}, and her
  credence in \emph{D} is greater than \emph{t}.
\item
  \emph{S}'s credence in \emph{C} ∧ \emph{D} is less than \emph{t}.
\item
  \emph{S} knows that she has all these states, and reflectively
  endorses them.
\end{itemize}

David Christensen (\citeproc{ref-Christensen2005}{2005}) argues from
considerations about the preface paradox to the conclusion that triples
like〈\emph{S},~\emph{A},~\emph{B}〉are possible. His argument is
non-constructive; he doesn't state a particular triple that clearly
satisfies all the constraints, just argues that one must exist. I'm
sceptical about that argument, but even if it worked, it wouldn't show
what's needed. What's needed is that triples satisfying the constraints
I set out for 〈\emph{S},~\emph{A},~\emph{B}〉are just as common as
triples satisfying the constraints I set out for
〈\emph{S},~\emph{C},~\emph{D}〉, for at least some value \emph{t}.
Considerations about esoteric cases like the preface paradox can't show
that, and I haven't seen any other argument that even attempts to show
it.

\section{Solving the Challenges}\label{sec-solving}

Critiquing other theories for their inability to meet a challenge that
one's own theory cannot meet is unfair. So I'll conclude this chapter by
showing that the six problems I have presented for Lockeans do not pose
a problem for my interest-relative theory of (rational) belief. I've
already discussed the points about correctness in
Section~\ref{sec-mecorrect}, and about closure in chapters
\ref{sec-knowledge} and \ref{sec-ties}, and there isn't much to be
added. However, I would like to briefly touch upon the remaining four
problems.

\subsection{Coins}\label{coins}

To believe \emph{p}, one must have a disposition to take it for granted.
A rational person prefers to bet on logically weaker propositions
instead of logically stronger ones in the coin case. They would not take
the logically stronger propositions for granted because if they did,
they would be indifferent between the bets. Therefore, they would not
believe that one of the coin flips after the second will land heads or
even that one of the coin flips after the first will land heads. This is
the correct outcome. The rational person assigns probability one to
these propositions but does not believe them.

\subsection{Games}\label{games}

In the Up-Down game, if the rational person believed that the other
player would play \emph{up}, they would be indifferent between UP and
DOWN. But it's irrational to be indifferent between those options, so
they wouldn't have the belief. They will think the probability that the
other person will play up is one - what else could it be? But they will
not believe it on pain of incoherence.

\subsection{Arbitrariness}\label{arbitrariness}

According to IRT, the difference between belief and non-belief is the
difference between willingness and unwillingness to take something as
given in inquiry. This is far from an arbitrary difference. Moreover, it
is a difference that supports lawlike generalisations. If someone
believes that \emph{p}, and believes that given \emph{p}, \emph{A} is
better than \emph{B}, they will prefer \emph{A} to \emph{B}. This isn't
a universal truth; people make mistakes. But nor is it merely a
statistical generalisation. Counterexamples to it are things to be
explained, while instances are explained by the underlying pattern.

\subsection{Moore}\label{moore}

In many ways the guiding aim of this project was to avoid the kind of
Moore-paradoxicality the Lockean falls into. So it shouldn't be a
surprise that we avoid it here. If someone shouldn't do something
because \emph{p} might be false, that's conclusive evidence that they
don't know that \emph{p}. And it's conclusive evidence that either they
don't rationally believe \emph{p}, or they are making some very serious
mistake in their reasoning. In the latter case, the reason they are
making a mistake is not that \emph{p} might be false, but that they have
a seriously mistaken belief about the kind of choice they are facing. So
we can never say that someone knows, or rationally believes,~\emph{p},
but their choice is irrational because \emph{p} might be false.

\bookmarksetup{startatroot}

\chapter{Evidence}\label{sec-evidence}

\section{A Puzzle About Evidence}\label{sec-evpuzzle}

In Section~\ref{sec-orthodoxevidence}, I argued that evidence can be
interest-relative. The key example involved someone I called Parveen.
Recall that she's in a restaurant and notices an old friend, Rahul,
across the restaurant. The conditions for detecting people aren't
perfect, and she's surprised Rahul is here. Still, we'd ordinarily say
it is part of her evidence that Rahul is in this restaurant. She doesn't
infer this from other facts, and she would not be called on to defend it
if she relies on it in ordinary circumstances. She then plays the
Red-Blue game, with these sentences.

\begin{itemize}
\tightlist
\item
  The red sentence is \emph{Two plus two equals four}.
\item
  The blue sentence is \emph{Rahul is in this restaurant}.
\end{itemize}

The key premises for the argument that evidence is interest-relative
are:

\begin{itemize}
\tightlist
\item
  The unique rational play for Parveen is Red-True; and
\item
  If evidence is interest-invariant, it is rational for Parveen to play
  Blue-True.
\end{itemize}

That argument shows that evidence is interest-relative. But it raises,
without answering, two big questions:

\begin{enumerate}
\def\labelenumi{\arabic{enumi}.}
\tightlist
\item
  When do interests matter for evidence?
\item
  When do interests matter for knowledge?
\end{enumerate}

I used to think that there was an easy answer to the second question. A
change in interest causes one to lose knowledge that \emph{p} iff one
becomes interested in a question which, given one's evidence, is
rationally answered differently depending on whether or not one answers
the question conditional on \emph{p}. This answer is true as far as it
goes, but it isn't particularly explanatory unless one holds fixed the
evidence between the earlier and later set of interests. And that is
just what I said should not be held fixed.

The aim of this chapter is to answer both questions simultaneously.

\section{A Simple, but Incomplete, Solution}\label{sec-simplesolution}

To keep things relatively simple, I'll assume in this chapter that
Parveen is an expected utility maximiser. More carefully, I'll assume
that the reasons covered in Chapter~\ref{sec-ties} about why expected
utility theory is only an approximation to the correct theory of
rational choice are not relevant. From here on, we'll assume we're in a
situation where expected utility theory is close enough to the true
theory of rational choice.

At a very high level of abstraction, we can think about the problem
facing Parveen (or anyone else whose evidence might be
interest-sensitive), as follows. They have some option \emph{o}, and
given their interests it matters whether the expected value of \emph{o}
is above or below \emph{x}. I'll write \emph{v}(•) for the function from
options to their expected value, so the question here is whether or not
\emph{v}(\emph{o}) us at least \emph{x}.

There is some background \emph{K} that is uncontroversially in Parveen's
evidence. There is some further proposition \emph{p} which might or
might not be in her evidence; that's what the change of interests calls
into question. It is uncontroversial that her evidence includes some
background \emph{K}, and controversial whether it includes some
contested proposition \emph{p}. For any \emph{q} in
\emph{K},~\emph{v}(\emph{o}~\textbar~\emph{q})~=~\emph{v}(\emph{o}).
That is, expected values are conditional on evidence.

A common idealisation helps capture this last idea. Assume there is a
prior value function \emph{v}\textsuperscript{-}, with a similar
metaphysical status to the prior probability function. Then for any
choice
\emph{c},~\emph{v}(\emph{c})~=~\emph{v}\textsuperscript{-}(\emph{c}~\textbar~\emph{E}),
where \emph{E} is the evidence Parveen has.

Now I can offer a simple, but incomplete, solution to question 2,
assuming \emph{p} is the only proposition whose status as evidence is
put into question by the interests-shift, and the only shift in
interests is that the question of whether \emph{v}(\emph{o}) ≥ \emph{x}
is now relevant. Then she knows \emph{p} only if
{[}\emph{v}\textsuperscript{-}(\emph{o}\textbar{}\emph{K})~+~\emph{v}\textsuperscript{-}(\emph{o}\textbar{}\emph{K}~∧~\emph{p}){]}/2~≥~\emph{x}.
That is, if \emph{p}'s status as evidence is questionable, the relevant
`value' for \emph{o} is the average of its expected value with and
without \emph{p} being evidence.

That gets the right answer about what Parveen should do. Her evidence
may or may not include that Rahul is in the restaurant. If it does, then
Blue-True has a value of \$50. If it does not, then Blue-True's value is
somewhat lower. Even if the evidence includes that someone who looks a
lot like Rahul is in the restaurant, the value of Blue-True might only
be \$45. Averaging them out, the value is less than \$50. It would only
be rational to play Blue-True if was worth \$50. So she shouldn't play
Blue-True.

Great! Well, great except for two monumental problems. The first is that
it only handles this very special case. The second is that the formula
used, take the arithmetic mean of the values with and without the
evidence, is barely better than arbitrary. It gets one thing right, it
says Parveen shouldn't play Blue-True, but it's hardly alone in having
that virtue.

Pragmatic encroachment starts with a very elegant, very intuitive,
principle: you only know the things you can reasonably take to be
settled for the purposes of current deliberation. This arbitrary
averaging formula is not elegant or intuitive.

Happily, the two problems have a common solution. Setting it out
requires going over recent work on coordination games.

\section{The Radical Interpreter}\label{sec-radicalinterpretation}

William Harper (\citeproc{ref-Harper1986}{1986}) pointed out that many
decision problems are really better thought of as games. For instance,
Newcomb's Problem can be represented by the game in
Table~\ref{tbl-Newcomb}, with the human as Row and the demon as Column.

\begin{longtable}[]{@{}rcc@{}}
\caption{Newcomb's Problem as a game.}\label{tbl-Newcomb}\tabularnewline
\toprule\noalign{}
\endfirsthead
\endhead
\bottomrule\noalign{}
\endlastfoot
& Predict 1 Box & Predict 2 Boxes \\
Choose 1 Box & 1000, 1 & 0, 0 \\
Choose 2 Boxes & 1001, 0 & 1, 1 \\
\end{longtable}

There is a unique equilibrium of this game: the bottom right corner. The
reason it's the unique equilibrium is similar to the reason that
two-boxers say to take two boxes: no other option is ratifiable for both
players.

This section will be centered around a game that is only slightly more
complicated. I call it The Interpretation Game. The game has two
players. As in Newcomb's Problem, they are a human and a mythical
creature. Here the mythical creature is The Radical Interpreter.

In any game, the payouts are a function of what will happen to the
players in each situation, and the players' values over those outcomes.
To turn a physical situation into a game, we need to know the players'
goals. Here are the goals I'll assume our players have:

\begin{itemize}
\tightlist
\item
  The Radical Interpreter assigns mental states to Human with the aim of
  making the action Human actually chooses the rational choice. I assume
  here that the `mental states' include Human's evidence. Indeed, the
  main thing I'll have The Radical Interpreter do is assign evidence to
  Human.
\item
  Human aims to maximise expected utility given their evidence. That
  last phrase, `their evidence', should be read de re. More precisely,
  they aim to do the thing that is expected utility maximising given the
  evidence they actually have. (So their own views about their evidence
  don't matter; all that matters is what their evidence really is.)
\end{itemize}

Given these aims, The Radical Interpreter and Human often play
coordination games. They will both achieve their aims if they act the
`same' way. That is, when it is uncertain whether \emph{p} is part of
Human's evidence, the coordination outcomes are:

\begin{itemize}
\tightlist
\item
  The Radical Interpreter says that \emph{p} is part of Human's
  evidence, and Human maximises expected utility given \emph{K}
  ∧~\emph{p}.
\item
  The Radical Interpreter says that \emph{p} is part of Human's
  evidence, and Human maximises expected utility given \emph{K}.
\end{itemize}

Coordination games typically have multiple equilibria, and that will
also be the case here.

Let's focus on one example. Human is offered a bet on \emph{p}. If the
bet wins, it wins 1 util; if the bet loses, it loses 100 utils. Human's
only choice is to Take or Decline the bet. The proposition \emph{p}, the
subject of the bet, is like the claim that Rahul is in the restaurant.
That is, it is unclear whether it is in Human's evidence. Again, let
\emph{K} be the rest of Human's evidence, and stipulate that Pr(\emph{p}
\textbar{}\emph{K})~=~0.9. Each party now faces a choice.

\begin{itemize}
\tightlist
\item
  The Radical Interpreter has to choose whether \emph{p} is part of
  Human's evidence or not.
\item
  Human has to decide whether to Take or Decline the bet.
\end{itemize}

The payouts for the game are given in
Table~\ref{tbl-radical-interpreter}.

\begin{longtable}[]{@{}rcc@{}}
\caption{The Radical Interpreter
game.}\label{tbl-radical-interpreter}\tabularnewline
\toprule\noalign{}
\endfirsthead
\endhead
\bottomrule\noalign{}
\endlastfoot
& \emph{p}~∈~\emph{E} & \emph{p}~∉~\emph{E} \\
Take the Bet & 1, 1 & -9.1, 0 \\
Decline the Bet & 0, 0 & 0, 1 \\
\end{longtable}

Why is this the right table? Let's start with The Radical Interpreter.

The Radical Interpreter achieves their aim iff the following
biconditional obtains: Human takes the bet iff \emph{p} is part of their
evidence. That's why they get payout 1 in the cells where that obtains,
and 0 otherwise.

Most of Human's payouts are obvious. In the bottom row, they are
guaranteed 0, since the bet is declined. In the top left, the bet wins
with probability 1, so their expected return is 1. In the top right, the
bet wins with probability 0.9, so the expected return of taking it is
1~×~0.9~-~100~×~0.1~=~-9.1.

There are two Nash equilibria for the game - the top left and the bottom
right. We could stop here and say that according to IRT it is
indeterminate whether \emph{p} is part of Human's evidence. But we can
do better.

But to do that, I need to survey more contested areas of game theory. In
particular, I need to introduce some work on equilibrium choice. To do
that, it helps to think about a game that is inspired by an example of
Rousseau's.

\section{Risk-Dominant Equilibria}\label{sec-globalgame}

Table~\ref{tbl-generic-game} is the abstract version of a two-player,
two-option game.

\begin{longtable}[]{@{}rcc@{}}
\caption{A generic 2 by 2 by 2
game.}\label{tbl-generic-game}\tabularnewline
\toprule\noalign{}
\endfirsthead
\endhead
\bottomrule\noalign{}
\endlastfoot
& \emph{a} & \emph{b} \\
\emph{A} & \emph{r}\textsubscript{11},~\emph{c}\textsubscript{11} &
\emph{r}\textsubscript{12},~\emph{c}\textsubscript{12} \\
\emph{B} & \emph{r}\textsubscript{21},~\emph{c}\textsubscript{21} &
\emph{r}\textsubscript{22},~\emph{c}\textsubscript{22} \\
\end{longtable}

What are usually called Stag Hunt games have the following eight
characteristics.

\begin{enumerate}
\def\labelenumi{\arabic{enumi}.}
\tightlist
\item
  \emph{r}\textsubscript{11}~\textgreater~\emph{r}\textsubscript{21}
\item
  \emph{r}\textsubscript{22}~\textgreater~\emph{r}\textsubscript{12}
\item
  \emph{c}\textsubscript{11}~\textgreater~\emph{c}\textsubscript{12}
\item
  \emph{c}\textsubscript{22}~\textgreater~\emph{c}\textsubscript{21}
\item
  \emph{r}\textsubscript{11}~\textgreater~\emph{r}\textsubscript{22}
\item
  \emph{c}\textsubscript{11}~≥~\emph{c}\textsubscript{22}
\item
  \emph{r}\textsubscript{21}~+~\emph{r}\textsubscript{22}~\textgreater~\emph{r}\textsubscript{11}~+~\emph{r}\textsubscript{12}
\item
  \emph{c}\textsubscript{12}~+~\emph{c}\textsubscript{22}~≥~\emph{c}\textsubscript{11}~+~\emph{c}\textsubscript{21}
\end{enumerate}

The first four conditions say that the game has two (strict) Nash
equilibria: \emph{Aa} and \emph{Bb}. The next two conditions say that
the \emph{Aa} equilibria is \emph{Pareto-optimal}: neither player
prefers \emph{Aa} to \emph{Bb}. In fact it says something a bit
stronger: one of the players strictly prefers the \emph{Aa} equilibria,
and the other player does not prefer \emph{Bb}. The last two conditions
say that the \emph{Bb} equilibria is \emph{risk-optimal}.

Hans Carlsson and Eric van Damme
(\citeproc{ref-CarlssonVanDamme1993}{1993}) offer an argument that in
any such game, rational players will end up at \emph{Bb}. The game that
Human and The Radical Interpreter are playing fits these eight
conditions, and The Radical Interpreter is perfectly rational. So if
Carlsson and van Damme are right, The Radical Interpreter will say that
\emph{p}~∉~\emph{E}. Indeed, if Carlsson and van Damme are right, the
toy theory I offered Section~\ref{sec-simplesolution} will be correct in
all cases where it applies.

The rest of this chapter would be much simpler if I thought Carlsson and
van Damme's argument worked in full generality. Unfortunately, I don't
think it does. In particular, I think it fails in the important case
where it is common knowledge that both players are rational, and both
players know precisely the values of each of the eight payoffs. But I
think it does work in the special case where one player has imperfect
access to what the payouts are. And that, it turns out, is the special
case that matters to us. That's getting ahead of the story though; let's
start with their argument.

I said games satisfying these conditions are called Stag Hunt games. The
name comes from a thought experiment in Rousseau's \emph{Discourse on
Inequality}.

\begin{quote}
They were perfect strangers to foresight, and were so far from troubling
themselves about the distant future, that they hardly thought of the
morrow. If a deer was to be taken, every one saw that, in order to
succeed, he must abide faithfully by his post: but if a hare happened to
come within the reach of any one of them, it is not to be doubted that
he pursued it without scruple, and, having seized his prey, cared very
little, if by so doing he caused his companions to miss theirs.
~(\citeproc{ref-Rousseau1913}{Rousseau, 1913: 209--10})
\end{quote}

Brian Skyrms (\citeproc{ref-Skyrms2001}{2001}) has argued that these
Stag Hunt games are important across philosophy; they are good models
for many real-life situations that are often (incorrectly) modeled as
Prisoners' Dilemmas. But going over why that is would be a needless
digression. Our focus is on Carlsson and van Damme's argument that
Rousseau was right: a `stranger to foresight', who is just focussing on
this game, should take the rabbit.

To make matters a little easier, we'll focus on a very particular
instance of Stag Hunt, the one in Table~\ref{tbl-stag-hunt}.

\begin{longtable}[]{@{}rcl@{}}
\caption{A simple version of Stag
Hunt.}\label{tbl-stag-hunt}\tabularnewline
\toprule\noalign{}
\endfirsthead
\endhead
\bottomrule\noalign{}
\endlastfoot
& \emph{a} & \emph{b} \\
\emph{A} & 4, 4 & 0, 3 \\
\emph{B} & 3, 0 & 3, 3 \\
\end{longtable}

The equilibrium \emph{Aa} is Pareto-optimal: it is the best outcome for
each individual. But it is risky, and Carlsson and van Damme suggest a
way to turn that risk into an argument for choosing \emph{Bb}.

Embed Table~\ref{tbl-stag-hunt} game in what they call a \emph{global
game}. Our first version of a global game is that each player knows that
they will play Table~\ref{tbl-global-game}, with \emph{x} to be selected
at random from a flat distribution over {[}-1, 5{]}.

\begin{longtable}[]{@{}rcc@{}}
\caption{The global game.}\label{tbl-global-game}\tabularnewline
\toprule\noalign{}
\endfirsthead
\endhead
\bottomrule\noalign{}
\endlastfoot
& \emph{a} & \emph{b} \\
\emph{A} & 4, 4 & 0,~\emph{x} \\
\emph{B} & \emph{x}, 0 & \emph{x},~\emph{x} \\
\end{longtable}

There isn't much to say about Table~\ref{tbl-global-game} with this
prior knowledge. Let's give the players a little more knowledge. (And
we'll call the players Row and Column to make it easier to refer to each
of them.)

Before they play the game, each player will get a noisy signal about the
value of \emph{x}. There will be signals \emph{s\textsubscript{R}} and
\emph{s\textsubscript{C}} chosen (independently) from a flat
distribution over {[}\emph{x}~-~0.25,~\emph{x}~+~0.25{]}, and shown to
Row and Column respectively. So each player will know the value of
\emph{x} to within ¼, and know that the other player knows it to within
¼ as well. This is a margin of error model, and in those models there is
very little that is common knowledge. That, Carlsson and van Damme
argue, makes a huge difference.

They go on to prove that iterated deletion of strictly dominated
strategies (almost) removes all but one strategy pair. (I'll go over the
proof of this in the next subsection.) Each player will play
\emph{A}/\emph{a} if the signal is greater than 2, and \emph{B}/\emph{b}
otherwise.\footnote{Strictly speaking, we can't rule out various mixed
  strategies when the signal is precisely 2, but this makes little
  difference, since that occurs with probability 0.} Surprisingly, this
shows that players should play the risk-optimal strategy even when they
know the other strategy is Pareto-optimal. When a player gets a signal
in (2, 3.75), then they know that \emph{x}~\textless~4, so \emph{Bb} is
the Pareto-optimal equilibrium. But the logic of the global game
suggests the risk-dominant equilibrium is what to play.

Carlsson and van Damme go on to show that many of the details of this
case don't matter. Most importantly, it doesn't matter that the margin
of error in the signal was ¼; as long as it is positive the argument
goes through.

Now what does this show about the game where players know precisely what
the value of \emph{x} is? Equivalently, what does it show about the game
where the margin of error is zero?

Carlsson and van Damme argue that it shows that the risk-dominant choice
is the right choice there as well. After all, the game where there is
perfect knowledge just is a margin of error game, where the margin of
error is 0. In previous work I'd endorsed this argument
(\citeproc{ref-Weatherson2018-WEAIEA-2}{Weatherson, 2018}). I now think
this was a mistake. The limit case, where the players know the value of
\emph{x}, is special. But, I'll argue, this doesn't actually undermine
the argument that in the game between Human and The Radical Interpreter,
both parties should choose the risk-dominant equilibria.

If the game between Human and The Radical Interpreter is meant to model
a real situation, Human won't know precisely what the payoffs are.
That's because real Humans don't know precisely what their evidence is.
They only know precisely what their evidence is if both positive and
negative introspection hold for evidence, and that's no more plausible
than that positive and negative introspection hold for knowledge. As L.
Humberstone (\citeproc{ref-Humberstone2016}{2016: 380--402}) shows,
that's not particularly plausible, even if one doesn't accept the
arguments in Williamson (\citeproc{ref-Williamson2000}{2000}) against
positive introspection.

If Human doesn't know precisely what their evidence is, they don't know
the payoffs in games like Table~\ref{tbl-global-game}, because those
payoffs are expected values. It turns out that's enough for the iterated
dominance argument that Human should play the risk-dominant equilibrium
to go through.

To be sure, The Radical Interpreter, who is just an idealisation,
presumably does know the payouts in the different states of the game. It
turns out, as I'll go over in Section~\ref{sec-perfectri}, that Carlsson
and van Damme's result only needs that one player is uncertain of the
payouts. Given the failure of at least negative introspection (and, I'd
say, positive introspection), that's something we can assume.

If Human should play the risk-dominant strategy in
Table~\ref{tbl-radical-interpreter}, they should decline the bet. So The
Radical Interpreter, who can figure this out, should say that \emph{p}
is not part of their evidence. Since one's evidence just is what The
Radical Interpreter says it is, that means that in
Table~\ref{tbl-radical-interpreter}, \emph{p} is not part of Human's
evidence.

Applied to the case of Parveen and Rahul, that means that The Radical
Interpreter is best off saying it is no part of Parveen's evidence that
Rahul is in the restaurant. More generally, in the simple cases
described in Section~\ref{sec-simplesolution}, The Radical Interpreter
should say that \emph{p} is not part of Human's evidence just in case
the equation used there holds.

The result is an interest-relative theory of evidence that is somewhat
well motivated. At least, it can be incorporated into a broader theory
of rational action.

This model keeps what was good about the pragmatic encroachment theory
developed in the previous chapters, while also allowing that evidence
can be interest-relative. It does require a considerably more complex
theory of rationality than was previously used. Rather than just model
rational agents as utility maximisers, they are modelled as playing
playing risk-dominant strategies in coordination games under uncertainty
about what the payouts are. Still, it turns out that this is little more
than assuming that they maximise evidential expected utility, and they
expect others (at least perfectly rational abstract others) to do the
same, and they expect those others to expect they will maximise expected
utility, and so on.

The rest of this section goes into more technical detail about Carlsson
and van Damme's example. Readers not interested in these details can
skip ahead to Section~\ref{sec-evsolution}. In
Section~\ref{sec-cvdproof} I summarise their argument that we only need
iterated deletion of strictly dominated strategies to get the result
that rational players will play the risk-dominant strategies. Then in
Section~\ref{sec-perfectri} I offer a small generalisation of their
argument, showing that it still goes through when one of the players
gets a precise signal, and the other gets a noisy signal.

\subsection{The Dominance Argument for Risk-Dominant
Equilibria}\label{sec-cvdproof}

Two players, Row (or R) and Column (or C) will play the game depicted in
Table~\ref{tbl-global-game}. They won't be told what \emph{x} is, but
they will get a noisy signal of \emph{x}, drawn from an even
distribution over {[}\emph{x}~-~0.25,~\emph{x}~+~0.25{]}. Call these
signals \emph{s\textsubscript{R}} and \emph{s\textsubscript{C}}. Each
player must then choose \emph{A}, getting either 4 or 0 depending on the
other player's choice, or choose \emph{B}, getting \emph{x} for sure.

Before getting the signal, the players must choose a strategy. In this
context, a strategy is a function from signals to choices. Since the
higher the signal is, the better it is to play \emph{B}, we can more or
less equate strategies with `tipping points', where the player plays
\emph{B} if the signal is above the tipping point, and \emph{A} below
the tipping point.\footnote{I'm ignoring mixed strategies here, and
  strategies that differ in cases where the signal is right at the
  tipping point. It's trivial but tedious to extend the proof to cover
  these cases.}

Call the tipping points for Row and Column respectively
\emph{T\textsubscript{R}} and \emph{T\textsubscript{C}}. Since this game
is symmetric, we'll just have to show that in conditions of common
knowledge of rationality, \emph{T\textsubscript{R}}~=~2. It follows by
symmetry that \emph{T\textsubscript{C}}~=~2 as well. The only rule that
will be used is iterated deletion of strictly dominated strategies.

The return to a strategy is uncertain, even given the other player's
strategy. But given the strategies of each player, each players'
expected return can be computed. That will be treated as the return to
the strategy pair.

Note first that \emph{T\textsubscript{R}}~=~4.25 strictly dominates any
strategy where \emph{T\textsubscript{R}}~=~\emph{y}~\textgreater~4.25.
If \emph{s\textsubscript{R}}~∈~(4.25,~\emph{y}), then
\emph{T\textsubscript{R}} is guaranteed to return above 4, and the
alternative strategy is guaranteed to return 4. In all other cases, the
strategies have the same return. There is some chance that
\emph{s\textsubscript{R}}~∈~(4.25,~\emph{y}). So we can delete all
strategies \emph{T\textsubscript{R}}~=~\emph{y}~\textgreater~4.25, and
similarly all strategies
\emph{T\textsubscript{C}}~=~\emph{y}~\textgreater~4.25. By similar
reasoning, we can rule out \emph{T\textsubscript{R}}~\textless~-0.25 and
\emph{T\textsubscript{C}}~\textless~-0.25.

If \emph{s\textsubscript{R}}~∈~{[}-0.75,~4.75{]}, then it is equally
likely that \emph{x} is above \emph{sR} as it is below it. Indeed, the
posterior distribution of \emph{x} is flat over
{[}\emph{s\textsubscript{R}}~-~0.25,~\emph{s\textsubscript{R}}~+~0.25{]}.
From this it follows that the expected return of playing \emph{B} after
seeing signal \emph{sR} is just \emph{sR}.

Now comes the important step. For arbitrary \emph{y}~\textgreater~2,
assume we know that \emph{T\textsubscript{C}}~≤~\emph{y}. Consider the
expected return of playing \emph{A} given various values for
\emph{s\textsubscript{R}}~\textgreater~2. Given that the lower
\emph{T\textsubscript{C}} is, the higher the expected return is of
playing \emph{A}, we'll just work on the simple case where
\emph{T\textsubscript{C}}~=~\emph{y}, realizing that this is an upper
bound on the expected return of \emph{A} given
\emph{T\textsubscript{C}}~≤~y. The expected return of \emph{A} is 4
times the probability that Column will play \emph{a}, i.e., 4 times the
probability that
\emph{s\textsubscript{C}}~\textless~\emph{T\textsubscript{C}}. Given all
the symmetries that have been built into the puzzle, we know that the
probability that
\emph{s\textsubscript{C}}~\textless~\emph{s\textsubscript{R}} is 0.5. So
the expected return of playing \emph{A} is at most 2 if
\emph{s\textsubscript{R}} ≥ \emph{y}. But the expected return of playing
\emph{B} is, as we showed in the last paragraph,
\emph{s\textsubscript{R}}, which is greater than 2. So it is better to
play \emph{B} than \emph{A} if \emph{s\textsubscript{R}}~≥~\emph{y}. And
the difference is substantial, so even if \emph{s\textsubscript{R}} is
epsilon less than that \emph{y}, it will still be better to play
\emph{B}. (This is rather hand-wavy, but I'll go over the more rigorous
version presently.)

So for any \emph{y}~\textgreater~2 if
\emph{T\textsubscript{C}}~≤~\emph{y} we can prove that
\emph{T\textsubscript{R}} should be lower still, because given that
assumption it is better to play \emph{B} even if the signal is just less
than \emph{y}. Repeating this reasoning over and over again pushes us to
it being better to play \emph{B} than \emph{A} as long as
\emph{s\textsubscript{R}}~\textgreater~2. The same kind of reasoning
from the opposite end pushes us to it being better to play \emph{A} than
\emph{B} as long as \emph{s\textsubscript{R}}~\textless~2. So we get
\emph{s\textsubscript{R}}~=~2 as the uniquely rational solution to the
game.

Let's make that a touch more rigorous. Assume that
\emph{T\textsubscript{C}}~=~y, and \emph{s\textsubscript{R}} is slightly
less than \emph{y}. In particular, we'll assume that
\emph{z}~=~y~-~\emph{s\textsubscript{R}} is in (0, 0.5). Then the
probability that \emph{s\textsubscript{C}}~\textless~\emph{y} is
0.5~+~2\emph{z}~‑~2\emph{z}\textsuperscript{2}. So the expected return
of playing \emph{A} is 2~+~8\emph{z}~‑~8\emph{z}\textsuperscript{2}. And
the expected return of playing \emph{B} is, again, \emph{sR}. These will
be equal iff \(\frac{\sqrt{145-32y}-9}{16}\). So if we know that
\emph{T\textsubscript{C}}~≥~\emph{y}, we know that
\emph{T\textsubscript{R}}~≥~\emph{y}~+~\(\frac{\sqrt{145-32y}-9}{16}\),
which will be less than \emph{y} if \emph{y}~\textgreater~2. Then by
symmetry, we know that \emph{T\textsubscript{C}} must be at most as
large as that as well. Then we can use that fact to derive a further
upper bound on \emph{T\textsubscript{R}} and hence on
\emph{T\textsubscript{C}}, and so on. And this will continue until we
push both down to 2. It does require quite a number of steps of iterated
deletion. Table~\ref{tbl-threshold} shows the upper bound on the
threshold after \emph{n} rounds of deletion of dominated strategies.
(The numbers in Table~\ref{tbl-threshold} are precise for the first two
rounds, and correct to three significant figures after that.)

\begin{longtable}[]{@{}cc@{}}
\caption{How the threshold moves towards
2.}\label{tbl-threshold}\tabularnewline
\toprule\noalign{}
\endfirsthead
\endhead
\bottomrule\noalign{}
\endlastfoot
\emph{Round} & \emph{Upper Bound on Threshold} \\
1 & 4.250 \\
2 & 3.875 \\
3 & 3.599 \\
4 & 3.378 \\
5 & 3.195 \\
6 & 3.041 \\
7 & 2.910 \\
8 & 2.798 \\
9 & 2.701 \\
10 & 2.617 \\
\end{longtable}

That is, \emph{T\textsubscript{R}}~=~4.25 dominates any strategy with a
tipping point above 4.25. And \emph{T\textsubscript{R}}~=~3.875
dominates any strategy with a higher tipping point than 3.875, assuming
\emph{T\textsubscript{C}}~≤~4.25. And \emph{T\textsubscript{R}}~≈~3.599
dominates any strategy with a higher tipping point than 3.599, assuming
\emph{T\textsubscript{C}}~≤~3.875. And so on.

Similar reasoning shows that at each stage not only are all strategies
with higher tipping points dominated, but so are strategies that assign
positive probability (whether it is 1 or less than 1), to playing
\emph{A} when the signal is above the `tipping point'.\footnote{If we're
  careful about how we state this, we can use this to rule out all mixed
  strategies except those that respond probabilistically to
  \emph{s\textsubscript{R}}~=~2}.

So it has been shown that iterated deletion of dominated strategies will
rule out all strategies except the risk-optimal equilibrium. The
possibility that \emph{x} is greater than the maximal return for
\emph{A} is needed to get the iterated dominance going. We also need the
signal to have an error bar to it, so that each round of iteration
removes more strategies. But that's all that was needed; the particular
values used are irrelevant to the proof.

\subsection{Making One Signal Precise}\label{sec-perfectri}

So far I've just been setting out Carlsson and van Damme's results. It's
time to prove something just slightly stronger. I'll show that the
result in Section~\ref{sec-cvdproof} did not require that both parties
receive a noisy signal. It's enough that just one party does.

More precisely, I'll change the game so that it is common knowledge that
the signal Column gets, \emph{s\textsubscript{C}}, equals \emph{x}.
Since the game is no longer symmetric, I can't just appeal to the
symmetry of the game as frequently as in the previous subsection. This
slows the proof down, but doesn't stop it.

This change actually helps us at the first stage of the argument. Since
Column could not be wrong about \emph{x}, Column knows that if
\emph{s\textsubscript{C}}~\textgreater~4 then playing \emph{b} dominates
playing \emph{a}. So one round of deleting dominated strategies rules
out \emph{T\textsubscript{C}}~\textgreater~4, as well as ruling out
\emph{T\textsubscript{R}}~\textgreater~4.25.

At any stage for any \emph{y}~\textgreater~2 such that we know
\emph{T\textsubscript{C}}~≤~y, the strategy
\emph{T\textsubscript{R}}~=~\emph{y} dominates
\emph{T\textsubscript{R}}~\textgreater~\emph{y}. That's because if
\emph{s\textsubscript{R}}~≥~\emph{y}, and
\emph{T\textsubscript{C}}~≤~\emph{y}, the probability that Column will
play \emph{a} (given Row's signal) is less than 0.5. After all, the
signal is just as likely to be above \emph{x} as below it.\footnote{This
  isn't strictly true if the signal is close enough to 5, but in that
  case we have an independent reason to think Column will play \emph{a}.}
So if \emph{s\textsubscript{R}} is at or above
\emph{T\textsubscript{C}}, the probability that Column's signal is above
Column's tipping point is at least 0.5. So the probability that Column
will play \emph{b} is at least 0.5. So the expected return to Row of
playing \emph{A}, which is 4 times the probability that Column will play
\emph{a}, is at most 2. Since the expected return to Row of playing
\emph{B} equals the value of the signal\footnote{Unless the signal is
  very close to 5, in which case they should play \emph{B} anyway.},
that means that if the signal is above 2, they should play \emph{B}.

Summing up, if Row knows \emph{T\textsubscript{C}}~≤~\emph{y}, for any
\emph{y}~\textgreater~2, Row also knows it is better to play \emph{B} if
\emph{s\textsubscript{R}} ≥ \emph{y}. That is, if Row knows
\emph{T\textsubscript{C}}~≤~\emph{y}, for any \emph{y}~\textgreater~2,
Row's tipping point should be at most \emph{y}.

Assume now that it is common knowledge that
\emph{T\textsubscript{R}}~≤~\emph{y}, for some \emph{y}~\textgreater~2.
Assume Column's signal, which we'll call \emph{x}, is just a little less
than \emph{y}. In particular, define \emph{z}~=~\emph{y}~-~\emph{x}, and
assume \emph{z}~∈~(0, 0.25). We want to work out the upper bound on the
expected return to Column of playing \emph{a}. (The return of playing
\emph{b} is known, it is \emph{x}.)

The expected return to Column of playing \emph{a} will be highest when
\emph{T\textsubscript{R}} is highest. So we can work out an upper bound
on that expected return by assuming that
\emph{T\textsubscript{R}}~=~\emph{y}. Given that assumption, the
probability that Row plays \emph{A} is (1~+~2\emph{z})/2. (That's the
probability that Row's signal, which is a random draw from
{[}\emph{x}~-~¼,~\emph{x}~+~¼{]}, is above \emph{y}.) So the expected
return of playing \emph{a} is 2~+~4\emph{z}, i.e.,
2~+~4(\emph{y}~-~\emph{x}). That will be greater than \emph{x} only when
\emph{x}~\textless~(2~+~4\emph{y})/5.

So if it is common knowledge that \emph{T\textsubscript{R}}~≤~\emph{y},
then it is best for Column to play \emph{b} unless
\emph{x}~\textless~(2~+~4\emph{y})/5. That is, if it is common knowledge
that \emph{T\textsubscript{R}}~≤~\emph{y}, then
\emph{T\textsubscript{C}} must be at most (2~+~4\emph{y})/5.

The rest of the proof proceeds in a zig-zag fashion. At one stage, we
show that \emph{T\textsubscript{R}} must be no greater than
\emph{T\textsubscript{C}}. So whatever value we've shown to be an upper
bound for \emph{T\textsubscript{C}} is also an upper bound for
\emph{T\textsubscript{R}}. At the next stage, we show that given any
upper bound on \emph{T\textsubscript{R}} greater than 2, we can derive a
new upper bound on \emph{T\textsubscript{C}} which is lower still. This
process will eventually rule out all values for
\emph{T\textsubscript{R}} and \emph{T\textsubscript{C}} greater than 2.
So just using iterated deletion of dominated strategies, we eventually
rule out all strategies that are involve tipping points above 2.

There is one last point to be careful about. It takes infinitely many
steps to rule out all tipping points above 2. Since it isn't obviously
sound to have infinitely many steps of iterated deletion, one might
worry about the soundness of the proof at this point. The key thing to
note is that for any tipping point above 2, it is ruled out in a finite
number of steps. So purely finitary reasoning rules out all tipping
points above 2. It's just that there is no upper bound to the (finite!)
number of steps needed.

This completes the mathematical part of the argument; I'll return to
discussing whether this result matters for thinking about evidence and
rational action, and reply to some objections to thinking that it does.

\section{Objections and Replies}\label{sec-evsolution}

\emph{Objection}: The formal argument requires that in the `global game'
there are values for \emph{x} that make \emph{A} the dominant choice.
These cases serve as a base step for an inductive argument that follows.
But in Parveen's case, there is no such setting for \emph{x}, so the
inductive argument can't get going.

\emph{Reply}: What matters is that there are values of \emph{x} such
that \emph{A} is the strictly dominant choice, and Human (or Parveen)
doesn't know that they know that they know, etc., that those values are
not actual. And that's true in our case. For all Human (or Parveen)
knows that they know that they know that they know\ldots, the
proposition in question is not part of their evidence under a maximally
expansive verdict on The Radical Interpreter's part. So the relevant
cases are there in the model, even if both players know that they know
that they know \ldots{} that the models don't obtain, for a high but
finite number of repetitions of `that they know'.

\emph{Objection}: This model is much more complex than the simple
motivation for pragmatic encroachment.

\emph{Reply}: Sadly, this is true. I would like to have a simpler model,
but I don't know how to create one. I suspect any such simple model will
just be incomplete; it won't say what Parveen's evidence is. In this
respect, any simple model will look just like applying tools like Nash
equilibria to coordination games. So more complexity will be needed, one
way or another. I think paying this price in complexity is worth it
overall, but I can see how some people might think otherwise.

\emph{Objection}: Change the case involving Human so that the bet loses
15 utils if \emph{p} is false, rather than 100. Now the risk-dominant
equilibrium is that Human takes the bet, and The Radical Interpreter
says that \emph{p} is part of Human's evidence. But note that if it was
clearly true that \emph{p} was not part of Human's evidence, then this
would still be too risky a situation for them to know \emph{p}. So
whether it is possible that \emph{p} is part of Human's evidence, and
not just part of their knowledge, matters.

\emph{Reply}: This is all true, and it shows that the view I'm putting
forward is incompatible with some programs in epistemology. In
particular, it is incompatible with E=K, since the what it takes to be
evidence on this story is slightly different from what it takes to be
knowledge. The next section argues that this is independently plausible.

\section{Evidence, Knowledge and Cut-Elimination}\label{sec-cutelim}

In the previous section I noted that my theory of evidence is committed
to denying Williamson's E=K thesis. This is the thesis that says one's
evidence is all and only what one knows. What I say is consistent with,
and arguably committed to, one half of that thesis. Nothing I've said
here provides a reason to reject the implication that if \emph{p} is
part of one's evidence, then one knows \emph{p}. Indeed, the story I'm
telling would have to be complicated even further if that fails. But I
am committed to denying the other direction. On my view, there can be
cases where someone knows \emph{p}, but \emph{p} is not part of their
evidence.

My main reason for this comes from the kind of cases that Shyam Nair
(\citeproc{ref-Nair2019}{2019}) describes as failures of
`cut-elimination'. I'll quickly set out what Nair calls cut-elimination,
and why it fails, and then look at how it raises problems for E=K.

Start by assuming that we have an operator ⊨ such that Γ~⊨~\emph{A}
means that \emph{A} can be rationally inferred from Γ. I'm following
Nair (and many others) in using a symbol usually associated with logical
entailment here, though this is potentially misleading. A big plotline
in what follows will be that ⊨, so understood, behaves very differently
from familiar notions of entailment.

For the purposes of this section, I'm staying somewhat neutral on what
it means to be able to rationally infer \emph{A} from Γ. In particular,
I want everything that follows to be consistent with the interpretation
that an inference is rational only if it produces knowledge. I don't
think that's true; I think folks with misleading evidence can rationally
form false beliefs, and I think the traveler in Dharmottara's example
rationally believes there is a fire. But there is a dialectical reason
for staying neutral here. I'm arguing against one important part of the
`knowledge first' program, and I don't want to do so by assuming the
falsity of other parts of it. So for this section (only), I'll write in
a way that is consistent with saying rational belief requires knowledge.

Given that, one way to interpret Γ~⊨~\emph{A} is that \emph{A} can be
known on the basis of Γ. What can be known on the basis of what is a
function of, among other things, who is doing the knowing, what their
background evidence is, what their capacities are, and so on. Strictly
speaking, that suggests we should have some subscripts on ⊨ for who is
the knower, what their background evidence is, and so on. In the
interests of readability, I'm going to leave all those implicit. In the
next section it will be important to come back and look at whether the
force of some of these arguments is diminished if we are careful about
this relativisation.

That's our important notation. The principle \emph{Cut} that Nair
focuses on is that if 1 and 2 are true, so is 3.

\begin{enumerate}
\def\labelenumi{\arabic{enumi}.}
\tightlist
\item
  Γ~⊨~A
\item
  \{A\}~∪~Δ~⊨~B
\item
  Γ~∪~Δ~⊨~B
\end{enumerate}

The principle is intuitive. Indeed, it is often implicit in a lot of
reasoning. Here is one instance of it in action.

\begin{quote}
I heard from a friend that Jack went up the hill. This friend is
trustworthy, so I'm happy to infer that Jack did indeed go up the hill.
I heard from another friend that Jack and Jill did the same thing. This
friend is also trustworthy, so I'm happy to infer that Jill did the same
thing as Jack, i.e., go up the hill.
\end{quote}

Normally we wouldn't spell out the `happy to infer' steps, but I've
included them in here to make the reasoning a bit more explicit. But
note what I didn't need to make explicit, even in this laborious
reconstruction. I didn't need to note a change of status of the claim
that Jack went up the hill. That goes from being a conclusion to being a
premise. What matters for our purposes is that there doesn't seem to be
a gap between the rationality of inferring that Jack went up the hill,
and the rationality of using that as a premise in later reasoning. The
idea that there is no gap here just is the idea that the principle
\emph{Cut} is true.

Wwhile \emph{Cut} seems intuitive in cases like this, Nair argues that
it can't be right in general. (If that's right we have a duty, one Nair
takes up, to explain why cases like Jack and Jill seem like cases of
good reasoning.) For my purposes, it is helpful to divide the putative
counterexamples to \emph{Cut} into two categories. I'll call them
\emph{monotonic} and \emph{non-monotonic} counterexamples. The
categorisation turns on whether Γ~∪~Δ~⊨~\emph{A} is true assuming that
Γ~⊨~A is true. I'll call cases where it is true monotonic instances of
\emph{Cut}, and cases where it is false non-monotonic instances.

That \emph{Cut} fails in non-monotonic cases is fairly obvious. We can
see this with an example that was hackneyed a generation ago.

\begin{quote}
Γ = \{Tweety is a bird\}\\
Δ = \{Tweety is a penguin\}\\
A = B = Tweety can fly

From Tweety is a bird we can rationally infer that Tweety flies. And
given that Tweety is a flying penguin, we can infer that she flies. But
given that Tweety is a penguin and a bird, we cannot infer this. So
principles 1 and 2 in \emph{Cut} are true, but 3 is false. And the same
pattern will recur any time Δ provides a defeater for the link between Γ
and \emph{A}.
\end{quote}

These cases will matter in what follows, but they are rather different
from the monotonic examples.The monotonic example I'll set out (in the
next three paragraphs) is very similar to one used in an argument
against E=K by Alvin Goldman (\citeproc{ref-Goldman2009}{2009}). In many
ways the argument against E=K I'm going to give is just a notational
variant on Goldman's, but I think the notation I'm borrowing from Nair
helps bring out the argument's strength.

Here is the crucial background assumption for the example. (I'll come
back to how plausible this is after setting the example up.) The nature
of \emph{F} around here varies, but it varies very very slowly. If we
find a pattern in common to all the \emph{F} within distance (in miles)
\emph{d} of here, we can rationally infer that the pattern extends
another mile. That's just boring induction. But we can't infer that it
extends to infinity, that would be a radical step. If we can't infer
that the pattern goes to infinity, there must be a point beyond which we
can't infer the pattern goes. Let's say that's one mile. So if we know
the pattern holds within distance \emph{d} of here, we can infer that it
holds within distance \emph{d}~+~1, but no more.\footnote{In any
  remotely realistic case, it would make more sense to say we can infer
  it holds in some multiple of \emph{d} rather than adding some value to
  \emph{d}. But I'm simplifying a lot to make a point, and this is just
  one more simplification.}

To see a case like this, imagine we're doing work that's more like
working out the diet of local wildlife than working out the mass of an
electron. If you know the mass of electrons around here, and what
pigeons around here eat, there are some inferences you can make. You can
come to know what the mass of electrons will be in the next town over,
and what pigeons eat in the next town over. But there is a difference
between the cases. You can also infer from this evidence what the mass
of electrons will be on the other side of the world. But you can't make
very confident inferences about what pigeons eat on the other side of
the world; they may have adapted their diet to local conditions. In our
case \emph{F} and \emph{G} concern things more like pigeon diets than
electron masses.

Now here is the counterexample.

\begin{quote}
Γ = Δ = \{Every \emph{F} within 3 miles of here is \emph{G}.\}\\
\emph{A} = Every \emph{F} between 3 and 4 miles of here is \emph{G}.\\
\emph{B} = Every \emph{F} between 4 and 5 miles of here is \emph{G}.
\end{quote}

If what I said was right, then this is a counterexample to \emph{Cut}.
Γ~⊨~\emph{A} is true because it says given evidence about all the
\emph{F} within 3 miles of here, we can infer that all the \emph{F}
within 4 miles are like them. And \{\emph{A}\}~∪~Δ~⊨~\emph{B} is true
because because it says given evidence about all the \emph{F} within 4
miles of here, we can infer that all the \emph{F} within 5 miles are
like them. But Γ~∪~Δ~⊨~\emph{A} is false, because it purports to say
that given evidence about the \emph{F} within 3 miles of here, we can
infer that all the \emph{F} within 5 miles are alike. And that's an
inductive bridge too far.

This particular example involving distances was an extreme idealisation.
But all we need for the larger argument is that there is some similarity
metric such that inductive inference is rational across short jumps in
that similarity metric, but not across long jumps. One kind of
similarity is physical distance from a salient point. That's not the
only kind of similarity, and rarely the most important kind.

As long as there is some `inductive margin of inference', the argument
works. What I mean by an inductive margin of inference is that given
that all the \emph{F} that differ from a salient point (along this
metric) by amount \emph{d} are \emph{G}, it is rational to infer that
all the \emph{F} that differ from that salient point by amount
\emph{d}~+~\emph{m} are \emph{G}, but not that all the \emph{F} that
differ from that salient point by amount \emph{d}~+~2\emph{m} are
\emph{G}. And it seems very plausible to me that there are some metrics,
and values of \emph{F}, \emph{G}, \emph{d}, \emph{m} such that that's
true.

For example, given what I know about Miami's weather, I can infer that
it won't snow there for the next few hundred Christmases. Indeed, I know
that. But I can't know that it won't snow there for the next few million
Christmases. There is some point, and I don't know what it is, where my
inductive knowledge about Miami's snowfall (or lack thereof) gives out.

While it is plausible that such cases are possible, any particular case
fitting this pattern is weird. Here's what is weird about them. It will
be easier to go back to the case where the metric is physical distance
to set this out, but the weirdness will extend to all cases. Imagine we
investigate the area within 3 miles of here thoroughly, and find that
all the \emph{F} are \emph{G}s. We infer, and now know, that all the
\emph{F} within 4 miles of here are \emph{G}s. We keep investigating,
and keep observing, and after a while we've observed all the \emph{F}
within 4 miles. And they are all \emph{G}, as we knew they would be. But
now we are in a position to infer that all the \emph{F} within 5 miles
are \emph{G}. Observing something that we knew to be true gives us a
reason to do something, i.e., make a further inference, that we couldn't
do before. That's weird, and I'm going to come back in the next section
to how it relates to the story I told about knowledge in
Chapter~\ref{sec-knowledge}.

The key point now is that this possibility undermines E=K. There is a
difference between knowing \emph{A} and being able to use \emph{A} to
support further inductive inferences. It is very natural to call that
the difference between knowing \emph{A} and having \emph{A} as evidence.

The reasoning that I've been criticising violates a principle Jonathan
Weisberg calls \emph{No Feedback} (\citeproc{ref-Weisberg2010}{Weisberg,
2010: 533--4}). This principle says that if a conclusion is derived from
some premises, plus some intermediary conclusions, then it is only
justified if it could, at least in principle, be derived from those
premises alone. A natural way to read this is that we have some
evidence, and things that we know on the basis of that evidence have a
different functional role from the evidence. They can't do what the
evidence itself can do, even if known. This looks like a problem for
E=K, as Weisberg himself notes (\citeproc{ref-Weisberg2010}{2010: 536}).

If any monotonic instances of failures of Cut exist, we need to
distinguish between things the thinker knows by inference, and things
they know by observation, in order to assess their inferences. That's to
say, some knowledge will not play the characteristic role of evidence. T
suggests that E=K is false.

\section{Basic Knowledge and Non-Inferential Knowledge}\label{sec-basic}

It would be natural to conclude from the examples I've discussed that
evidence is something like non-inferential knowledge. This is very
similar to a view defended by Patrick Maher
(\citeproc{ref-Maher1996}{1996}). And it is, I will argue, close to the
right view. But it can't be exactly right, for reasons Alexander Bird
(\citeproc{ref-Bird2004}{2004}) brings out.

I will argue that evidence is not non-inferential knowledge, but rather
basic knowledge. The primary difference between these two notions is
that \emph{being non-inferential} is a diachronic notion, it depends on
the causal source of the knowledge, while being basic is a synchronic
notion, it depends on how the knowledge is currently supported. In
general, non-inferential knowledge will be basic knowledge, and basic
knowledge will be non-inferential. But the two notions can come apart,
and when they do, the evidence is what is basic, not what is
non-inferential.

The following kind of case is central to Bird's objection to the idea
that evidence is non-inferential knowledge. Assume that our inquirer
sees that \emph{A} and rationally infers \emph{B}. On the view that
evidence is non-inferential knowledge, \emph{A} is evidence but \emph{B}
is not. Now imagine that at some much later time, the inquirer remembers
\emph{B}, but has forgotten that it is based on \emph{A}. This isn't
necessarily irrational. As Harman (\citeproc{ref-Harman1986}{1986})
stresses, an obligation to remember our evidence is wildly unrealistic.
The inquirer learns \emph{C} and infers \emph{B}~∧ \emph{C}. This seems
perfectly rational. But why is it rational?

If evidence is non-inferential knowledge, then this is a mystery. Since
\emph{B} was inferred, that can't be the evidence that justifies
\emph{B}~∧~\emph{C}. So the only other option is that the evidence is
the, now forgotten, \emph{A}. It is puzzling how something that is
forgotten can now justify. But a bigger problem is that if \emph{A} is
the inquirer's evidence, then they should also be able to infer
\emph{A}~∧~\emph{C}. But this would be an irrational inference.

So I agree with Bird that we can't identify evidence with
non-inferential knowledge, if by that we mean knowledge that was not
originally gained through inference. (And what else could it mean?) But
a very similar theory of evidence can work. The thing about evidence is
that it can play a distinctive role in reasoning, it provides a
distinctive kind of reason. In particular, it provides basic reasons.

Evidence stops regresses. That's why we can say that our fundamental
starting points are self-evident. Now there is obviously a controversy
about what things are self-evident. I don't find it particularly likely
that claims about the moral rights we were endowed with by our Creator
are self-evident. But I do think it is true that a lot of things are
self-evident. (Even including, perhaps, that we have moral rights.) We
should take this notion of self-evidence seriously. Our evidence is that
knowledge which provides basic reasons.

What is it for a reason to be basic? It isn't that it was not originally
inferred. Something that was once inferred from long forgotten premises
may now be a basic reason. Rather, it is something that needs no further
reason given as support. (Its support is itself, since it is
self-evident.) What makes a reason need further support? I'm an
interest-relative epistemologist, so I think this will be sensitive to
the agent's interests. For example, I think facts reported in a reliable
history book are pieces of basic evidence when we are thinking about
history, but not when we are thinking about the reliability of that
book. But this kind of interest-relativity is inessential to the story.
What is essential is that evidence provides a reason that does not in
turn require more justification.

This picture suggests an odd result about cases of forgotten evidence.
There is a much discussed puzzle about forgotten evidence that was set
in motion by Gilbert Harman (\citeproc{ref-Harman1986}{1986}). He argued
that if someone irrationally believes \emph{p} in the basis of some
evidence, then later forgets the evidence but retains the belief, the
belief may now be rational. It would not be rational if they remembered
both the evidence, and that it was the evidence for \emph{p}. But, and
this is what I want to take away from the case, there is no obligation
on thinkers to keep track of why they believe each of the things they
do.

There is a large literature now on this case; Sinan Dogramaci
(\citeproc{ref-Dogramaci2015}{2015}) both provides a useful guide to the
debate and moves it forward by considering what we might aim to achieve
by offering one or other evaluation of the believer in this case. The
view I'm offering here is, as far as I can tell, completely neutral on
Harman's original case. But it has something striking to say about a
similar case.

Imagine an inquirer, call him Jaidyn, believes \emph{p} for the
excellent reason that he read it in a book from a reliable historian
\emph{H}. Six months later, he has forgotten that that's where he
learned that \emph{p}, though he still believes that \emph{p}. In a
discussion about historians, a friend of Jaidyn's says that \emph{H} is
really unreliable. Jaidyn is a bit shocked, and literally can't believe
it. This is for the best since \emph{H} is in fact reliable, and his
friend is suffering from a case of mistaken identity. But he is moved
enough by the testimony to suspend judgment on \emph{H}'s reliabity, and
so he forms a disposition to not believe anything \emph{H} says without
corroboration. Since he doesn't know that he believes \emph{p} because
\emph{H} says so, he doesn't do anything about this belief. What should
we say about Jaidyn's belief that \emph{p}?

Here's what I want to say. I don't claim this is particularly intuitive,
but I'm not sure there is anything particularly intuitive; it's best to
just see what a theory says about the case. My theory says that Jaidyn
still knows that \emph{p}. This knowledge was once based on \emph{H}'s
testimony, but it is no longer based on that. Indeed, it is no longer
based on anything. Presumably, if Jaidyn is rational, the knowledge will
be sensitive to the absence of counter-evidence, or to incoherence with
the rest of his world-view. But these are checks and balances in
Jaidyn's doxastic system, they aren't the basis of the belief. Since the
belief is knowledge, and is a basic reason for Jaidyn, it is part of his
evidence.

Note three things about that last conclusion. First, this is a case
where a piece of inferential knowledge can be in someone's evidence. By
(reasonably) forgetting the source of the knowledge, it converts to
being evidence. Second, almost any knowledge could make this jump.
Whenever someone has no obligation to remember the source or basis of
some knowledge, they can reasonably forget the source, and the basis,
and the knowledge will become basic. And then it is evidence. The
picture I'm working with is that pieces of knowledge can easily move in
and out of one's evidence set; sometimes all it takes is forgetting
where the knowledge came from. But third, if Jaidyn had done better
epistemically, and remembered the source, he would no longer know that
\emph{p}.

It is somewhat surprising that knowledge can be dependent on forgetting.
Jaidyn knows that \emph{p}, but if he'd done better at remembering why
he believes \emph{p}, he wouldn't know it. Still, the knowledge isn't
grounded in forgetting. It's originally grounded in testimony from an
actually reliable source, and Jaidyn did as good a job as he needed to
in checking the reliability of the source before accepting the
testimony. Now since Jaidyn is finite, he doesn't have any obligation to
remember everything. It seems odd to demand that Jaidyn adjust his
beliefs on the basis of where they are from if he isn't even required to
track where they are from. It would be very odd to say that Jaidyn's
evidence now includes neither \emph{p} (because it is undermined by his
friend's testimony), nor the fact that someone said that \emph{p}. That
suggests any \emph{p}-related inferences Jaidyn makes are totally
unsupported by his evidence, which doesn't seem right.

So the picture of evidence as basic knowledge, combined with a plausible
theory of when forgetting is permissible, suggests that the forgetful
reader knows more than the reader with a better memory. I suspect the
same thing will happen in versions of Goldman's explosive inductive
argument. Imagine a thinker observes all the \emph{F}s within 3 miles,
sees they are all \emph{G}, and rationally infers that all the \emph{F}s
within 4 miles are \emph{G}. Some time later they retain the belief, the
knowledge actually, that all \emph{F}s within 4 miles are \emph{G}. But
they forget that this was partially inferential knowledge, like Jaidyn
forgot the source of his knowledge that \emph{p}. They then make the
seemingly sensible inductive inference that all \emph{F}s within 5 miles
are \emph{G}. Is this rational, and can it produce knowledge? I think
the answer is yes; if they (not unreasonably) forget the source of their
knowledge that the \emph{F}s 3 to 4 miles away are \emph{G}, then this
knowledge becomes basic. If it's basic, it is evidence. And if it is
evidence, it can support one round of inductive reasoning.

I've drifted a fair way from discussing interest-relativity. And a lot
of what I say here is inessential to defending IRT. So I'll return to
the main plotline with a discussion of how my view of evidence helps
respond to a challenge Ram Neta issues to IRT, and implies a rejection
of a key principle in Jeremy Fantl and Matthew McGrath's theory of
knowledge.

\section{Holism and Defeaters}\label{sec-neta}

The picture of evidence I've outlined here grounds a natural response to
a nice puzzle case due to Ram Neta
(\citeproc{ref-Neta2007}{2007}).\footnote{This section draws my
  (\citeproc{ref-Weatherson2011-WEADIR}{2011: 5}).}

\begin{quote}
Kate needs to get to Main Street by noon: her life depends upon it. She
is desperately searching for Main Street when she comes to an
intersection and looks up at the perpendicular street signs at that
intersection. One street sign says ``State Street'' and the
perpendicular street sign says ``Main Street.'' Now, it is a matter of
complete indifference to Kate whether she is on State Street--nothing
whatsoever depends upon it. (\citeproc{ref-Neta2007}{Neta, 2007: 182})
\end{quote}

Neta argues that IRT implies Kate knows that she is on State Street, but
does not know that she is on Main Street. He suggests this is
intuitively implausible. I think I agree with that intuition, so let's
take it for granted and ask whether IRT has this problematic
implication.

Let's also assume that it is not rational for Kate to take the street
sign's word for it. I'm not sure that's true actually, but let's assume
it to get the argument going. I think Neta is reasoning that since
Kate's life depends on it, then IRT must say that she can't trust street
signs, because the stakes are so high.

That claim about the relation between stakes and what one can take for
granted can't be right. I often take actions that my life depends on
going by the say so of signs. For example, I often turn onto the freeway
ramp labelled on ramp, and not the ramp labelled off ramp, without
really double checking. If I was wrong about this there is a very high
chance I'd be very quickly killed. (Wrong-way crashes on freeways are a
very common kind of fatal collisions.) If Kate can't take the sign for
granted, it isn't just because her life is at stake; somewhat
disconcertingly, that doesn't make the case any different from everyday
driving.

But maybe Kate has some other way of checking where she is - like a map
on a phone in her pocket - and it would be irrational to take the sign
for granted and not check that other map. So I'm not going to push on
this assumption.

So what evidence should The Radical Interpreter assign to Kate? It
doesn't seem to be at issue that Kate sees that the signs say State and
Main. The big question is whether she can simply take it as evidence
that she is on State and Main. That is, do the contents of the sign
simply become part of Kate's evidence? (Assume that the signs are
accurate and there is no funny business going on, so it is plausible
that the signs contribute this evidence.) There are three natural
options.

\begin{enumerate}
\def\labelenumi{\arabic{enumi}.}
\tightlist
\item
  Both signs supply evidence directly to Kate, so her evidence includes
  that she is on State and that she is on Main.
\item
  Neither sign contributes evidence directly to Kate, so her evidence
  includes what the signs say, but nothing directly about her location.
\item
  One sign contributes evidence directly to Kate, but the other does
  not.
\end{enumerate}

Option 1 implies that Kate is rational to not check further whether she
is on Main Street. And that's irrational, so option 1 is out.

Option 3 implies that the signs behave differently, and that The
Rational Interpreter will assign them different roles in Kate's
cognitive architecture. But this will be true even though the signs are
equally reliable, and Kate's evidence about their reliability is
identical. So Kate treating them differently would be irrational, and
The Radical Interpreter does not want to make Kate irrational if it can
be helped. So option 3 is out.

That leaves Option 2. Kate's evidence does not include that she is on
State, and does not include that she is on Main. The latter
`non-inclusion' is directly explained by pragmatic factors. The former
is explained by those factors plus the requirement that Kate's evidence
is what The Radical Interpreter says it is, and The Radical
Interpreter's desire to make Kate rational.

So Kate's evidence doesn't distinguish between the streets. It does,
however, include that the signs say she is on State and that she is on
Main. Could she be entitled in inferring that she is on State, but not
that she is on Main?

It is hard to see how this could be so. Street signs are hardly basic
epistemic sources. They are the kind of evidence we should be
`conservative' about in the sense of Pryor
(\citeproc{ref-Pryor2004}{2004}). We should only use them if we
antecedently believe they are correct. So for Kate to believe she's on
State, she'd have to believe the street signs she can see are correct.
If not, she'd incoherently be relying on a source she doesn't trust,
even though it is not a basic source. But if she believes the street
signs are correct, she'd believe she was on Main, and that would lead to
practical irrationality. So there's no way to coherently add the belief
that she's on State Street to her stock of beliefs. So she doesn't know,
and can't know, that she's either on State or on Main. This is, in a
roundabout way, due to the practical situation Kate faces.

Neta thinks that the best way for IRT to handle this case is to say that
the high stakes associated with the proposition that Kate is on Main
Street imply that certain methods of belief formation do not produce
knowledge. And he argues, plausibly, that such a restriction will lead
to implausibly sceptical results.

What to say about this suggestion turns on how we understand what a
`method' is. If methods are individuated very finely, like \emph{Trust
street signs right here}, then it's plausible that Kate should restrict
what methods she uses, but implausible that this is badly sceptical. If
methods are individuated very coarsely, like \emph{Trust written
testimony}, then it's plausible that this is badly sceptical, but
implausible that Kate should give up on methods this general. I can
rationally treat some parts of a book as providing direct evidence about
the world, and other, more speculative, parts as providing direct
evidence about what the author says, and hence indirect evidence about
the world. Similarly, Kate can treat these street signs as indirect
evidence about her location, while still treating other signs around her
as providing direct evidence. So there is no sceptical threat here.

But while the case doesn't show IRT is false, it does tell us something
interesting about the implications of IRT. When a practical
consideration defeats a claim to know that \emph{p}, it will often also
knock out nearby knowledge claims. Some of these are obvious, like that
the practical consideration defeats the claim to know 0=0 → \emph{p}.
But some of these are more indirect. When the inquirer knows what her
evidence is, and knows that she has just the same evidence for \emph{q}
as for \emph{p}, then if a practical consideration defeats a claim to
know \emph{p}, it also defeats a claim to know \emph{q}. In practice,
this makes IRT a somewhat more sceptical theory than it may have first
appeared. It's not so sceptical as to be implausible, but it's more
sceptical than is immediately obvious. This kind of result, where IRT
ends up being somewhat sceptical but not implausibly so, has been a
theme of many different cases throughout the book.

\section{Epistemic Weakness}\label{sec-weakness}

The cases where cut-elimination fails raise a problem for the way that
Jeremy Fantl and Matthew McGrath spell out their version of IRT. Here is
a principle they rely on in motivating IRT.

\begin{quote}
When you know a proposition \emph{p}, no weaknesses in your epistemic
position with respect to \emph{p}---no weaknesses, that is, in your
standing on any truth-relevant dimension with respect to
\emph{p}---stand in the way of \emph{p} justifying you in having further
beliefs. (\citeproc{ref-FantlMcGrath2009}{Fantl \& McGrath, 2009: 64})
\end{quote}

And a few pages later they offer the following gloss on this principle.

\begin{quote}
We offer no analysis of the intuitive notion of `standing in the way'.
But we do think that, when Y does not obtain, the following
counterfactual condition is sufficient for a subject's position on some
dimension d to be something that stands in the way of Y obtaining:
whether Y obtains can vary with variations in the subject's position on
d, holding fixed all other factors relevant to whether Y obtains.
(\citeproc{ref-FantlMcGrath2009}{Fantl \& McGrath, 2009: 67})
\end{quote}

This gloss suggests that the difference between knowledge and evidence
is something that stands in the way of an inference. The inquirer who
knows that nearby \emph{F}s are \emph{G}s, but does not know that
somewhat distant \emph{F}s are \emph{G}s, has many things standing in
the way of this knowledge. One of them is, according to this test, that
her evidence does not include that all nearby \emph{F}s are \emph{G}s.
Yet this is something she knows. So a weakness in her epistemic position
with respect to the nature of nearby \emph{F}s, that it is merely
evidence and not knowledge, stands in the way of it justifying further
beliefs.

The same thing will be true in the monontonic cases of cut-elimination
failure. The thinker whose evidence includes Γ ∪ Δ, and whose
inferential knowledge includes \emph{A}, cannot infer \emph{B}. But if
they had \emph{A} as evidence, and not merely as knowledge, then they
could infer \emph{B}. So the weakness in their epistemic position, the
gap between evidence and knowledge, stands in the way of something.

I didn't endorse the principle of Fantl and McGrath's quoted above, but
I did endorse very similar principles, and one might wonder whether they
are subject to the same criticism. The main principle I endorsed was
that if one knows that \emph{p}, one is immune from criticism for using
\emph{p} on the grounds that \emph{p} might be false, or is too risky to
use. Equivalently, if the use of \emph{p} in an inference is defective,
but \emph{p} is known, the explanation of why it is defective cannot be
that \emph{p} is too risky. But now won't the same problem arise? Our
inquirer in the monotonic cut-elimination example can't use \emph{A} in
reasoning to \emph{B}. If \emph{A} was part of their evidence, then it
wouldn't be risky, and they would be able to use it. So the risk is part
of what makes the use of it mistaken.

I reject the very last step in that criticism. The fact that something
is wrong, and that it wouldn't have been wrong if X, does not mean the
non-obtaining of X is part of the ground, or explanation, for why it is
wrong. If I break a law, then what I do is illegal. Had the law in
question been struck down by a constitutional court, then my action
wouldn't have been illegal. Similarly, if the law had been repealed, my
action would not have been illegal. But that doesn't imply that the
ground or explanation of the illegality of my action is the court's not
striking the law down, or the later legislature not repealing the law.
That is to put too much into the notion of ground or explanation. No,
what makes the act illegal is that a particular piece of legislation was
passed, and this act violates it. This explanation is defeasible - it
would be defeated if a court or later legislature had stepped in - but
it is nonetheless complete.

The same thing is true in the case of knowledge and evidence. Imagine an
inquirer who observes all the \emph{F}s within 3 miles being \emph{G},
and infers both that all the \emph{F}s within 4 miles are \emph{G}, and,
therefore, that all the \emph{F}s within 5 miles are \emph{G}. The
intermediate step is, in a sense, risky. And the final step is bad. And
the final step wouldn't have been bad if the intermediate step hadn't
been risky. But it's not the riskiness that makes the second inference
bad. No, what makes the second inference bad is that it violates
Weisberg's No Feedback principle. That's what the reasoner can be
criticised for, not for taking an epistemic risk.

There are two differences then between the core principle I rely on -
using reasons that are known provides immunity to criticism for taking
epistemic risks - and the principle Fantl and McGrath rely on. I use a
concept of epistemic risk where they use a concept of strength of
epistemic position. I don't think these are quite the same thing, but
they are clearly similar. But the bigger difference is that they endorse
a counterfactual gloss of their principle, and I reject any such
counterfactual gloss. I don't say that the person who uses known
\emph{p} is immune to all criticisms that would have been vitiated had
\emph{p} been less risky. I just say that the risk can't be the ground
of the criticism; something else must be. In some cases, including this
one, that `something else' might be correlated with risk. But it must be
the explanation.

Of course, this difference between my version of IRT and Fantl and
McGrath's is tiny compared to how much our theories have in common. And
indeed, it's tiny compared to how much my theory simply borrows from
theirs. But it's helpful I think to highlight the differences to
understand the choice points within versions of IRT.

\bookmarksetup{startatroot}

\chapter{Power}\label{sec-power}

Knowledge is power. That is, grounds our ability to do things. What
things? Not, typically, bodily movements. If Lupin knows the passcode
for the phone, he can unlock the phone. But even without that knowledge,
he could make the bodily movement of typing in 220348 or whatever the
passcode was. What power does he get from knowing the code is 220348? He
gets the ability to deliberately unlock the phone. He gets the ability
to unlock the phone by typing 220348, and this not be a lucky guess, but
a rational action. Knowledge makes rational action possible. That's why
it is powerful. That's what its power consists in. That's why the Nyāya
philosophers were right to base anti-sceptical arguments on the
possibility of rational action. Knowledge matters in everyday life; it
explains why we have the power to act rationally.\footnote{The story I'm
  telling in this paragraph deliberately echoes the view that ability
  modals do not express possibility but necessity
  (\citeproc{ref-MandelkernEtAl2017}{Mandelkern, Schultheis, \& Boylan,
  2017}). To say Lupin can unlock the phone is not to say he might
  unlock the phone - anyone might unlock it hitting random numbers -
  it's to say he has a method that would unlock the phone if deployed.}

But it turns out to be surprisingly hard to articulate the magnitude of
that power. One might want to say that the actions which knowing that
\emph{p} makes rational include all the actions whatsoever that make
sense if \emph{p} can be taken as given. This, as we've seen many times
over, can't be right. If it were right it would entail either that some
actions that seem horribly reckless are in fact rational, or that an
absurd form of scepticism is true, and almost no actions are in fact
rational. If our pre-theoretic judgments of which actions are rational
is even close to being right, there must be limits to the power of
knowledge.

Are those limits sensitive to the interests of the would-be knower, or
are they independent of those interests? I've argued that they are
sensitive to the knower's interests, against what I called the orthodox
view that they are not. There are two primary reasons I've given for
this.

One reason is that the interest-relative answer, unlike the orthodox
answer, makes it clear why the boundary between knowledge and
non-knowledge matters. On the interest-relative view, the boundary
between knowledge and non-knowledge is philosophically and practically
significant. On the orthodox view, it's like the boundary between heavy
things and non-heavy things. How heavy something is matters a lot; but
which side of the heavy/non-heavy boundary it falls on does not.

According to orthodoxy, to know something is to have enough power to use
that knowledge to act rationally in, you know, a wide enough range of
cases. How wide? It's sort of wide enough. Why is it this range rather
than some other? Oh, no good reason. We just talk that way, and it's the
job of epistemologists to explain why we do. No, not explain - because
it's arbitrary and inexplicable. It's to describe the limits, limits
which are ultimately quite arbitrary. This is no good at all. The
heterodox interest-relative view has a better answer. The limits are set
just where they need to be to make it true that what this person knows
rationalises the actions that it should in fact rationalise.

A second reason for favoring the interest-relative view is that it makes
some core principles of action theory and decision theory turn out to be
correct, while the orthodox view makes them not quite right. Here are
the (versions of the) principles I have in mind.

\begin{description}
\tightlist
\item[Means-End Rationality]
If \emph{X} should be aiming for end \emph{E}, and \emph{X} knows both
that action \emph{A} is the only means to end \emph{E}, and that
\emph{A} will indeed lead to end \emph{E}, to then \emph{X} should
intend to perform \emph{A}.
\item[Strict Dominance Reasoning]
If \emph{X} has to choose between \emph{A} and \emph{B}, and there is
some partition \emph{P} of a space of possibilities such that \emph{X}
knows both that precisely one member of \emph{P} is actual, that
\emph{X}'s actions make no difference to which member is actual, and
that conditional on each member of \emph{P}, \emph{A} is better than
\emph{B}, then \emph{X} should prefer \emph{A} to \emph{B}.
\end{description}

Given orthodoxy, neither of these can be correct. In each case, \emph{X}
might know that \emph{A} will lead to a good outcome, but it might also
be probable enough that \emph{A} will lead to a disastrous outcome for
it to be irrational for \emph{X} to choose \emph{A}. That's absurd, and
so we should reject orthodoxy. The interest-relative view makes these,
and any number of related principles, turn out unrestrictedly
true.\footnote{Note that it's not much help to say that the principles
  are approximately true on the orthodox picture. In game theory we want
  to be able to iterate principles like this, and approximately true
  principles can't be applied iteratively.}

If these are the two reasons for adopting an interest-relative view, the
appeal of some kind of contextualist version of an interest-invariant
view should be clear. The contextualist can, and does, say that there is
something special about the boundary between knowledge and
non-knowledge. When one say that someone knows something, one means
(more or less) that their evidence for that thing is good enough for
one's own purposes. Any speaker of the language can truly say, ``When I
use `knows', it means what I need it to mean, neither more nor less.''
We could quibble here, but that's something like a solution to the
problem of arbitrariness.

The problems come with the principles like Means-End Rationality and
Strict Dominance Reasoning. These are both false on standard,
interest-invariant, contextualist views.\footnote{For the rest of this
  chapter, when I talk about contextualism, I mean the standard version
  of contextualism that does not entail that knowledge is
  interest-relative.} Contextualists are aware of this fact, and say
that they have something nearly as good: a meta-linguistic replacement.
Both principles are true if we replace talk about what \emph{X} knows
with talk about what \emph{X} can truly self-predicate `knowledge' of.
It doesn't matter whether \emph{X} knows that \emph{A} will lead to end
\emph{E}, but whether \emph{X} can say the words \emph{I know that
performing A will lead to end E}. But it's absurd to think that
fundamental principles of rationality should make reference to a
particular language, e.g., English, like this. So contextualism cannot
get us out of this jam. This is not to say that contextualism is false.
Maybe there are other reasons to believe in contextualism. It might be
supported by general principles about the way that names behave in
attitude ascriptions. Or it might be a consequence of the view that
`knows' is polysemous, a view I floated in a footnote in
Section~\ref{sec-neutrality}. Or it might be supported by reflection on
cases where people talk about others with lower evidential needs than
their own. I don't take a stance on these questions here; I'm just
arguing that contextualism doesn't save orthodoxy.

All these benefits of the interest-relative view are of no use if the
interest-relative view has even worse defects. And most of this book has
involved pushing back against some of the alleged defects of the view.
In some case, I've avoided possible criticisms by adopting a form of
interest-relative epistemology that doesn't allow the criticisms to take
hold. Many existing critiques of interest-relative epistemology focus on
forms of the view where being in a `high stakes' situation is relevant
to what one knows. Since my version of the view does not say that
interests matter iff they put the chooser in a high stakes situation,
those criticisms don't have any bite. Other critiques target a part-time
version of interest-relative epistemology, where some but not all key
notions are interest-relative. I used to endorse such a part-time
interest-relative theory, but eventually decided that the criticisms of
such a view were decisive.

Still, many criticisms do need replies, and much of this book has
consisted in replying to them. In Chapter~\ref{sec-belief} I replied to
(successful) criticisms of how I'd previously handled knowledge of
propositions that are not relevant to the thinker's current inquiries.
In Chapter~\ref{sec-inquiry} I replied to arguments that started with
the observation that it sometimes seems right to double check things
that we know. And in Chapter~\ref{sec-ties} I replied to arguments based
on how some versions of interest-relative epistemology (including the
version I'd previously endorsed) handled choices between nearly
indistinguishable options.

Many criticisms of interest-relative epistemology do not turn on
detailed engagement with how the view handles this or that case, but on
the very idea of interests being relevant to epistemology. One way you
see this is with arguments that just start from the implausibility of
two people who are alike in evidence differing with respect to
knowledge. But you also see it in the surprisingly common position in
the literature that there is some extra large burden of proof on
defenders of interest-relativity. It seems to be a common presupposition
that unless there is a super compelling argument that knowledge is
interest-relative, we should reject the idea that it is, even if there
is no equally compelling argument against the idea. I'm in general very
sceptical of burden of proof arguments; it always looks like an attempt
to win by bribing the umpires. But there is extra reason to be sceptical
of these particular burden of proof arguments. It's hard to even state a
principle about the (alleged) independence of knowledge and interests
without saying something false about double luck cases, or variable
reliability cases, or deviant causal chain cases.

We've known since 1963 (if not many centuries before) that knowledge
depends on more than just the evidential basis of the thing purportedly
known. Since then there has been a dizzying array of proposals about
what else knowledge might depend on. And on reflection, many of these
proposals have been very plausible. This book aims to defend one such
proposal, that knowledge depends on interests. In particular, how much
evidential support one needs to have knowledge depends on what inquiries
one is, and should be, engaged in. Given the tight relationships between
knowledge and rational action, adding this to the vast list of things
knowledge is sensitive to should not be surprising.

\bookmarksetup{startatroot}

\chapter*{References}\label{references}
\addcontentsline{toc}{chapter}{References}

\markboth{References}{References}

\phantomsection\label{refs}
\begin{CSLReferences}{1}{0}
\bibitem[\citeproctext]{ref-Adamson2015}
Adamson, Peter. (2015). \emph{Philosophy in the hellenistic and roman
worlds: A history of philosophy without any gaps, volume 2} (Oxford).
Oxford University Press.

\bibitem[\citeproctext]{ref-Adamson2019}
Adamson, Peter. (2019). \emph{Medieval philosophy: A history of
philosophy without any gaps, volume 4}. {O}xford {U}niversity {P}ress.

\bibitem[\citeproctext]{ref-AdamsonGaneri2020}
Adamson, Peter, and Jonardon Ganeri. (2020). \emph{Classical indian
philosophy: A history of philosophy without any gaps, volume 5}. Oxford
University Press.

\bibitem[\citeproctext]{ref-AndersonHawthorne2019a}
Anderson, Charity, and John Hawthorne. (2019a). Knowledge, practical
adequacy, and stakes. \emph{Oxford Studies in Epistemology}, \emph{6},
234--257.

\bibitem[\citeproctext]{ref-AndersonHawthorne2019b}
Anderson, Charity, and John Hawthorne. (2019b). Pragmatic encroachment
and closure. In Brian Kim \& Matthew McGrath (Eds.), \emph{Pragmatic
encroachment in epistemology} (107--115). Routledge.

\bibitem[\citeproctext]{ref-ArmourGarb2011}
Armour-Garb, B. (2011). Contextualism without pragmatic encroachment.
\emph{Analysis}, \emph{71}(4), 667--676.
doi:\href{https://doi.org/10.1093/analys/anr083}{10.1093/analys/anr083}

\bibitem[\citeproctext]{ref-Aumann1999}
Aumann, Robert J. (1999). Interactive epistemology i: knowledge.
\emph{International Journal of Game Theory}, \emph{28}(3), 263--300.
doi:\href{https://doi.org/10.1007/s001820050111}{10.1007/s001820050111}

\bibitem[\citeproctext]{ref-BasuSchroeder2019}
Basu, Rima, and Mark Schroeder. (2019). Doxastic wrongings. In Brian Kim
\& Matthew McGrath (Eds.), \emph{Pragmatic encroachment in epistemology}
(181--205). Routledge.

\bibitem[\citeproctext]{ref-Bennett2017}
Bennett, Karen. (2017). \emph{Making things up}. Oxford University
Press.

\bibitem[\citeproctext]{ref-Bhatt1999}
Bhatt, Rajesh. (1999). \emph{Covert modality in non-finite contexts}
(PhD thesis). University of Pennsylvania.

\bibitem[\citeproctext]{ref-Bird2004}
Bird, Alexander. (2004). Is evidence non-inferential?
\emph{Philosophical Quarterly}, \emph{54}(215), 252--265.
doi:\href{https://doi.org/10.1111/j.0031-8094.2004.00350.x}{10.1111/j.0031-8094.2004.00350.x}

\bibitem[\citeproctext]{ref-BlomeTillmann2009}
Blome-Tillmann, Michael. (2009). Contextualism, subject-sensitive
invariantism, and the interaction of {`knowledge'}-ascriptions with
modal and temporal operators. \emph{Philosophy and Phenomenological
Research}, \emph{79}(2), 315--331.
doi:\href{https://doi.org/10.1111/j.1933-1592.2009.00280.x}{10.1111/j.1933-1592.2009.00280.x}

\bibitem[\citeproctext]{ref-Boyd2015}
Boyd, Kenneth. (2016). Pragmatic encroachment and epistemically
responsible action. \emph{Synthese}, \emph{193}(9), 2721--2745.
doi:\href{https://doi.org/10.1007/s11229-015-0878-y}{10.1007/s11229-015-0878-y}

\bibitem[\citeproctext]{ref-Brittain2021}
Brittain, Charles, and Peter Osorio. (2021). {Philo of Larissa}. In
Edward N. Zalta (Ed.), \emph{The {Stanford} encyclopedia of philosophy}
({S}ummer 2021).
\url{https://plato.stanford.edu/archives/sum2021/entries/philo-larissa/};
Metaphysics Research Lab, Stanford University.

\bibitem[\citeproctext]{ref-Brown2008}
Brown, Jessica. (2008). Subject-sensitive invariantism and the knowledge
norm for practical reasoning. \emph{No{û}s}, \emph{42}(2), 167--189.
doi:\href{https://doi.org/10.1111/j.1468-0068.2008.00677.x}{10.1111/j.1468-0068.2008.00677.x}

\bibitem[\citeproctext]{ref-BuchakRisk}
Buchak, Lara. (2013). \emph{Risk and rationality}. Oxford University
Press.

\bibitem[\citeproctext]{ref-Caplin2011}
Caplin, Andrew, Mark Dean, and Daniel Martin. (2011). Search and
satisficing. \emph{American Economic Review}, \emph{101}(7), 2899--2922.
doi:\href{https://doi.org/10.1257/aer.101.7.2899}{10.1257/aer.101.7.2899}

\bibitem[\citeproctext]{ref-CarlssonVanDamme1993}
Carlsson, Hans, and Eric van Damme. (1993). Global games and equilibrium
selection. \emph{Econometrica}, \emph{61}(5), 989--1018.
doi:\href{https://doi.org/10.2307/2951491}{10.2307/2951491}

\bibitem[\citeproctext]{ref-Chakravarti2017}
Chakravarti, Ashok. (2017). Imperfect information and opportunism.
\emph{Journal of Economic Issues}, \emph{51}(4), 1114--1136.
doi:\href{https://doi.org/10.1080/00213624.2017.1391594}{10.1080/00213624.2017.1391594}

\bibitem[\citeproctext]{ref-ChernevEtAl2015}
Chernev, Alexander, Ulf Böckenholt, and Joseph Goodman. (2015). Choice
overload: A conceptual review and meta-analysis. \emph{Journal of
Consumer Psychology}, \emph{25}(2), 333--358.
doi:\href{https://doi.org/10.1016/j.jcps.2014.08.002}{10.1016/j.jcps.2014.08.002}

\bibitem[\citeproctext]{ref-Cherniak1986}
Cherniak, Christopher. (1986). \emph{Minimal rationality}. MIT Press.

\bibitem[\citeproctext]{ref-ChoKreps1987}
Cho, In-Koo, and David M. Kreps. (1987). Signalling games and stable
equilibria. \emph{The Quarterly Journal of Economics}, \emph{102}(2),
179--221. doi:\href{https://doi.org/10.2307/1885060}{10.2307/1885060}

\bibitem[\citeproctext]{ref-Christensen2005}
Christensen, David. (2005). \emph{Putting logic in its place}. Oxford
University Press.

\bibitem[\citeproctext]{ref-Christensen2007}
Christensen, David. (2007). Does murphy's law apply in epistemology?
Self-doubt and rational ideals. \emph{Oxford Studies in Epistemology},
\emph{2}, 3--31.

\bibitem[\citeproctext]{ref-Christensen2011}
Christensen, David. (2011). Disagreement, question-begging and epistemic
self-criticism. \emph{Philosophers' Imprint}, \emph{11}(6), 1--22.
Retrieved from \url{http://hdl.handle.net/2027/spo.3521354.0011.006}

\bibitem[\citeproctext]{ref-Christensen2019}
Christensen, David. (2019). Formulating independence. In Mattias Skipper
\& Asbjørn Steglich-Petersen (Eds.), \emph{Higher-order evidence: New
essays} (13--34). Oxford University Press.

\bibitem[\citeproctext]{ref-Clark2012}
Clark, Christopher. (2012). \emph{The sleepwalkers: How europe went to
war in 1914}. Harper Collins.

\bibitem[\citeproctext]{ref-Cohen2002}
Cohen, Stewart. (2002). Basic knowledge and the problem of easy
knowledge. \emph{Philosophy and Phenomenological Research},
\emph{65}(2), 309--329.
doi:\href{https://doi.org/10.1111/j.1933-1592.2002.tb00204.x}{10.1111/j.1933-1592.2002.tb00204.x}

\bibitem[\citeproctext]{ref-Cohen2004}
Cohen, Stewart. (2004). Knowledge, assertion, and practical reasoning.
\emph{Philosophical Issues}, \emph{14}(1), 482--491.
doi:\href{https://doi.org/10.1111/j.1533-6077.2004.00040.x}{10.1111/j.1533-6077.2004.00040.x}

\bibitem[\citeproctext]{ref-Conlisk1996}
Conlisk, John. (1996). Why bounded rationality? \emph{Journal of
Economic Literature}, \emph{34}(2), 669--700.

\bibitem[\citeproctext]{ref-sep-well-being}
Crisp, Roger. (2021). {Well-Being}. In Edward N. Zalta (Ed.), \emph{The
{Stanford} encyclopedia of philosophy} ({W}inter 2021). Metaphysics
Research Lab, Stanford University.

\bibitem[\citeproctext]{ref-sep-questions}
Cross, Charles, and Floris Roelofsen. (2018). Questions. In Edward N.
Zalta (Ed.), \emph{The stanford encyclopedia of philosophy} (Spring
2018).
\url{https://plato.stanford.edu/archives/spr2018/entries/questions/};
Metaphysics Research Lab, Stanford University.

\bibitem[\citeproctext]{ref-DasThesis}
Das, Nilanjan. (2016). \emph{Epistemic stability} (PhD thesis). {MIT}.

\bibitem[\citeproctext]{ref-DeRose2002}
DeRose, Keith. (2002). Assertion, knowledge and context.
\emph{Philosophical Review}, \emph{111}(2), 167--203.
doi:\href{https://doi.org/10.2307/3182618}{10.2307/3182618}

\bibitem[\citeproctext]{ref-DiabEtAl2008}
Diab, Dalia L., Michael A. Gillespie, and Scott Highhouse. (2008). Are
maximizers really unhappy? The measurement of maximizing tendency.
\emph{Judgment and Decision Making}, \emph{3}(5), 364--370. Retrieved
from \url{http://journal.sjdm.org/8320/jdm8320.pdf}

\bibitem[\citeproctext]{ref-Dogramaci2015}
Dogramaci, Sinan. (2015). Forget and forgive: A practical approach to
forgotten evidence. \emph{Ergo}, \emph{2}(26), 645--677.
doi:\href{https://doi.org/10.3998/ergo.12405314.0002.026}{10.3998/ergo.12405314.0002.026}

\bibitem[\citeproctext]{ref-Dylan2016}
Dylan, Bob. (2016). \emph{The lyrics: 1961-2012}. Simon \& Schuster.

\bibitem[\citeproctext]{ref-EatonPickavance2015}
Eaton, Daniel, and Timothy Pickavance. (2015). Evidence against
pragmatic encroachment. \emph{Philosophical Studies}, \emph{172},
3135--3143.
doi:\href{https://doi.org/10.1007/s11098-015-0461-x}{10.1007/s11098-015-0461-x}

\bibitem[\citeproctext]{ref-Egan2008}
Egan, Andy. (2008). {Seeing and Believing: Perception, Belief Formation
and the Divided Mind}. \emph{Philosophical Studies}, \emph{140}(1),
47--63.
doi:\href{https://doi.org/10.1007/s11098-008-9225-1}{10.1007/s11098-008-9225-1}

\bibitem[\citeproctext]{ref-Elster1979}
Elster, Jon. (1979). \emph{Ulysses and the sirens: Studies in
rationality and irrationality}. Cambridge University Press.

\bibitem[\citeproctext]{ref-Falbo2021}
Falbo, Arianna. (2021). Inquiry and confirmation. \emph{Analysis},
\emph{81}(4), 622--631.
doi:\href{https://doi.org/10.1093/analys/anab037}{10.1093/analys/anab037}

\bibitem[\citeproctext]{ref-FantlMcGrath2002}
Fantl, Jeremy, and Matthew McGrath. (2002). Evidence, pragmatics, and
justification. \emph{Philosophical Review}, \emph{111}(1), 67--94.
doi:\href{https://doi.org/10.2307/3182570}{10.2307/3182570}

\bibitem[\citeproctext]{ref-FantlMcGrath2009}
Fantl, Jeremy, and Matthew McGrath. (2009). \emph{Knowledge in an
uncertain world}. Oxford University Press.

\bibitem[\citeproctext]{ref-Foley1993}
Foley, Richard. (1993). \emph{Working without a net}. Oxford University
Press.

\bibitem[\citeproctext]{ref-Friedman2017}
Friedman, Jane. (2017). Why suspend judging? \emph{No{û}s},
\emph{51}(2), 302--326.
doi:\href{https://doi.org/10.1111/nous.12137}{10.1111/nous.12137}

\bibitem[\citeproctext]{ref-Friedman2019b}
Friedman, Jane. (2019a). Checking again. \emph{Philosophical Issues},
\emph{29}(1), 84--96.
doi:\href{https://doi.org/10.1111/phis.12141}{10.1111/phis.12141}

\bibitem[\citeproctext]{ref-Friedman2019a}
Friedman, Jane. (2019b). Inquiry and belief. \emph{No{û}s},
\emph{53}(2), 296--315.
doi:\href{https://doi.org/10.1111/nous.12222}{10.1111/nous.12222}

\bibitem[\citeproctext]{ref-Friedman2020}
Friedman, Jane. (2020). The epistemic and the zetetic.
\emph{Philosophical Review}, \emph{129}(4), 501--536.
doi:\href{https://doi.org/10.1215/00318108-8540918}{10.1215/00318108-8540918}

\bibitem[\citeproctext]{ref-Friedman2024debate}
Friedman, Jane. (2024a). Suspension of judgment is a question-directed
attitude. In Blake Roeber, Ernest Sosa, Matthias Steup, \& John Turri
(Eds.), \emph{Contemporary debates in epistemology} (3rd ed., 66--78).
Wiley Blackwell.

\bibitem[\citeproctext]{ref-Friedman2024}
Friedman, Jane. (2024b). The aim of inquiry? \emph{{P}hilosophy and
{P}henomenological {R}esearch}, \emph{108}(2), 506--523.
doi:\href{https://doi.org/10.1111/phpr.12982}{10.1111/phpr.12982}

\bibitem[\citeproctext]{ref-Ganson2008}
Ganson, Dorit. (2008). Evidentialism and pragmatic constraints on
outright belief. \emph{Philosophical Studies}, \emph{139}(3), 441--458.
doi:\href{https://doi.org/10.1007/s11098-007-9133-9}{10.1007/s11098-007-9133-9}

\bibitem[\citeproctext]{ref-Ganson2019}
Ganson, Dorit. (2019). Great expectations: Belief and the case for
pragmatic encroachment. In Brian Kim \& Matthew McGrath (Eds.),
\emph{Pragmatic encroachment in epistemology}. Routledge.

\bibitem[\citeproctext]{ref-Gao2023}
Gao, Jie. (2023). Should credence be sensitive to practical factors? A
cost-benefit analysis. \emph{Mind and Language}, \emph{38}(5),
1238--1257.
doi:\href{https://doi.org/10.1111/mila.12451}{10.1111/mila.12451}

\bibitem[\citeproctext]{ref-Gendler2005}
Gendler, Tamar Szabó, and John Hawthorne. (2005). The real guide to fake
barns: A catalogue of gifts for your epistemic enemies.
\emph{Philosophical Studies}, \emph{124}(3), 331--352.
doi:\href{https://doi.org/10.1007/s11098-005-7779-8}{10.1007/s11098-005-7779-8}

\bibitem[\citeproctext]{ref-Gettier1963}
Gettier, Edmund L. (1963). Is justified true belief knowledge?
\emph{Analysis}, \emph{23}(6), 121--123.
doi:\href{https://doi.org/10.2307/3326922}{10.2307/3326922}

\bibitem[\citeproctext]{ref-GigerenzerSelton2001}
Gigerenzer, Gerd, and Reinhard Selten. (2001). \emph{Bounded
rationality: The adaptive toolbox}. MIT Press.

\bibitem[\citeproctext]{ref-Gillies2010}
Gillies, Anthony S. (2010). Iffiness. \emph{Semantics and Pragmatics},
\emph{3}(4), 1--42.
doi:\href{https://doi.org/10.3765/sp.3.4}{10.3765/sp.3.4}

\bibitem[\citeproctext]{ref-Goldman2009}
Goldman, Alvin. (2009). Williamson on knowledge and evidence. In
\emph{{Williamson on Knowledge}} (73--91).

\bibitem[\citeproctext]{ref-Harman1973}
Harman, Gilbert. (1973). \emph{Thought}. Princeton University Press.

\bibitem[\citeproctext]{ref-Harman1986}
Harman, Gilbert. (1986). \emph{Change in view}. {MIT} Press.

\bibitem[\citeproctext]{ref-Harper1986}
Harper, William. (1986). Mixed strategies and ratifiability in causal
decision theory. \emph{Erkenntnis}, \emph{24}(1), 25--36.
doi:\href{https://doi.org/10.1007/BF00183199}{10.1007/BF00183199}

\bibitem[\citeproctext]{ref-Hawthorne2004}
Hawthorne, John. (2004). \emph{Knowledge and lotteries}. Oxford
University Press.

\bibitem[\citeproctext]{ref-Hawthorne2005}
Hawthorne, John. (2005). Knowledge and evidence. \emph{Philosophy and
Phenomenological Research}, \emph{70}(2), 452--458.
doi:\href{https://doi.org/10.1111/j.1933-1592.2005.tb00540.x}{10.1111/j.1933-1592.2005.tb00540.x}

\bibitem[\citeproctext]{ref-HawthorneEtAl2015}
Hawthorne, John, Daniel Rothschild, and Levi Spectre. (2016). Belief is
weak. \emph{Philosophical Studies}, \emph{173}, 1393--1404.
doi:\href{https://doi.org/10.1007/s11098-015-0553-7}{10.1007/s11098-015-0553-7}

\bibitem[\citeproctext]{ref-HawthorneSrinivasan2013}
Hawthorne, John, and Amia Srinivasan. (2013). Disagreement without
transparency: Some bleak thoughts. In David Christensen \& Jennifer
Lackey (Eds.), \emph{The epistemology of disagreement: New essays}
(9--30). Oxford University Press.

\bibitem[\citeproctext]{ref-HawthorneStanley2008}
Hawthorne, John, and Jason Stanley. (2008). {Knowledge and Action}.
\emph{Journal of Philosophy}, \emph{105}(10), 571--590.
doi:\href{https://doi.org/10.5840/jphil20081051022}{10.5840/jphil20081051022}

\bibitem[\citeproctext]{ref-Hedden2012}
Hedden, Brian. (2012). Options and the subjective ought.
\emph{Philosophical Studies}, \emph{158}(2), 343--360.
doi:\href{https://doi.org/10.1007/s11098-012-9880-0}{10.1007/s11098-012-9880-0}

\bibitem[\citeproctext]{ref-Hieronymi2013}
Hieronymi, Pamela. (2013). The use of reasons in thought (and the use of
earmarks in arguments). \emph{Ethics}, \emph{124}(1), 114--127.
doi:\href{https://doi.org/10.1086/671402}{10.1086/671402}

\bibitem[\citeproctext]{ref-Hills2009}
Hills, Alison. (2009). Moral testimony and moral epistemology.
\emph{Ethics}, \emph{120}(1), 94--127.
doi:\href{https://doi.org/10.1086/648610}{10.1086/648610}

\bibitem[\citeproctext]{ref-HollidayMandelkern2024}
Holliday, Wesley H., and Matthew Mandelkern. (2024). The orthologic of
epistemic modals. \emph{Journal of Philosophical Logic},
\emph{53}(831-907).
doi:\href{https://doi.org/s10992-024-09746-7}{s10992-024-09746-7}

\bibitem[\citeproctext]{ref-Hotelling1929}
Hotelling, Harold. (1929). Stability in competition. \emph{The Economic
Journal}, \emph{39}(153), 41--57.
doi:\href{https://doi.org/10.2307/2224214}{10.2307/2224214}

\bibitem[\citeproctext]{ref-Humberstone1981}
Humberstone, I. L. (1981). From worlds to possibilities. \emph{Journal
of Philosophical Logic}, \emph{10}(3), 313--339.
doi:\href{https://doi.org/10.1007/BF00293423}{10.1007/BF00293423}

\bibitem[\citeproctext]{ref-Humberstone2016}
Humberstone, Lloyd. (2016). \emph{Philsophical applications of modal
logic}. College Publications.

\bibitem[\citeproctext]{ref-Hunter1996}
Hunter, Daniel. (1996). On the relation between categorical and
probabilistic belief. \emph{No{û}s}, \emph{30}, 75--98.
doi:\href{https://doi.org/10.2307/2216304}{10.2307/2216304}

\bibitem[\citeproctext]{ref-Ichikawa2017}
Ichikawa, Jonathan. (2017). \emph{Contextualising knowledge}. {O}xford
{U}niversity {P}ress.

\bibitem[\citeproctext]{ref-IyengarEtAl2006}
Iyengar, Sheena S., Rachael E. Wells, and Barry Schwartz. (2006). Doing
better but feeling worse: Looking for the {`best'} job undermines
satisfaction. \emph{Psychological Science}, \emph{17}(2), 143--150.
doi:\href{https://doi.org/10.1111/j.1467-9280.2006.01677.x}{10.1111/j.1467-9280.2006.01677.x}

\bibitem[\citeproctext]{ref-Jackson1987}
Jackson, Frank. (1987). \emph{Conditionals}. Oxford.

\bibitem[\citeproctext]{ref-Joyce2018}
Joyce, James. (n.d.). \emph{Deliberation and stability in newcomb
problems and pseudo-newcomb problems}.

\bibitem[\citeproctext]{ref-Joyce1999}
Joyce, James M. (1999). \emph{The foundations of causal decision
theory}. Cambridge University Press.

\bibitem[\citeproctext]{ref-Kelly2010-KELPDA}
Kelly, Thomas. (2010). Peer disagreement and higher order evidence. In
Ted Warfield \& Richard Feldman (Eds.), \emph{Disagreement} (111--174).
Oxford University Press.

\bibitem[\citeproctext]{ref-Keynes1936Foxwell}
Keynes, John Maynard. (1936). Herbert somerton foxwell. \emph{The
Economic Journal}, \emph{46}(184), 589--619.
doi:\href{https://doi.org/10.2307/2224674}{10.2307/2224674}

\bibitem[\citeproctext]{ref-Keynes1937}
Keynes, John Maynard. (1937). The general theory of employment.
\emph{Quarterly Journal of Economics}, \emph{51}(2), 209--223.

\bibitem[\citeproctext]{ref-Kim2023}
Kim, Brian. (2023). Pragmatic infallibilism. \emph{Asian Journal of
Philosophy}, \emph{2}(2), 1--22.
doi:\href{https://doi.org/10.1007/s44204-023-00097-9}{10.1007/s44204-023-00097-9}

\bibitem[\citeproctext]{ref-Kimball2015}
Kimball, Miles. (2015). Cognitive economics. \emph{The Japanese Economic
Review}, \emph{66}(2), 167--181.
doi:\href{https://doi.org/10.1111/jere.12070}{10.1111/jere.12070}

\bibitem[\citeproctext]{ref-Knight1921}
Knight, Frank. (1921). \emph{Risk, uncertainty and profit}. University
of Chicago Press.

\bibitem[\citeproctext]{ref-KohlbergMertens1986}
Kohlberg, Elon, and Jean-Francois Mertens. (1986). On the strategic
stability of equilibria. \emph{Econometrica}, \emph{54}(5), 1003--1037.
doi:\href{https://doi.org/10.2307/1912320}{10.2307/1912320}

\bibitem[\citeproctext]{ref-Kratzer2012}
Kratzer, Angelika. (2012). \emph{Modals and conditionals}. Oxford
University Press.

\bibitem[\citeproctext]{ref-KripkeNozick}
Kripke, Saul. (2011). Nozick on knowledge. In \emph{Philosophical
troubles: Collected papers, volume 1} (161--224). {O}xford {U}niversity
{P}ress.

\bibitem[\citeproctext]{ref-Kroedel2012}
Kroedel, Thomas. (2012). The lottery paradox, epistemic justification
and permissibility. \emph{Analysis}, \emph{72}(1), 57--60.
doi:\href{https://doi.org/10.1093/analys/anr129}{10.1093/analys/anr129}

\bibitem[\citeproctext]{ref-Lasonen-Aarnio2010b}
Lasonen-Aarnio, Maria. (2010). Unreasonable knowledge.
\emph{Philosophical Perspectives}, \emph{24}, 1--21.
doi:\href{https://doi.org/10.1111/j.1520-8583.2010.00183.x}{10.1111/j.1520-8583.2010.00183.x}

\bibitem[\citeproctext]{ref-Lasonen-Aarnio2014}
Lasonen-Aarnio, Maria. (2014). Higher-order evidence and the limits of
defeat. \emph{Philosophy and Phenomenological Research}, \emph{88}(2),
314--345.
doi:\href{https://doi.org/10.1111/phpr.12090}{10.1111/phpr.12090}

\bibitem[\citeproctext]{ref-Lederman2018}
Lederman, Harvey. (2018). Two paradoxes of common knowledge: Coordinated
attack and electronic mail. \emph{No{û}s}, \emph{52}(4), 921--945.
doi:\href{https://doi.org/10.1111/nous.12186}{10.1111/nous.12186}

\bibitem[\citeproctext]{ref-Lee2017a}
Lee, Matthew. (2017a). Credence and correctness: In defense of credal
reductivism. \emph{Philosophical Papers}, \emph{46}(2), 273--296.
doi:\href{https://doi.org/10.1080/05568641.2017.1364142}{10.1080/05568641.2017.1364142}

\bibitem[\citeproctext]{ref-Lee2017b}
Lee, Matthew. (2017b). On the arbitrariness objection to the threshold
view. \emph{Dialogue}, \emph{56}(1), 143--158.
doi:\href{https://doi.org/10.1017/S0012217317000154}{10.1017/S0012217317000154}

\bibitem[\citeproctext]{ref-Lewis1969a}
Lewis, David. (1969). \emph{Convention: A philosophical study}. Harvard
University Press.

\bibitem[\citeproctext]{ref-Lewis1976b}
Lewis, David. (1976). Probabilities of conditionals and conditional
probabilities. \emph{Philosophical Review}, \emph{85}(3), 297--315.
doi:\href{https://doi.org/10.2307/2184045}{10.2307/2184045}

\bibitem[\citeproctext]{ref-Lewis1982c}
Lewis, David. (1982). Logic for equivocators. \emph{No{û}s},
\emph{16}(3), 431--441.
doi:\href{https://doi.org/10.1017/cbo9780511625237.009}{10.1017/cbo9780511625237.009}

\bibitem[\citeproctext]{ref-Lewis1986h}
Lewis, David. (1986). Probabilities of conditionals and conditional
probabilities {II}. \emph{Philosophical Review}, \emph{95}(4), 581--589.
doi:\href{https://doi.org/10.2307/2185051}{10.2307/2185051}

\bibitem[\citeproctext]{ref-Lewis1988}
Lewis, David. (1988). Desire as belief. \emph{Mind}, \emph{97}(387),
323--332.
doi:\href{https://doi.org/10.1093/mind/xcvii.387.323}{10.1093/mind/xcvii.387.323}

\bibitem[\citeproctext]{ref-Lewis1996}
Lewis, David. (1996). Desire as belief {II}. \emph{Mind},
\emph{105}(418), 303--313.
doi:\href{https://doi.org/10.1093/mind/105.418.303}{10.1093/mind/105.418.303}

\bibitem[\citeproctext]{ref-Lewis2004a}
Lewis, David. (2004). Causation as influence. In John Collins, Ned Hall,
\& L. A. Paul (Eds.), \emph{Causation and counterfactuals} (75--106).
{MIT} Press.

\bibitem[\citeproctext]{ref-LipseyLancaster}
Lipsey, R. G., and Kelvin Lancaster. (1956-1957). The general theory of
second best. \emph{Review of Economic Studies}, \emph{24}(1), 11--32.
doi:\href{https://doi.org/10.2307/2296233}{10.2307/2296233}

\bibitem[\citeproctext]{ref-Littlejohn2015}
Littlejohn, Clayton. (2018). Stop making sense? On a puzzle about
rationality. \emph{Philosophy and Phenomenological Research},
\emph{96}(2), 257--272.
doi:\href{https://doi.org/10.1111/phpr.12271}{10.1111/phpr.12271}

\bibitem[\citeproctext]{ref-MacFarlane2005-Knowledge}
MacFarlane, John. (2005). The assessment sensitivity of knowledge
attributions. \emph{Oxford Studies in Epistemology}, \emph{1}, 197--233.

\bibitem[\citeproctext]{ref-MachamerEtAl2000}
Machamer, Peter, Lindley Darden, and Carl F. Craver. (2000). Thinking
about mechanisms. \emph{Philosophy of Science}, \emph{67}(1), 1--25.
doi:\href{https://doi.org/10.1086/392759}{10.1086/392759}

\bibitem[\citeproctext]{ref-Maher1996}
Maher, Patrick. (1996). Subjective and objective confirmation.
\emph{Philosophy of Science}, \emph{63}(2), 149--174.
doi:\href{https://doi.org/10.1086/289906}{10.1086/289906}

\bibitem[\citeproctext]{ref-sep-abilities}
Maier, John. (2022). {Abilities}. In Edward N. Zalta \& Uri Nodelman
(Eds.), \emph{The {Stanford} encyclopedia of philosophy} ({F}all 2022).
Metaphysics Research Lab, Stanford University.

\bibitem[\citeproctext]{ref-MaitraWeatherson2010}
Maitra, Ishani, and Brian Weatherson. (2010). Assertion, knowledge and
action. \emph{Philosophical Studies}, \emph{149}(1), 99--118.
doi:\href{https://doi.org/10.1007/s11098-010-9542-z}{10.1007/s11098-010-9542-z}

\bibitem[\citeproctext]{ref-MandelkernEtAl2017}
Mandelkern, Matthew, Ginger Schultheis, and David Boylan. (2017).
Agentive modals. \emph{Philosophical Review}, \emph{126}(3), 301--343.
doi:\href{https://doi.org/10.1215/00318108-3878483}{10.1215/00318108-3878483}

\bibitem[\citeproctext]{ref-ManganEtAl2010}
Mangan, Jean, Amanda Hughes, and Kim Slack. (2010). Student finance,
information and decision making. \emph{Higher Education}, \emph{60}(5),
459--472.
doi:\href{https://doi.org/10.1007/s10734-010-9309-7}{10.1007/s10734-010-9309-7}

\bibitem[\citeproctext]{ref-Manski2017}
Manski, Charles F. (2017). Optimize, satisfice, or choose without
deliberation? A simple minimax-regret assessment. \emph{Theory and
Decision}, \emph{83}(2), 155--173.
doi:\href{https://doi.org/10.1007/s11238-017-9592-1}{10.1007/s11238-017-9592-1}

\bibitem[\citeproctext]{ref-McGrath2021}
McGrath, Matthew. (2021). Being neutral: Agnosticism, inquiry and the
suspension of judgment. \emph{No{û}s}, \emph{55}(2), 463--484.
doi:\href{https://doi.org/10.1111/nous.12323}{10.1111/nous.12323}

\bibitem[\citeproctext]{ref-McGrathKim2019}
McGrath, Matthew, and Brian Kim. (2019). Introduction. In Brian Kim \&
Matthew McGrath (Eds.), \emph{Pragmatic encroachment in epistemology}
(1--9). Routledge.

\bibitem[\citeproctext]{ref-Melchior2019}
Melchior, Guido. (2019). \emph{Knowing and checking: An epistemological
investigation}. Routledge.

\bibitem[\citeproctext]{ref-Mercier2020}
Mercier, Hugo. (2020). \emph{Not born yesterday: The science of who we
trust and what we believe}. Princeton University Press.

\bibitem[\citeproctext]{ref-Nagel2010}
Nagel, Jennifer. (2010). Epistemic anxiety and adaptive invariantism.
\emph{Philosophical Perspectives}, \emph{24}(1), 407--435.
doi:\href{https://doi.org/10.1111/j.1520-8583.2010.00198.x}{10.1111/j.1520-8583.2010.00198.x}

\bibitem[\citeproctext]{ref-Nagel2013-Williamson}
Nagel, Jennifer. (2013). Motivating williamson's model gettier cases.
\emph{Inquiry}, \emph{56}(1), 54--62.
doi:\href{https://doi.org/10.1080/0020174X.2013.775014}{10.1080/0020174X.2013.775014}

\bibitem[\citeproctext]{ref-Nagel2014}
Nagel, Jennifer. (2014). \emph{Knowledge: A very short introduction}.
Oxford University Press.

\bibitem[\citeproctext]{ref-Nair2019}
Nair, Shyam. (2019). Must good reasoning satisfy cumulative
transitivity? \emph{Philosophy and Phenomenological Research},
\emph{98}(1), 123--146.
doi:\href{https://doi.org/10.1111/phpr.12431}{10.1111/phpr.12431}

\bibitem[\citeproctext]{ref-Neta2007}
Neta, Ram. (2007). Anti-intellectualism and the knowledge-action
principle. \emph{Philosophy and Phenomenological Research},
\emph{75}(1), 180--187.
doi:\href{https://doi.org/10.1111/j.1933-1592.2007.00069.x}{10.1111/j.1933-1592.2007.00069.x}

\bibitem[\citeproctext]{ref-NewmanEtAl2018}
Newman, David B., Joanna Schug, Masaki Yuki, Junko Yamada, and John B.
Nezlek. (2018). The negative consequences of maximizing in friendship
selection. \emph{Journal of Personality and Social Psychology},
\emph{114}(5), 804--824.
doi:\href{https://doi.org/10.1037/pspp0000141}{10.1037/pspp0000141}

\bibitem[\citeproctext]{ref-North2010}
North, Jill. (2010). An empirical approach to symmetry and probability.
\emph{Studies In History and Philosophy of Science Part B: Studies In
History and Philosophy of Modern Physics}, \emph{41}(1), 27--40.
doi:\href{https://doi.org/10.1016/j.shpsb.2009.08.008}{10.1016/j.shpsb.2009.08.008}

\bibitem[\citeproctext]{ref-Nozick1981}
Nozick, Robert. (1981). \emph{Philosophical explorations}. Harvard
University Press.

\bibitem[\citeproctext]{ref-Odell2002}
Odell, John S. (2002). Bounded rationality and world political economy.
In David M. Andrews, C. Randall Henning, \& Louis W. Pauly (Eds.),
\emph{Governing the world's money} (168--193). Cornell University Press.

\bibitem[\citeproctext]{ref-OgakiTanaka2017}
Ogaki, Masao, and Saori C. Tanaka. (2017). \emph{Behavioral economics:
Toward a new economics by integration with traditional economics}.
Springer.

\bibitem[\citeproctext]{ref-Papi2013}
Papi, Mario. (2013). Satisficing and maximizing consumers in a
monopolistic screening model. \emph{Mathematical Social Sciences},
\emph{66}(3), 385--389.
doi:\href{https://doi.org/10.1016/j.mathsocsci.2013.08.005}{10.1016/j.mathsocsci.2013.08.005}

\bibitem[\citeproctext]{ref-Pasnau2017}
Pasnau, Robert. (2017). \emph{After certainty: A history of our
epistemic ideals and illusions}. Oxford University Press.

\bibitem[\citeproctext]{ref-Pingle2006}
Pingle, Mark. (2006). Deliberation cost as a foundation for behavioral
economics. In Morris Altman (Ed.), \emph{In handbook of contemporary
behavioral economics: Foundations and developments} (340--355).
Routledge.

\bibitem[\citeproctext]{ref-Pryor2004}
Pryor, James. (2004). What's wrong with moore's argument?
\emph{Philosophical Issues}, \emph{14}(1), 349--378.
doi:\href{https://doi.org/10.1111/j.1533-6077.2004.00034.x}{10.1111/j.1533-6077.2004.00034.x}

\bibitem[\citeproctext]{ref-Quiggin1982}
Quiggin, John. (1982). A theory of anticipated utility. \emph{Journal of
Economic Behavior \& Organization}, \emph{3}(4), 323--343.
doi:\href{https://doi.org/10.1016/0167-2681(82)90008-7}{10.1016/0167-2681(82)90008-7}

\bibitem[\citeproctext]{ref-Quong2017}
Quong, Jonathan. (2018). {Public Reason}. In Edward N. Zalta (Ed.),
\emph{The {Stanford} encyclopedia of philosophy} ({S}pring 2018).
\url{https://plato.stanford.edu/archives/spr2018/entries/public-reason/};
Metaphysics Research Lab, Stanford University.

\bibitem[\citeproctext]{ref-Railton1984}
Railton, Peter. (1984). Alienation, consequentialism and the demands of
morality. \emph{Philosophy and Public Affairs}, \emph{13}(2), 134--171.

\bibitem[\citeproctext]{ref-RamseyGeneralProp}
Ramsey, Frank. (1990). General propositions and causality. In D. H.
Mellor (Ed.), \emph{Philosophical papers} (145--163). Cambridge
University Press.

\bibitem[\citeproctext]{ref-Reutskaja2011}
Reutskaja, Elena, Rosemarie Nagel, Colin F. Camerer, and Antonio Rangel.
(2011). Search dynamics in consumer choice under time pressure: An
eye-tracking study. \emph{American Economic Review}, \emph{101}(2),
900--926.
doi:\href{https://doi.org/10.1257/aer.101.2.900}{10.1257/aer.101.2.900}

\bibitem[\citeproctext]{ref-Roberts2012}
Roberts, Craige. (2012). Information structure in discourse: Towards an
integrated formal theory of pragmatics. \emph{Semantics and Pragmatics},
\emph{5}(6), 1--69.
doi:\href{https://doi.org/10.3765/sp.5.6}{10.3765/sp.5.6}

\bibitem[\citeproctext]{ref-RossSchroeder2014}
Ross, Jacob, and Mark Schroeder. (2014). Belief, credence, and pragmatic
encroachment. \emph{Philosophy and Phenomenological Research},
\emph{88}(2), 259--288.
doi:\href{https://doi.org/10.1111/j.1933-1592.2011.00552.x}{10.1111/j.1933-1592.2011.00552.x}

\bibitem[\citeproctext]{ref-Rousseau1913}
Rousseau, Jean-Jacques. (1913). \emph{Social contract \& discourses} (G.
D. H. Cole, Trans.). J. M. Dent \& Sons.

\bibitem[\citeproctext]{ref-RussellDoris2008}
Russell, Gillian, and John M. Doris. (2009). Knowledge by indifference.
\emph{Australasian Journal of Philosophy}, \emph{86}(3), 429--437.
doi:\href{https://doi.org/10.1080/00048400802001996}{10.1080/00048400802001996}

\bibitem[\citeproctext]{ref-Russell1997}
Russell, Stuart J. (1997). Rationality and intelligence.
\emph{Artificial Intelligence}, \emph{94}(1-2), 57--77.
doi:\href{https://doi.org/10.1016/S0004-3702(97)00026-X}{10.1016/S0004-3702(97)00026-X}

\bibitem[\citeproctext]{ref-Savage1967}
Savage, Leonard. (1967). Difficulties in the theory of personal
probability. \emph{Philosophy of Science}, \emph{34}(4), 305--310.
doi:\href{https://doi.org/10.1086/288168}{10.1086/288168}

\bibitem[\citeproctext]{ref-ScheibehenneEtAl2010}
Scheibehenne, Benjamin, Rainer Greifeneder, and Peter M. Todd. (2010).
Can there ever be too many options? A meta-analytic review of choice
overload. \emph{Journal of Consumer Research}, \emph{37}(3), 409--425.
doi:\href{https://doi.org/10.1086/651235}{10.1086/651235}

\bibitem[\citeproctext]{ref-Schmidt2024}
Schmidt, Eva. (forthcoming). Reasons, attenuators, and virtue: A novel
account of pragmatic encroachment. \emph{Analytic Philosophy}, 1--22.
doi:\href{https://doi.org/10.1111/phib.12314}{10.1111/phib.12314}

\bibitem[\citeproctext]{ref-Schoenfield2013}
Schoenfield, Miriam. (2013). Permission to believe: Why permissivism is
true and what it tells us about irrelevant influences on belief.
\emph{No{û}s}, \emph{47}(1), 193--218.
doi:\href{https://doi.org/10.1111/nous.12006}{10.1111/nous.12006}

\bibitem[\citeproctext]{ref-Schroeder2009}
Schroeder, Mark. (2009). Means-end coherence, stringency, and subjective
reasons. \emph{Philosophical Studies}, \emph{143}(2), 223--248.
doi:\href{https://doi.org/10.1007/s11098-008-9200-x}{10.1007/s11098-008-9200-x}

\bibitem[\citeproctext]{ref-Schroeder2012}
Schroeder, Mark. (2012). Stakes, withholding and pragmatic encroachment
on knowledge. \emph{Philosophical Studies}, \emph{160}(2), 265--285.
doi:\href{https://doi.org/10.1007/s11098-011-9718-1}{10.1007/s11098-011-9718-1}

\bibitem[\citeproctext]{ref-Schwartz2004}
Schwartz, Barry. (2004). \emph{The paradox of choice: Why more is less}.
Harper Collins.

\bibitem[\citeproctext]{ref-SchwartzEtAl2002}
Schwartz, Barry, Andrew Ward, John Monterosso, Sonja Lyubomirsky,
Katherin White, and Darrin R. Lehman. (2002). Maximizing versus
satisficing: Happiness is a matter of choice. \emph{Journal of
Personality and Social Psychology}, \emph{83}(5), 1178--1197.
doi:\href{https://doi.org/10.1037/0022-3514.83.5.1178}{10.1037/0022-3514.83.5.1178}

\bibitem[\citeproctext]{ref-Skyrms2001}
Skyrms, Brian. (2001). The stag hunt. \emph{Proceedings and Addresses of
the American Philosophical Association}, \emph{75}(2), 31--41.
doi:\href{https://doi.org/10.2307/3218711}{10.2307/3218711}

\bibitem[\citeproctext]{ref-Sosa1999}
Sosa, Ernest. (1999). How to defeat opposition to moore.
\emph{Philosophical Perspectives}, \emph{13}, 141--153.
doi:\href{https://doi.org/10.1111/0029-4624.33.s13.7}{10.1111/0029-4624.33.s13.7}

\bibitem[\citeproctext]{ref-SperberEtAl2010}
Sperber, Dan, Fabrice Clément, Christophe Heintz, Olivier Mascaro, Hugo
Mercier, Gloria Origgi, and Deirdre Wilson. (2010). Epistemic vigilance.
\emph{Mind and Language}, \emph{25}(4), 359--393.
doi:\href{https://doi.org/10.1111/j.1468-0017.2010.01394.x}{10.1111/j.1468-0017.2010.01394.x}

\bibitem[\citeproctext]{ref-Staffel2019}
Staffel, Julia. (2019). How do beliefs simplify reasoning?
\emph{No{û}s}, \emph{53}(4), 937--962.
doi:\href{https://doi.org/10.1111/nous.12254}{10.1111/nous.12254}

\bibitem[\citeproctext]{ref-Stalnaker1975}
Stalnaker, Robert. (1975). Indicative conditionals. \emph{Philosophica},
\emph{5}(3), 269--289.
doi:\href{https://doi.org/10.1007/bf02379021}{10.1007/bf02379021}

\bibitem[\citeproctext]{ref-Stalnaker1984}
Stalnaker, Robert. (1984). \emph{Inquiry}. MIT Press.

\bibitem[\citeproctext]{ref-Stalnaker1994}
Stalnaker, Robert. (1994). On the evaluation of solution concepts.
\emph{Theory and Decision}, \emph{37}(1), 49--73.
doi:\href{https://doi.org/10.1007/BF01079205}{10.1007/BF01079205}

\bibitem[\citeproctext]{ref-Stalnaker1996}
Stalnaker, Robert. (1996). Knowledge, belief and counterfactual
reasoning in games. \emph{Economics and Philosophy}, \emph{12},
133--163.
doi:\href{https://doi.org/10.1017/S0266267100004132}{10.1017/S0266267100004132}

\bibitem[\citeproctext]{ref-Stalnaker1998}
Stalnaker, Robert. (1998). Belief revision in games: Forward and
backward induction. \emph{Mathematical Social Sciences}, \emph{36}(1),
31--56.
doi:\href{https://doi.org/10.1016/S0165-4896(98)00007-9}{10.1016/S0165-4896(98)00007-9}

\bibitem[\citeproctext]{ref-Stalnaker1999}
Stalnaker, Robert. (1999). Extensive and strategic forms: Games and
models for games. \emph{Research in Economics}, \emph{53}(3), 293--319.
doi:\href{https://doi.org/10.1006/reec.1999.0200}{10.1006/reec.1999.0200}

\bibitem[\citeproctext]{ref-Stanley2005}
Stanley, Jason. (2005). \emph{{Knowledge and Practical Interests}}.
Oxford University Press.

\bibitem[\citeproctext]{ref-Stanley2011}
Stanley, Jason. (2011). \emph{Know how}. Oxford University Press.

\bibitem[\citeproctext]{ref-Steglich-Petersen2024}
Steglich-Petersen, Asbjørn. (2024). An instrumentalist explanation of
pragmatic encroachment. \emph{Analytic Philosophy}, \emph{65}(3),
374--392.
doi:\href{https://doi.org/10.1111/phib.12283}{10.1111/phib.12283}

\bibitem[\citeproctext]{ref-Strevens2020}
Strevens, Michael. (2020). \emph{The knowledge machine: How
irrationality created modern science}. Liveright.

\bibitem[\citeproctext]{ref-Tucker2016}
Tucker, Chris. (2016). Satisficing and motivated submaximization (in the
philosophy of religion). \emph{Philosophy and Phenomenological
Research}, \emph{93}(1), 127--143.
doi:\href{https://doi.org/10.1111/phpr.12191}{10.1111/phpr.12191}

\bibitem[\citeproctext]{ref-Unger1975}
Unger, Peter. (1975). \emph{Ignorance: A case for scepticism}. Oxford
University Press.

\bibitem[\citeproctext]{ref-Weatherson2005-WEACWD}
Weatherson, Brian. (2005a). {Can We Do Without Pragmatic Encroachment?}
\emph{Philosophical Perspectives}, \emph{19}(1), 417--443.
doi:\href{https://doi.org/10.1111/j.1520-8583.2005.00068.x}{10.1111/j.1520-8583.2005.00068.x}

\bibitem[\citeproctext]{ref-Weatherson2005b}
Weatherson, Brian. (2005b). True, truer, truest. \emph{Philosophical
Studies}, \emph{123}(1-2), 47--70.
doi:\href{https://doi.org/10.1007/s11098-004-5218-x}{10.1007/s11098-004-5218-x}

\bibitem[\citeproctext]{ref-Weatherson2011-WEADIR}
Weatherson, Brian. (2011). Defending interest-relative invariantism.
\emph{Logos \& Episteme}, \emph{2}(4), 591--609.
doi:\href{https://doi.org/10.5840/logos-episteme2011248}{10.5840/logos-episteme2011248}

\bibitem[\citeproctext]{ref-Weatherson2012}
Weatherson, Brian. (2012). Knowledge, bets and interests. In Jessica
Brown \& Mikkel Gerken (Eds.), \emph{Knowledge ascriptions} (75--103).
Oxford University Press.

\bibitem[\citeproctext]{ref-Weatherson2016}
Weatherson, Brian. (2016a). Games, beliefs and credences.
\emph{Philosophy and Phenomenological Research}, \emph{92}(2), 209--236.
doi:\href{https://doi.org/10.1111/phpr.12088}{10.1111/phpr.12088}

\bibitem[\citeproctext]{ref-Weatherson2016-WEARTE}
Weatherson, Brian. (2016b). Reply to eaton and pickavance.
\emph{Philosophical Studies}, \emph{173}(12), 3231--3233.

\bibitem[\citeproctext]{ref-Weatherson2017-WEAII}
Weatherson, Brian. (2017). Interest-relative invariantism. In Jonathan
Ichikawa (Ed.), \emph{The routledge handbook of epistemic contextualism}
(240--253). Routledge.

\bibitem[\citeproctext]{ref-Weatherson2018-WEAIEA-2}
Weatherson, Brian. (2018). Interests, evidence and games.
\emph{Episteme}, \emph{15}(3), 329--344.

\bibitem[\citeproctext]{ref-Weatherson2019}
Weatherson, Brian. (2019). \emph{Normative externalism}. Oxford
University Press.

\bibitem[\citeproctext]{ref-Weisberg2010}
Weisberg, Jonathan. (2010). Bootstrapping in general. \emph{Philosophy
and Phenomenological Research}, \emph{81}(3), 525--548.
doi:\href{https://doi.org/10.1111/j.1933-1592.2010.00448.x}{10.1111/j.1933-1592.2010.00448.x}

\bibitem[\citeproctext]{ref-Weisberg2013}
Weisberg, Jonathan. (2013). Knowledge in action. \emph{Philosophers'
Imprint}, \emph{13}(22), 1--23.
doi:\href{https://doi.org/2027/spo.3521354.0013.022}{2027/spo.3521354.0013.022}

\bibitem[\citeproctext]{ref-Weisberg2020}
Weisberg, Jonathan. (2020). Belief in psyontology. \emph{Philosophers'
Imprint}, xx--xx.

\bibitem[\citeproctext]{ref-White2005-WHIEP}
White, Roger. (2005). Epistemic permissiveness. \emph{Philosophical
Perspectives}, \emph{19}, 445--459.
doi:\href{https://doi.org/10.1111/j.1520-8583.2005.00069.x}{10.1111/j.1520-8583.2005.00069.x}

\bibitem[\citeproctext]{ref-Williams1976}
Williams, Peter. (1976). Indeterminate probabilities. In M. Przelecki,
K. Szaniawski, \& R. Wojcicki (Eds.), \emph{Formal methods in the
methodology of the empriccal sciences} (229--246). Reidel.

\bibitem[\citeproctext]{ref-Williamson1994}
Williamson, Timothy. (1994). \emph{{Vagueness}}. Routledge.

\bibitem[\citeproctext]{ref-Williamson2000}
Williamson, Timothy. (2000). \emph{{Knowledge and its Limits}}. Oxford
University Press.

\bibitem[\citeproctext]{ref-Williamson2005}
Williamson, Timothy. (2005). {Contextualism, Subject-Sensitive
Invariantism and Knowledge of Knowledge}. \emph{The Philosophical
Quarterly}, \emph{55}(219), 213--235.
doi:\href{https://doi.org/10.1111/j.0031-8094.2005.00396.x}{10.1111/j.0031-8094.2005.00396.x}

\bibitem[\citeproctext]{ref-Williamson2007}
Williamson, Timothy. (2007). How probable is an infinite sequence of
heads? \emph{Analysis}, \emph{67}(295), 173--180.
doi:\href{https://doi.org/10.1111/j.1467-8284.2007.00671.x}{10.1111/j.1467-8284.2007.00671.x}

\bibitem[\citeproctext]{ref-WilliamsonLofoten}
Williamson, Timothy. (2013). Gettier cases in epistemic logic.
\emph{Inquiry}, \emph{56}(1), 1--14.
doi:\href{https://doi.org/10.1080/0020174X.2013.775010}{10.1080/0020174X.2013.775010}

\bibitem[\citeproctext]{ref-Williamson2022}
Williamson, Timothy. (2022). Knowledge, credence, and the strength of
belief. In Amy Flowerree \& Baron Reed (Eds.), \emph{Expansive
epistemology: Norms, action, and the social world} (xx--xx). Routledge.

\bibitem[\citeproctext]{ref-Wittgenstein1953}
Wittgenstein, Ludwig. (1953). \emph{Philosophical investigations}.
Macmillan.

\bibitem[\citeproctext]{ref-Woodard2021}
Woodard, Elise. (2020). \emph{Why double-check}.

\bibitem[\citeproctext]{ref-Wright2002}
Wright, Crispin. (2002). (Anti-)sceptics simple and subtle: G.e. Moore
and john McDowell. \emph{Philosophy and Phenomenological Research},
\emph{65}(2), 330--348.
doi:\href{https://doi.org/10.1111/j.1933-1592.2002.tb00205.x}{10.1111/j.1933-1592.2002.tb00205.x}

\bibitem[\citeproctext]{ref-Wright2018}
Wright, Crispin. (2018). \emph{A plague on all your houses: Some
reflections on the variable behaviour of "knows"} (A. Coliva, P.
Leonardi, \& S. Moruzzi, Eds.). Palgrave Macmillan.

\bibitem[\citeproctext]{ref-Wu2024}
Wu, Jenny Yi-Chen. (forthcoming). A defense of impurist permissivism.
\emph{Episteme}, 1--21.
doi:\href{https://doi.org/10.1017/epi.2023.22}{10.1017/epi.2023.22}

\bibitem[\citeproctext]{ref-Yalcin2018}
Yalcin, Seth. (2018). Belief as question-sensitive. \emph{Philosophy and
Phenomenological Research}, \emph{97}(1), 23--47.
doi:\href{https://doi.org/10.1111/phpr.12330}{10.1111/phpr.12330}

\bibitem[\citeproctext]{ref-Yalcin2021}
Yalcin, Seth. (2021). Fragmented but rational. In Christina Borgoni,
Dirk Kindermann, \& Andrea Onofori (Eds.), \emph{The fragmented mind}
(156--179). Oxford University Press.

\bibitem[\citeproctext]{ref-Ye2024}
Ye, Ru. (forthcoming). Knowledge-action principles and
threshold-impurism. \emph{Erkenntnis}, 1--18.
doi:\href{https://doi.org/10.1007/s10670-022-00626-7}{10.1007/s10670-022-00626-7}

\bibitem[\citeproctext]{ref-Zweber2016}
Zweber, Adam. (2016). Fallibilism, closure, and pragmatic encroachment.
\emph{Philosophical Studies}, \emph{173}(10), 2745--2757.
doi:\href{https://doi.org/10.1007/s11098-016-0631-5}{10.1007/s11098-016-0631-5}

\end{CSLReferences}


\backmatter

\end{document}
